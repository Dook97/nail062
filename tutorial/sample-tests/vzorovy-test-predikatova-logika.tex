\documentclass[a4paper]{article}
\usepackage{a4wide}
\usepackage{amssymb}
\usepackage[utf8]{inputenc}
\usepackage[czech]{babel}

\begin{document}

\begin{center}
    \large{Vzorový zápočtový test: predikátová logika}    
\end{center}

Časový limit: 45 minut. Celkem bodů: 100.

\bigskip

\begin{enumerate}

\item Víme, že:
\begin{enumerate}
    \item[$(i)$] Aristoteles je Řek, César je Říman a Didó je Kartáginka.
    \item[$(ii)$] Žádný Řek není Říman.
    \item[$(iii)$] Žádný Kartáginec není Řek.
    \item[$(iv)$] V Kartágu se narodili pouze Kartáginci.
\end{enumerate}
    
Pomocí rezoluce chceme dokázat, že
    
\begin{enumerate}
    \item[$(v)$] Existuje někdo, kdo se nenarodil v Kartágu a není to Říman.
\end{enumerate}
    
Konkrétně:
\begin{enumerate}
\item Uvedená tvrzení vyjádřete \underline{sentencemi} $\varphi_1, \dots, \varphi_5$ v jazyce $L=\langle R, M, K, N, a, c, d \rangle$ bez rovnosti, kde $R, M, K, N$ jsou unární relační symboly a $R(x), M(x), K(x)$ resp. $N(x)$ znamenají (po řadě) ``$x$ je Řek / Říman / Kartágin[ec/ka]'' resp. ``$x$ se narodil v Kartágu'', a $a, c, d$ jsou konstanty označující Aristotela, Césara, Didó. {\it (15b)}
    \item Pomocí skolemizace nalezněte otevřenou teorii $T$ (případně ve větším jazyce), která je nesplnitelná, právě když  $\{\varphi_1, \varphi_2, \varphi_3, \varphi_4\} \models \varphi_5$. Převeďte $T$ do CNF a napište ji v množinové reprezentaci. {\it (10b)}
    \item Rezolucí dokažte, že $T$ není splnitelná. Rezoluční zamítnutí znázorněte rezolučním stromem. U každého kroku uveďte použitou unifikaci. {\it (20b)}
    %\item Nalezněte konjunkci základních instancí axiomů $T$, která je nesplnitelná. {\it (2b)}
    %\item Vyplývá z tvrzení $(i)$ až $(iv)$, že ``César se nenarodil v Kartágu''? Uveďte zdůvodnění. {\it (2b)}
    \end{enumerate}

\smallskip

\item Nechť $T=\{(\exists x)(P(x)\to Q(x)),\ (\exists x)(\neg R(x)\to \neg Q(x))\}$ je teorie jazyka $L=\langle P,Q,R\rangle$ bez rovnosti, kde $P,Q,R$ jsou unární relační symboly, a označme $\varphi$ sentenci $(\exists x)(P(x) \to R(x))$.
\begin{enumerate}
    \item Zkonstruujte dokončené tablo z teorie $T$ s položkou $F\varphi$ v kořeni. {\it (25b)}
    \item Je $\varphi$ pravdivá v $T$? Je lživá v $T$? Je nezávislá v $T$? Zdůvodněte všechny odpovědi. {\it (10b)}
    \item Má teorie $T$ konzervativní kompletní extenzi? Uveďte příklad nebo zdůvodněte, proč ne. {\it (10b)}
\end{enumerate}

\smallskip

\item Nechť $\mathcal{A}=\langle\mathbb{Z},\mathrm{abs}^A \rangle$ je struktura jazyka $L=\langle \mathrm{abs} \rangle$ s rovností, kde $\mathrm{abs}$ je unární funkční symbol a $\mathrm{abs}^A$ je funkce absolutní hodnoty v $\mathbb{Z}$. Najděte příklad netriviální (t.j. jiné než $\emptyset$ a $\mathbb{Z}$) množiny definovatelné v $\mathcal{A}$ bez parametrů. Uveďte definující formuli. (10 bodů)
\end{enumerate}

\end{document} 