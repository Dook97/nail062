\documentclass[a4paper,12pt]{article}

\usepackage{a4wide}
\usepackage{amsmath}
\usepackage{amssymb}
\usepackage{amsthm}
\usepackage[czech]{babel}
\usepackage{bookmark}
\usepackage{enumerate}
\usepackage[T1]{fontenc}
\usepackage{forest}
\usepackage{hyperref}
\usepackage[utf8]{inputenc}
\usepackage{lmodern}
\usepackage{multicol}
\usepackage{tikz}

\theoremstyle{definition}
    \newtheorem{problem}{Příklad}

% \theoremstyle{remark}
%     \newtheorem*{steps}{Postup řešení}

\theoremstyle{plain}
    \newtheorem*{solution}{Řešení}
    

\DeclareRobustCommand\proves{\mathrel{|}\joinrel\mkern-.5mu\mathrel{-}}
\DeclareMathOperator{\Conseq}{Csq}
\DeclareMathOperator{\M}{M}

% hide solutions
\newif\ifhidesolutions
    \hidesolutionstrue
    % \hidesolutionsfalse

\ifhidesolutions
    \usepackage{environ}
    \NewEnviron{hide}{}
    \let\solution\hide
    \let\endsolution\endhide
\fi








\begin{document}

\section*{NAIL062 V\&P Logika: 2. cvičení}
% po 1. přednášce

% podzim 2023: 1, 2a, 2d (částečně)

\textbf{Témata:} 
Formalizace ve výrokové logice. Syntaxe a sémantika výrokové logiky. Ukázka tablo metody a rezoluční metody.


\medskip\begin{problem}
Uvažme následující tvrzení:
\begin{itemize}\it
\item Ten, kdo je dobrý běžec a má dobrou kondici, uběhne maraton.
\item Ten, kdo nemá štěstí a nemá dobrou kondici, neuběhne maraton.
\item Ten, kdo uběhne maraton, je dobrý běžec.
\item Budu-li mít štěstí, uběhnu maraton.
\item Mám dobrou kondici.
\end{itemize}
\begin{enumerate}
\item Formalizujte tato tvrzení jako teorii $T$ ve výrokové logice v jazyce $L=\langle b, k, m, s\rangle$, kde výrokové proměnné mají po řadě význam ``být dobrý běžec'', ``mít dobrou kondici'', ``uběhnout maraton'' a ``mít štěstí''.
\item Najděte všechny modely teorie $T$. Pokuste se využít k tomu \emph{tablo}.
\item Napište několik různých důsledků teorie $T$.
\item Najděte CNF teorii ekvivalentní teorii $T$.
\item Výrok je v \emph{disjunktivní normální formě (DNF)}, je-li disjunkcí konjunkcí literálů. Najděte DNF výrok ekvivalentní teorii $T$. (Pokuste se najít co nejkratší.)
\end{enumerate}
\end{problem}


\medskip\begin{problem}
Uvažme \emph{vrcholová pokrytí} následujícího grafu:

\begin{center}
    \begin{tikzpicture}[every node/.style={circle,fill=blue!10,draw,minimum size=0.5cm,node distance=1.5cm}]
    \node (1) {$1$};
    \node[right of=1] (2) {$2$};
    \node[below of=2] (3) {$3$};
    \node[left of=3] (4) {$4$};
    \path[draw] (1) -- (2) -- (3) -- (4) -- (1) -- (3);
    \node[right of=3] (5) {$5$};
    \path[draw] (2) -- (5) -- (3);
    \end{tikzpicture}
\end{center}

\begin{enumerate}
    \item Formalizujte ve výrokové logice problém, zda graf na obrázku má nejvýše $k$-prvkové vrcholové pokrytí, pro pevně zvolené $k$. Označme výslednou teorii jako $T_k$.
    \item Ukažte, že $T_2$ nemá žádné modely, tj. graf nemá 2-prvkové vrcholové pokrytí.
    \item Uměli byste k tomu využít tablo metodu? Připomeňte si ji.
    \item Uměli byste k tomu využít rezoluční metodu? Připomeňte si ji.
    \item Najděte všechna 3-prvková vrcholová pokrytí.
\end{enumerate}
\end{problem}


\medskip\begin{problem}
    Sestrojte strom výrazu resp. výroku, zapište v prefixovém, infixovém a postfixovém formátu:
    \begin{enumerate}
        \item $(3+5)*(-2)+(2*3)$
        \item $(p \to q) \leftrightarrow \neg (p \wedge \neg q)$
        \item $(p \leftrightarrow q) \leftrightarrow ((p \vee q) \to (p \wedge q))$
    \end{enumerate}
\end{problem}


\medskip\begin{problem}
Sestrojte pravdivostní tabulky a Vennův diagram pro následující výrokové formule. Najděte jejich množiny modelů. Které z nich jsou tautologie?%\footnote{Venn in doubt, draw a diagram.}
\begin{enumerate}
\item $(p \to q) \leftrightarrow \neg p \vee q$
\item $(p \to q) \leftrightarrow \neg (p \wedge \neg q)$
\item $((p\to q)\to p)\to p$
\item $\neg (p\vee q)\leftrightarrow \neg p\wedge \neg q$
\end{enumerate}
\end{problem}


\medskip\begin{problem}
    Uveďte příklad výroku v jazyce $\mathbb P=\{p,q,r\}$, který
    \begin{enumerate}
    \item je pravdivý,
    \item je sporný,
    \item je nezávislý,
    \item je ekvivalentní s, ale různý od, výroku $(p\wedge q)\to\neg r$,
    \item má za modely právě $\{(1,0,0),(1,0,1),(0,0,1)\}$.
    \end{enumerate}
\end{problem}


\end{document}