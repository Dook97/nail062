\documentclass[a4paper]{article}

\usepackage{a4wide}
\usepackage{amsmath}
\usepackage{amssymb}
\usepackage{amsthm}
\usepackage{enumitem}
    \setlist[enumerate]{label=(\alph*),itemsep=3pt,topsep=6pt}
    \setlist[itemize]{itemsep=3pt,topsep=6pt}
\usepackage{tikz}
\usepackage[utf8]{inputenc}


\theoremstyle{definition}
\newtheorem{problem}{Příklad}
\newtheorem*{ukol}{Domácí úkol}


\begin{document}

\section*{NAIL062 V\&P Logika: 7. cvičení}
% po 5. přednášce


\textbf{Témata:}
(Zápočtový test z výrokové logiky.) Syntaxe a sémantika predikátové logiky.


\medskip\begin{problem}
    Určete volné a vázané výskyty proměnných v následujících formulích. Poté je převeďte na varianty, ve kterých nebudou proměnné s volným i vázaným výskytem zároveň.
    \begin{enumerate}
       \item $(\exists x)(\forall y)P(y,z) \vee (y=0)$
       \item $(\exists x)(P(x) \wedge (\forall x)Q(x)) \vee (x=0)$
       \item $(\exists x)(x>y) \wedge (\exists y)(y>x)$
    \end{enumerate}
    \end{problem}
    
    \medskip\begin{problem}
    Označme $\varphi$ formuli $(\forall x)((x=z) \vee (\exists y)(f(x)=y) \vee (\forall z)(y=f(z)))$. Které z následujících termů jsou substituovatelné do $\varphi$?
    \begin{enumerate}
       \item term $z$ za proměnnou $x$, term $y$ za proměnnou $x$,
       \item term $z$ za proměnnou $y$, term $2*y$ za proměnnou $y$,
       \item term $x$ za proměnnou $z$, term $y$ za proměnnou $z$,
    \end{enumerate}
\end{problem}
    
\medskip\begin{problem}
    Jsou následující formule variantami formule $(\forall x)(x<y \vee (\exists z)(z=y \wedge z\ne x))$?
    \begin{enumerate}
    \item $(\forall z)(z<y \vee (\exists z)(z=y \wedge z\ne z))$
    \item $(\forall y)(y<y \vee (\exists z)(z=y \wedge z\ne y))$
    \item $(\forall u)(u<y \vee (\exists z)(z=y \wedge z\ne u))$
    \end{enumerate}
\end{problem}
    
    
\medskip\begin{problem}
    Mějme strukturu $\mathcal{A}=(\{a,b,c,d\},\vartriangleright^{A})$ v jazyce s jediným binárním relačním symbolem $\vartriangleright$, kde $\vartriangleright^{A}=\{(a,c), (b,c), (c,c), (c,d)\}$. 
    \begin{itemize}
        \item Které z následujících formulí jsou pravdivé v $\mathcal A$? 
        \item Pro každou formuli najděte strukturu $\mathcal{B}$ (existuje-li) takovou, že $\mathcal{B}\models \varphi$ právě když $\mathcal{A}\not\models \varphi$.
    \end{itemize}    
    \begin{enumerate}
       \item $x \vartriangleright y$
       \item $(\exists x)(\forall y)(y \vartriangleright x)$
    %   \item $(\exists x)(\forall y)(x \ntriangleright y)$, kde podformule $x \ntriangleright y$ je zkratkou za $\neg(x \vartriangleright y)$
       \item $(\exists x)(\forall y)((y \vartriangleright x) \to (x \vartriangleright x))$
    %   \item $(\forall x)(\exists y)((y \ntriangleright x)\to(x \vartriangleright y))$
       \item $(\forall x)(\forall y)(\exists z)((x \vartriangleright z)\wedge(z \vartriangleright y))$
       \item $(\forall x)(\exists y)((x \vartriangleright z)\vee(z \vartriangleright y))$
    %   \item $(x \ntriangleright z) \vee (\exists y)(y \ntriangleright z)$
    \end{enumerate}
\end{problem}
    
    
\medskip\begin{problem}
    Jsou následující sentence pravdivé / lživé / nezávislé (v logice)?
    \begin{enumerate}
     \item $(\exists x)(\forall y)(P(x) \vee \neg P(y))$
    \item $(\forall x)(P(x)\to Q(f(x))) \wedge (\forall x)P(x) \wedge (\exists x)\neg Q(x)$
      \item $(\forall x)(P(x) \vee Q(x)) \to ((\forall x)P(x) \vee (\forall x)Q(x))$
       \item $(\forall x)(P(x)\to Q(x)) \to ((\exists x)P(x)\to(\exists x)Q(x))$
    \item $(\exists x)(\forall y)P(x,y) \to (\forall y)(\exists x)P(x,y)$
    %\item $(\forall x)(P(x,z) \to Q(x,y)) \to ((\forall x)(P(x,z) \to (\forall x)Q(x,y))$
    \end{enumerate}
\end{problem}
    
    
\medskip\begin{problem}
    Dokažte (sémanticky) nebo najděte protipříklad: Pro každou strukturu $\mathcal{A}$, formuli $\varphi$, a sentenci $\psi$,
    \begin{enumerate}
    \item $\mathcal{A}\models (\psi \to (\exists x)\varphi) \Leftrightarrow \mathcal{A}\models (\exists x)(\psi \to \varphi)$
    \item $\mathcal{A}\models (\psi \to (\forall x)\varphi) \Leftrightarrow \mathcal{A}\models (\forall x)(\psi \to \varphi)$
    \item $\mathcal{A}\models ((\exists x)\varphi \to \psi) \Leftrightarrow \mathcal{A}\models (\forall x)(\varphi \to \psi)$
    \item $\mathcal{A}\models ((\forall x)\varphi \to \psi ) \Leftrightarrow \mathcal{A}\models (\exists x)(\varphi \to \psi)$
    \end{enumerate}
    Platí to i pro každou formuli $\psi$ s volnou proměnnou $x$? A pro každou formuli $\psi$ ve které $x$ není volná?
\end{problem}
    

\medskip\begin{problem}
    Rozhodněte, zda následující platí pro každou formuli $\varphi$. Dokažte (sémanticky, z definic) nebo najděte protipříklad.
    \begin{enumerate}
       \item $\varphi \models (\forall x)\varphi$
       \item $\models \varphi \to (\forall x)\varphi$
       \item $\varphi \models (\exists x)\varphi$
       \item $\models \varphi \to (\exists x)\varphi$
    \end{enumerate}
\end{problem}
    
    
\medskip\begin{problem}
    Buď $L=\langle +, -, 0\rangle$ jazyk teorie grup (s rovností). Teorie grup $T$ sestává z těchto axiomů:
    \begin{align*}
    x+(y+z)&=(x+y)+z\\
    0+x&=x=x+0\\
    x+(-x)&=0=(-x)+x
    \end{align*}
    Rozhodněte, zda jsou následující formule pravdivé / lživé / nezávislé v $T$. Zdůvodněte.
    \begin{enumerate}
        \item $x+y=y+x$
        \item $x+y=x\ \rightarrow\ y=0$
        \item $x+y=0\ \rightarrow\ y=-x$
        \item $-(x+y)=(-y)+(-x)$
    \end{enumerate}
\end{problem}


\medskip\begin{ukol}
Tentokrát žádný není. Místo toho řešte příklady zbývající ze cvičení.
\end{ukol}

\end{document}