\documentclass[a4paper,12pt]{article}

\usepackage{a4wide}
\usepackage{amsmath}
\usepackage{amssymb}
\usepackage{amsthm}
\usepackage[czech]{babel}
\usepackage{bookmark}
\usepackage{enumerate}
\usepackage[T1]{fontenc}
\usepackage{forest}
\usepackage{hyperref}
\usepackage[utf8]{inputenc}
\usepackage{lmodern}
\usepackage{multicol}
\usepackage{tikz}

\theoremstyle{definition}
    \newtheorem{problem}{Příklad}

% \theoremstyle{remark}
%     \newtheorem*{steps}{Postup řešení}

\theoremstyle{plain}
    \newtheorem*{solution}{Řešení}
    

\DeclareRobustCommand\proves{\mathrel{|}\joinrel\mkern-.5mu\mathrel{-}}
\DeclareMathOperator{\Conseq}{Csq}
\DeclareMathOperator{\M}{M}

% hide solutions
\newif\ifhidesolutions
    \hidesolutionstrue
    % \hidesolutionsfalse

\ifhidesolutions
    \usepackage{environ}
    \NewEnviron{hide}{}
    \let\solution\hide
    \let\endsolution\endhide
\fi









\begin{document}

\section*{NAIL062 V\&P Logika: 7. cvičení}
% po 5. přednášce

% 2023: 2a, 3, 4, 6b (částečně), 7 včetně konstrukce zamítnutí (z důkazu úplnosti), 9b
% 2a může v závislosti na převodu do CNF vyjít moc jednoduché, raději 2b



\textbf{Témata:}
Rezoluce ve výrokové logice. Aplikace věty o kompaktnosti. Hilbertův kalkulus.


\medskip\begin{problem}
    Označme jako $\varphi$ výrok $\neg (p \vee q) \to (\neg p \wedge \neg q)$. Ukažte, že $\varphi$ je tautologie:
    \begin{enumerate}
        \item Převeďte $\neg \varphi$ do CNF a zapište výsledný výrok jako formuli $S$ v množinové reprezentaci.
        \item Najděte rezoluční zamítnutí $S$.
    \end{enumerate}
    \end{problem}
    
    
    \medskip\begin{problem}
    Najděte rezoluční zamítnutí následujících výroků:
    \begin{enumerate}
        \item $\neg(((p\to q)\to \neg q)\to \neg q)$
        \item $(p\leftrightarrow (q\to r))\wedge((p\leftrightarrow q)\wedge(p\leftrightarrow \neg r))$
        
    \end{enumerate}
    \end{problem}
        
        
    \medskip\begin{problem}
    Dokažte rezolucí, že v teorii $T=\{\neg p \to \neg q,\neg q \to \neg r, (r\to p)\to s\}$ platí výrok $s$.
    \end{problem}
    
    
    \medskip\begin{problem}Nechť prvovýroky $r$, $s$, $t$  reprezentují (po řadě), že \emph{``Radka / Sára / Tom je ve škole''} a označme $\mathbb{P}=\{r,s,t\}$. Víme, že
        \begin{itemize}\it
        \item Není-li Tom ve škole, není tam ani Sára.
        \item Radka bez Sáry do školy nechodí.
        \item Není-li Radka ve škole, je tam Tom.
        \end{itemize}
        \begin{enumerate}
        \item Formalizujte naše znalosti jako teorii $T$ v jazyce $\mathbb P$.
        \item Rezoluční metodou dokažte, že z $T$ vyplývá, že \emph{Tom je ve škole}: Napište formuli $S$ v množinové reprezentaci, která je nesplnitelná, právě když to platí, a najděte rezoluční zamítnutí $S$. Nakreslete rezoluční strom.
        \item Určete množinu modelů teorie $T$.
        \end{enumerate}
    \end{problem}
    
    
    \medskip\begin{problem} Máme k dispozici MgO, H$_2$, O$_2$, a C, a můžeme provádět následující reakce:
        \begin{itemize}
            \item MgO\ +\ H$_2$\ \ $\to$\ \ Mg\ +\ H$_2$O
            \item C\ +\ O$_2$\ \ $\to$\ \ CO$_2$
            \item CO$_2$\ +\ H$_2$O\ \ $\to$\ \ H$_2$CO$_3$
        \end{itemize}
        \begin{enumerate}
            \item Reprezentujte naše možnosti výrokem %(nad vhodně zvoleným jazykem) 
            a převeďte ho do množinové reprezentace.
            \item Pomocí rezoluce dokažte, že můžeme získat H$_2$CO$_3$. Lze najít LI-důkaz téhož?
        \end{enumerate}
    \end{problem}
    
    
    \medskip\begin{problem}
        Najděte rezoluční uzávěry $\mathcal{R}(S)$ pro následující výroky $S$:
        \begin{enumerate}
            \item $\{\{p,q\},\{p,\neg q\},\{\neg p,\neg q\}\}$
            \item $\{\{p,\neg q,r\},\{q,r\},\{\neg p, r\},\{q,\neg r\},\{\neg q\}\}$ (uzávěr je poměrně velký, vymyslete systematický postup a vygenerujte jen část)
        \end{enumerate}
    \end{problem}
        
        
    \medskip\begin{problem}
        Zkonstruujte \emph{strom dosazení} pro následující formuli. Na základě tohoto stromu sestrojte rezoluční zamítnutí, dle postupu z důkazu Věty o úplnosti rezoluce.
        $$
        S=\{\{p,r\},\{q,\neg r\},\{\neg q\},\{\neg p,t\},\{\neg s\},\{s,\neg t\}\}
        $$
        
    \end{problem}
    
    
    \medskip\begin{problem}
        Dokažte podrobně, že je-li $S=\{C_1,C_2\}$ splnitelná a $C$ je rezolventa $C_1$ a $C_2$, potom je i $C$ splnitelná.
    \end{problem}
    
        
    \medskip\begin{problem} Dokažte pomocí věty o kompaktnosti a variant tvrzení pro konečné objekty:
    \begin{enumerate}
        \item Každý spočetný rovinný graf je obarvitelný čtyřmi barvami.
        \item Každé spočetné částečné uspořádání lze rozšířit na úplné (lineární) uspořádání.
        %\item Hallova věta platí i pro nekonečné množiny.
    \end{enumerate}
    
    \end{problem}
        
    
    \medskip\begin{problem}
    V Hilbertově kalkulu dokažte pro libovolné formule následující vztahy:
    \begin{enumerate}
        %\item $_H\ p\to p$
        \item $\{\neg p\}\proves_H\ p\to q$
        \item $\{\neg(\neg p)\}\proves_H\ p$
        \item $\{p\to q,q \to r\}\proves_H\ p\to r$
    \end{enumerate}    
    \end{problem}
    
    \medskip\begin{problem}
        Dokažte korektnost Hilbertova kalkulu:
        \begin{itemize}
            \item Dokažte, že logické axiomy jsou tautologie.
            \item Dokažte, že modus ponens je korektní, tj. když $T\models\varphi$ a $T\models\varphi\to\psi$, tak $T\models\psi$.
            \item Ukažte, že $T\proves_H\ \varphi$ implikuje $T\models\varphi$.
        \end{itemize}
        \end{problem}
        
    \medskip\begin{problem}
        Vyslovte a dokažte větu o dedukci pro Hilbertův kalkul.
    \end{problem}
       



\end{document}