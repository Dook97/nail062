\documentclass[a4paper,12pt]{article}

\usepackage{a4wide}
\usepackage{amsmath}
\usepackage{amssymb}
\usepackage{amsthm}
\usepackage[czech]{babel}
\usepackage{bookmark}
\usepackage{enumerate}
\usepackage[T1]{fontenc}
\usepackage{forest}
\usepackage{hyperref}
\usepackage[utf8]{inputenc}
\usepackage{lmodern}
\usepackage{multicol}
\usepackage{tikz}

\theoremstyle{definition}
    \newtheorem{problem}{Příklad}

% \theoremstyle{remark}
%     \newtheorem*{steps}{Postup řešení}

\theoremstyle{plain}
    \newtheorem*{solution}{Řešení}
    

\DeclareRobustCommand\proves{\mathrel{|}\joinrel\mkern-.5mu\mathrel{-}}
\DeclareMathOperator{\Conseq}{Csq}
\DeclareMathOperator{\M}{M}

% hide solutions
\newif\ifhidesolutions
    \hidesolutionstrue
    % \hidesolutionsfalse

\ifhidesolutions
    \usepackage{environ}
    \NewEnviron{hide}{}
    \let\solution\hide
    \let\endsolution\endhide
\fi









\begin{document}

\section*{NAIL062 V\&P Logika: První domácí  úkol}
% výroková logika

\subsubsection*{Termín odevzdání: 20. 11. v 10:40.}
Celkem 10 bodů. Řešení odevzdejte v papírové podobě na cvičení nebo, pokud nebudete moci přijít, emailem před začátkem cvičení. Řešení musí být rozumně čitelné, a v případě odevzdání emailem musí mít bílé pozadí. Je zakázáno o úkolech až do termínu odevzdání jakýmkoliv způsobem komunikovat s kýmkoliv kromě mne. Řešení musí být 100\% vaší vlastní prací, a je vaší povinností zajistit, že nikdo nebude mít přístup k vašemu řešení.

\bigskip

\noindent {\textbf{Poznámka:} Používáte-li tablo metodu, musí řešení obsahovat tablo korektně sestrojené přesně podle definice: nedělejte žádné zkratky, nevynechávejte vrcholy nad rámec konvence z přednášky, apod. Podobně, používáte-li rezoluční metodu, musí řešení obsahovat korektní rezoluční strom. (Rezoluční důkaz zapisovat nemusíte.)


\medskip\begin{problem}[4 body]
    Adam, Barbora a Cyril jsou vyslýcháni, při jejich výslechu bylo zjištěno následující:
    \begin{itemize}\it
        \item Alespoň jeden z vyslýchaných říká pravdu a alespoň jeden lže.
        \item Adam říká: ``Barbora nebo Cyril lžou.''
        \item Barbora říká: ``Cyril lže.''
        \item Cyril říká: ``Adam nebo Barbora lžou.''
    \end{itemize}
    \begin{enumerate}
        \item Vyjádřete naše znalosti jako výroky $\varphi_1$ až $\varphi_4$ nad množinou prvovýroků $\mathbb{P}=\{a,b,c\}$, přičemž $a,b,c$ znamená (po řadě), že {\it ``Adam/Barbora/Cyril říká pravdu''}.
        \item Najděte všechny modely teorie $T = \{\varphi_1, \dots, \varphi_4\}$.
        \item Dokažte tablo metodou, že z teorie $T$ plyne, že: {\it Adam říká pravdu.}
        \item Dokažte totéž rezoluční metodou.
    \end{enumerate}
\end{problem}


\medskip\begin{problem}[2 body]
    Uvažme následující výroky $\varphi$ a $\psi$ nad $\mathbb P=\{p, q, r, s\}$:
    \begin{align*}
        \varphi &= (\neg p \vee  q)\to(p\wedge r)\\
        \psi &= s\to q
    \end{align*}
    \begin{enumerate}
        \item Určete počet (až na ekvivalenci) výroků $\chi$ nad $\mathbb P$ takových, že $\varphi\wedge\psi\models\chi$.
        \item Určete počet (až na ekvivalenci) úplných teorií $T$ nad $\mathbb P$ takových, že $T\models\varphi\wedge\psi$.
        \item Najděte nějakou axiomatizaci pro každou (až na ekvivalenci) kompletní teorii $T$ nad $\mathbb P$ takovou, že $T\models\varphi\wedge\psi$.
    \end{enumerate}
\end{problem}

\medskip\begin{problem}[2 body]
    Pomocí algoritmu implikačního grafu najděte všechny modely následující teorie:
    $$
    T=\{p,\neg q \to \neg r,\neg q \to \neg s,r \to p,\neg s \to \neg p\}
    $$
\end{problem}

\medskip\begin{problem}[1 bod]
    Pomocí algoritmu jednotkové propagace najděte všechny modely následující teorie:
    \begin{align*}
        &(\neg a \vee \neg b \vee c \vee \neg d)\wedge(\neg b \vee c)\wedge d \wedge (\neg a \vee \neg c \vee e)\wedge \\
        &(\neg c \vee \neg d)\wedge(\neg a \vee \neg d \vee \neg e)\wedge(a\vee \neg b \vee\neg e)
    \end{align*}
\end{problem}

\medskip\begin{problem}[1 bod]  
Převeďte následující výrok do CNF a DNF:
$$
((p\to \neg q) \to \neg r) \to \neg p.
$$
\begin{enumerate}
    \item tabulkou (určením modelů),
    \item ekvivalentními úpravami (pokuste se najít co nejkratší CNF a DNF ekvivalenty).
\end{enumerate}
\end{problem}


\end{document}


% \medskip\begin{problem}[1 bod] 
% Uvažte  teorii $S=\{p_i \to (p_{i+1} \vee q_{i+1}), q_i \to (p_{i+1} \vee q_{i+1}) \mid i\in \mathbb{N}\}$ nad $\mathrm{var}(S)$.
% \begin{enumerate}
%     \item Které výroky ve tvaru  $p_i \to p_j$ jsou důsledky $S$?
%     \item Které výroky ve tvaru  $p_i \to (p_j \vee q_j)$ jsou důsledky $S$?
%     \item Určete všechny modely $S$.
% \end{enumerate}
% \end{problem}
