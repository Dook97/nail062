\documentclass{amsart}
\usepackage[utf8]{inputenc}
\usepackage{enumerate}

\theoremstyle{definition}
\newtheorem{problem}{Příklad}



\begin{document}

\section*{Cvičení z logiky: 0. sada příkladů}

\bigskip\bigskip

\subsection*{Témata:} 
Predikátová logika (jazyk, teorie, logika prvního řádu, vyšších řádů). Vyjadřování různých vlastností ve výrokové a predikátové logice.

\bigskip

\medskip\hrule\medskip


\begin{problem}
Ztratili jsme se v labyrintu a před námi jsou troje dveře - červené, zelené a modré. Víme, že za právě jedněmi dveřmi je cesta ven, za ostatními je drak. Na dveřích jsou nápisy:
\begin{itemize}
    \item Červené dveře: ``Cesta ven je za těmito dveřmi.''
    \item Modré dveře: ``Cesta ven není za těmito dveřmi.''
    \item Zelené dveře: ``Cesta ven není za modrými dveřmi.'' 
\end{itemize}
Víme, že alespoň jeden z nápisů je pravdivý a alespoň jeden je lživý. Formalizujte naše znalosti.
\end{problem}

\begin{problem} Víme, že:
\begin{enumerate}
\item[$(i)$] Každý zná sám sebe.
\item[$(ii)$] Když člověk studuje na škole, musel se na ni hlásit a ta škola ho přijala.
\item[$(iii)$] Alfons se nehlásil na školu, která přijala někoho, kdo Alfonse zná.
\end{enumerate}
Formalizujte naše znalosti. Umíte ukázat, že ``Alfons nestuduje na žádné škole.''?
\end{problem}
\smallskip


\begin{problem}
Najděte vhodný jazyk a teorii prvního řádu pro
\begin{enumerate}[a)]
    \item orientované grafy (bez násobných hran)
    \item grafy (neorientované, bez smyček),
  %  \item grafovou souvislost (tj. s predikátem pro fakt, že $u$ a $v$ jsou spojeny cestou), %nelze, preformulovat zadani
    \item relace ekvivalence,
    \item částečná uspořádání.
\end{enumerate}
\end{problem}


\smallskip
\begin{problem}
Mějme graf $G$ (neorientovaný, bez smyček). Najděte formule (v jazyce grafů), které vyjadřují následující vlastnosti. Kdy to lze v logice prvního řádu, a kdy je třeba logika druhého řádu?
\begin{enumerate}[(a)]
    \item $G$ obsahuje vrchol stupně 1
   \item existuje cesta délky $k$ z $u$ do $v$ v $G$ (pro nějaké fixní $k$)
    \item $G$ je regulární stupně 3,
    \item $G$ obsahuje a $k$-kliku (pro nějaké fixní $k$),
    \item $G$ má vrcholové 3-obarvení.
    \item $G$ je bipartitní,
    \item $G$ má perfektní párování,
    \item existuje cesta z $u$ do $v$ v $G$
\end{enumerate}
\end{problem}


\smallskip
\begin{problem} Najděte FO-formule (v jazyce $\le$) vyjadřující následující vlastnosti uspořádaných množin:
\begin{enumerate}[(a)]
\item ``$x$ je nejmenší prvek'', ``$x$ je minimální prvek'',
\item ``$x$ má bezprostředního následníka'',
\item ``každé dva prvky mají největšího společného předchůdce''.
\end{enumerate}
\end{problem}


\smallskip
\begin{problem} Najděte formule prvního řádu (v jazyce rovnosti), které vyjadřují pro dané $n>0$, že
\begin{enumerate}[(a)]
\item ``existuje alespoň $n$ prvků'',
\item ``existuje nejvíce $n$ prvků'',
\item ``existuje právě  $n$ prvků''
\end{enumerate}
Je možné vyjádřit, za pomoci (možná nekonečné) množiny formulí, že ``existuje nekonečně mnoho prvků''?
\end{problem}

\smallskip
\begin{problem}
Lze $\mathbb N$, $\mathbb Z$, $\mathbb R$ a $\mathbb C$ rozlišit pomocí vlastností prvího řádu
\begin{enumerate}[(a)]
\item v jazyce uspořádaných množin?
\item v jazyce aritmetiky?
\end{enumerate}
\end{problem}



\smallskip
\begin{problem}
Uvažte konečnou hru dvou střídajících se hráčů, Alice a Boba. Hra končí po $n$ kolech výhrou jednoho z hráčů. První tah má Alice. Hra je daná formulí
$\varphi(x_1, y_1, x_2, y_2,\dots,x_n, y_n)$ vyjadřující, že hra z tahy $x_1, y_1, x_2, y_2,\dots,x_n, y_n$ končí výhrou Alice. Najděte formule (v logice prvního řádu), které vyjadřují
\begin{enumerate}[(a)]
    \item ``Alice nemůže prohrát'',
    \item ``Bob nemůže prohrát'',
    \item ``Alice má vyhrávající strategii'',
    \item ``Bob má vyhrávající strategii''.
\end{enumerate}
\end{problem}




\end{document}
