\documentclass[a4paper]{article}

\usepackage{a4wide}
\usepackage{amssymb}
\usepackage{amsthm}
\usepackage{enumitem}
    \setlist[enumerate]{label=(\alph*),itemsep=3pt,topsep=6pt}
    \setlist[itemize]{itemsep=3pt,topsep=6pt}
\usepackage[utf8]{inputenc}


\theoremstyle{definition}
\newtheorem{problem}{Příklad}
\newtheorem*{ukol}{Domácí úkol}


\begin{document}

\section*{NAIL062 V\&P Logika: 0. cvičení}


\textbf{Témata:} Úvod. Vyjadřování různých vlastností ve výrokové a predikátové logice. Připomenutí matematických pojmů.


\medskip\begin{problem}
Ztratili jsme se v labyrintu a před námi jsou troje dveře - červené, zelené a modré. Víme, že za právě jedněmi dveřmi je cesta ven, za ostatními je drak. Na dveřích jsou nápisy:
\begin{itemize}
    \item Červené dveře: {\it ``Cesta ven je za těmito dveřmi.''}
    \item Modré dveře: {\it ``Cesta ven není za těmito dveřmi.''}
    \item Zelené dveře: {\it ``Cesta ven není za modrými dveřmi.''}
\end{itemize}
Víme, že alespoň jeden z nápisů je pravdivý a alespoň jeden je lživý. Formalizujte naše znalosti. Určete, za kterými dveřmi je cesta ven.
\end{problem}


\medskip\begin{problem} Víme, že:
\begin{itemize}\it
\item Každý zná sám sebe.
\item Když člověk studuje na škole, musel se na ni hlásit a ta škola ho přijala.
\item Alfons se nehlásil na školu, která přijala někoho, kdo Alfonse zná.
\end{itemize}
Formalizujte naše znalosti. (Uměli byste ukázat, že {\it ``Alfons nestuduje na žádné škole.''}?)
\end{problem}


\medskip\begin{problem}
Mějme daný graf $G$ (neorientovaný, bez smyček) a dva jeho vrcholy $u,v$. Formalizujte následující vlastnosti ve výrokové logice:
\begin{enumerate}
    \item $G$ je bipartitní,
    \item $G$ má perfektní párování,
    \item $u$ a $v$ leží v jedné komponentě souvislosti,
    \item $G$ je souvislý.
\end{enumerate}
\end{problem}


\medskip\begin{problem}
Najděte formule v predikátové logice v jazyce grafů, které v \emph{teorii grafů} (neorientovaných, bez smyček) vyjadřují následující vlastnosti. Kdy to lze v logice prvního řádu, a kdy je třeba logika druhého řádu?
\begin{enumerate}
    \item graf obsahuje vrchol stupně 1
    \item graf je regulární stupně 3,
    \item graf obsahuje $k$-kliku (pro nějaké fixní $k$),
    \item existuje cesta délky $k$ z vrcholu $u$ do vrcholu $v$ (pro nějaké fixní $k$),
    \item vrcholy $u$ a $v$ mají alespoň jednoho společného souseda,
    \item graf je bipartitní,
    \item graf má perfektní párování,
    \item vrcholy $u$ a $v$ leží v jedné komponentě souvislosti,
    \item graf je souvislý.    
\end{enumerate}
\end{problem}


\medskip\begin{problem} Najděte formule prvního řádu vyjadřující následující vlastnosti v jazyce uspořádaných množin:
\begin{enumerate}
    \item $x$ je nejmenší prvek,
    \item $x$ je minimální prvek,
    \item $x$ má bezprostředního následníka,
    \item každé dva prvky mají největšího společného předchůdce.
\end{enumerate}
\end{problem}


\medskip\begin{problem} Najděte formule prvního řádu (v jazyce rovnosti), které vyjadřují pro dané $k>0$, že
\begin{enumerate}
    \item existuje alespoň $k$ prvků,
    \item existuje nejvýše $k$ prvků,
    \item existuje právě $k$ prvků.
\end{enumerate}
\end{problem}


\medskip\begin{problem} Vyjádřete v logice prvního řádu v jazyce s jedním unárním funkčním symbolem $f$, že funkce je
\begin{enumerate}
    \item prostá,
    \item na,
    \item bijekce.
\end{enumerate}
\end{problem}


\medskip\begin{problem} Vyjádřete v logice druhého řádu, že daná binární relace je 
\begin{enumerate}
    \item (unární) funkce,
    \item prostá funkce,
    \item funkce na,
    \item bijekce (vyjádřete tak, že k ní existuje inverzní funkce).
\end{enumerate}
\end{problem}


\medskip\begin{problem}
Lze $\mathbb N$, $\mathbb Z$, $\mathbb R$ a $\mathbb C$ rozlišit pomocí vlastností prvního řádu
\begin{enumerate}
    \item v jazyce uspořádaných množin?
    \item v jazyce těles?
\end{enumerate}
\end{problem}


\medskip\begin{ukol}
Zatím žádný není. Místo toho řešte příklady zbývající ze cvičení.
\end{ukol}


\end{document}
