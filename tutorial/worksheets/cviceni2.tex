\documentclass[a4paper,11pt]{amsart}

\usepackage{a4wide}
\usepackage{amsmath}
\usepackage{amssymb}
\usepackage{amsthm}
\usepackage[czech]{babel}
\usepackage{bookmark}
\usepackage{enumerate}
\usepackage[T1]{fontenc}
\usepackage{forest}
\usepackage{hyperref}
\usepackage[utf8]{inputenc}
\usepackage{lmodern}
\usepackage{multicol}
\usepackage{tikz}

\theoremstyle{definition}
    \newtheorem{problem}{Příklad}

% \theoremstyle{remark}
%     \newtheorem*{steps}{Postup řešení}

\theoremstyle{plain}
    \newtheorem*{solution}{Řešení}
    

\DeclareRobustCommand\proves{\mathrel{|}\joinrel\mkern-.5mu\mathrel{-}}
\DeclareMathOperator{\Conseq}{Csq}
\DeclareMathOperator{\M}{M}

% hide solutions
\newif\ifhidesolutions
    \hidesolutionstrue
    % \hidesolutionsfalse

\ifhidesolutions
    \usepackage{environ}
    \NewEnviron{hide}{}
    \let\solution\hide
    \let\endsolution\endhide
\fi








\begin{document}

\section*{NAIL062 V\&P Logika: 2. cvičení}
% po 1. přednášce



\subsection*{Výukové cíle:} Po absolvování cvičení student

    \begin{itemize}\setlength{\itemsep}{0pt}
        \item rozumí pojmům sémantiky výrokové logiky (pravdivostní hodnota, pravdivostní funkce, model, platnost, tautologie, spornost, nezávislost, splnitelnost, ekvivalence), umí je formálně definovat a uvést příklady
        \item umí rozhodnout, zda je množina logických spojek univerzální
        \item zná terminologii pro výroky v CNF a DNF %([pozitivní/negativní/opačný] literál, [prázdná/jednotková] klauzule, [prázdná/jednotková] elemntární konjunkce, prázdný výrok v CNF/DNF)
        \item umí převést daný výrok resp. konečnou teorii do CNF a do DNF, a to pomocí množiny modelů i pomocí ekvivalentních úprav
        \item rozumí terminologii týkající se vlastností teorií (sporná, bezesporná/splnitelná, kompletní, důsledky, $T$-ekvivalence), umí pojmy formálně definovat a uvést příklady
        \item rozumí pojmu [jednoduchá, konzervativní] extenze, umí je formálně definovat, uvést příklady
        \item umí v konkrétním případě rozhodnout, zda jde o [jednoduchou, konzervativní] extenzi, a zdůvodnit jak z definice, tak i pomocí sémantického kritéria
    \end{itemize}


\end{document}
