\documentclass[a4paper]{article}

\usepackage{a4wide}
\usepackage{amsmath}
\usepackage{amssymb}
\usepackage{amsthm}
\usepackage{enumitem}
    \setlist[enumerate]{label=(\alph*),itemsep=3pt,topsep=6pt}
    \setlist[itemize]{itemsep=3pt,topsep=6pt}
\usepackage{tikz}
\usepackage[utf8]{inputenc}


\theoremstyle{definition}
\newtheorem{problem}{Příklad}
\newtheorem*{ukol}{Domácí úkol}

\DeclareMathOperator{\Conseq}{Csq}


\begin{document}

\section*{NAIL062 V\&P Logika: 2. cvičení}


\textbf{Témata:} 
Syntaxe a sémantika výrokové logiky. Převod do CNF a DNF.

\medskip\begin{problem}
    Mějme teorii $T=\{\neg q \to (\neg p \vee q),\ \neg p \to q,\ r \to q\}$ v jazyce $\{p, q, r, s\}$.
    \begin{enumerate}
        \item Uveďte příklad následujícího: výrok pravdivý v $T$, lživý v $T$, nezávislý v $T$, splnitelný v $T$, a dvojice $T$-ekvivalentních výroků.
        \item Které z následujících výroků jsou pravdivé, lživé, nezávislé, splnitelné v $T$? $T$-ekvivalentní? 
        $$
        p, \ \neg q, \ \neg p\vee q, \ p\to r,\ \neg q\to r, \ p\vee q\vee r\vee s
        $$
    \end{enumerate}
\end{problem}
    
    
\medskip\begin{problem} Určete množinu modelů dané formule. Využijte toho, že je v DNF resp. v CNF.
\begin{enumerate}
    \item $(\neg p_1 \wedge \neg p_2)\vee( \neg p_1 \wedge p_2)\vee( p_1 \wedge \neg p_2)\vee( p_2 \wedge \neg p_3)\vee( p_1 \wedge p_3)$
    \item $(\neg p_1 \vee \neg p_2)\wedge( \neg p_1 \vee p_2)\wedge( p_1 \vee \neg p_2)\wedge( p_2 \vee \neg p_3)\wedge( p_1 \vee p_3)$
    \item $(p_1 \wedge  \neg p_2 \wedge  p_3 \wedge  \neg p_4 )\vee(p_2 \wedge  p_3 \wedge  \neg p_4 )\vee(\neg p_3)\vee(p_2 \wedge  p_4)\vee(p_1 \wedge  p_3 \wedge  p_5 )\vee(p_3 \wedge  \neg p_4 \wedge  p_2 )$
    \item $(p_1 \vee \neg p_2 \vee p_3 \vee \neg p_4 )\wedge(p_2 \vee p_3 \vee \neg p_4 )\wedge(\neg p_3)\wedge(p_2 \vee p_4)\wedge(p_1 \vee p_3 \vee p_5 )\wedge(p_3 \vee \neg p_4 \vee p_2 )$
\end{enumerate}
\end{problem}
    
    
\medskip\begin{problem} Převeďte následující výroky do CNF a DNF (I) tabulkou (určením modelů), (II) ekvivalentními úpravami.
\begin{enumerate}
    \item $(\neg p \vee q)\to (\neg q \wedge r)$,
    \item $(\neg p \to (\neg q \to r))\to p$,
\end{enumerate}
\end{problem}
    
    
\medskip\begin{problem} Pro danou formuli $\varphi$ v CNF najděte a 3-CNF formuli $\varphi'$ takovou, že $\varphi'$ je splnitelná, právě když $\varphi$ je splnitelná. Popište efektivní algoritmus konstrukce $\varphi'$ je-li dána $\varphi$ (tj. redukci z problému SAT do problému 3-SAT).
\end{problem}
    
    
\medskip\begin{problem} Najděte (co nejkratší) CNF a DNF reprezentace Booleovské funkce $\mathrm{maj}: {^3}2\to 2$, která vrací převládající hodnotu mezi 3 vstupy.
\end{problem}
    
    
\medskip\begin{problem} Uměli byste nalézt CNF a DNF reprezentace $n$-ární parity, tj. Booleovské funkce $\mathrm{par}: {^n}2\to 2$ definované pomocí $\mathrm{par}(x_1,\dots,x_n)=(x_1+\dots+x_n)\bmod 2$,
která vrací XOR všech vstupních hodnot? Zkuste to pro malé hodnoty $n$.
\end{problem}
    
    
\medskip\begin{problem} Buď $\mathbb P$ spočetně nekonečná množina prvovýroků. Ukažte, že již neplatí, že každou $K\subseteq \mathrm{M}_\mathbb P$ lze axiomatizovat výrokem v CNF i výrokem v DNF. Najděte množinu modelů $K$, kterou nelze axiomatizovat ani výrokem v CNF, ani výrokem v DNF.
\end{problem}


\medskip\begin{problem}
Uvažte následující dvě teorie:
\begin{enumerate}[label=(\Roman*)]
    \item $T=\{p\wedge q,p\to\neg q,q\}$ v jazyce $\mathbb P=\{p,q\}$
    \item $T=\{(p\wedge q)\to r, \neg r\vee(p\wedge q)\}$ v jazyce $\mathbb P=\{p,q,r\}$        
\end{enumerate}
\begin{enumerate}
    \item Rozhodněte, zda je teorie $T$ [konzistentní/splnitelná/kompletní]. (konzistentní=bezesporná, kompletní=úplná)
    \item Uveďte příklad výroku $\varphi$, který je [platný/nesplnitelný/nezávislý] v $T$
    \item Uveďte příklad rozšíření $T'$ teorie $T$ (pokud existuje, a pokud možno, neekvivalentního s $T$), které je [jednoduché/konzervativní/kompletní/konz. jedn./kompl. jedn./kompl. konz.].
\end{enumerate}

\end{problem}
        
    
\medskip\begin{problem}
Uvažme nekonečnou výrokovou teorii $T=\{p_i \to p_{i+1}\mid i\in \mathbb{N}\}$ nad $\mathrm{var}(T)$. 
\begin{enumerate}
    \item Které výroky ve tvaru  $p_i \to p_j$ jsou důsledky $T$?
    \item Určete všechny modely $T$.
\end{enumerate}
\end{problem}


\medskip\begin{problem}
Dokažte nebo vyvraťte (nebo uveďte správný vztah), že pro každou teorii $T$ a výroky $\varphi$, $\psi$ v jazyce $\mathbb{P}$ platí:
\begin{enumerate}
    \item $T \models \varphi$,\ \  právě když \ \ $T \not\models \neg \varphi$
    \item $T \models \varphi$ a $T \models \psi$,\ \ právě když \ \ $T \models \varphi \wedge \psi$
    \item $T \models \varphi$ nebo $T \models \psi$,\ \ právě když \ \ $T \models \varphi \vee \psi$
    \item $T \models \varphi \to \psi$ and $T \models \psi \to \chi$,\ \ právě když \ \ $T \models \varphi \to \chi$
\end{enumerate}
\end{problem}
    

\medskip\begin{problem}
Dokažte nebo vyvraťte (nebo uveďte správný vztah), že pro libovolné teorie $T$, $S$ nad~$\mathbb{P}$ platí:
\begin{enumerate}
    \item $S\subseteq T \Rightarrow \Conseq(T) \subseteq \Conseq(S)$
    \item $\Conseq(S\cup T)=\Conseq(S) \cup \Conseq(T)$
    \item $\Conseq(S\cap T)=\Conseq(S) \cap \Conseq(T)$
\end{enumerate}
\end{problem}


\medskip\begin{ukol}[2 body]{\,}
\begin{enumerate}[label=\arabic*.]
    \item Převeďte následující výrok do CNF a DNF:
    $$
    ((p\to \neg q) \to \neg r) \to \neg p.
    $$
    \begin{enumerate}
        \item tabulkou (určením modelů),
        \item ekvivalentními úpravami (pokuste se najít co nejkratší CNF a DNF ekvivalenty).
    \end{enumerate}
    \item Uvažte  teorii $S=\{p_i \to (p_{i+1} \vee q_{i+1}), q_i \to (p_{i+1} \vee q_{i+1}) \mid i\in \mathbb{N}\}$ nad $\mathrm{var}(S)$.
    \begin{enumerate}
            \item Které výroky ve tvaru  $p_i \to p_j$ jsou důsledky $S$?
            \item Které výroky ve tvaru  $p_i \to (p_j \vee q_j)$ jsou důsledky $S$?
            \item Určete všechny modely $S$.
    \end{enumerate}
\end{enumerate} 
\end{ukol}


\end{document}