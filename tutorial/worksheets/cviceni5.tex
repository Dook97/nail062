\documentclass[a4paper,11pt]{amsart}

\usepackage{a4wide}
\usepackage{amsmath}
\usepackage{amssymb}
\usepackage{amsthm}
\usepackage[czech]{babel}
\usepackage{bookmark}
\usepackage{enumerate}
\usepackage[T1]{fontenc}
\usepackage{forest}
\usepackage{hyperref}
\usepackage[utf8]{inputenc}
\usepackage{lmodern}
\usepackage{multicol}
\usepackage{tikz}

\theoremstyle{definition}
    \newtheorem{problem}{Příklad}

% \theoremstyle{remark}
%     \newtheorem*{steps}{Postup řešení}

\theoremstyle{plain}
    \newtheorem*{solution}{Řešení}
    

\DeclareRobustCommand\proves{\mathrel{|}\joinrel\mkern-.5mu\mathrel{-}}
\DeclareMathOperator{\Conseq}{Csq}
\DeclareMathOperator{\M}{M}

% hide solutions
\newif\ifhidesolutions
    \hidesolutionstrue
    % \hidesolutionsfalse

\ifhidesolutions
    \usepackage{environ}
    \NewEnviron{hide}{}
    \let\solution\hide
    \let\endsolution\endhide
\fi








\begin{document}

\section*{NAIL062 V\&P Logika: 5. cvičení}
% po 1. přednášce


\subsection*{Výukové cíle:} Po absolvování cvičení student

    \begin{itemize}\setlength{\itemsep}{0pt}
        \item zná potřebné pojmy z rezoluční metody (rezoluční pravidlo, rezolventa, rezoluční důkaz/zamítnutí, rezoluční strom), umí je formálně definovat, uvést příklady
        \item umí pracovat s výroky v CNF a jejich modely v množinové reprezentaci
        \item umí sestrojit rezoluční zamítnutí dané (i nekonečné) CNF formule (existuje-li), a také nakreslit příslušný rezoluční strom
        \item zná pojem stromu dosazení, umí ho formálně definovat a pro konkrétní CNF formuli sestrojit
        \item umí aplikovat rezoluční metodu k řešení daného problému (slovní úlohy, aj.)
    \end{itemize}
    
\end{document}