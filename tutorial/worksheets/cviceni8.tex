\documentclass[a4paper]{article}

\usepackage{a4wide}
\usepackage{amsmath}
\usepackage{amssymb}
\usepackage{amsthm}
\usepackage{enumerate}
\usepackage{tikz}
\usepackage[utf8]{inputenc}


\theoremstyle{definition}
\newtheorem{problem}{Příklad}
\newtheorem*{ukol}{Domácí úkol}


\begin{document}

\section*{NAIL062 V\&P Logika: 8. cvičení}


\textbf{Témata:}
Struktury a podstruktury. Extenze teorií, extenze o definice. Definovatelnost ve struktuře.


\medskip\begin{problem}
    Uvažme $\underline{\mathbb{Z}}_4=\langle\{0,1,2,3\},+,-,0 \rangle$ kde $+$ je binární sčítání modulo $4$ a $-$ je unární funkce, která vrací \emph{inverzní} prvek $+$ vzhledem k \emph{neutrálnímu} prvku $0$.
    \begin{enumerate}[(a)]
        \item Je $\underline{\mathbb{Z}}_4$ model teorie grup (tj. je to \emph{grupa})?
        \item Určete všechny podstruktury $\underline{\mathbb{Z}}_4\langle a\rangle$ generované nějakým $a\in \mathbb{Z}_4$.
        \item Obsahuje $\underline{\mathbb{Z}}_4$ ještě nějaké další podstruktury?
        \item Je každá podstruktura $\underline{\mathbb{Z}}_4$ modelem teorie grup?
        \item Je každá podstruktura $\underline{\mathbb{Z}}_4$ elementárně ekvivalentní $\underline{\mathbb{Z}}_4$?
        \item Je každá podstruktura \emph{komutativní} grupy (tj. grupy, která splňuje $x+y=y+x$) také komutativní grupa?
    \end{enumerate}
\end{problem}
 
        
\medskip\begin{problem}
    Buď $\underline{\mathbb{Q}}=\langle\mathbb{Q},+,-,\cdot,0,1 \rangle$ těleso racionálních čísel se standardními operacemi.
    \begin{enumerate}[(a)]
    \item Existuje redukt $\underline{\mathbb{Q}}$, který je modelem teorie grup?
    \item Lze redukt $\langle\mathbb{Q},\cdot,1\rangle$ rozšířit na model teorie grup?
    \item Obsahuje $\underline{\mathbb{Q}}$ podstrukturu, která není elementárně ekvivalentní $\underline{\mathbb{Q}}$?
    \item Označmě $Th(\underline{\mathbb{Q}})$ množinu všech sentencí pravdivých v $\underline{\mathbb{Q}}$. Je $Th(\underline{\mathbb{Q}})$ úplná teorie?
    \end{enumerate}
\end{problem}
    

\medskip\begin{problem}
    Mějme teorii $T=\{x=c_1 \vee x=c_2 \vee x=c_3\}$ v jazyce $L=\langle c_1,c_2,c_3\rangle$ s rovností.
    \begin{enumerate}[(a)]
    \item Je $T$ (sémanticky) konzistentní?
    \item Jsou všechny modely $T$ elementárně ekvivalentní? Tj. je $T$ (sémanticky) úplná?
    \item Najděte všechny jednoduché úplné extenze $T$.
    \item Je teorie $T'=T\cup\{x=c_1 \vee x=c_4\}$ v jazyce $L=\langle c_1,c_2,c_3,c_4\rangle$ extenzí $T$? Je $T'$ jednoduchá extenze $T$? Je $T'$ konzervativní extenze $T$?
    \end{enumerate}
\end{problem}


\medskip\begin{problem}
Buď $T'$ extenze teorie $T=\{(\exists y)(x+y=0),(x+y=0)\wedge (x+z=0)\rightarrow y=z\}$ v jazyce $L=\langle +,0,\le\rangle$ s rovností o definice $<$ a unárního $-$ s axiomy
\begin{align*}
    -x=y\ \ &\leftrightarrow\ \ x+y=0\\
    x<y\ \ &\leftrightarrow\ \ x\le y\ \wedge\ \neg(x=y)
\end{align*}
Najděte formule v jazyce $L$, které jsou ekvivalentní v $T'$ s následujícími formulemi.
\begin{enumerate}[(a)]
    \item $x+(-x)=0$
    \item $x+(-y)<x$
    \item $-(x+y)<-x$
\end{enumerate}
\end{problem}


\medskip\begin{problem}
Mějme jazyk $L=\langle F \rangle$ s rovností, kde $F$ je binární funkční symbol. Najděte formule definující následující množiny (bez parametrů):
\begin{enumerate}[(a)]
    \item interval $(0,\infty)$ v $\mathcal A=\langle\mathbb R, \cdot\rangle$ kde $\cdot$ je násobení reálných čísel,
    \item množina $\{(x, 1/x)\mid x\neq 0\}$ ve stejné struktuře $\mathcal A$,
    \item množina všech nejvýše jednoprvkových podmnožin $\mathbb N$ v $\mathcal B=\langle\mathcal P(\mathbb N),\cup\rangle$,
    \item množina všech prvočísel v $\mathcal C=\langle \mathbb N\cup\{0\}, \cdot\rangle$.
\end{enumerate}
\end{problem}



\medskip\begin{ukol}[2 body]

\end{ukol}

\end{document}