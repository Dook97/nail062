\documentclass[a4paper,12pt]{article}

\usepackage{a4wide}
\usepackage{amsmath}
\usepackage{amssymb}
\usepackage{amsthm}
\usepackage[czech]{babel}
\usepackage{bookmark}
\usepackage{enumerate}
\usepackage[T1]{fontenc}
\usepackage{forest}
\usepackage{hyperref}
\usepackage[utf8]{inputenc}
\usepackage{lmodern}
\usepackage{multicol}
\usepackage{tikz}

\theoremstyle{definition}
    \newtheorem{problem}{Příklad}

% \theoremstyle{remark}
%     \newtheorem*{steps}{Postup řešení}

\theoremstyle{plain}
    \newtheorem*{solution}{Řešení}
    

\DeclareRobustCommand\proves{\mathrel{|}\joinrel\mkern-.5mu\mathrel{-}}
\DeclareMathOperator{\Conseq}{Csq}
\DeclareMathOperator{\M}{M}

% hide solutions
\newif\ifhidesolutions
    \hidesolutionstrue
    % \hidesolutionsfalse

\ifhidesolutions
    \usepackage{environ}
    \NewEnviron{hide}{}
    \let\solution\hide
    \let\endsolution\endhide
\fi









\begin{document}

\section*{NAIL062 V\&P Logika: 11. cvičení}
% po 9. přednášce
% 1 a 2 sám stručně, 3 přeskočit.



\textbf{Témata:} Aplikace Věty o kompaktnosti. Ještě tablo metoda. Převod do PNF. Skolemizace. Herbrandova věta.



\medskip\begin{problem} Buď $L$ jazyk s rovností obsahující binární relační symbol $\le$ a $T$ $L$-teorie, která má nekončený model a platí v ní axiomy lineárního uspořádání. Pomocí věty o kompaktnosti ukažte, že $T$ má model 
    s \emph{nekonečným klesajícím řetězcem}.
    %$\mathcal{A}$ s \emph{nekonečným klesajícím řetězcem}; tj. že existují prvky $c_i$ pro každé $i\in \mathbb{N}$ v $A$ takové, že: $\dots < c_{n+1} < c_n< \dots <c_0$. 
    (Z toho plyne, že \emph{dobré uspořádání}, tj. lineární a každá neprázdná podmnožina má nejmenší prvek, není definovatelné v logice prvního řádu.)
\end{problem}

\medskip\begin{problem} Dokažte syntakticky, pomocí transformací tabel:
    \begin{enumerate}
        \item {\it Větu o konstantách:} Buď $\varphi$ $L$-formule s volnými proměnnými $x_1,\dots,x_n$ a $T$ $L$-teorie. Označme $L'$ extenzi $L$ o nové konstantní symboly $c_1,\dots,c_n$ a $T'$ teorii $T$ v $L'$. Potom platí:
        $T \proves (\forall x_1)\dots(\forall x_n)\varphi$ právě když $T'\proves\varphi(x_1/c_1,\dots,x_n/c_n)$
        \item {\it Větu o dedukci:} Pro každou (uzavřenou) teorii $T$ a sentence $\varphi$, $\psi$ platí: $T\proves \varphi\to\psi$ právě když $T,\varphi\proves\psi$
    \end{enumerate}
    \end{problem} 


\medskip\begin{problem}
    Ukažme, že platí následující pravidla, kde
   $\varphi$ a $\psi$ jsou sentence nebo formule s volnou proměnnou $x$ (značíme $\varphi(x)$, $\psi(x)$). Najděte tablo důkazy dané formule.  (Viz převod do PNF, stejně lze dokázat i ostatní pravidla o vytýkání kvantifikátorů.)
\begin{enumerate}
    \item $\neg(\exists x)\varphi(x)\to (\forall x)\neg \varphi(x)$,
    \item $(\forall x)\neg \varphi(x)\to \neg(\exists x)\varphi(x)$,
    % \item $(\exists x)(\varphi(x)\vee \psi(x))\leftrightarrow (\exists x)\varphi(x)\vee (\exists x)\psi(x)$,
    % \item $(\forall x)(\varphi(x)\wedge\psi(x))\leftrightarrow (\forall x)\varphi(x)\wedge(\forall x)\psi(x)$,
    % \item $(\varphi \vee (\forall x)\psi(x))\to (\forall x)(\varphi \vee \psi(x))$ kde $x$ není volná v $\varphi$,
    % \item $(\varphi \wedge (\exists x)\psi(x))\to (\exists x)(\varphi \wedge \psi(x))$ kde $x$ není volná v $\varphi$.
    % \item $(\exists x)(\varphi \to \psi(x))\to(\varphi \to (\exists x)\psi(x))$ kde $x$ není volná v $\varphi$,
    % \item $(\exists x)(\varphi \wedge \psi(x))\to(\varphi \wedge (\exists x)\psi(x))$ kde $x$ není volná v $\varphi$,
    \item $(\exists x)(\varphi(x)\to\psi)\to((\forall x)\varphi(x)\to \psi)$ kde $x$ není volná v $\psi$,
    \item $((\exists x)\varphi(x)\to\psi)\to(\forall x)(\varphi(x)\to \psi)$ kde $x$ není volná v $\psi$.
\end{enumerate}
\end{problem}


\medskip\begin{problem} Převeďte následující formule do PNF. Poté najděte jejich Skolemovy varianty.
    \begin{enumerate}
        \item $(\forall y)((\exists x)P(x,y)\to Q(y,z))\wedge (\exists y)((\forall x)R(x,y)\vee Q(x,y))$
        \item $(\exists x)R(x,y)\leftrightarrow (\forall y)P(x,y)$
        \item $\neg((\forall x)(\exists y)P(x,y)\to (\exists x)(\exists y)R(x,y))\wedge(\forall x)\neg(\exists y)Q(x,y)$
    \end{enumerate}
\end{problem}


\medskip\begin{problem} Převeďte na ekvisplnitelnou CNF formuli, zapište v množinové reprezentaci.
    \vspace{-6pt}
{\setlength{\columnsep}{-1cm}
\begin{multicols}{2}
\begin{enumerate}
    \item $(\forall y)(\exists x)P(x,y)$
    \item $\neg (\forall y)(\exists x)P(x,y)$
    \item $\neg (\exists x)((P(x)\to P(c))\wedge (P(x)\to P(d)))$
    \item $(\exists x)(\forall y)(\exists z)(P(x,z)\wedge P(z,y) \to R(x,y))$
\end{enumerate}
\end{multicols}
}
\end{problem}

    
\medskip\begin{problem} Skolemova varianta nemusí být ekvivalentní původní formuli, ověřte, že platí:
\begin{enumerate}
    \item $\models (\forall x)P(x,f(x)) \to (\forall x)(\exists y)P(x,y)$
    \item $\not\models (\forall x)(\exists y)P(x,y)\to (\forall x)P(x,f(x))$
\end{enumerate}

\end{problem}


\medskip\begin{problem}
Nechť $T=\{\varphi_1,\varphi_2\}$ je teorie v jazyce $L=\langle R\rangle$ s~rovností, kde:
\begin{align*}
\varphi_1=&\quad (\exists y)R(y,x)\\
\varphi_2=&\quad (\exists z)(R(z,x)\wedge R(z,y)\wedge (\forall w)(R(w,x) \wedge R(w,y)\to R(w,z)))
\end{align*}
\begin{enumerate}
\item Pomocí skolemizace sestrojte otevřeně axiomatizovanou teorii $T'$ (případně v širším jazyce $L'$) ekvisplnitelnou s $T$. {\it (2b)} 
\item Buď $\mathcal{A}=\langle\mathbb{N}\cup\{0\},R^A\rangle$, kde $(n,m)\in R^A$ právě když $n$ dělí $m$.  Nalezněte expanzi $\mathcal{A}'$ $L$-struktury $\mathcal{A}$ do jazyka $L'$ takovou, že $\mathcal{A}'\models T'$. {\it (2b)}
\end{enumerate}
\end{problem}


% \medskip\begin{problem}
% Nechť $T=\{\varphi_1,\varphi_2,\varphi_3\}$ je teorie v jazyce $L=\langle<,f,g,h\rangle$ s~rovností, kde:
% \begin{align*}
%     \varphi_1=&\quad (\forall u)(\exists v)(\forall x)(v<x \to u<f(x))\\
%     \varphi_2=&\quad (\exists u)(\forall v)(\exists x)(v<x \wedge \neg u<g(x))\\
%     \varphi_3=&\quad (\exists u)(\forall x)\neg u<h(x)
% \end{align*}
% \begin{enumerate}
%     %\item Nalezněte formule $\varphi'_1$, $\varphi'_2$ v prenexním tvaru a ekvivalentní s $\varphi_1$ resp. $\varphi_2$.
%     \item Pomocí skolemizace sestrojte otevřenou teorii $T'$ ekvisplnitelnou s $T$.
%     \item Buď $\mathcal{A}=\langle\mathbb{R},<,\mathrm{id},\mathrm{tg}',\sin\rangle$, kde $<$ má svůj obvyklý význam na $\mathbb{R}$, $\mathrm{id}(r)=r$ pro všechna $r\in\mathbb{R}$, $\mathrm{tg}'(k\pi/2)=0$ pro $k\in\mathbb{Z}$, $\mathrm{tg}'(r)=\mathrm{tg}(r)$ ($\mathrm{tg}$ je funkce tangens) pro $r\neq k\pi/2$ s $k\in\mathbb{Z}$ a $\sin$ je funkce sinus. Nalezněte expanzi $\mathcal{A}'$ struktury $\mathcal{A}$ takovou, že $\mathcal{A}'\models T'$.
% \end{enumerate}
% \end{problem}


\medskip\begin{problem} % the 2nd part could be moved to tutorial 9
Teorie těles $T$ jazyka $L=\langle +,-,\cdot,0,1\rangle$ obsahuje jeden axiom $\varphi$, který není otevřený: $x\neq 0\ \to\ (\exists y)(x\cdot y=1)$. Víme, že $T\models 0\cdot y=0$ a $T\models\ (x\ne 0\ \wedge\ x\cdot y=1\ \wedge\ x\cdot z=1)\ \to\ y=z$.
\begin{enumerate}
    \item Najděte Skolemovu variantu $\varphi_S$ formule $\varphi$ s novým funkčním symbolem $f$.
    \item Uvažme teorii $T'$ vzniklou z $T$ nahrazením $\varphi$ za $\varphi_S$. Platí $\varphi$ v $T'$?
    \item Lze každý model $T$ \emph{jednoznačně} rozšířit na model $T'$?
\end{enumerate}
Nyní uvažme formuli $\psi=x\cdot y=1\vee  (x=0 \wedge y=0)$.
\begin{enumerate}
    \setcounter{enumi}{3}
    \item Platí v $T$ axiomy existence a jednoznačnosti pro $\psi(x,y)$ a proměnnou $y$?
    \item Sestrojte extenzi $T''$ teorie $T$ o definici symbolu $f$ formulí $\psi$.
    \item Je $T''$ ekvivalentní teorii $T'$?
    \item Najděte $L$-formuli, která je v $T''$-ekvivalentní s formulí:
    $f(x\cdot y)=f(x)\cdot f(y)$
\end{enumerate}
\end{problem}


% \medskip\begin{problem} Popište Herbrandovo univerzum a uveďte příklad Herbrandovy struktury pro následující jazyky:
% \begin{enumerate}
%     \item $L=\langle P,Q,f,a,b \rangle$ kde  $P,Q$ jsou relační symboly, $P$ unární a $Q$ binární, $f$ je unární funkční symbol, a $a,b$ jsou konstantní symboly.
%     \item $L=\langle P,f,g,a \rangle$ kde $P$ je binární relační symbol, $f,g$ jsou unární funkční symboly, a symbol $a$ je konstantní.
% \end{enumerate}
% \end{problem}


\medskip\begin{problem} Sestrojte Herbrandův model dané teorie, nebo najděte nesplnitelnou konjunkci základních instancí jejích axiomů ($a,b$ jsou konstantní symboly v daném jazyce).
\begin{enumerate}
    \item $T=\{\neg P(x)\vee Q(f(x),y), \neg Q(x,b), P(a)\}$
    \item $T=\{\neg P(x)\vee Q(f(x),y), Q(x,b), P(a)\}$
    \item $T=\{P(x,f(x)),\neg P(x,g(x))\}$
    \item $T=\{P(x,f(x)),\neg P(x,g(x)), P(g(x),f(y)) \to P(x,y)\}$
\end{enumerate}
\end{problem}


\end{document}