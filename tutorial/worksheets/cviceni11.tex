\documentclass[a4paper]{article}

\usepackage{a4wide}
\usepackage{amsmath}
\usepackage{amssymb}
\usepackage{amsthm}
\usepackage{enumerate}
\usepackage{tikz}
\usepackage[utf8]{inputenc}


\theoremstyle{definition}
\newtheorem{problem}{Příklad}
\newtheorem*{ukol}{Domácí úkol}


\begin{document}

\section*{NAIL062 V\&P Logika: 11. cvičení}


\textbf{Témata:}
Unifikace. Rezoluce v predikátové logice.


\medskip\begin{problem}
 
\end{problem}


\medskip\begin{problem} Víme, že platí následující:
    \begin{itemize}
        \item Je-li cihla na (jiné) cihle, potom není na zemi.
        \item Každá cihla je na (jiné) cihle nebo na zemi.
        \item Žádná cihla není na cihle, která by byla na (jiné) cihle.
    \end{itemize}
    Vyjádřete tato fakta ve vhodném jazyce logiky prvního řádu a dokažte rezolucí následující tvrzení: ``Je-li cihla na (jiné) cihle, spodní cihla je na zemi.''
\end{problem}
        
    
\medskip\begin{problem} Víme, že platí následující:
    \begin{enumerate}[(a)]
        \item Každý holič holí všechny, kdo neholí sami sebe
        \item Žádný holič neholí nikoho, kdo holí sám sebe.
    \end{enumerate}
    Vyjádřete tato fakta ve vhodném jazyce logiky prvního řádu a dokažte rezolucí, že neexistují žádní holiči.
\end{problem}
        
    
\medskip\begin{problem}
    Ukažte, že daná množina klauzulí je zamítnutelná (rezolucí). Popište zamítnutí pomocí rezolučního stromu. V každém kroku rezoluce napište použitou unifikaci a podtrhněte rezolvované literály.
    \begin{align*}
        S=\{
            &\{P(a,x,f(y)),P(a,z,f(h(b))),\neg Q(y,z)\},\\
            &\{\neg Q(h(b),w),H(w,a)\},\\
            &\{\neg P(a,w,f(h(b))),H(x,a)\},\\
            &\{P(a,u,f(h(u))),H(u,a),Q(h(b),b)\},\\
            &\{\neg H(v,a)\}
        \}
    \end{align*}
\end{problem}


\medskip\begin{ukol}[3 body]


Kromě tohoto úkolu se připravte na zápočtový test. Vyřešte vzorový test (na webu).
\end{ukol}

\end{document}