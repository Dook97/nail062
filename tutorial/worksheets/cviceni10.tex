\documentclass[a4paper,12pt]{article}

\usepackage{a4wide}
\usepackage{amsmath}
\usepackage{amssymb}
\usepackage{amsthm}
\usepackage[czech]{babel}
\usepackage{bookmark}
\usepackage{enumerate}
\usepackage[T1]{fontenc}
\usepackage{forest}
\usepackage{hyperref}
\usepackage[utf8]{inputenc}
\usepackage{lmodern}
\usepackage{multicol}
\usepackage{tikz}

\theoremstyle{definition}
    \newtheorem{problem}{Příklad}

% \theoremstyle{remark}
%     \newtheorem*{steps}{Postup řešení}

\theoremstyle{plain}
    \newtheorem*{solution}{Řešení}
    

\DeclareRobustCommand\proves{\mathrel{|}\joinrel\mkern-.5mu\mathrel{-}}
\DeclareMathOperator{\Conseq}{Csq}
\DeclareMathOperator{\M}{M}

% hide solutions
\newif\ifhidesolutions
    \hidesolutionstrue
    % \hidesolutionsfalse

\ifhidesolutions
    \usepackage{environ}
    \NewEnviron{hide}{}
    \let\solution\hide
    \let\endsolution\endhide
\fi









\begin{document}

\section*{NAIL062 V\&P Logika: 10. cvičení}
% po 8. přednášce

% 2023 : 1, 2abd, 3, 4ab a první část c, (plán: 6a příště)
% příklady 5 a 7 přeskočit, tabla jsou trochu moc velká

\textbf{Témata:}
Extenze o definice. Definovatelné množiny. Tablo metoda v predikátové logice, jazyky s rovností.


\medskip\begin{problem}
Buď $T'$ extenze teorie $T=\{(\exists y)(x+y=0),(x+y=0)\wedge (x+z=0)\rightarrow y=z\}$ v jazyce $L=\langle +,0,\le\rangle$ s rovností o definice $<$ a unárního $-$ s axiomy
\begin{align*}
    -x=y\ \ &\leftrightarrow\ \ x+y=0\\
    x<y\ \ &\leftrightarrow\ \ x\le y\ \wedge\ \neg(x=y)
\end{align*}
Najděte formule v jazyce $L$, které jsou ekvivalentní v $T'$ s následujícími formulemi.
\medskip

(a) $x+(-x)=0$ \hfill (b) $x+(-y)<x$ \hfill (c) $-(x+y)<-x$\hfill{}

\end{problem}


\medskip\begin{problem}
    Mějme jazyk $L=\langle F \rangle$ s rovností, kde $F$ je binární funkční symbol. Najděte formule definující následující množiny (bez parametrů):
    \begin{enumerate}
        \item interval $(0,\infty)$ v $\mathcal A=\langle\mathbb R, \cdot\rangle$ kde $\cdot$ je násobení reálných čísel,
        \item množina $\{(x, 1/x)\mid x\neq 0\}$ ve stejné struktuře $\mathcal A$,
        \item množina všech nejvýše jednoprvkových podmnožin $\mathbb N$ v $\mathcal B=\langle\mathcal P(\mathbb N),\cup\rangle$,
        \item množina všech prvočísel v $\mathcal C=\langle \mathbb N\cup\{0\}, \cdot\rangle$.
    \end{enumerate}
\end{problem}


\medskip\begin{problem}
    Předpokládejme, že:
    \begin{itemize}\it
    \item Všichni viníci jsou lháři.
    \item Alespoň jeden z obviněných je také svědkem.
    \item Žádný svědek nelže.
    \end{itemize}
    Dokažte tablo metodou, že: {\it Ne všichni obvinění jsou viníci.}
\end{problem} 
    

\medskip\begin{problem}
Uvažte následující tvrzení:
\begin{enumerate}[label=(\roman*)] \it 
    \item Nula je malé číslo.
    \item Číslo je malé, právě když je blízko nuly.
    \item Součet dvou malých čísel je malé číslo.
    \item Je-li $x$ blízko $y$, potom $f(x)$ je blízko $f(y)$.
\end{enumerate}
Chceme dokázat, že platí: {\it (v) Jsou-li $x$ a $y$ malá čísla, potom $f(x+y)$ je blízko $f(0)$.}

\begin{enumerate}
\item Formalizujte %tvrzení po řadě 
jako sentence $\varphi_1,\dots,\varphi_5$ v jazyce $L=\langle M,B,f,+,0\rangle$ bez rovnosti.
%, kde $M$ je unární relační symbol ($M(x)$ značí, že ``$x$ je malé''), $B$ je binární relační symbol ($B(x,y)$ značí, že ``$x$ je blízko $y$''), $f$ je unární funkční symbol, $+$ je binární funkční symbol (označující součet) a $0$ je konstantní symbol.
\item Sestrojte dokončené tablo z teorie $T=\{\varphi_1,\varphi_2,\varphi_3,\varphi_4\}$ s položkou $F\varphi_5$ v kořeni.
% {\it (Nápověda: Axiomy rovnosti nejsou v tablu nezbytné.)}
\item Rozhodněte, zda platí $T\models \varphi_5$, a zda platí $T\models M(f(0))$.

\item Pokud existují, uveďte alespoň dvě kompletní jednoduché extenze teorie $T$.
\end{enumerate}
\end{problem}

    
\medskip\begin{problem} Nechť $L(x,y)$ reprezentuje \emph{``existuje let z $x$ do $y$''} a $S(x,y)$ reprezentuje \emph{``existuje spojení z $x$ do $y$''}. Předpokládejme, že
    \begin{itemize}  
    \item Z Prahy lze letět do Bratislavy, Londýna a New Yorku, a z New Yorku do Paříže,
    \item $(\forall x)(\forall y)(L(x,y) \to L(y,x))$,
    \item $(\forall x)(\forall y)(L(x,y)\to S(x,y))$,
    \item $(\forall x)(\forall y)(\forall z)(S(x,y)\wedge L(y,z)\to S(x,z))$.
    \end{itemize}
    Dokažte tablo metodou, že existuje spojení z Bratislavy do Paříže.
\end{problem}


\medskip\begin{problem} Mějme teorii $T^*$ s axiomy rovnosti. Pomocí tablo metody ukažte, že:
\begin{enumerate} 
    \item $T^*\models x=y\ \to\ y=x$\hfill(symetrie)
    \item $T^*\models (x=y\ \wedge\ y=z)\ \to\ x=z$\hfill(tranzitivita)
\end{enumerate}
{\it Hint:} Pro (a) použijte axiom rovnosti $(iii)$ pro $x_1=x$, $x_2=x$, $y_1=y$ a $y_2=x$, \newline
    na (b) použijte $(iii)$ pro $x_1=x$, $x_2=y$, $y_1=x$ a $y_2=z$.
\end{problem}


\medskip\begin{problem} 
Buď $T$ následující teorie v jazyce $L=\langle R,f,c,d\rangle$ s rovností, kde $R$ je binární relační symbol,  $f$ unární funkční symbol, a $c,d$ konstantní symboly:
$$
T=\{R(x,x),R(x,y)\wedge R(y,z)\to R(x,z),R(x,y)\wedge R(y,x)\to x=y,R(f(x),x)\}
$$
Označme jako $T'$ generální uzávěr $T$. Nechť $\varphi$ a $\psi$ jsou následující formule:
\begin{align*}
    \varphi &= R(c,d) \wedge (\forall x)(x=c\vee x=d)\\
    \psi &= (\exists x)R(x,f(x))
\end{align*}
\begin{enumerate}
    \item Sestrojte tablo důkaz formule $\psi$ z teorie $T'\cup\{\varphi\}$. (Tablo může vyjít poměrně velké. Pro zjednodušení můžete kromě axiomů rovnosti v tablu přímo používat axiom $(\forall x)(\forall y)(x=y\to y=x)$, což je jejich důsledek.)
    \item Ukažte, že $\psi$ není důsledek teorie $T$, tím že najdete model $T$, ve kterém $\psi$ neplatí.
    \item Kolik kompletních jednoduchých extenzí (až na ekvivalenci) má teorie $T\cup \{\varphi\}$? Uveďte dvě.
    \item Buď $S$ následující teorie v jazyce $L'=\langle R\rangle$ s rovností. Je $T$ konzervativní extenzí $S$?
     $$S=\{R(x,x),R(x,y)\wedge R(y,z)\to R(x,z),R(x,y)\wedge R(y,x)\to x=y\}$$     
\end{enumerate}
\end{problem}




\end{document}