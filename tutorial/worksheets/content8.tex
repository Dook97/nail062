\section*{NAIL062 V\&P Logika: 8. sada příkladů -- Tablo metoda v predikátové logice}
% po 5. přednášce


\subsection*{Výukové cíle:} Po absolvování cvičení student

    \begin{itemize}\setlength{\itemsep}{0pt}
        \item rozumí tomu, jak se liší tablo metoda v predikátové logice od výrokové logiky, umí formálně definovat všechny potřebné pojmy
        \item zná atomická tabla pro kvantifikátory, rozumí jejich použití
        \item umí sestrojit dokončené tablo pro danou položku z dané (i nekonečné) teorie
        \item umí popsat kanonický model pro danou dokončenou bezespornou větev tabla
        \item zná axiomy rovnosti a rozumí jejich souvislosti s pojmy kongruence, faktorstruktura
        \item umí aplikovat tablo metodu k řešení daného problému (slovní úlohy, aj.)
        \item rozumí použití tablo metody pro jazyky s rovností, umí aplikovat na jednoduchých příkladech
        \item zná větu o kompaktnosti predikátové logiky, umí ji aplikovat
    \end{itemize}
    

\section*{Příklady na cvičení}


\begin{problem}


    \begin{solution}
                    
    \end{solution}

\end{problem}

        
        
\section*{Další příklady k procvičení}

        
\section*{K zamyšlení}


