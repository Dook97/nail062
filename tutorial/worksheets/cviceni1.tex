\documentclass[a4paper,11pt]{amsart}

\usepackage{a4wide}
\usepackage{amsmath}
\usepackage{amssymb}
\usepackage{amsthm}
\usepackage[czech]{babel}
\usepackage{bookmark}
\usepackage{enumerate}
\usepackage[T1]{fontenc}
\usepackage{forest}
\usepackage{hyperref}
\usepackage[utf8]{inputenc}
\usepackage{lmodern}
\usepackage{multicol}
\usepackage{tikz}

\theoremstyle{definition}
    \newtheorem{problem}{Příklad}

% \theoremstyle{remark}
%     \newtheorem*{steps}{Postup řešení}

\theoremstyle{plain}
    \newtheorem*{solution}{Řešení}
    

\DeclareRobustCommand\proves{\mathrel{|}\joinrel\mkern-.5mu\mathrel{-}}
\DeclareMathOperator{\Conseq}{Csq}
\DeclareMathOperator{\M}{M}

% hide solutions
\newif\ifhidesolutions
    \hidesolutionstrue
    % \hidesolutionsfalse

\ifhidesolutions
    \usepackage{environ}
    \NewEnviron{hide}{}
    \let\solution\hide
    \let\endsolution\endhide
\fi








\begin{document}

\section*{NAIL062 V\&P Logika: 1. cvičení}
% po 1. přednášce

\subsection*{Výukové cíle:} Po absolvování cvičení student

    \begin{itemize}\setlength{\itemsep}{0pt}
        \item rozumí pojmům syntaxe výrokové logiky (jazyk, prvovýrok, výrok, strom výroku, podvýrok, teorie), umí je formálně definovat a uvést příklady
        \item rozumí pojmům model, důsledek teorie, umí je formálně definovat a uvést příklady
        \item umí formalizovat daný systém (slovní/výpočetní úlohu, apod.) ve výrokové logice
        \item umí najít modely dané teorie
        \item umí rozhodnout, zda je daný výrok důsledkem dané teorie
        \item má zkušenost s použitím (s pomocí instruktora) tablo metody a rezoluční metody k důkazu vlastností daného systému (např. k řešení slovní úlohy)
    \end{itemize}


\section*{Příklady na cvičení}


\begin{problem}\label{problem:dragons}

    Ztratili jsme se v labyrintu a před námi jsou troje dveře: červené, zelené a modré. Víme, že za právě jedněmi dveřmi je cesta ven, za ostatními je drak. Na dveřích jsou nápisy:
    \begin{itemize}
        \item Červené dveře: {\it ``Cesta ven je za těmito dveřmi.''}
        \item Modré dveře: {\it ``Cesta ven není za těmito dveřmi.''}
        \item Zelené dveře: {\it ``Cesta ven není za modrými dveřmi.''}
    \end{itemize}
    Víme, že alespoň jeden z nápisů je pravdivý a alespoň jeden je lživý. Kudy vede cesta ven?

    \begin{enumerate}[(a)]
        \item Zvolte vhodný jazyk (množinu prvovýroků) $\mathbb P$.
        \item Formalizujte všechny znalosti jako teorii $T$ v jazyce $\mathbb P$. (Pozor: Axiomy nejsou nápisy na dveřích, ty nemusí být pravdivé.)
        \item Najděte všechny modely teorie $T$.
        \item Formalizujte tvrzení ``Cesta ven je za červenými/modrými/zelenými dveřmi'' jako výroky $\varphi_1,\varphi_2,\varphi_3$ nad $\mathbb P$. Je některý z těchto výroků důsledkem $T$?
        \item Vyzkoušejte si použití tablo metody: Zkonstruujte tablo z teorie $T$ s položkou $F\varphi_i$ v kořeni, budou všechny větve sporné? (Pokuste se vymyslet správné kroky konstrukce tabla, inspirujte se příkladem z přednášky.)
        \item Vyzkoušejte si použití rezoluční metody: Převeďte axiomy teorie $T$, a také výrok $\neg\varphi_i$, do konjunktivní normální formy (CNF). Pokuste se sestrojit rezoluční zamítnutí, zakreslete ho ve formě rezolučního stromu. (Pozor: Nezapomeňte znegovat dokazovaný výrok $\varphi_i$.)
    \end{enumerate}  

    \begin{solution}
        %todo
    \end{solution}

\end{problem}


\begin{problem}

    Uvažme \emph{vrcholová pokrytí} následujícího grafu:    
    \begin{center}
        \begin{tikzpicture}[every node/.style={circle,fill=blue!10,draw,minimum size=0.5cm,node distance=1.5cm}]
        \node (1) {$1$};
        \node[right of=1] (2) {$2$};
        \node[below of=2] (3) {$3$};
        \node[left of=3] (4) {$4$};
        \path[draw] (1) -- (2) -- (3) -- (4) -- (1) -- (3);
        \node[right of=3] (5) {$5$};
        \path[draw] (2) -- (5) -- (3);
        \end{tikzpicture}
    \end{center}
    Chceme pro dané $k>0$ zjistit, zda má tento graf nejvýše $k$-prvkové vrcholové pokrytí.

    \begin{enumerate}[(a)]
        \item Zvolte vhodný jazyk (množinu prvovýroků) $\mathbb P$.
        \item Formalizujte ve výrokové logice problém, zda graf na obrázku má nejvýše $k$-prvkové vrcholové pokrytí, pro pevně zvolené $k$. Označme výslednou teorii jako $T_k$.
        \item Ukažte, že $T_2$ nemá žádné modely, tj. graf nemá 2-prvkové vrcholové pokrytí.
        \item Uměli byste k tomu využít tablo metodu?
        \item Uměli byste k tomu využít rezoluční metodu?
        \item Najděte všechna 3-prvková vrcholová pokrytí.            
    \end{enumerate}

    \begin{solution}
        %todo
        Vrcholové pokrytí grafu je množina vrcholů $S$ taková, že každý vrchol grafu je buď v $S$, nebo sousedí s nějakým vrcholem z $S$.
    \end{solution}

\end{problem}


\section*{Další příklady k procvičení}


\begin{problem}
    
    Uvažme následující tvrzení:
    \begin{enumerate}[(i)]\it
        \item Ten, kdo je dobrý běžec a má dobrou kondici, uběhne maraton.
        \item Ten, kdo nemá štěstí a nemá dobrou kondici, neuběhne maraton.
        \item Ten, kdo uběhne maraton, je dobrý běžec.
        \item Budu-li mít štěstí, uběhnu maraton.
        \item Mám dobrou kondici.
    \end{enumerate}
    Podobně jako v Příkladu \ref{problem:dragons} popište situaci pomocí výrokové logiky:
    \begin{enumerate}[(a)]
        \item Formalizujte tato tvrzení jako teorii $T$ nad vhodnou množinou prvovýroků.
        \item Najděte všechny modely teorie $T$. 
        \item Pokuste se využít k hledání modelů také \emph{tablo metodu}.
        \item Napište několik různých důsledků teorie $T$.
        \item Najděte CNF teorii ekvivalentní teorii $T$.
    \end{enumerate}
    
\end{problem}


\begin{problem}

    Mějme tři bratry, každý z nich buď vždy říká pravdu anebo vždy lže.
    \begin{enumerate}[(i)]
        \item Nejstarší říká: \emph{``Oba mí bratři jsou lháři.''}
        \item Prostřední říká: \emph{``Nejmladší je lhář.''}
        \item Nejmladší říká: \emph{``Nejstarší je lhář.''}
    \end{enumerate}
    Pomocí výrokové logiky ukažte, že nejmladší bratr je pravdomluvný.
     
\end{problem}


\begin{problem}
    Mějme pevně dané Sudoku. Popište, jak vytvořit teorii (ve výrokové logice), jejíž modely jednoznačně odpovídají validním řešením.
\end{problem}


\begin{problem}

    Formalizujte následující tvrzení ve výrokové logice:
    \begin{itemize}
        \item Borůvky podél cesty jsou zralé, ale králíčci v oblasti nebyli pozorováni.

        \item Králíčci v oblasti nebyli pozorováni a procházení po cestě je bezpečné, ale borůvky podél cesty jsou zralé.
        
        \item Pokud jsou borůvky podél cesty zralé, pak je procházení po cestě bezpečné pouze tehdy, pokud králíčci nebyli v oblasti pozorováni.
        
        \item Procházet se podél cesty není bezpečné, ale v oblasti nebyli pozorováni králíčci a borůvky podél cesty jsou zralé.
        
        \item Aby bylo procházení po cestě bezpečné, je nezbytné, ale nedostačující, aby borůvky podél cesty nebyly zralé a králíčci nebyli v oblasti pozorováni.
        
        \item Procházení po cestě není bezpečné, kdykoli jsou borůvky podél cesty jsou zralé a v oblasti byli pozorováni králíčci.
    
    \end{itemize}
    
\end{problem}





\begin{problem}

    Formalizujte následující vlastnosti matematických objektů ve výrokové logice:
    \begin{enumerate}[(a)]
        \item Pro pevně daný (konečný) graf $G$, že je regulární stupně 3.
        \item Pro pevně daný (konečný) graf $G$, že má perfektní párování.
        \item Pro pevně danou částečně uspořádanou množinu, že je totálně (lineárně) uspořádaná.
        \item Pro pevně danou částečně uspořádanou množinu, že má nejmenší prvek.
    \end{enumerate}

\end{problem}


\begin{problem}

    Nakreslete strom výroku pro následující výroky:
    \begin{enumerate}
        \item $(p \to q) \leftrightarrow \neg (p \wedge \neg q)$
        \item $(p \leftrightarrow q) \leftrightarrow ((p \vee q) \to (p \wedge q))$
    \end{enumerate}

\end{problem}


\begin{problem}

    Najděte množinu modelů následujících výroků:
    \begin{enumerate}
        \item $(p \to q) \leftrightarrow \neg (p \wedge \neg q)$
        \item $(p \leftrightarrow q) \leftrightarrow ((p \vee q) \to (p \wedge q))$
    \end{enumerate}

\end{problem}


\section*{K zamyšlení}


\begin{problem}

    Připomeňte si definici \emph{stromu výroku}.
    \begin{enumerate}[(a)]
        \item Dokažte podrobně, že každý výrok má jednoznačně určený strom.
        \item Platilo by to, i kdybychom v definici výroku nahradili symboly pro levou a pravou závorku `(', `)' symbolem `|'?
        \item Co by se stalo, pokud bychom závorky vůbec nepsali?
    \end{enumerate}

\end{problem}


\begin{problem}

    Připomeňte si definici \emph{výroku}. Jaké jsou možné délky výroků, tj. pro jaká $n\in\mathbb N$ existuje výrok délky právě $n$? (Uvažujte konečný jazyk, každý prvovýrok je jen jeden symbol. Výrok obsahuje všechny závorky dané definicí, konvence o vynechávání závorek se týká jen toho, jak výrok zapisujeme my.)

\end{problem}





\end{document}


\medskip\begin{problem}
Mějme daný graf $G$ (neorientovaný, bez smyček) a dva jeho vrcholy $u,v$. Formalizujte následující vlastnosti ve výrokové logice:
\begin{enumerate}
    \item $G$ je bipartitní,
    \item $G$ má perfektní párování,
    \item $u$ a $v$ leží v jedné komponentě souvislosti,
    \item $G$ je souvislý.
\end{enumerate}
\end{problem}


\medskip\begin{problem}
Najděte formule v predikátové logice v jazyce grafů, které v \emph{teorii grafů} (neorientovaných, bez smyček) vyjadřují následující vlastnosti. Kdy to lze v logice prvního řádu, a kdy je třeba logika druhého řádu?
\begin{enumerate}
    \item graf obsahuje vrchol stupně 1
    \item graf je regulární stupně 3,
    \item graf obsahuje $k$-kliku (pro nějaké fixní $k$),
    \item existuje cesta délky $k$ z vrcholu $u$ do vrcholu $v$ (pro nějaké fixní $k$),
    \item vrcholy $u$ a $v$ mají alespoň jednoho společného souseda,
    \item graf je bipartitní,
    \item graf má perfektní párování,
    \item vrcholy $u$ a $v$ leží v jedné komponentě souvislosti,
    \item graf je souvislý.    
\end{enumerate}
\end{problem}



\end{document}
