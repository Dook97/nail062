\section*{NAIL062 V\&P Logika: 6. sada příkladů -- Základy predikátové logiky}
% po 5. přednášce


\subsection*{Výukové cíle:} Po absolvování cvičení student

\begin{itemize}\setlength{\itemsep}{0pt}
    \item rozumí pojmu struktura, signatura, umí je formálně definovat a uvést příklady
    \item rozumí pojmům syntaxe predikátové logiky (jazyk, term, atomická formule, formule, teorie, volná proměnná, otevřená formule, sentence, instance, varianta) umí je formálně definovat a uvést příklady
    \item rozumí pojmům sémantiky predikátové logiky (hodnota termu, pravdivostní hodnota, platnost [při ohodnocení], model, pravdivost/lživost v modelu/v teorii, nezávislost [v teorii], důsledek teorie) umí je formálně definovat a uvést příklady
    \item rozumí pojmu kompletní teorie a jeho souvislosti s elementární ekvivalencí struktur, umí obojí definovat, aplikovat na příkladě
    \item zná základní příklady teorií (teorie grafů, uspořádání, algebraické teorie)
    \item umí popsat modely dané teorie    
\end{itemize}
    

\section*{Příklady na cvičení}


\begin{problem}


    \begin{solution}
                    
    \end{solution}

\end{problem}

        
        
\section*{Další příklady k procvičení}

        
\section*{K zamyšlení}


