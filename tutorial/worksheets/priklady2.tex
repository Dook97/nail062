\documentclass[a4paper,11pt]{amsart}

\usepackage{a4wide}
\usepackage{amsmath}
\usepackage{amssymb}
\usepackage{amsthm}
\usepackage[czech]{babel}
\usepackage{bookmark}
\usepackage{enumerate}
\usepackage[T1]{fontenc}
\usepackage{forest}
\usepackage{hyperref}
\usepackage[utf8]{inputenc}
\usepackage{lmodern}
\usepackage{multicol}
\usepackage{tikz}

\theoremstyle{definition}
    \newtheorem{problem}{Příklad}

% \theoremstyle{remark}
%     \newtheorem*{steps}{Postup řešení}

\theoremstyle{plain}
    \newtheorem*{solution}{Řešení}
    

\DeclareRobustCommand\proves{\mathrel{|}\joinrel\mkern-.5mu\mathrel{-}}
\DeclareMathOperator{\Conseq}{Csq}
\DeclareMathOperator{\M}{M}

% hide solutions
\newif\ifhidesolutions
    \hidesolutionstrue
    % \hidesolutionsfalse

\ifhidesolutions
    \usepackage{environ}
    \NewEnviron{hide}{}
    \let\solution\hide
    \let\endsolution\endhide
\fi








\begin{document}

\section*{NAIL062 V\&P Logika: 2. sada příkladů}
% po 2. přednášce

\subsection*{Výukové cíle:} Po absolvování cvičení student

    \begin{itemize}\setlength{\itemsep}{0pt}
        \item rozumí pojmům sémantiky výrokové logiky (pravdivostní hodnota, pravdivostní funkce, model, platnost, tautologie, spornost, nezávislost, splnitelnost, ekvivalence), umí je formálně definovat a uvést příklady
        \item umí rozhodnout, zda je množina logických spojek univerzální
        \item zná terminologii pro výroky v CNF a DNF %([pozitivní/negativní/opačný] literál, [prázdná/jednotková] klauzule, [prázdná/jednotková] elemntární konjunkce, prázdný výrok v CNF/DNF)
        \item umí převést daný výrok resp. konečnou teorii do CNF a do DNF, a to pomocí množiny modelů i pomocí ekvivalentních úprav
        \item rozumí terminologii týkající se vlastností teorií (sporná, bezesporná/splnitelná, kompletní, důsledky, $T$-ekvivalence), umí pojmy formálně definovat a uvést příklady
        \item rozumí pojmu [jednoduchá, konzervativní] extenze, umí je formálně definovat, uvést příklady
        \item umí v konkrétním případě rozhodnout, zda jde o [jednoduchou, konzervativní] extenzi, a zdůvodnit jak z definice, tak i pomocí sémantického kritéria
    \end{itemize}


\section*{Příklady na cvičení}


\begin{problem}

    Uveďte příklad výroku v jazyce $\mathbb P=\{p,q,r\}$, který je (a) pravdivý, (b) sporný, (c) nezávislý, (d) ekvivalentní s $(p\wedge q)\to\neg r$, (e) má za modely právě $\{(1,0,0),(1,0,1),(0,0,1)\}$.


    \begin{solution}
            
    \end{solution}

\end{problem}


\begin{problem} 
    
    Jsou tyto množiny logických spojek univerzální? (a) $\{\vee, \rightarrow, \leftrightarrow\}$, (b) $\{\vee, \wedge, \rightarrow\}$    

    \begin{solution}
            
    \end{solution}


\end{problem}


\begin{problem} 
    
    Převeďte následující výrok do CNF a DNF. Proveďte to (a) sémanticky (pomocí pravdivostní tabulky), (b) ekvivalentními úpravami:
    $$
    (\neg p \vee q)\to (\neg q \wedge r)
    $$

    \begin{solution}
            
    \end{solution}
        
\end{problem}


\begin{problem}\label{problem:properties-of-theories}

    Mějme teorii $T=\{p\wedge q,p\to\neg q,q\}$ v jazyce $\mathbb P=\{p,q\}$.
    \begin{enumerate}[(a)]
        \item Rozhodněte, zda je teorie $T$ [konzistentní/splnitelná/kompletní]. %(konzistentní=bezesporná, kompletní=úplná)
        \item Uveďte příklad výroku $\varphi$, který je [platný/nesplnitelný/nezávislý] v $T$
        \item Uveďte příklad extenze $T'$ teorie $T$ (pokud existuje, a pokud možno neekvivalentní s $T$), která je [jednoduchá / konzervativní/kompletní/konzervativní jednoduchá/kompletní jednoduchá/kompletní konzervativní].
        \item Na příkladech extenzí ukažte, že platí sémantické kritérium.
    \end{enumerate}

    \begin{solution}
            
    \end{solution}
    
\end{problem}


\begin{problem}

    Dokažte nebo vyvraťte (nebo uveďte správný vztah), že pro každou teorii $T$ a výroky $\varphi$, $\psi$ v jazyce $\mathbb{P}$ platí:
    \begin{enumerate}[(a)]
        \item $T \models \varphi$,\ \  právě když \ \ $T \not\models \neg \varphi$
        \item $T \models \varphi$ a $T \models \psi$,\ \ právě když \ \ $T \models \varphi \wedge \psi$
        \item $T \models \varphi$ nebo $T \models \psi$,\ \ právě když \ \ $T \models \varphi \vee \psi$
        \item $T \models \varphi \to \psi$ and $T \models \psi \to \chi$,\ \ právě když \ \ $T \models \varphi \to \chi$
    \end{enumerate}

    \begin{solution}
            
    \end{solution}
    
\end{problem}


\section*{Další příklady k procvičení}


\begin{problem}
        
    Mějme teorii $T=\{\neg q \to (\neg p \vee q),\ \neg p \to q,\ r \to q\}$ v jazyce $\{p, q, r\}$.
    \begin{enumerate}[(a)]
        \item Uveďte příklad následujícího: výrok pravdivý v $T$, lživý v $T$, nezávislý v $T$, splnitelný v $T$, a dvojice $T$-ekvivalentních výroků.
        \item Které z následujících výroků jsou pravdivé, lživé, nezávislé, splnitelné v $T$? $T$-ekvivalentní? 
        $$
        p, \ \neg q, \ \neg p\vee q, \ p\to r,\ \neg q\to r, \ p\vee q\vee r
        $$
    \end{enumerate}

\end{problem}


\begin{problem} 
    
    Jsou následující množiny logických spojek univerzální? Zdůvodněte.
    \begin{enumerate}[(a)]
        \item $\{\downarrow\}$ kde $\downarrow$ je Peirce arrow (NOR),
        \item $\{\uparrow\}$ kde $\uparrow$ je Sheffer stroke (NAND),
    \end{enumerate}

\end{problem}


\begin{problem} 
        
    Určete množinu modelů dané formule. Využijte toho, že je v DNF resp. v CNF.
    \begin{enumerate}[(a)]
        \item $(\neg p_1 \wedge \neg p_2)\vee( \neg p_1 \wedge p_2)\vee( p_1 \wedge \neg p_2)\vee( p_2 \wedge \neg p_3)$
        \item $(\neg p_1 \vee \neg p_2)\wedge( \neg p_1 \vee p_2)\wedge( p_1 \vee \neg p_2)\wedge( p_2 \vee \neg p_3)$
    \end{enumerate}

\end{problem}


\begin{problem} 
    
    Převeďte do CNF a DNF oběma metodami: $(\neg p \to (\neg q \to r))\to p$
    
\end{problem}


\begin{problem} 
    
    Najděte (co nejkratší) CNF a DNF reprezentace Booleovské funkce $\mathrm{maj}\colon\{0,1\}^3\to \{0,1\}$, která vrací převládající hodnotu mezi 3 vstupy.

\end{problem}


\begin{problem}
    
    Stejné zadání, jako Příklad~\ref{problem:properties-of-theories}, ale pro teorii $T=\{(p\wedge q)\to r, \neg r\vee(p\wedge q)\}$ v jazyce $\mathbb P=\{p,q,r\}$.
    
\end{problem}


\begin{problem}
    
    Dokažte nebo vyvraťte (nebo uveďte správný vztah), že pro libovolné teorie $T$, $S$ nad~$\mathbb{P}$ platí:
    \begin{enumerate}[(a)]
        \item $S\subseteq T \Rightarrow \Conseq(T) \subseteq \Conseq(S)$
        \item $\Conseq(S\cup T)=\Conseq(S) \cup \Conseq(T)$
        \item $\Conseq(S\cap T)=\Conseq(S) \cap \Conseq(T)$
    \end{enumerate}

\end{problem}


\section*{K zamyšlení}


\begin{problem}
    
    Ukažte, že $\wedge$ a $\vee$ nestačí k definování všech Booleovských operátorů, tj. že $\{\wedge,\vee\}$ není \emph{univerzální} množina logických spojek.

\end{problem}


\begin{problem}
    
    Uvažte Booleovský operátor $\mathrm{IFTE}(p, q, r)$ definovaný jako `if $p$ then $q$ else $r$'. 
    \begin{enumerate}[(a)]
        \item Zkonstruujte pravdivostní tabulku.
        \item Ukažte, že všechny základní Booleovské operátory ($\neg, \to, \wedge,\vee,\dots$) lze vyjádřit pomocí IFTE a konstant TRUE a FALSE.
    \end{enumerate}

\end{problem}


\begin{problem} 
    
    Buď $\mathbb P$ spočetně nekonečná množina prvovýroků.    
    \begin{enumerate}[(a)]
        \item Ukažte, že již neplatí, že každou $K\subseteq \mathrm{M}_\mathbb P$ lze axiomatizovat výrokem v CNF i výrokem v DNF.
        \item  Uveďte příklad množiny modelů $K$, kterou nelze axiomatizovat ani výrokem v CNF, ani výrokem v DNF.
    \end{enumerate}

\end{problem}


\begin{problem} 
    
    Najděte CNF a DNF reprezentaci $n$-ární parity, tj. Booleovské funkce $\mathrm{par}\colon\{0,1\}^n\to \{0,1\}$,
    která vrací XOR všech vstupních hodnot:
    $$
    \mathrm{par}(x_1,\dots,x_n)=(x_1+\dots+x_n)\bmod 2
    $$
    Zkuste to pro malé hodnoty $n$.

\end{problem}


\begin{problem}

    Uvažme nekonečnou výrokovou teorii $T=\{p_i \to p_{i+1}\mid i\in \mathbb{N}\}$ nad $\mathrm{var}(T)$. 
    \begin{enumerate}[(a)]
        \item Najděte všechny modely $T$.
        \item Které výroky ve tvaru  $p_i \to p_j$ jsou důsledky $T$?        
    \end{enumerate}

\end{problem}


\end{document}
















