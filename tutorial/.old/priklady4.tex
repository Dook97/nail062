\documentclass{amsart}
\usepackage[utf8]{inputenc}
\usepackage{enumerate}
\usepackage{graphicx}

\usepackage{a4wide}

\theoremstyle{definition}
\newtheorem{problem}{Příklad}



\begin{document}

\section*{Cvičení z logiky: 4. sada příkladů}

\bigskip\bigskip\bigskip

\subsection*{Témata:} Více o Tablo metodě. Věta o dedukci. Věta o kompaktnosti a aplikace. Rezoluční metoda, množinová reprezentace formulí v CNF, rezoluční uzávěr, strom dosazení. Hilbertovský kalkul.

\medskip\hrule

\begin{problem} Navrhněte vhodná atomická tabla pro Peirceovu spojku $\downarrow$ (NOR), pro Shefferovu spojku $\uparrow$ (NAND), a pro ternárí operátor ``if p then q else r'' (IFTE).
\end{problem}\medskip

\begin{problem}
Dokažte přímo (transformací tabel) větu o dedukci, tj. že pro každou teorii $T$ a výroky $\varphi$, $\psi$ platí
$$T \vdash \varphi\to \psi\text{\ \ právě když\ \ }T,\varphi \vdash \psi.$$
\end{problem}\medskip

% \begin{problem}
% V důkazu lemmatu o úplnosti v přednášce jsme ověřili, že shoduje-li se ohodnocení $v$ s každou položkou dokončené větve $V$ do dané hloubky $i$ vytvořujícího stromu, shoduje se i s položkou tvaru $T(\varphi \wedge \psi)$ nebo $F(\varphi \wedge \psi)$ na $V$, kde $\varphi \wedge \psi$ má hloubku  $i+1$. Dokažte obdobné tvrzení pro ostatní logické spojky.
% \end{problem}\medskip

\hrule

\begin{problem}
Ukažte, že každý spočetný rovinný graf je obarvitelný čtyřmi barvami.
\end{problem}\medskip

\begin{problem}
Ukažte, že každé spočetné částečné uspořádání lze rozšířit na úplné (lineární) uspořádání.
\end{problem}\medskip

% \begin{problem}
% Nechť $\mathcal S$ je spočetný neprázdný systém (množina) neprázdných konečných množin. \emph{Prostý selektor} na $\mathcal S$ je prostá funkce $f\colon \mathcal{S} \to \bigcup \mathcal{S}$ taková, že $f(S)\in S$ pro každou $S\in \mathcal{S}$. Ukažte, že $\mathcal{S}$ má prostý selektor, právě když ho má každá neprázdná konečná část $\mathcal{S}$.
% \end{problem}\medskip

\hrule

\begin{problem}
Označme jako $\varphi$ výrok $\neg (p \vee q) \to (\neg p \wedge \neg q)$.
\begin{enumerate}[(a)]
\item Převeďte $\neg \varphi$ do CNF a množinové reprezentace.
\item Najděte rezoluční zamítnutí $\neg \varphi$, tj. důkaz $\varphi$.
\end{enumerate}
\end{problem}\medskip


\begin{problem}
Najděte rezoluční uzávěry $\mathcal{R}(S)$ pro následující výroky $S$:
\begin{enumerate}
\item $\{\{p,q\},\{\neg p, \neg q\},\{\neg p, q\}\}$
\item $\{\{p,q\},\{p,\neg q\},\{\neg p,\neg q\}\}$
\item $\{\{p,\neg q,r\},\{q,r\},\{\neg p, r\},\{q,\neg r\},\{\neg q\}\}$
\end{enumerate}
\end{problem}\medskip

\begin{problem}
Najděte rezoluční zamítnutí následujících výroků:
\begin{enumerate}
\item $(p\leftrightarrow (q\to r))\wedge((p\leftrightarrow q)\wedge(p\leftrightarrow \neg r))$
\item $\neg(((p\to q)\to \neg q)\to \neg q)$
\end{enumerate}
\end{problem}\medskip


\begin{problem}
Dokažte rezolucí, že v teorii $T=\{\neg p \to \neg q,\neg q \to \neg r, (r\to p)\to s\}$ platí výrok $s$.
\end{problem}\medskip

\hrule


\begin{problem}
Dokažte, že je-li $S=\{C_1,C_2\}$ splnitelná a $C$ je rezolventa $C_1$ a $C_2$, potom je i $C$ splnitelná.
\end{problem}\medskip

\begin{problem}
Zkonstruujte \emph{strom dosazení} pro formuli $S=\{\{p,r\},\{q,\neg r\},\{\neg q\},\{\neg p,t\},\{\neg s\},\{s,\neg t\}\}$.
\end{problem}\medskip


\begin{problem}Předpokládejme, že máme k dispozici MgO, H$_2$, O$_2$, a C a můžeme provádět následující reakce:
\begin{align*}&(1)\quad\text{MgO\ +\ H$_2$\ \ $\to$\ \ Mg\ +\ H$_2$O}\\
&(2)\quad\text{C\ +\ O$_2$\ \ $\to$\ \ CO$_2$}\\
&(3)\quad\text{CO$_2$\ +\ H$_2$O\ \ $\to$\ \ H$_2$CO$_3$}
\end{align*}
\begin{enumerate}
\item Reprezentujte naše možnosti výrokem (nad vhodně zvoleným jazykem) a převeďte ho do množinové reprezentace
\item Pomocí LI-rezoluce dokažte, že můžeme získat H$_2$CO$_3$.
\end{enumerate}
\end{problem}\medskip

\hrule

\begin{problem}
V Hilbertově kalkulu dokažte pro libovolné formule následující vztahy:
\begin{enumerate}[(a)]
\item $\vdash_H\ p\to p$
\item $\{\neg p\}\ \vdash_H\ p\to q$
\item $\{\neg(\neg p)\}\ \vdash_H\ p$
\item $\{p\to q,q \to r\}\ \vdash_H\ p\to r$
\item $\{p, q \to (p\to r)\}\ \vdash_H\ q\to r$

\end{enumerate}
\end{problem}\medskip

\begin{problem}
Dokažte korektnost Hilbertova kalkulu:
\begin{itemize}
    \item Dokažte, že logické axiomy jsou tautologie.
    \item Dokažte, že modus ponens je korektní, tj. když $T\models\varphi$ a $T\models\varphi\to\psi$, tak $T\models\psi$.
    \item Ukažte, že $T\ \vdash_H\ \varphi$ implikuje $T\models\varphi$.
\end{itemize}
\end{problem}\medskip

\begin{problem}
Vyslovte a dokažte větu o dedukci pro Hilbertův kalkul.
\end{problem}\medskip

\bigskip

\hrule

\bigskip


\subsection*{Hilbert's calculus}
The \emph{Hilbert's propositional calculus} is a proof system for propositional logic where 
\begin{itemize}
    \item we only use the logical connectives $\neg,\to$
    \item we have the following (schemes of) \emph{logical axioms}:
    \begin{enumerate}[(i)]
        \item $\varphi \to (\psi \to \varphi)$
        \item $(\varphi\to (\psi \to \chi)) \to ((\varphi \to \psi)\to(\varphi \to \chi))$
        \item $(\neg \varphi \to \neg \psi)\to(\psi \to \varphi)$
    \end{enumerate}
    \item and the following \emph{rule of inference}:
    $$\frac{\varphi,\ \varphi \to \psi}{\psi}$$
    i.e. ``from $\varphi$ and $\varphi\to\psi$ infer $\psi$'' (called ``modus ponens'')
\end{itemize}
In Hilbert's calculus, a \emph{proof} of a proposition $\varphi$ from a theory $T$ is a finite sequence $\varphi_0,\dots,\varphi_n=\varphi$ of formulas such that for every $i\leq n$,
\begin{itemize}
\item $\varphi_i$ is a logical axiom, or 
\item $\varphi_i \in T$ (an axiom of the theory), or
\item $\varphi_i$ can be inferred from a pair of preceding propositions $\varphi_j$, $\varphi_k$ ($j<i,k<i$) by applying the rule of inference.
\end{itemize}
If such a proof exists, we write $T\ \vdash_H\ \varphi$.




\end{document}