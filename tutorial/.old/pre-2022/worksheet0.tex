\documentclass{amsart}
\usepackage[utf8]{inputenc}
\usepackage{enumerate}

\theoremstyle{definition}
\newtheorem{problem}{Problem}

\title{\sc Logic Tutorial: Worksheet 0}

\date{}


\begin{document}

\maketitle


\subsection*{Topics:} Predicate logic (language, theory, first order, higher order). Expressing various properties in propositional and predicate logic.

\smallskip
\begin{problem}
\item We are lost in a labyrinth. In front of us are three doors: a red door, a green door, and a blue door. We know that exactly one of the doors leads out of the labyrinth while the two other doors lead to a hungry dragon. We read the following inscriptions on the doors:
\begin{itemize}
    \item Red door: ``The way out is behind this door.''
    \item Green door: ``The way out is not behind the blue door.''
    \item Blue door: ``The way out is not behind this door.''
\end{itemize}
We know that at least one of the inscriptions is true and at least one is false. Formalize our knowledge.
\end{problem}

\smallskip
\begin{problem} We know the following:
\begin{enumerate}
    \item[$(i)$] Everyone knows themselves.
\item[$(ii)$] If a person studies at a school, they had to apply to the school and the school had to accept them.
\item[$(iii)$] Alfonso did not apply to any school that accepted someone who knows Alfonso.
\end{enumerate}
Formalize our knowledge. Can you show that ``Alfonso does not study at any school''?
\end{problem}




\smallskip
\begin{problem}
Find a suitable first-order language and theory for
\begin{enumerate}[a)]
    \item oriented graphs (without multiple edges)
    \item graphs (undirected, loopless),
%    \item connectivity in graphs (i.e., with a predicate for $u$ and $v$ being connected),
    \item equivalence relations,
    \item partial orders.
\end{enumerate}
\end{problem}


\smallskip
\begin{problem}
We are given an (undirected, loopless) graph $G$. Find formul{\ae} (in the language of graphs) which express the following properties. When is it possible in FO logic and when do we need second order?
\begin{enumerate}[(a)]
    \item $G$ contains a vertex of degree 1
    \item there is a path of length $k$ between $u$ and $v$ in $G$ (for some fixed $k$)
    \item $G$ is regular of degree 3,
    \item $G$ contains a $k$-clique (for some fixed $k$),
    \item $G$ is vertex 3-colorable.
    \item $G$ is bipartite,
    \item $G$ has a perfect matching,
    \item there is a path between $u$ and $v$ in $G$
\end{enumerate}
\end{problem}


\smallskip
\begin{problem} Find first-order formul{\ae} (in the language of $\le$) expressing the following properties in a partially ordered set:
\begin{enumerate}[(a)]
\item ``$x$ is the smallest element'', ``$x$ is a minimal element'',
\item ``$x$ has an immediate successor'',
\item ``every two elements have the greatest common predecessor''.
\end{enumerate}
\end{problem}

\smallskip
\begin{problem} Find first-order formul{\ae} (in the language of equality) expressing for a fixed $n>0$ that
\begin{enumerate}[(a)]
\item ``there exist at least $n$ elements'',
\item ``there exist at most $n$ elements'',
\item ``there exist exactly  $n$ elements''
\end{enumerate}
Is it possible to express using a (possibly infinite) set of formul{\ae} that  ``there are infinitely many elements''?
\end{problem}


\smallskip
\begin{problem}
Can $\mathbb N$, $\mathbb Z$, $\mathbb R$, and $\mathbb C$ be distinguished by first-order properties
\begin{enumerate}[(a)]
\item in the language of ordered sets?
\item in the language of arithmetic?
\end{enumerate}
\end{problem}


% \smallskip
% \begin{problem} Find a SO formula expressing ``there exist finitely many elements'':
% \begin{enumerate}[(a)]
%   \item Find first-order formul{\ae} (with a symbol $f$ for a function) expressing ``$f$ is injective'', ``$f$ is surjective''.
%   \item Find a second-order formula expressing ``every surjective function is injective''.
% \end{enumerate}
% \end{problem}


% \smallskip
% \begin{problem} Recall the definition of (rooted) trees.
% \begin{enumerate}[(a)]
% \item Show that for every rooted tree $T$ and for every two nodes $x$, $y$ with $x<_T y$ there exists an \emph{immediate successor} (a child) of $x$ between $x$ and $y$.
% \item Give an example of a tree in which some node (except the root) has no immediate predecessor (parent).
% \item Show that every finitely branching tree in which every node (except the root) has a parent is countable.
% \end{enumerate}
% \end{problem}

\smallskip
\begin{problem}
Consider a finite game of two alternating players, Alice and Bob. Assume that the game ends after $n$ rounds by
a win of one of the players and that Alice goes first. The game is given by the formula
$\varphi(x_1, y_1, x_2, y_2,\dots,x_n, y_n)$ expressing that the game with moves $x_1, y_1, x_2, y_2,\dots,x_n, y_n$ ends
by a win of Alice. Find formul{\ae} (in first-order logic) which express
\begin{enumerate}[(a)]
    \item ``Alice cannot lose'',
    \item ``Bob cannot lose'',
    \item ``Alice has a winning strategy'',
    \item ``Bob has a winning strategy''.
\end{enumerate}
\end{problem}


\end{document}
