\documentclass{amsart}
\usepackage[utf8]{inputenc}
\usepackage{enumerate}

\theoremstyle{definition}
\newtheorem{problem}{Problem}

\title{\sc Logic Tutorial: Worksheet 1}

\date{}


\begin{document}

\maketitle


\subsection*{Topics:} Introduction, math background, syntax of propositional logic (expression and formation tree, prefix, infix, postfix notation), semantics of propositional logic (Boolean operators, truth table, Venn diagram, tautologies), 

\smallskip
\begin{problem}
Construct the expression (and formation) tree, write out the expression in prefix, infix, and postfix format:
\begin{enumerate}[(a)]
    \item $(3+5)*(-2)+(2*3)$
    \item $p \to q \leftrightarrow \neg (p \wedge \neg q)$
    \item $(p \leftrightarrow q) \leftrightarrow (p \vee q) \to (p \wedge q)$
\end{enumerate}
\end{problem}


\smallskip
\begin{problem}
Construct truth tables and Venn diagrams for the following propositional formul\ae:
\begin{enumerate}[(a)]
\item $p \to q \leftrightarrow \neg (p \wedge \neg q)$
\item $((p\to q)\to p)\to p$
\item $\neg (p\vee q)\leftrightarrow \neg p\wedge \neg q$
%\item $(p\to q \to r)\to(p\to q)\to (p\to r)$
\end{enumerate}
\end{problem}


\smallskip
\begin{problem}
Show that the following propositional formul{\ae} are tautologies:
\begin{enumerate}[(a)]
    \item $p \to q \leftrightarrow \neg p \vee q$
    \item $p \to q \leftrightarrow \neg (p \wedge \neg q)$,
    \item $p \to q \leftrightarrow (p \leftrightarrow (p\wedge q))$,
    \item $p \to q \leftrightarrow (q \leftrightarrow (p\vee q))$,
    \item $p \wedge q \leftrightarrow ((p \leftrightarrow q) \leftrightarrow (p \vee q))$,
    \item $(p \leftrightarrow q) \leftrightarrow (p \vee q) \to (p \wedge q)$.
\end{enumerate}
\end{problem}


\smallskip
\begin{problem}
Show that $\wedge$ and $\vee$ are not enough to define all Boolean operators.
\end{problem}

\begin{problem} Are the following sets of logical connectives adequate? 
\begin{enumerate}[(a)]
    \item $\{\downarrow\}$ where $\downarrow$ is the Peirce arrow (NOR),
    \item $\{\uparrow\}$ where $\uparrow$ is the Sheffer stroke (NAND),
    \item $\{\vee, \rightarrow, \leftrightarrow\}$,
    \item $\{\vee, \wedge, \rightarrow\}$.
\end{enumerate}
\end{problem}

\smallskip
\begin{problem}
Consider the ternary Boolean operator $\mathrm{IFTE}(p, q, r)$ meaning ``if $p$ then $q$ else $r$''. 
\begin{enumerate}[(a)]
    \item Construct its truth table.
    \item Show that all basic Boolean operators ($\neg, \to, \wedge,\vee,\dots$) can be expressed using IFTE and the constants TRUE and FALSE.
\end{enumerate}  
\end{problem}



\begin{problem}
Give an example of a proposition in the language $\mathbb P=\{p,q,r\}$ which
\begin{enumerate}[(a)]
\item is true,
\item is contradictory,
\item is independent,
\item is equivalent to, but different from, the proposition $(p\wedge q)\to\neg r$,
\item has exactly the following models: $\{(1,0,0),(1,0,1),(0,0,1)\}$.
\end{enumerate}
\end{problem}


\begin{problem}
Find the set of all models of the given proposition in the language $\mathbb P=\{p,q,r\}$.
\begin{enumerate}[(a)]
    \item $p\leftrightarrow \neg r$
    \item $((p\wedge q)\to\neg r)\to r$
    \item $(p\wedge q\wedge \neg r)\vee (\neg p\wedge \neg q\wedge \neg r)$
    \item $(p\vee q\vee \neg r)\wedge (\neg p\vee \neg q\vee \neg r)$
    
\end{enumerate}


\end{problem}

\medskip\hrule\medskip





\end{document}
