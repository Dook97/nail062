\documentclass{amsart}
\usepackage[utf8]{inputenc}
\usepackage{enumerate}
\usepackage{graphicx}

\usepackage{a4wide}
\usepackage[shortlabels]{enumitem}

\theoremstyle{definition}
\newtheorem{problem}{Příklad}
\theoremstyle{remark}
\newtheorem*{theorem*}{Věta}


\begin{document}


\section*{Cvičení z logiky: 6. sada příkladů}

\medskip

\subsection*{Témata:}  Definovatelné množiny, tablo metoda, rovnost, věta o konstantách, věta o dedukci, prenexní forma, Skolemova varianta.

\medskip
\hrule

\begin{problem}
Mějme jazyk $L=\langle F \rangle$ s rovností, kde $F$ je binární funkční symbol. Najděte formule definující následující množiny (bez parametrů):
\begin{enumerate}[(a)]
\itemsep6pt
    \item interval $(0,\infty)$ v $\mathcal A=\langle\mathbb R, \cdot\rangle$ kde $\cdot$ je násobení reálných čísel,
    \item množina $\{(x, 1/x)\mid x\neq 0\}$ ve stejné struktuře $\mathcal A$,
    \item množina všech nejvýše jednoprvkových podmnožin $\mathbb N$ v $\mathcal B=\langle\mathcal P(\mathbb N),\cup\rangle$,
    \item množina všech prvočísel v $\mathcal C=\langle \mathbb N\cup\{0\}, \cdot\rangle$.
\end{enumerate}
\end{problem}

\medskip
\hrule

\begin{problem}
Předpokládejme, že
\begin{enumerate}[(a)]
\itemsep6pt
\item všichni viníci jsou lháři,
\item aspoň jeden z obviněných je také svědkem,
\item žádný svědek nelže.
\end{enumerate}
Dokažte tablo metodou, že ne všichni obvinění jsou viníci.
\end{problem} 



\begin{problem} Nechť $L(x,y)$ reprezentuje \emph{``existuje let z $x$ do $y$''} a $S(x,y)$ reprezentuje \emph{``existuje spojení z $x$ do $y$''}. Předpokládejme, že
\begin{enumerate}[(a)]
\itemsep6pt
\item Z Prahy lze letět do Bratislavy, Londýna a New Yorku, a z New Yorku do Paříže,
\item $(\forall x)(\forall y)(L(x,y) \to L(y,x))$,
\item $(\forall x)(\forall y)(L(x,y)\to S(x,y))$,
\item $(\forall x)(\forall y)(\forall z)(S(x,y)\wedge L(y,z)\to S(x,z))$.
\end{enumerate}
Dokažte tablo metodou, že existuje spojení z Bratislavy do Paříže.
\end{problem} 

\medskip

\begin{problem} V následujících příkladech jsou $\varphi$ a $\psi$ sentence nebo formule s volnou proměnnou $x$ (což značíme $\varphi(x)$, $\psi(x)$). Najděte tablo důkazy dané formule:
\begin{enumerate}[(a)]
\itemsep6pt
\item $(\exists x)(\varphi(x)\vee \psi(x))\leftrightarrow (\exists x)\varphi(x)\vee (\exists x)\psi(x)$,
\item $(\forall x)(\varphi(x)\wedge\psi(x))\leftrightarrow (\forall x)\varphi(x)\wedge(\forall x)\psi(x)$,
\item $(\varphi \vee (\forall x)\psi(x))\to (\forall x)(\varphi \vee \psi(x))$ kde $x$ není volná v $\varphi$,
\item $(\varphi \wedge (\exists x)\psi(x))\to (\exists x)(\varphi \wedge \psi(x))$ kde $x$ není volná v $\varphi$.
\item $(\exists x)(\varphi \to \psi(x))\to(\varphi \to (\exists x)\psi(x))$ kde $x$ není volná v $\varphi$,
\item $(\exists x)(\varphi \wedge \psi(x))\to(\varphi \wedge (\exists x)\psi(x))$ kde $x$ není volná v $\varphi$,
%\item $\neg(\exists x)\varphi(x)\to (\forall x)\neg \varphi(x)$,
%\item $(\forall x)\neg \varphi(x)\to \neg(\exists x)\varphi(x)$,
\item $(\exists x)(\varphi(x)\to\psi)\to((\forall x)\varphi(x)\to \psi)$ kde $x$ není volná v $\psi$,
\item $((\exists x)\varphi(x)\to\psi)\to(\forall x)(\varphi(x)\to \psi)$ kde $x$ není volná v $\psi$.
\end{enumerate}
\end{problem} 

\medskip
\hrule



\medskip
\hrule

\begin{problem} Buď $L$ jazyk s rovností obsahující binární relační symbol $\le$ a $T$ teorie v tomto jazyce taková, že $T$ má nekončený model a platí v ní axiomy lineárního uspořádání $T$. Pomocí věty o kompaktnosti ukažte, že $T$ má model $\mathcal{A}$ s \emph{nekonečným klesajícím řetězcem}; tj. že existují prvky $c_i$ pro každé $i\in \mathbb{N}$ v $A$ takové, že 
    $$\dots < c_{n+1} < c_n< \dots <c_0.$$
    (Z toho plyne, že pojem \emph{dobrého uspořádání} není definovatelný v logice prvního řádu.)
\end{problem}

\medskip
\hrule

\begin{problem}
Buď $T'$ extenze teorie $T=\{(\exists y)(x+y=0),(x+y=0)\wedge (x+z=0)\rightarrow y=z\}$ v jazyce $L=\langle +,0,\le\rangle$ s rovností o definice $<$ a unárního $-$ s axiomy
\begin{align*}
-x=y\ \ &\leftrightarrow\ \ x+y=0\\
x<y\ \ &\leftrightarrow\ \ x\le y\ \wedge\ \neg(x=y)
\end{align*}
Najděte formule v jazyce $L$, které jsou ekvivalentní v $T'$ s následujícími formulemi.
\begin{enumerate}[(a)]
\itemsep6pt
\item $x+(-x)=0$
\item $x+(-y)<x$
\item $-(x+y)<-x$
\end{enumerate}
\end{problem}

\medskip


\begin{problem} Převeďte následující formule do prenexní normální formy.
\begin{enumerate}
\itemsep6pt
\item $(\forall y)((\exists x)P(x,y)\to Q(y,z))\wedge (\exists y)((\forall x)R(x,y)\vee Q(x,y))$
\item $(\exists x)R(x,y)\leftrightarrow (\forall y)P(x,y)$
\item $\neg((\forall x)(\exists y)P(x,y)\to (\exists x)(\exists y)R(x,y))\wedge(\forall x)\neg(\exists y)Q(x,y)$
\end{enumerate}
\end{problem}

\medskip

\begin{problem} Najděte Skolemovy varianty formulí v PNF z předchozího příkladu.
\end{problem}

\medskip

\begin{problem} Ověřte, že platí následující:
\begin{enumerate}
\itemsep6pt
\item $\models (\forall x)P(x,f(x)) \to (\forall x)(\exists y)P(x,y)$
\item $\not\models (\forall x)(\exists y)P(x,y)\to (\forall x)P(x,f(x))$
\end{enumerate}
(Z toho plyne, že Skolemova varianta nemusí být ekvivalentní původní formuli.)
\end{problem}




\end{document}