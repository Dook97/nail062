\documentclass[a4paper]{amsart}
\usepackage[utf8]{inputenc}
\usepackage{enumerate}
\usepackage{url}
\usepackage{hyperref}

\usepackage{a4wide}
\theoremstyle{definition}
\newtheorem{task}{Úkol}


\title{\sc Cvičení z logiky: Domácí úkol č. 1}
%\author{Termín odevzdání: úterý 10. listopadu v 10:40}

\begin{document}

\maketitle

\thispagestyle{empty}

Úkol odevzdávejte v Moodle. Ponechte si dostatečný čas pro odevzdání, tak aby vám krátkodobé technické potíže s Moodle nezabránily úkol odevzdat. Pozdě odevzdané úkoly nebudou hodnoceny, kromě případů hodných zvláštního zřetele. Odevzdané řešení musí být vaše vlastní, není dovoleno hledat nápovědy ani řešení konzultovat s kýmkoliv kromě mne. Své odpovědi dostatečně podrobně zdůvodněte, uveďte všechny pomocné výpočty apod.


\begin{task}
Převeďte následující výrok do CNF a do DNF. (Uveďte celý postup, ne jen odpověď.)
$$
((p\to \neg q) \to \neg r) \vee \neg p.
$$
\end{task}

\begin{task}
Rozhodněte, zda je následující 2-CNF výrok splnitelný. Pokud ano, najděte nějaké splňující ohodnocení. Nakreslete příslušný implikační graf, a graf silně souvislých komponent v topologickém uspořádání.
\begin{align*}
&(a \vee  c) \wedge  (a \vee  \neg d) \wedge  (b \vee  \neg d) 
\wedge  (b \vee  \neg e) \wedge  (\neg c \vee  \neg e) 
\wedge  (\neg a \vee  \neg f)
\wedge \\ &\wedge (b\vee\neg c)\wedge
(\neg b \vee  f) \wedge  (c \vee  \neg f) \wedge \neg f
\end{align*}
\end{task}

\begin{task}
Rozhodněte, zda je následující výrok v Hornově tvaru splnitelný. Pokud ano, najděte nějaké splňující ohodnocení. (Uveďte celý postup, ne jen odpověď.)
\begin{align*}
&(\neg a \vee \neg b \vee c \vee \neg d)\wedge(\neg b \vee c)\wedge d \wedge (\neg a \vee \neg c \vee e)\wedge \\
&\wedge(\neg c \vee \neg d)\wedge(\neg a \vee \neg d \vee \neg e)\wedge(a\vee \neg b \vee\neg e)
\end{align*}
\end{task}

\begin{task}
Uvažme následující výroky $\varphi$ a $\psi$ nad $\mathbb P=\{p, q, r, s\}$:
\begin{align*}
    \varphi &= (\neg p \vee  q)\to(p\wedge r)\\
    \psi &= s\to q
\end{align*}
\begin{enumerate}[(a)]
    \item Určete počet (až na ekvivalenci) výroků $\chi$ nad $\mathbb P$ takových, že $\varphi\wedge\psi\models\chi$.
    \item Určete počet (až na ekvivalenci) úplných teorií $T$ nad $\mathbb P$ takových, že $T\models\varphi\wedge\psi$.
    \item Najděte nějakou axiomatizaci pro každou (až na ekvivalenci) úplnou teorii $T$ nad $\mathbb P$ takovou, že $T\models\varphi\wedge\psi$.
\end{enumerate}


\end{task}


\begin{task}
Uvažme následující tvrzení:
\begin{enumerate}
\item[$(i)$] {\it Ten, kdo je dobrý bežec a má dobrou kondici, uběhne maraton.}
\item[$(ii)$] {\it Ten, kdo nemá štěstí a nemá dobrou kondici, neuběhne maraton.}
\item[$(iii)$]{\it Ten, kdo uběhne maraton, je dobrý běžec.}
\item[$(iv)$] {\it Budu-li mít štěstí, uběhnu maraton.}
\item[$(v)$] {\it Mám dobrou kondici.}
\end{enumerate}


\begin{enumerate}[(a)]
\item Přeložte tvrzení (i) až (v) po řadě do výroků $\varphi_1$ až $\varphi_5$ v jazyce $L=\langle b, k, m, s\rangle$, kde výrokové proměnné mají po řadě význam ``být dobrý běžec'', ``mít dobrou kondici'', ``uběhnout maraton'' a ``mít štěstí''.
\item Sestrojte dokončené tablo z teorie $T=\{\varphi_1,\dots,\varphi_5\}$ s položkou $F (k \wedge \neg k)$ v kořeni. Sestrojte kanonický model pro nejlevější bezespornou větev tohoto tabla.

\item Najděte příklad výroků v jazyce $L$, které jsou $T$-ekvivalentní, ale ne logicky ekvivalentní. 
\item Určete počet navzájem neekvivalentních jednoduchých extenzí teorie $T$.
\end{enumerate}
\end{task}



\end{document}
