\documentclass[11pt,a4paper]{amsart}
\usepackage[utf8]{inputenc}
\usepackage[shortlabels]{enumitem}
\usepackage{graphicx}
\usepackage{a4wide}

\theoremstyle{definition}
\newtheorem{problem}{Příklad}
\theoremstyle{remark}
\newtheorem*{theorem*}{Věta}


\begin{document}


\section*{Cvičení z logiky: 7. sada příkladů}

\bigskip\bigskip

\subsection*{Témata:}  Skolemizace, extenze o definice, Herbrandův model, ekvisplnitelnost, množinová reprezentace, rezoluce, unifikace.

\bigskip\bigskip

\begin{problem} Teorie těles $T$ jazyka $L=\langle +,-,\cdot,0,1\rangle$ obsahuje jeden axiom $\varphi$, který není otevřený:
$$x\ne 0\ \to\ (\exists y)(x\cdot y=1).$$
\medskip

Víme, že $T\models 0\cdot y=0$ a $T\models\ (x\ne 0\ \wedge\ x\cdot y=1\ \wedge\ x\cdot z=1)\ \to\ y=z$.
\medskip
\begin{enumerate}[(a)]
\itemsep12pt
\item Najděte Skolemovu variantu $\varphi_S$ formule $\varphi$ s novým funkčním symbolem $f$.
\item Uvažme teorii $T'$ vzniklou z $T$ nahrazením $\varphi$ za $\varphi_S$. Platí $\varphi$ v $T'$?
\item Lze každý model $T$ \emph{jednoznačně} rozšířit na model $T'$?
\end{enumerate}
\end{problem}

\bigskip

\begin{problem} Nechť $T$ je teorie těles z předchozího příkladu a mějme následující formuli $\psi$: $$x\cdot y=1\vee  (x=0 \wedge y=0)$$
\begin{enumerate}[(a)]
\itemsep12pt
\item Platí v $T$ axiomy existence a jednoznačnosti pro $\psi(x,y)$ a proměnnou $y$?
\item Sestrojte extenzi $T^*$ teorie $T$ o definovaný symbol $f$ formulí $\psi$.
\item Je $T^*$ ekvivalentní teorii $T'$ z předchozího příkladu?
\item Najděte formuli v původním jazyce $L$, která je v $T^*$ ekvivalentní s formulí
$$f(x\cdot y)=f(x)\cdot f(y)$$
\end{enumerate}
\end{problem}

\bigskip

\begin{problem} Sestrojte Herbrandovo univerzum a příklad Herbrandovy struktury pro následující jazyky:
\medskip
\begin{enumerate}[(a)]
\itemsep12pt
\item $L=\langle P,Q,f,a,b \rangle$ kde  $P,Q$ jsou relační symboly, $P$ unární a $Q$ binární, $f$ je unární funkční symbol, a $a,b$ jsou konstantní symboly.
\item $L=\langle P,f,g,a \rangle$ kde $P$ je binární relační symbol, $f,g$ jsou unární funkční symboly, a symbol $a$ je konstantní.
\end{enumerate}
\end{problem}

\bigskip



\begin{problem} Sestrojte Herbrandův model, nebo najděte nesplnitelnou konjunkci základních instancí jejich axiomů ($a,b$ jsou konstantní symboly v daném jazyce).
\medskip
\begin{enumerate}[(a)]
\itemsep12pt
\item $T=\{\neg P(x)\vee Q(f(x),y), \neg Q(x,b), P(a)\}$
\item $T=\{\neg P(x)\vee Q(f(x),y), Q(x,b), P(a)\}$
\item $T=\{P(x,f(x)),\neg P(x,g(x))\}$
\item $T=\{P(x,f(x)),\neg P(x,g(x)), P(g(x),f(y)) \to P(x,y)\}$
\end{enumerate}
\end{problem}

\bigskip

\begin{problem} Převeďte následující formule na ekvisplnitelné formule v množinové reprezentaci.
\medskip
\begin{enumerate}[(a)]
\itemsep12pt
\item $(\forall y)(\exists x)P(x,y)$
\item $\neg (\forall y)(\exists x)P(x,y)$
\item $\neg (\exists x)((P(x)\to P(a))\wedge (P(x)\to P(b)))$
\item $(\exists x)(\forall y)(\exists z)(P(x,z)\wedge P(z,y) \to R(x,y))$
\end{enumerate}
\end{problem}

\bigskip

\begin{problem} Víme, že platí následující:
\medskip
\begin{enumerate}[(a)]
\itemsep12pt
\item Je-li cihla na (jiné) cihle, potom není na zemi.
\item Každá cihla je na (jiné) cihle nebo na zemi.
\item Žádná cihla není na cihle, která by byla na (jiné) cihle.
\end{enumerate}
\medskip
Vyjádřete tato fakta ve vhodném jazyce logiky prvního řádu a dokažte rezolucí následující tvrzení: ``Je-li cihla na (jiné) cihle, spodní cihla je na zemi.''
\end{problem}

\bigskip

\begin{problem} Víme, že platí následující:
\medskip
\begin{enumerate}[(a)]
\itemsep12pt
\item Každý holič holí všechny, kdo neholí sami sebe
\item Žádný holič neholí nikoho, kdo holí sám sebe.
\end{enumerate}
\medskip
Vyjádřete tato fakta ve vhodném jazyce logiky prvního řádu a dokažte rezolucí, že neexistují žádní holiči.
\end{problem}



\medskip\begin{problem}
Ukažte, že daná množina klauzulí je zamítnutelná (rezolucí). Popište zamítnutí pomocí rezolučního stromu. V každém kroku rezoluce napište použitou unifikaci a podtrhněte rezolvované literály.
\begin{enumerate}[(a)]
\item $\{P(a,x,f(y)),P(a,z,f(h(b))),\neg Q(y,z)\}$
\item $\{\neg Q(h(b),w),H(w,a)\}$
\item $\{\neg P(a,w,f(h(b))),H(x,a)\}$
\item $\{P(a,u,f(h(u))),H(u,a),Q(h(b),b)\}$
\item $\{\neg H(v,a)\}$
\end{enumerate}
\end{problem}


\end{document}