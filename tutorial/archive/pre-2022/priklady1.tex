\documentclass{amsart}
\usepackage[utf8]{inputenc}
\usepackage{enumerate}

\theoremstyle{definition}
\newtheorem{problem}{Příklad}



\begin{document}

\section*{Cvičení z logiky: 1. sada příkladů}

\bigskip\bigskip

\subsection*{Témata:} 
Úvod, matematické základy, syntax výrokové logiky (strom výrazu, vytvořující strom, prefixový, infixový a postfixový zápis), sémantika výrokové logiky (Booleovské operátory, pravdivostní tabulka, Vennův diagram, tautologie, modely).

\bigskip

\medskip\hrule\medskip

\begin{problem}
Sestrojte strom výrazu resp. formule, zapište je v prefixovém, infixovém a postfixovém formátu:
\begin{enumerate}[(a)]
    \item $(3+5)*(-2)+(2*3)$
    \item $p \to q \leftrightarrow \neg (p \wedge \neg q)$
    \item $(p \leftrightarrow q) \leftrightarrow ((p \vee q) \to (p \wedge q))$
\end{enumerate}
\end{problem}


\smallskip
\begin{problem}
Sestrojte pravdivostní tabulky a Vennův diagram pro následující výrokové formule:
\begin{enumerate}[(a)]
\item $p \to q \leftrightarrow \neg (p \wedge \neg q)$
\item $((p\to q)\to p)\to p$
\item $\neg (p\vee q)\leftrightarrow \neg p\wedge \neg q$
%\item $(p\to q \to r)\to(p\to q)\to (p\to r)$
\end{enumerate}
\end{problem}


\smallskip
\begin{problem}
Ukažte, že následující výrokové formule jsou tautologie:
\begin{enumerate}[(a)]
    \item $p \to q \leftrightarrow \neg p \vee q$
    \item $p \to q \leftrightarrow \neg (p \wedge \neg q)$,
    \item $p \to q \leftrightarrow (p \leftrightarrow (p\wedge q))$,
    \item $p \to q \leftrightarrow (q \leftrightarrow (p\vee q))$,
    \item $p \wedge q \leftrightarrow ((p \leftrightarrow q) \leftrightarrow (p \vee q))$,
    \item $(p \leftrightarrow q) \leftrightarrow (p \vee q) \to (p \wedge q)$.
\end{enumerate}
\end{problem}


\smallskip
\begin{problem}
Ukažte, že $\wedge$ a $\vee$ nestačí k definování všech Booleovských operátorů.
\end{problem}

\smallskip
\begin{problem} Jsou následující množiny logických spojek univerzální? Zdůvodněte.
\begin{enumerate}[(a)]
    \item $\{\downarrow\}$ kde $\downarrow$ je Peirce arrow (NOR),
    \item $\{\uparrow\}$ kde $\uparrow$ je Sheffer stroke (NAND),
    \item $\{\vee, \rightarrow, \leftrightarrow\}$,
    \item $\{\vee, \wedge, \rightarrow\}$.
\end{enumerate}
\end{problem}

\smallskip
\begin{problem}
Uvažte ternární Booleovský operátor $\mathrm{IFTE}(p, q, r)$ definovaný jako ``if $p$ then $q$ else $r$''. 
\begin{enumerate}[(a)]
    \item Zkonstruujte pravdivostní tabulku.
    \item Ukažte, že všechny základní Booleovské operátory ($\neg, \to, \wedge,\vee,\dots$) lze vyjádřit pomocí IFTE a konstant TRUE a FALSE.
\end{enumerate}  
\end{problem}


\begin{problem}
Uveďte příklad výroku v jazyce $\mathbb P=\{p,q,r\}$, který
\begin{enumerate}[(a)]
\item je pravdivý,
\item je sporný,
\item je nezávislý,
\item je ekvivalentní s, ale různý od, výroku $(p\wedge q)\to\neg r$,
\item má za modely právě $\{(1,0,0),(1,0,1),(0,0,1)\}$.
\end{enumerate}
\end{problem}


\begin{problem}
Najděte množinu modelů daného výroku v jazyce $\mathbb P=\{p,q,r\}$.
\begin{enumerate}[(a)]
    \item $p\leftrightarrow \neg r$
    \item $((p\wedge q)\to\neg r)\to r$
    \item $(p\wedge q\wedge \neg r)\vee (\neg p\wedge \neg q\wedge \neg r)$
    \item $(p\vee q\vee \neg r)\wedge (\neg p\vee \neg q\vee \neg r)$
    
\end{enumerate}


\end{problem}



\end{document}
