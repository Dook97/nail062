% veta o konstantach - stary dukaz

Předpokládejme nejprve, že $\varphi$ platí v každém modelu teorie $T$. Chceme ukázat, že $\varphi(x_1/c_1,\dots,x_n/c_n)$ platí v každém modelu $\A'$ teorie $T'$. 
    
    Označme jako $\A$ redukt $\A'$ na jazyk $L$, potom $\A$ je model teorie $T$. Protože platí $\A\models\varphi[e]$ pro \emph{libovolné} ohodnocení $e$, platí i pro ohodnocení  $\A\models\varphi[e(x_1/c_1^{\A'},\dots,x_n/c_n^{\A'})]$ ve kterém ohodnotíme proměnnou $x_i$ interpretací konstantního symbolu $c_i$ ve struktuře $\A'$. To ale znamená, že $\A'\models\varphi(x_1/c_1,\dots,x_n/c_n)$, což jsme chtěli dokázat.
    
    Naopak, předpokládejme, že $\varphi(x_1/c_1,\dots,x_n/c_n)$ platí v každém modelu teorie $T'$ a ukažme, že $\varphi$ platí v každém modelu $\A$ teorie $T$. Zvolme tedy takový model $\A$ a nějaké ohodnocení $e\colon\Var\to A$ a ukažme, že $\A\models\varphi[e]$.

    Označme jako $\A'$ expanzi $\A$ do jazyka $L'$, kde konstantní symbol $c_i$ interpretujeme jako prvek $c_i^{\A'}=e(x_i)$, pro všechna $i$. Protože platí $\A'\models\varphi(x_1/c_1,\dots,x_n/c_n)[e']$ pro všechna ohodnocení $e'$, platí i $\A'\models\varphi(x_1/c_1,\dots,x_n/c_n)[e]$, což ale znamená, že $\A'\models\varphi[e]$ (neboť $e=e(x_1/c_1^{\A'},\dots,x_n/c_n^{\A'})je totéž jako $\A'\models\varphi[e(x_1/c_1^{\A'},\dots,x_n/c_n^{\A'})]$
    $\A'\models\varphi[e(x_1/c_1^{\A'},\dots,x_n/c_n^{\A'})]$, $\A'\models\varphi[e(x_1/c_1^{\A'},\dots,x_n/c_n^{\A'})]$ což znamená $\A\models\varphi[e]$. 