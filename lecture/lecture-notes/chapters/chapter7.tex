\chapter{Tablo metoda v predikátové logice}

V této kapitole ukážeme, jak lze zobecnit \emph{metodu analytického tabla} z výrokové na predikátovou logiku.\footnote{Na tomto místě je dobré připomenout si tablo metodu ve výrokové logice, viz Kapitola \ref{chapter:tableau-method-propositional}.} Metoda funguje velmi podobně, musíme si ale poradit \emph{kvantifikátory}.

\section{Neformální úvod}

V této sekci tablo metodu neformálně představíme. K formálním definicím se vrátíme později. Začneme dvěma příklady, na kterých ilustruje, jak tablo metoda v predikátové logice funguje, a jak se vypořádá s kvantifikátory.

\begin{example} Na Obrázku \ref{figure:predicate-tableau-intro-examples} jsou znázorněna dvě tabla. Jsou to tablo důkazy (v logice, tj. z prázdné teorie) \emph{sentencí} $(\exists x)\neg P(x)\limplies\neg(\forall x)P(x)$ (vpravo) a $\neg(\forall x)P(x)\limplies(\exists x)\neg P(x)$ (vlevo) jazyka $L=\langle P\rangle$ (bez rovnosti), kde $P$ je unární relační symbol. Symbol $c_0$ je \emph{pomocný konstantní symbol}, který do jazyka při konstrukci tabla přidáváme.

\begin{figure}[htbp]
\begin{minipage}{.49\textwidth}
\centering
\begin{forest}
    for tree={math content}
    [\F(\exists x)\neg P(x)\limplies\neg(\forall x)P(x)
        [\textcolor{red}{\T(\exists x)\neg P(x)}
            [\F\neg(\forall x)P(x)
                [\textcolor{blue}{\T(\forall x)P(x)}
                    [\T\neg P(c_0)
                        [\F P(c_0)
                            [\textcolor{blue}{\T(\forall x)P(x)}
                                [\T P(c_0), tikz={\node[fit to=tree,label=below:$\otimes$] {};}]
                            ]
                        ]
                    ]                
                ]
            ]
        ]
    ]
\end{forest}
\end{minipage}
\begin{minipage}{.49\textwidth}
\centering
\begin{forest}
    for tree={math content}
    [\F\neg(\forall x)P(x)\limplies(\exists x)\neg P(x)
        [\T\neg(\forall x) P(x)
            [\textcolor{blue}{\F(\exists x)\neg P(x)}
                [\textcolor{red}{\F(\forall x)P(x)}
                    [\F P(c_0)
                        [\textcolor{blue}{\F (\exists x)\neg P(x)}
                            [\F\neg P(c_0)
                                [\T P(c_0), tikz={\node[fit to=tree,label=below:$\otimes$] {};}]
                            ]
                        ]
                    ]                
                ]
            ]
        ]
    ]
\end{forest}
\end{minipage}
\label{figure:predicate-tableau-intro-examples}
\caption{Příklady tabel. Položky typu `svědek' jsou znázorněny červeně, položky typu `všichni' modře.}
\end{figure}
\end{example}


\subsubsection{Položky}
Formule v položkách musí být vždy \emph{sentence}, neboť potřebujeme, aby měly v daném modelu \emph{pravdivostní hodnotu} (nezávisle na ohodnocení proměnných). To ale není zásadní omezení, chceme-li dokázat, že formule $\varphi$ platí v teorii $T$, můžeme nejprve nahradit formuli $\varphi$ a všechny axiomy $T$ jejich \emph{generálními uzávěry} (tj. univerzálně kvantifikujeme všechny volné proměnné). Získáme tak \emph{uzavřenou} teorii $T'$ a sentenci $\varphi'$ a platí: $T'\models\varphi'$ právě když $T\models\varphi$.

\subsubsection{Kvantifikátory}
Redukce položek funguje stejně, použijeme tatáž atomická tabla pro logické spojky (viz Tabulka \ref{table:atomic-tableaux}, kde místo výroků jsou $\varphi,\psi$ sentence). Musíme ale přidat 4 nová atomická tabla pro $\mathrm T/\mathrm F$ a univerzální/existenční kvantifikátor. Tyto položky dělíme na dva typy:
\begin{itemize}
    \item typ ``\emph{svědek}'': položky tvaru $\mathrm{T}(\exists x)\varphi(x)$ a $\mathrm{F}(\forall x)\varphi(x)$
    \item typ ``\emph{všichni}'': položky tvaru $\mathrm{T}(\forall x)\varphi(x)$ a $\mathrm{F}(\exists x)\varphi(x)$    
\end{itemize}
Příklady vidíme v tablech na Obrázku \ref{figure:predicate-tableau-intro-examples} (`svědci' jsou červeně, `všichni' modře).

Kvantifikátor nemůžeme pouze odstranit, neboť výsledná formule $\varphi(x)$ by nebyla sentencí. Místo toho současně s odstraněním kvantifikátoru \emph{substituujeme} za $x$ nějaký \emph{konstantní term}, v nové položce tedy bude \emph{sentence} $\varphi(x/t)$. Jaký konstantní term $t$ substituujeme záleží na tom, zda jde o položku typu ``svědek'' nebo ``všichni''. 

\subsubsection{Pomocné konstantní symboly}
Jazyk $L$ teorie $T$, ve které dokazujeme, rozšíříme o spočetně mnoho \emph{nových (pomocných) konstantních symbolů} $C=\{c_0,c_1,c_2,\dots\}$ (ale budeme psát i $c,d,\dots$), výsledný rozšířený jazyk označíme $L_C$. Konstantní termy v jazyce $L_C$ tedy existují, i pokud původní jazyk $L$ nemá žádné konstanty. A vždy při konstrukci tabla máme k dispozici nějaký \emph{nový}, dosud \emph{nepoužitý} (ani v teorii, ani v konstruovaném tablu) pomocný konstantní symbol $c\in C$.

\subsubsection{Svědci}
Při redukci položky typu ``svědek'' substituujeme za proměnnou jeden z těchto nových, pomocných symbolů, a to takový, který \emph{dosud nebyl na dané větvi použit}. V případě položky $\T(\exists x)\varphi(x)$ tedy máme $\T\varphi(x/c)$. Tento konstantní symbol $c$ bude hrát roli (nějakého) prvku, který danou formuli splňuje (resp. vyvrací, jde-li o položku tvaru $\F(\forall x)\varphi(x)$). Srovnej s Větou o konstantách (Věta \ref{theorem:on-constants}). Je důležité, že symbol $c$ dosud nebyl na větvi ani v teorii nijak použit. Typicky ale poté použijeme položky typu ``všichni'', abychom se dozvěděli, co musí \emph{o tomto svědku platit}.

Na Obrázku \ref{figure:predicate-tableau-intro-examples} vidíme příklad: položka $\T(\exists x)\neg P(x)$ v levém tablu je redukovaná, její redukcí vznikla položka $\T\neg P(c_0)$; $c_0\in C$ je pomocný symbol, na větvi se dosud nevyskytoval (a je první takový). Podobně pro položku $\F(\forall x)P(x)$ a $\F P(c_0)$ v pravém tablu.

\subsubsection{Všichni}
Při redukci položky typu ``všichni'' substituujeme za proměnnou $x$ libovolný \emph{konstantní term} $t$ rozšířeného jazyka $L_C$. Z položky tvaru $\T(\forall x)\varphi(x)$ tedy získáme položku $\T\varphi(x/t)$. 

Aby byla bezesporná větev \emph{dokončená}, budou na ní ale muset být položky $\T\varphi(x/t)$ pro \emph{všechny} konstantní $L_C$-termy $t$. (Musíme `použít' vše, co položka $\T(\forall x)\varphi(x)$ `říká'.) A stejně pro položku tvary $\mathrm{F}(\exists x)\varphi(x)$.

Ve výrokové logice jsme používali konvenci, že při připojování atomických tabel vynecháváme jejich kořeny (jinak bychom opakovali na větvi tutéž položku dvakrát). V predikátové logice použijeme stejnou konvenci, ale \emph{s výjimkou položek typu `všichni'}. U těch zapíšeme i kořen připojovaného atomického tabla. Proč to děláme? Abychom si připomněli, že s touto položkou ještě nejsme hotovi, že musíme připojit atomická tabla s jinými konstantními termy.

Na Obrázku \ref{figure:predicate-tableau-intro-examples} v levém tablu \emph{není} položka $\T(\forall x)P(x)$ \emph{redukovaná}. Její \emph{první výskyt} (4. vrchol shora) jsme zredukovali, substituujeme term $t=c_0$, máme tedy $\varphi(x/t)=P(c_0)$. Připojili jsme atomické tablo v sestávající z téže položky v kořeni $\T(\forall x)P(x)$, kterou do tabla \emph{zapíšeme}, a z položky $\T P(c_0)$ pod ní. Zatímco \emph{první výskyt} položky $\T(\forall x)P(x)$ je tímto redukovaný, \emph{druhý výskyt} (7. vrchol shora) redukovaný není. Podobně pro položku $\F(\exists x)\neg P(x)$ v pravém tablu.

Tento poněkud technický přístup k definici \emph{redukovanosti} (výskytů) položek typu `všichni' se nám bude hodit v definici \emph{systematického tabla}.

\subsubsection{Jazyk}

Nadále budeme předpokládat, že jazyk $L$ je \emph{spočetný}.\footnote{Z hlediska výpočetní logiky to není velké omezení.} Z toho plyne, že každá $L$-teorie $T$ má jen spočetně mnoho axiomů, a také že konstantních termů v jazyce $L_C$ je jen spočetně mnoho. Toto omezení potřebujeme, neboť každé, i nekonečné tablo má jen spočetně mnoho položek, a musíme být schopni použít všechny axiomy dané teorie, a substituovat všechny konstantní termy jazyka $L_C$.

Nejprve také budeme předpokládat, že jde o jazyk \emph{bez rovnosti}, což je jednodušší. Problémem je, že \emph{tablo} je čistě syntaktický objekt, ale \emph{rovnost} má speciální sémantický význam, totiž musí být v každém modelu interpretována relací identity. Jak adaptovat metodu pro jazyky s rovností si ukážeme později. 

\section{Formální definice}

V této sekci definujeme všechny pojmy potřebné pro tablo metodu pro jazyky bez rovnosti. K jazykům s rovností se vrátíme v Sekci \ref{section:tableaux-equality}. 

Buď $L$ \emph{spočetný} jazyk bez rovnosti. Označme jako $L_C$ rozšíření jazyka $L$ o spočetně mnoho nových \emph{pomocných} konstantních symbolů $C=\{c_i\mid i\in \mathbb N\}$. Zvolme nějaké očíslování konstantních termů jazyka $L_C$, označme tyto termy $\{t_i\mid i\in\mathbb N\}$.

Mějme nějakou $L$-teorii $T$ a $L$-sentenci $\varphi$.

\subsection{Atomická tabla}

\emph{Položka} je nápis $\T\varphi$ nebo $\F\varphi$, kde $\varphi$ je nějaká $L_C$-sentence. Položky tvaru $\T(\exists x)\varphi(x)$ a $\F(\forall x)\varphi(x)$ jsou \emph{typu `svědek'}, položky tvaru $\T(\forall x)\varphi(x)$ a $\F(\exists x)\varphi(x)$ jsou \emph{typu `všichni'}

\emph{Atomická tabla} jsou položkami označkované stromy znázorněné v Tabulkách \ref{table:predicate-atomic-tableaux-logical} a \ref{table:predicate-atomic-tableaux-quantifiers}.

\begin{table}[htbp]
\centering
\begin{tabular}{@{}c||c|c|c|c|c@{}}
 & $\neg$ & $\land$ & $\lor$ & $\limplies$ & $\liff$  \\ \midrule \midrule
True
&  
\begin{forest}
[$\mathrm{T}\neg\varphi$ [$\mathrm{F}\varphi$]]
\end{forest}
&  
\begin{forest}
[$\mathrm{T}\varphi\land\psi$ [$\mathrm{T}\varphi$ [$\mathrm{T}\psi$]]]
\end{forest}
& 
\begin{forest}
[$\mathrm{T}\varphi\lor\psi$ [$\mathrm{T}\varphi$] [$\mathrm{T}\psi$]]
\end{forest}
&
\begin{forest}
[$\mathrm{T}\varphi\limplies\psi$ [$\mathrm{F}\varphi$] [$\mathrm{T}\psi$]]
\end{forest}
&  
\begin{forest}
[$\mathrm{T}\varphi\liff\psi$ [$\mathrm{T}\varphi$ [$\mathrm{T}\psi$]] [$\mathrm{F}\varphi$ [$\mathrm{F}\psi$]]]
\end{forest}
\\ \midrule
False 
& 
\begin{forest}
[$\mathrm{F}\neg\varphi$ [$\mathrm{T}\varphi$]]
\end{forest}
&
\begin{forest}
[$\mathrm{F}\varphi\land\psi$ [$\mathrm{F}\varphi$] [$\mathrm{F}\psi$]]
\end{forest}
&
\begin{forest}
[$\mathrm{F}\varphi\lor\psi$ [$\mathrm{F}\varphi$ [$\mathrm{F}\psi$]]]
\end{forest}
&
\begin{forest}
[$\mathrm{F}\varphi\limplies\psi$ [$\mathrm{T}\varphi$ [$\mathrm{F}\psi$]]]
\end{forest}
&
\begin{forest}
[$\mathrm{F}\varphi\liff\psi$ [$\mathrm{T}\varphi$ [$\mathrm{F}\psi$]] [$\mathrm{F}\varphi$ [$\mathrm{T}\psi$]]]
\end{forest}
\end{tabular}
\caption{Atomická tabla pro logické spojky; $\varphi$ a $\psi$ jsou libovolné $L_C$-sentence.}
\label{table:predicate-atomic-tableaux-logical}
\end{table}


\begin{table}[htbp]
    \centering
    \begin{tabular}{@{}c||c|c@{}}
     & $\forall$ & $\exists$ \\ \midrule \midrule
    True
    &  
    \begin{forest}
        [$\T(\forall x)\varphi(x)$ [$\T\varphi(x/t_i)$]]
    \end{forest}
    &  
    \begin{forest}
        [$\T(\exists x)\varphi(x)$ [$\T\varphi(x/c_i)$]]
    \end{forest}
    \\ \midrule
    False 
    &  
    \begin{forest}
        [$\F(\forall x)\varphi(x)$ [$\F\varphi(x/c_i)$]]
    \end{forest}
    &  
    \begin{forest}
        [$\F(\exists x)\varphi(x)$ [$\F\varphi(x/t_i)$]]
    \end{forest} 
    \end{tabular}
    \caption{Atomická tabla pro kvantifikátory; $\varphi$ je $L_C$-sentence, $x$ proměnná, $t_i$ libovolný konstantní $L_C$-term, $c_i\in C$ je nový pomocný konstantní symbol (který se dosud nevyskytuje na dané větvi konstruovaného tabla).}
    \label{table:predicate-atomic-tableaux-quantifiers}
\end{table}

\subsection{Tablo důkaz}

Definice v této části jsou téměř identické odpovídajícím definicím z výrokové logiky. Hlavní technický problém je jak definovat redukovanost položek typu `všichni' na větvi tabla: chceme aby za proměnnou byly substituovány \emph{všechny} konstantní $L_C$-termy $t_i$. 

\begin{definition}[Tablo]
    \emph{Konečné tablo z teorie $T$} je uspořádaný, položkami označkovaný strom zkonstruovaný aplikací konečně mnoha následujících pravidel:
    \begin{itemize}
        \item jednoprvkový strom označkovaný libovolnou položkou je tablo z teorie $T$,
        \item pro libovolnou položkou $P$ na libovolné větvi $V$, můžeme na konec větve $V$ připojit atomické tablo pro položku $P$, přičemž je-li $P$ typu `svědek', můžeme použít jen pomocný konstantní symbol $c_i\in C$, který se na větvi $V$ dosud nevyskytuje (pro položky typu `všichni' můžeme použít libovolný konstantní $L_C$-term $t_i$),
        \item na konec libovolné větve můžeme připojit položku $\mathrm{T}\alpha$ pro libovolný axiom teorie $\alpha\in T$.
    \end{itemize}
    \emph{Tablo z teorie $T$} je buď konečné, nebo i \emph{nekonečné}: v tom případě vzniklo ve spočetně mnoha krocích. Můžeme ho formálně vyjádřit jako sjednocení $\tau=\bigcup_{i\geq 0}\tau_i$, kde $\tau_i$ jsou konečná tabla z $T$, $\tau_0$ je jednoprvkové tablo, a $\tau_{i+1}$ vzniklo z $\tau_i$ v jednom kroku.\footnote{Sjednocení proto, že v jednotlivých krocích přidáváme do tabla nové vrcholy, $\tau_i$ je tedy podstromem $\tau_{i+1}$.}
    
    Tablo \emph{pro položku $P$} je tablo, které má položku $P$ v kořeni.
    \end{definition}
    
    Připomeňme konvenci, že pokud $P$ \emph{není} typu `všichni', potom kořen atomického tabla nebudeme zapisovat (neboť vrchol s položkou $P$ už v tablu je).

\begin{exercise}
    Ukažte v jednotlivých krocích jak byla tabla z Obrázku \ref{figure:predicate-tableau-intro-examples} zkonstruována.
\end{exercise}


\begin{definition}[Tablo důkaz]
    \emph{Tablo důkaz} sentence $\varphi$ z teorie $T$ je \emph{sporné} tablo z teorie $T$ s položkou $\mathrm{F}\varphi$ v kořeni. Pokud existuje, je $\varphi$ \emph{(tablo) dokazatelná} z $T$, píšeme $T\proves\varphi$. (Definujme také \emph{tablo zamítnutí} jako sporné tablo s $\mathrm{T}\varphi$ v kořeni. Pokud existuje, je $\varphi$ \emph{(tablo) zamítnutelná} z $T$, tj. platí $T\proves\neg\varphi$.)  
    \begin{itemize}
        \item Tablo je \emph{sporné}, pokud je každá jeho větev sporná.
        \item Větev je \emph{sporná}, pokud obsahuje položky $\mathrm{T}\psi$ a $\mathrm{F}\psi$ pro nějakou sentenci $\psi$, jinak je \emph{bezesporná}.
        \item Tablo je \emph{dokončené}, pokud je každá jeho větev dokončená.
        \item Větev je \emph{dokončená}, pokud 
        \begin{itemize}
            \item je sporná, nebo
            \item je každá položka na této větvi \emph{redukovaná} a zároveň větev obsahuje položku $\mathrm{T}\alpha$ pro každý axiom $\alpha\in T$.
        \end{itemize}
         
        \item Položka $P$ je \emph{redukovaná} na větvi $V$ procházející touto položkou, pokud 
        \begin{itemize}
            \item je tvaru $\mathrm{T}\psi$ resp. $\mathrm{F}\psi$ pro \emph{atomickou sentenci} $\psi$ (tj. $R(t_1,\dots,t_n)$, kde $t_i$ jsou \emph{konstantní} $L_C$-termy), nebo
            \item není typu `všichni' a vyskytuje se na $V$ jako kořen atomického tabla\footnote{Byť podle konvence tento kořen nezapisujeme.} (tj., typicky, při konstrukci tabla již došlo k jejímu rozvoji na $V$), nebo
            \item je typu `všichni' a všechny její \emph{výskyty} na $V$ jsou na větvi $V$ \emph{redukované}.
        \end{itemize}
        \item Výskyt položky $P$ typu `všichni' na větvi $V$ je \emph{$i$-tý}, pokud má na $V$ právě $i-1$ předků označených touto položkou, a $i$-tý výskyt je \emph{redukovaný} na $V$, pokud
        \begin{itemize}
            \item položka $P$ má $(i+1)$-ní výskyt na $V$, a zároveň
            \item na $V$ se vyskytuje položka $\T\varphi(x/t_i)$ (je-li $P=\T(\forall x)\varphi(x)$) resp. $\F\varphi(x/t_i)$ (je-li $P=\F(\exists x)\varphi(x)$), kde $t_i$ je $i$-tý konstantní $L_C$-term.\footnote{Tj. (typicky) už jsme za $x$ substituovali term $t_i$.}
        \end{itemize} 
    \end{itemize}
\end{definition}
Všimněte si, že je-li položka typu `všichni' na nějaké větvi redukovaná, musí mít na této větvi nekonečně mnoho výskytů, a museli jsme v nich použít při substituci všechny možnosti, tj. všechny konstantní $L_C$-termy.
    
\begin{example}\label{example:tableau-proofs}
Jako příklad sestrojme tablo důkazy \emph{v logice} (z prázdné teorie) následujících sentencí: 
        \begin{enumerate}[(a)]
            \item $(\forall x)(P(x) \limplies Q(x)) \limplies ((\forall x)P(x) \limplies (\forall x)Q(x))$, kde $P,Q$ jsou unární relační symboly.
            \item $(\forall x)(\varphi(x) \land \psi(x)) \liff((\forall x)\varphi (x) \land (\forall x)\psi(x))$, kde $\varphi(x),\psi(x)$ jsou libovolné formule s jedinou volnou proměnnou $x$.
        \end{enumerate}
Výsledná tabla jsou na Obrázcích \ref{figure:predicate-tableau-more-examples-a} a \ref{figure:predicate-tableau-more-examples-b}. Dvojice sporných položek jsou znázorněny červeně. Rozmyslete si, jak byla tabla po krocích zkonstruována.
\begin{figure}[htbp]
\centering
\begin{forest}
    for tree={math content}
    [\F(\forall x)(P(x) \limplies Q(x)) \limplies ((\forall x)P(x) \limplies (\forall x)Q(x))
        [\T(\forall x)(P(x) \limplies Q(x))
            [\F(\forall x)P(x) \limplies (\forall x)Q(x)
                [\T(\forall x)P(x)
                    [\F(\forall x)Q(x)
                        [\textcolor{red}{\F Q(c_0)}
                            [\T(\forall x)P(x)
                                [\textcolor{red}{\T P(c_0)}
                                    [\T(\forall x)(P(x) \limplies Q(x))
                                        [\T P(c_0)\limplies Q(c_0)
                                            [\textcolor{red}{\F P(c_0)}, tikz={\node[fit to=tree,label=below:$\otimes$] {};}]
                                            [\textcolor{red}{\T Q(c_0)}, tikz={\node[fit to=tree,label=below:$\otimes$] {};}]            
                                        ]
                                    ]
                                ]
                            ]
                        ]
                    ]                
                ]
            ]
        ]
    ]
\end{forest}
\caption{Tablo důkaz z Příkladu \ref{example:tableau-proofs} (a).}
\label{figure:predicate-tableau-more-examples-a}
\end{figure} 

\begin{figure}[htbp]
\centering
\begin{forest}
for tree={math content}
[\F(\forall x)(\varphi(x) \land \psi(x)) \liff((\forall x)\varphi (x) \land (\forall x)\psi(x))
    [\T(\forall x)(\varphi(x) \land \psi(x))
        [\F(\forall x)\varphi (x) \land (\forall x)\psi(x)
            [\F(\forall x)\varphi (x)
                [\textcolor{red}{\F\varphi(c_0)}
                    [\T(\forall x)(\varphi(x) \land \psi(x))
                        [\T\varphi(c_0) \land \psi(c_0)
                            [\textcolor{red}{\T\varphi(c_0)}
                                [\T\psi(c_0), tikz={\node[fit to=tree,label=below:$\otimes$] {};}]
                            ]
                        ]
                    ]
                ]
            ]
            [\F(\forall x)\psi(x)
                [\textcolor{red}{\F\psi(c_0)}
                    [\T(\forall x)(\varphi(x) \land \psi(x))
                        [\T\varphi(c_0) \land \psi(c_0)
                            [\T\varphi(c_0)
                                [\textcolor{red}{\T\psi(c_0)}, tikz={\node[fit to=tree,label=below:$\otimes$] {};}]
                            ]
                        ]
                    ]
                ]
            ]
        ]
    ]
    [\F(\forall x)(\varphi(x) \land \psi(x))
        [\T(\forall x)\varphi (x) \land (\forall x)\psi(x)
            [\T(\forall x)\varphi (x)
                [\T(\forall x)\psi(x)
                    [\F(\varphi(c_0) \land \psi(c_0))
                        [\textcolor{red}{\F\varphi(c_0)}
                            [\T(\forall x)\varphi (x)
                                [\textcolor{red}{\T\varphi(c_0)}, tikz={\node[fit to=tree,label=below:$\otimes$] {};}]
                            ]
                        ]
                        [\textcolor{red}{\F\psi(c_0)}
                            [\T(\forall x)\psi (x)
                                [\textcolor{red}{\T\psi(c_0)}, tikz={\node[fit to=tree,label=below:$\otimes$] {};}]
                            ]
                        ]
                    ]                
                ]
            ]
        ]
    ]
]
\end{forest}
\caption{Tablo důkaz z Příkladu \ref{example:tableau-proofs} (b). Konstantu $c_0$ můžeme použít jako \emph{novou} ve všech třech případech. Stačí, že se zatím nevyskytuje \emph{na dané větvi}.}
\label{figure:predicate-tableau-more-examples-b}
\end{figure} 
\end{example}


\subsection{Systematické tablo a konečnost důkazů}

V Sekci \ref{section:finiteness-and-systematicity-of-proofs} jsme ukázali, že neprodlužujeme-li sporné větve (což nemusíme dělat), potom sporné tablo, speciálně tablo důkaz, bude vždy konečný. Stejný důkaz funguje i v logice predikátové.

\begin{corollary}[Konečnost důkazů]\label{corollary:finiteness-of-proofs-predicate}
    Pokud $T\proves\varphi$, potom existuje i \emph{konečný} tablo důkaz $\varphi$ z $T$.
\end{corollary}
\begin{proof}
Stejný jako ve ve výrokové logice, viz důkaz Důsledku \ref{corollary:finiteness-of-proofs}.
\end{proof}

Ve stejné sekci jsme si ukázali konstrukci \emph{systematického tabla}. Tu lze také snadno adaptovat na predikátovou logiku. Musíme zajistit, abychom někdy zredukovali každou položku, použili každý axiom, a nově v predikátové logice také substituovali každý $L_C$ term $t_i$ za proměnnou v položkách typu `všichni'.

\begin{definition}
Mějme položku $R$ a teorii $T=\{\alpha_0,\alpha_1,\alpha_2,\dots\}$. \emph{Systematické tablo} z teorie $T$ pro položku $R$ je tablo $\tau=\bigcup_{i\geq 0}\tau_i$, kde $\tau_0$ je jednoprvkové tablo s položkou $R$, a pro každé $i\geq 0$:

Buď $P$ položka v nejlevějším vrcholu $v$ na co nejmenší úrovni tabla $\tau_i$, která není redukovaná na nějaké bezesporné větvi procházející $P$ (resp. jde-li o položku typu `všichni', její \emph{výskyt} v tomto vrcholu není redukovaný). Potom $\tau_i'$ je tablo vzniklé z $\tau_i$ připojením atomického tabla pro $P$ na každou bezespornou větev procházející $v$, kde
\begin{itemize}
    \item je-li $P$ typu `všichni' a má-li ve vrcholu $v$ $k$-tý výskyt, potom za proměnnou substituujeme $k$-tý $L_C$-term $t_k$,
    \item je-li $P$ typu `svědek', potom na dané větvi $V$ za proměnnou substituujeme $c_i\in C$ s nejmenším možným $i$ (takovým, že na $V$ se $c_i$ dosud nevyskytuje).
\end{itemize}
Jinak, pokud taková položka $P$ a vrchol $v$ neexistují, tj. všechny položky jsou redukované, definujeme $\tau_i'=\tau_i$.

Tablo $\tau_{i+1}$ je potom tablo vzniklé z $\tau_i'$ připojením $\mathrm{T}\alpha_i$ na každou bezespornou větev $\tau_i'$, pokud $i\leq |T|$. Jinak (je-li $T$ konečná a už jsme použili všechny axiomy) tento krok přeskočíme a definujeme $\tau_{i+1}=\tau_i'$. 
\end{definition}

Stejně jako ve výrokové logice platí, že systematické tablo je vždy dokončené, a poskytuje konečný důkaz:

\begin{lemma}\label{lemma:systematic-is-finished-predicate}
    Systematické tablo je dokončené.
\end{lemma}
\begin{proof}
    Obdobný jako důkaz ve výrokové logice (Lemma \ref{lemma:systematic-is-finished}). Pro položky typu `všichni' si všimněte, že $k$-tý výskyt redukujeme v momentě, kdy na něj při konstrukci narazíme: připojením vrcholu s $(k+1)$-ním výskytem a substitucí $k$-tého $L_C$-termu $t_k$.
\end{proof}

\begin{corollary}[Systematičnost důkazů]\label{corollary:systematicity-of-proofs-predicate}
    Pokud $T\proves\varphi$, potom systematické tablo je (konečným) tablo důkazem $\varphi$ z $T$.
\end{corollary}
\begin{proof}
    Stejný jako důkaz ve výrokové logice (Důsledek \ref{corollary:systematicity-of-proofs}).
\end{proof}


\section{Jazyky s rovností} \label{section:tableaux-equality}

Nyní si ukážeme, jak aplikovat tablo metodu na jazyky s rovností. Co je to rovnost? V matematice může v různém kontextu znamenat různé relace. Platí $1+0=0+1$? Mluvíme-li o celých číslech, pak ano, ale máme-li na mysli aritmetické výrazy (nebo např. termy v jazyce těles), potom si levá a pravá strana nejsou rovny: jde o jiné výrazy. \footnote{Podobně např. $t_1=t_2$ v Prologu neznamená, že jde o tentýž term, ale že termy $t_1$ a $t_2$ jsou \emph{unifikovatelné}, viz kapitola o rezoluci v predikátové logice.}

Představte si, že máme teorii $T$ v jazyce s rovností obsahujícím konstantní symboly $c_1,c_2$, unární funkční symbol $f$ a unární relační symbol $P$. Mějme nějaké dokončené tablo z této teorie, a v něm bezespornou větev, na kterém najdeme položku $\T c_1=c_2$. Budeme chtít sestrojit \emph{kanonický model} $\A$ pro tuto větev, podobně jako ve výrokové logice. Položka bude znamenat, že v kanonickém modelu platí $c_1^\A =^\A c_2^\A$, tj. $(c_1^\A,c_2^\A)\in {=^\A}$. To nám ale nestačí, chceme také, aby platilo také např.:
\begin{itemize}
    \item $c_2^\A =^\A c_1^\A$,
    \item $f^\A(c_1^\A) =^\A f^\A(c_2^\A)$,
    \item $c_1^\A\in P^\A$, právě když $c_2^\A\in P^\A$.
\end{itemize}
Obecně tedy chceme, aby relace $=^\A$ byla tzv. \emph{kongruencí},\footnote{Název pochází z kongruence modulo $n$, která je kongruencí v tomto smyslu na množině všech celých čísel, např. splňuje: $a+b\equiv c+d\pmod n$ kdykoliv $a\equiv c\pmod n$ a $b\equiv d\pmod n$.} tj. ekvivalencí, která se chová `dobře' vůči funkcím a relacím struktury $\A$. Toho docílíme tak, že k teorii $T$ přidáme tzv. \emph{axiomy rovnosti}, které tyto vlastnosti vynutí, a tablo sestrojíme z výsledné teorie $T^*$.

V modelu $\A$ potom bude relace $=^\A$ kongruencí. To nám ale nestačí, chceme, aby rovnost byla \emph{identita}, tj. aby $(a,b)\in=^\A$ platilo jedině když $a$ a $b$ jsou týmž prvkem univerza. Toho docílíme identifikací všech $=^\A$-ekvivalentních prvků do jediného prvku. Této konstrukci se říká \emph{faktorstruktura} podle kongruence $=^\A$.\footnote{Stejně jako grupa $\mathbb Z_n$ je faktorstrukturou grupy $\mathbb Z$ podle $\equiv\pmod n$; např. prvek $2\in\mathbb Z_n$ představuje množinu všech celých čísel, jejichž zbytek po dělení $n$ je roven 2.} Nyní tyto pojmy formalizujeme.

\begin{definition}[Kongruence]
    Mějme ekvivalenci $\sim$ na množině $A$, funkci $f\colon A^n\to A$, a relaci $R\subseteq A^n$. Říkáme, že $\sim$ je
    \begin{itemize}
        \item \emph{kongruencí pro funkci $f$}, pokud pro všechna $x_i,y_i\in A$ taková, že $x_i\sim y_i$ ($1\leq i\leq n$) platí $f(x_1,\dots,x_n)\sim f(y_1,\dots,y_n)$,
        \item \emph{kongruencí pro relaci $R$}, pokud pro všechna $x_i,y_i\in A$ taková, že $x_i\sim y_i$ ($1\leq i\leq n$) platí $R(x_1,\dots,x_n)$ právě když $R(y_1,\dots,y_n)$.
    \end{itemize}    
    \emph{Kongruence struktury} $\A$ je ekvivalence $\sim$ na množině $A$, která je kongruencí pro všechny funkce a relace $\A$. 
\end{definition}

\begin{definition}[Faktorstruktura]
    Mějme strukturu $\A$ a její kongruenci $\sim$. \emph{Faktorstruktura (podílová struktura)} $\A$ podle $\sim$ je struktura $\A/_\sim$ v témž jazyce, jejíž univerzum $A/_\sim$ je množina všech rozkladových tříd $A$ podle $\sim$, a jejíž funkce a relace jsou definované \emph{pomocí reprezentantů}, tj:
    \begin{itemize}
        \item $f^{\A/_\sim}([x_1]_\sim,\dots,[x_n]_\sim)=[f^\A(x_1,\dots,x_n)]_\sim$, pro každý ($n$-ární) funkční symbol $f$, a
        \item $R^{\A/_\sim}([x_1]_\sim,\dots,[x_n]_\sim)$ právě když $R^\A(x_1,\dots,x_n)$, pro každý ($n$-ární) relační symbol~$R$.
    \end{itemize} 
    
\end{definition}


\begin{definition}[Axiomy rovnosti]
    \emph{Axiomy rovnosti} pro jazyk $L$ s rovností jsou následující:
    \begin{enumerate}[(i)]
        \item $x=x$,
        \item $x_1=y_1\land\cdots\land x_n=y_n\limplies f(x_1,\dots,x_n)=f(y_1,\dots,y_n)$ pro každý $n$-ární funkční symbol $f$ jazyka $L$,
        \item $x_1=y_1\land\cdots\land x_n=y_n\limplies (R(x_1,\dots,x_n)\limplies R(y_1,\dots,y_n))$ pro každý $n$-ární relační symbol $R$ jazyka $L$ \emph{včetně rovnosti}.
    \end{enumerate}
\end{definition}

\begin{exercise}
    První z axiomů rovnosti znamená reflexivitu relace $=^\A$. Kam se poděly symetrie a tranzitivita? Ukažte, že plynou z axiomu (iii) pro symbol rovnosti $=$.
\end{exercise}

Z axiomů $(i)$ a $(iii)$ tedy plyne, že relace $=^\A$ je ekvivalence na $A$, a axiomy $(ii)$ a $(iii)$ vyjadřují, že $=^\A$ je kongruencí $\A$. V tablo metodě v případě jazyka s rovností implicitně přidáme všechny axiomy rovnosti:

\begin{definition}[Tablo důkaz s rovností]
    Je-li $T$ teorie v jazyce $L$ s rovností, potom označme jako $T^*$ rozšíření teorie $T$ o generální uzávěry\footnote{Neboť v tablo metodě potřebujeme \emph{sentence}.} axiomů rovnosti pro jazyk $L$. \emph{Tablo důkaz} z teorie $T$ je \emph{tablo důkaz} z $T^*$, podobně pro tablo zamítnutí (a obecně jakékoliv tablo).    
\end{definition}

Platí následující jednoduché pozorování:
\begin{observation}
    Jestliže $\A\models T^*$, potom platí i $\A/_{=^\A}\models T^*$, a ve struktuře $\A/_{=^\A}$ je symbol rovnosti interpretován jako identita. Na druhou stranu, v každém modelu, ve kterém je symbol rovnosti interpretován jako identita, platí axiomy rovnosti.
\end{observation}

Toto pozorování využijeme při konstrukci \emph{kanonického modelu}, který budeme potřebovat v důkazu Věty o úplnosti. Nejprve ale dokážeme Větu o korektnosti.

\section{Korektnost a úplnost}

V této sekci dokážeme, že tablo metoda je i v predikátové logice korektní a úplná. Důkazy obou vět mají stejnou strukturu jako ve výrokové logice, liší se jen v implementačních detailech.

\subsection{Věta o korektnosti}

Model (struktura) $\A$ se \emph{shoduje} s položkou $P$, pokud
$P=\mathrm{T}\varphi$ a $\A\models\varphi$, nebo $P=\mathrm{F}\varphi$ a $\A\not\models\varphi$. Dále $\A$ se shoduje s větví $V$, pokud se shoduje s každou položkou na této větvi.

Ukážeme nejprve pomocné lemma analogické Lemmatu \ref{lemma:agrees-with-branch}:
\begin{lemma}\label{lemma:agrees-with-branch-predicate}
    Shoduje-li se model $\A$ teorie $T$ s položkou v kořeni tabla z teorie $T$ (v jazyce $L$), potom lze $\A$ expandovat do jazyka $L_C$ tak, že se shoduje s některou větví v tablu.
\end{lemma}
Všimněte si, že stačí expandovat $\A$ o nové konstanty $c^\A$ vyskytující se na větvi $V$. Ostatní konstantní symboly lze interpretovat libovolně.

\begin{proof}
    Mějme tablo $\tau=\bigcup_{i\geq 0}\tau_i$ z teorie $T$ a model $\A\in\M_L(T)$ shodující se s kořenem $\tau$, tedy s (jednoprvkovou) větví $V_0$ v (jednoprvkovém) $\tau_0$.
    
    Indukcí podle $i$ najdeme posloupnost větví $V_i$ a expanzí $\A_i$ modelu $\A$ o konstanty $c^\A\in C$ vyskytující se na $V_i$ takových, že $V_i$ je větev v tablu $\tau_i$ shodující se s modelem $\A_i$, $V_{i+1}$ je prodloužením $V_i$, a $\A_{i+1}$ je expanzí $\A_i$ (mohou si být i rovny). Požadovaná větev tabla $\tau$ je potom $V=\bigcup_{i\geq 0}V_i$. Expanzi modelu $\A$ do jazyka $L_C$ získáme jako `limitu' expanzí $\A_i$, tj. vyskytuje-li se symbol $c\in C$ na $V$, vyskytuje se na nějaké z větví $V_i$ a interpretujeme ho stejně jako v $\A_i$ (ostatní pomocné symboly interpretujeme libovolně).
    \begin{itemize}
        \item Pokud $\tau_{i+1}$ vzniklo z $\tau_i$ bez prodloužení větve $V_i$, definujeme $V_{i+1}=V_i$ a $\A_{i+1}=\A_{i}$.
        \item Pokud $\tau_{i+1}$ vzniklo z $\tau_i$ připojením položky $\mathrm{T}\alpha$ (pro nějaký axiom $\alpha\in T$) na konec větve  $V_i$, definujeme $V_{i+1}$ jako tuto prodlouženou větev a $\A_{i+1}=\A_i$ (nepřidali jsme žádný nový pomocný konstantní symbol). Protože $\A_{i+1}$ je modelem $T$, platí v něm axiom $\alpha$, tedy shoduje se i s novou položkou $\mathrm{T}\alpha$.
        \item Nechť $\tau_{i+1}$ vzniklo z $\tau_i$ připojením atomického tabla pro nějakou položku $P$ na konec větve $V_i$. Protože se model $\A_i$ shoduje s položkou $P$ (která leží na větvi $V_i$), shoduje se i s kořenem připojeného atomického tabla.
        \begin{itemize}
            \item Pokud jsme připojili atomické tablo pro logickou spojku, položíme $\A_{i+1}=\A_i$ (nepřidali jsme nový pomocný symbol). Protože $\A_{i+1}$ se shoduje s kořenem atomického tabla, shoduje se i s některou z jeho větví (stejně jako ve výrokové logice); definujeme $V_{i+1}$ jako prodloužení $V_i$ o tuto větev.
            \item Je-li položka $P$ typu `svědek': Pokud je $P=\T(\exists x)\varphi(x)$, potom $\A_i\models(\exists x)\varphi(x)$, tedy existuje $a\in A$ takové, že $\A_i\models\varphi(x)[e(x/a)]$. Větev $V_{i+1}$ definujeme jako prodloužení $V_i$ o nově přidanou položku $\T\varphi(x/c)$ a model $\A_{i+1}$ jako expanzi $\A_i$ o konstantu $c^A=a$. Případ $P=\F(\forall x)\varphi(x)$ je obdobný.
            \item Je-li položka $P$ typu `všichni', větev $V_{i+1}$ definujeme jako prodloužení $V_i$ o atomické tablo. Nově přidaná položka je $\T\varphi(x/t)$ nebo $\F\varphi(x/t)$ pro nějaký $L_C$-term $t$. Předpokládejme, že jde o první z těchto dvou možností, pro druhou je důkaz analogický. 
            Model $\A_{i+1}$ definujeme jako \emph{libovolnou} expanzi $\A_i$ o nové konstanty vyskytující se v $t$.    
            Protože $\A_i\models(\forall x)\varphi(x)$, platí i $\A_{i+1}\models(\forall x)\varphi(x)$ a tedy i $\A_{i+1}\models\varphi(x/t)$; model $\A_{i+1}$ se tedy shoduje s větví $V_i$.
        \end{itemize}       
    \end{itemize}
\end{proof}

Připomeňme stručně myšlenku důkazu Věty o korektnosti: Pokud by existoval důkaz a zároveň protipříklad, protipříklad by se musel shodovat s některou větví důkazu, ty jsou ale všechny sporné. Důkaz je tedy téměř stejný jako ve výrokové logice.

\begin{theorem}[O korektnosti]
Je-li sentence $\varphi$ tablo dokazatelná z teorie $T$, potom je $\varphi$ pravdivá v $T$, tj. $T\proves\varphi\ \Rightarrow\ T\models\varphi$.    
\end{theorem}

\begin{proof}
Předpokládejme pro spor, že $T\not\models\varphi$, tj. existuje $\A\in\M(T)$ takový, že $\A\not\models\varphi$. Protože $T\proves\varphi$,  existuje sporné tablo z $T$ s $\mathrm{F}\varphi$ v kořeni. Model $\A$ se shoduje s $\mathrm{F}\varphi$, tedy podle Lemmatu \ref{lemma:agrees-with-branch-predicate} lze expandovat do jazyka $L_C$ tak, že se expanze shoduje s nějakou větví $V$. Všechny větve jsou ale sporné.
\end{proof}

\subsection{Věta o úplnosti}

Stejně jako ve výrokové logice ukážeme, že \emph{bezesporná} větev v \emph{dokončeném} tablo důkazu poskytuje protipříklad: model teorie $T$, který se shoduje s položkou $\mathrm{F}\varphi$ v kořeni tabla, tj. neplatí v něm $\varphi$. Takových modelů může být více, definujeme proto opět jeden konkrétní, \emph{kanonický}.

Model musí mít nějakou doménu. Jak ji získat z tabla, což je čistě sémantický objekt? Využijeme standardní (v matematice) trik: ze syntaktických objektů uděláme sémantické. Konkrétně, za doménu zvolíme množinu všech \emph{konstantních termů} jazyka $L_C$.\footnote{Tj. termů zbudovaných aplikací funkčních symbolů jazyka $L$ na konstantní symboly jazyka $L$ (má-li nějaké) a pomocné konstantní symboly z $C$.} Ty chápeme jako konečné řetězce. V následujícím výkladu budeme někdy (neformálně) místo termu $t$ psát ``$t$'', abychom zdůraznili, že v daném místě chápeme $t$ jako řetězec znaků, a ne např. jako termovou funkci, kterou je třeba vyhodnotit.\footnote{Srovnejte aritmetický výraz ``1+1'' a 1+1=2.}

\begin{definition}[Kanonický model]\label{definition:canonical-model-predicate}
Mějme teorii $T$ v jazyce $L=\langle\mathcal F,\mathcal R\rangle$ a nechť $V$ je bezesporná větev nějakého dokončeného tabla z teorie $T$. Potom \emph{kanonický model} pro $V$ je $L_C$-struktura $\A=\langle A,\mathcal F^\mathcal A\cup C^\mathcal A,\mathcal R^\mathcal A\rangle$ definovaná následovně:

Je-li jazyk $L$ bez rovnosti, potom:
\begin{itemize}
    \item Doména $A$ je množina všech konstantních $L_C$-termů.
    \item Pro každý $n$-ární relační symbol $R\in\mathcal R$ a ``$s_1$'', \dots, ``$s_n$'' z $A$:
    $$
    (\text{``$s_1$''},\dots,\text{``$s_n$''})\in R^\mathcal A\text{ právě když na větvi $V$ je položka $\T R(s_1,\dots,s_n)$}
    $$
    \item Pro každý $n$-ární funkční symbol $f\in\mathcal F$ a ``$s_1$'', \dots, ``$s_n$'' z $A$:
    $$
    f^\mathcal A(\text{``$s_1$''},\dots,\text{``$s_n$''})=\text{``$f(s_1,\dots,s_n)$''}
    $$
    Speciálně, pro konstantní symbol $c$ máme $c^\mathcal A=\text{``$c$''}$.
\end{itemize}
Funkci $f^\mathcal A$ tedy interpretujeme jako `vytvoření' nového termu ze symbolu $f$ a vstupních termů (řetězců). 

Nechť je $L$ jazyk s rovností. Připomeňme, že naše tablo je nyní z teorie $T^*$, tj. z rozšíření~$T$ o axiomy rovnosti pro $L$. Nejprve vytvoříme kanonický model $\mathcal B$ pro $V$ jako by byl $L$ bez rovnosti (jeho doména $B$ je tedy množina všech konstantních $L_C$-termů). Dále definujeme relaci $=^B$ stejně jako pro ostatní relační symboly:
$$
\text{``$s_1$''}=^B\text{``$s_2$''}\text{ právě když na větvi $V$ je položka $\T s_1=s_2$}
$$
\emph{Kanonický model} pro $V$ potom definujeme jako faktorstrukturu $\A=\B/_{=^B}$.
\end{definition}

Jak plyne z diskuze v Sekci \ref{section:tableaux-equality}, relace $=^B$ je opravdu kongruence struktury $\B$, definice je tedy korektní, a relace $=^\A$ je identita na $A$. Platí následující jednoduché pozorování:


\begin{observation}\label{canonical-with-equality-satisfies-the-same}
    Pro každou formuli $\varphi$ máme $\B\models\varphi$ (kde symbol $=$ je interpretován jako binární relace $=^B$), právě když $\A=\B/_{=^B}\models\varphi$ (kde $=$ je interpretován jako identita).    
\end{observation}


Všimněte si, že v jazyce bez rovnosti je kanonický model vždy spočetně nekonečný. V jazyce s rovností může ale být konečný, jak uvidíme v následujících příkladech.

\begin{example}
    Nejprve si ukažme příklad kanonického modelu v jazyce bez rovnosti. Mějme teorii $T=\{(\forall x)R(f(x))\}$ v jazyce $L=\langle R,f,d \rangle$ bez rovnosti, kde $R$ je unární relační, $f$ unární funkční, a $d$ konstantní symbol. Najděme protipříklad ukazující, že $T\not\models\neg R(d)$. 
    
    Systematické tablo z $T$ s položkou $\F\neg R(d)$ v kořeni není sporné, obsahuje jedinou větev $V$, která je bezesporná. (Sestrojte si tablo sami!) Kanonický model pro $V$ je $L_C$-struktura $\A=\langle A,R^\A,f^\A,d^\A,c_0^\A,c_1^\A,c_2^\A,\dots\rangle$, jejíž doména je
    $$
    A=\{\text{``$d$''},\text{``$f(d)$''},\text{``$f(f(d))$''},\dots,\text{``$c_0$''},\text{``$f(c_0)$''},\text{``$f(f(c_0))$''},\dots,\text{``$c_1$''},\text{``$f(c_1)$''},\text{``$f(f(c_1))$''},\dots\}
    $$
    a interpretace symbolů jsou následující:
    \begin{itemize}
        \item $d^\A=\text{``$d$''}$,
        \item $c^\A_i=\text{``$c_i$''}$ pro všechna $i\in \mathbb N$,
        \item $f^\A(\text{``$d$''})=\text{``$f(d)$''}$, $f^\A(\text{``$f(d)$''})=\text{``$f(f(d))$''}$, \dots
        \item $R^\A=A\setminus C=\{\text{``$d$''},\text{``$f(d)$''},\text{``$f(f(d))$''},\dots,\text{``$f(c_0)$''},\text{``$f(f(c_0))$''},\dots,\text{``$f(c_1)$''},\text{``$f(f(c_1))$''},\dots\}$.
    \end{itemize}
    Redukt kanonického modelu $\A$ na původní jazyk $L$ je potom $\A'=\langle A, R^\A, f^\A, d^\A\rangle$.
\end{example}

\begin{example}
    Nyní příklad v jazyce s rovností: Mějme teorii $T=\{(\forall x)R(f(x)),(\forall x)(x=f(f(x)))\}$ v jazyce $L=\langle R,f,d \rangle$ s rovností. Opět najděme protipříklad ukazující, že $T\not\models\neg R(d)$. 

    Systematické tablo z teorie $T^*$ (tj. z $T$ rozšířené o axiomy rovnosti pro $L$) s položkou $\F\neg R(d)$ v kořeni obsahuje bezespornou větev $V$. (Sestrojte si tablo sami!) Nejprve sestrojíme kanonický model $\B$ pro tuto větev, jako by byl jazyk bez rovnosti:
    $$
    \B=\langle B,R^\B,f^\B,d^\B,c_0^\B,c_1^\B,c_2^\B,\dots\rangle
    $$
    kde $B$ je množina všech konstantních $L_C$-termů. Relace $=^B$ je definovaná, jako by symbol `$=$' byl `obyčejným' relačním symbolem v $L$. Je to kongruence struktury $\B$, a platí pro ni, že $s_1=^B s_2$ právě když $s_1=f(\cdots (f(s_2))\cdots)$ nebo $s_2=f(\cdots (f(s_1))\cdots)$ pro sudý počet aplikací $f$. Jako reprezentanty jednotlivých tříd tedy můžeme vybrat termy s žádným nebo jedním výskytem symbolu $f$:
    $$
        B/_{=^B} = \{[\text{``$d$''}]_{=^B},[\text{``$f(d)$''}]_{=^B},[\text{``$c_0$''}]_{=^B},[\text{``$f(c_0)$''}]_{=^B},[\text{``$c_1$''}]_{=^B},[\text{``$f(c_1)$''}]_{=^B},\dots\}
    $$
    Kanonický model pro větev $V$ je potom $L_C$-struktura 
    $$
    \A=\B/_{=^B}=\langle A,R^\A,f^\A,d^\A,c_0^\A,c_1^\A,c_2^\A,\dots\rangle
    $$
    kde $A=B/_{=^B}$ a interpretace symbolů jsou následující:
    \begin{itemize}
        \item $d^\A=[\text{``$d$''}]_{=^B}$,
        \item $c^\A_i=[\text{``$c_i$''}]_{=^B}$ pro všechna $i\in \mathbb N$,
        \item $f^\A([\text{``$d$''}]_{=^B})=[\text{``$f(d)$''}]_{=^B}$, $f^\A([\text{``$f(d)$''}]_{=^B})=[\text{``$f(f(d))$''}]_{=^B}=[\text{``$d$''}]_{=^B}$, \dots
        \item $R^\A=A=B/_{=^B}$.
    \end{itemize}
    Redukt kanonického modelu $\A$ na původní jazyk $L$ je opět $\A'=\langle A, R^\A, f^\A, d^\A\rangle$.
\end{example}

\begin{exercise}
    \begin{enumerate}[(a)]
        \item Sestrojte dokončené tablo s položkou $\T (\forall x)(\forall y)(x=y)$ v kořeni. Sestrojte kanonický model pro (jedinou, bezespornou) větev tohoto tabla.
        \item Sestrojte dokončené tablo s položkou $\T (\forall x)(\forall y)(\forall z)(x=y\lor x=z \lor y=z)$ v kořeni. Sestrojte kanonické modely pro několik bezesporných větví a porovnejte je.
    \end{enumerate}
\end{exercise} 

Nyní jsme připraveni dokázat Větu o úplnosti. Použijeme opět následující pomocné lemma, jehož znění je zcela stejné, jako znění Lemmatu \ref{lemma:canonical-model-agrees} a důkaz se liší jen v technických detailech.

\begin{lemma}\label{lemma:canonical-model-agrees-predicate}
    Kanonický model pro (bezespornou dokončenou) větev $V$ se shoduje s $V$.
\end{lemma}
\begin{proof}
Nejprve uvažme jazyky bez rovnosti. Ukážeme indukcí podle struktury sentencí v položkách, že kanonický model $\A$ se shoduje se všemi položkami $P$ na větvi $V$. 

Základ indukce, tj. případ, kdy $\varphi=R(s_1,\dots,s_n)$ je atomická sentence, je jednoduchý: Je-li na $V$ položka $\T\varphi$, potom $(s_1,\dots,s_n)\in R^\A$ plyne přímo z definice kanonického modelu, máme tedy $\A\models\varphi$.  Je-li na $V$ položka $\F\varphi$, potom na $V$ není položka $\T\varphi$ ($V$ je bezesporná), $(s_1,\dots,s_n)\not\in R^\A$, a $\A\not\models\varphi$

Nyní indukční krok. Rozebereme jen několik případů, ostatní se dokáží obdobně. 

Pro logické spojky je důkaz zcela stejný jako ve výrokové logice, například je-li $P=\mathrm{F}\varphi\land\psi$, potom protože je $P$ na $V$ redukovaná, vyskytuje se na $V$ položka $\mathrm{F}\varphi$ nebo položka $\mathrm{F}\psi$. Platí tedy $\A\not\models\varphi$ nebo $\A\not\models\psi$, z čehož plyne $\A\not\models\varphi\land\psi$ a $\A$ se shoduje s $P$.

Máme-li položku typu ``všichni'', například $P=\T(\forall x)\varphi(x)$ 
(případ $P=\F(\exists x)\varphi(x)$ je obdobný), potom jsou na $V$ i položky $T\varphi(x/t)$ pro každý konstantní $L_C$-term, tj. pro každý prvek $\text{``$t$''}\in A$. Dle indukčního předpokladu je $\A\models\varphi(x/t)$ pro každé $\text{``$t$''}\in A$, tedy $\A\models(\forall x)\varphi(x)$.

Máme-li položku typu ``svědek'', například $P=\T(\exists x)\varphi(x)$ 
(případ $P=\F(\forall x)\varphi(x)$ je obdobný), potom je na $V$ i položka $T\varphi(x/c)$ pro nějaké $\text{``$c$''}\in A$. Dle indukčního předpokladu je $\A\models\varphi(x/c)$, tedy i $\A\models(\exists x)\varphi(x)$.

Je-li jazyk s rovností, máme kanonický model $\A=\B/_{=^B}$, důkaz výše platí pro $\B$, a zbytek plyne z Pozorování \ref{canonical-with-equality-satisfies-the-same}.
\end{proof}

\begin{exercise}
    Ověřte zbývající případy v důkazu Lemmatu \ref{lemma:canonical-model-agrees-predicate}.
\end{exercise}

Důkaz Věty o úplnosti je také analogický její verzi pro výrokovou logiku:

\begin{theorem}[O úplnosti]\label{theorem:completeness-theorem-predicate}
    Je-li sentence $\varphi$ pravdivá v teorii $T$, potom je tablo dokazatelná z $T$, tj. $T\models\varphi\ \Rightarrow\ T\proves\varphi$.    
\end{theorem} 

\begin{proof}
Ukážeme, že libovolné \emph{dokončené} tablo z $T$ s položkou $\mathrm{F}\varphi$ v kořeni je nutně sporné. Důkaz provedeme sporem: kdyby takové tablo nebylo sporné, existovala by v něm bezesporná (dokončená) větev $V$. Uvažme kanonický model $\A$ pro tuto větev, a označme jako $\A'$ jeho redukt na jazyk $L$. Protože je $V$ dokončená, obsahuje $\mathrm{T}\alpha$ pro všechny axiomy $\alpha\in T$. Model $\A$ se podle Lemmatu \ref{lemma:canonical-model-agrees-predicate} shoduje se všemi položkami na $V$, splňuje tedy všechny axiomy a máme i $\A'\models T$. Protože se ale $\A$ shoduje i s položkou $\mathrm{F}\varphi$ v kořeni, platí i $\A'\not\models\varphi$, což znamená, že $\A'\in \M_L(T)\setminus\M_L(\varphi)$, tedy $T\not\models\varphi$, a to je spor. Tablo tedy muselo být sporné, tj. být tablo důkazem $\varphi$ z $T$.
\end{proof}


\section{Důsledky korektnosti a úplnosti}

Stejně jako ve výrokové logice, Věty o korektnosti a úplnosti dohromady říkají, že \emph{dokazatelnost} je totéž, co \emph{platnost}. To nám umožňuje obdobně zformulovat syntaktické analogie sémantických pojmů a vlastností.

Analogií \emph{důsledků} jsou \emph{teorémy} teorie $T$:
$$
\Thm_L(T)=\{\varphi\mid \varphi\text{ je $L$-sentence a } T\proves\varphi\}
$$

\begin{corollary}[Dokazatelnost = platnost]\label{corollary:corollary-of-soundness-and-completeness-predicate}
    Pro libovolnou teorii $T$ a sentence $\varphi,\psi$ platí:
    \begin{itemize}
        \item $T\proves\varphi$ právě když $T\models\varphi$
        \item $\Thm_L(T)=\Conseq_L(T)$
    \end{itemize}
\end{corollary}

Platí například:
\begin{itemize}
    \item Teorie je \emph{sporná}, jestliže je v ní dokazatelný spor (tj. $T\proves\bot$).
    \item Teorie je \emph{kompletní}, jestliže pro každou sentenci $\varphi$ je buď $T\proves\varphi$ nebo $T\proves\neg\varphi$ (ale ne obojí, jinak by byla sporná).
    \item Věta o dedukci: Pro teorii $T$ a sentence $\varphi,\psi$ platí $T,\varphi\proves\psi$, právě když $T\proves\varphi\to\psi$.
\end{itemize}

Na závěr této sekce si ukážeme několik aplikací Vět o úplnosti a korektnosti.

\subsection{Löwenheim-Skolemova věta}\label{subsection:loewenheim-skolem-theorem}

\begin{theorem}[Löwenheim-Skolemova]
    Je-li $L$ spočetný jazyk bez rovnosti, potom každá bezesporná $L$-teorie má spočetně nekonečný model.
\end{theorem}

\begin{proof}
Vezměme nějaké dokončené (např. systematické) tablo z teorie $T$ s položkou $\F\bot$ v kořeni. Protože $T$ je bezesporná, není v ní dokazatelný spor, tedy tablo musí obsahovat bezespornou větev. Hledaný spočetně nekonečný model je $L$-redukt kanonického modelu pro tuto větev.
\end{proof}

K této větě se ještě vrátíme v Kapitole \ref{chapter:model-theory}, kde si ukážeme silnější verzi zahrnující i jazyky s rovností (v nich je kanonický model spočetný, ale může být i konečný).

\subsection{Věta o kompaktnosti}

Stejně jako ve výrokové logice platí Věta o kompaktnosti, stejný je i její důkaz:

\begin{theorem}[O kompaktnosti]\label{theorem:compactness-theorem-predicate}
    Teorie má model, právě když každá její konečná část má model.    
\end{theorem}
\begin{proof}
Model teorie je zřejmě modelem každé její části. Naopak, pokud $T$ nemá model, je sporná, tedy $T\proves\bot$. Vezměme nějaký \emph{konečný} tablo důkaz $\bot$ z $T$. K jeho konstrukci stačí konečně mnoho axiomů $T$, ty tvoří konečnou podteorii $T'\subseteq T$, která nemá model.
\end{proof}


\subsection{Nestandardní model přirozených čísel}

Na závěr této sekce si ukážeme, že existuje tzv. \emph{nestandardní model} přirozených čísel. Klíčem je Věta o kompaktnosti.
    
Nechť $\underline{\mathbb N}=\langle\mathbb N,S,+,\cdot,0,\leq\rangle$ je standardní model přirozených čísel. Označme $\Th(\underline{\mathbb N})$ množinu všech sentencí \emph{pravdivých} ve struktuře $\underline{\mathbb N}$ (tzv. \emph{teorii struktury} $\underline{\mathbb N}$). Pro $n\in \mathbb N$ definujme \emph{$n$-tý numerál} jako term $\underline n=S(S(\cdots (S(0)\cdots))$, kde $S$ je aplikováno $n$-krát.

Vezměme nový konstantní symbol $c$ a vyjádřeme, že je ostře větší než každý $n$-tý numerál:
$$
T=\Th(\underline{\mathbb N})\cup\{\underline n<c\mid n\in \mathbb N\}
$$
Všimněte si, že každá konečná část teorie $T$ má model. Z věty o kompaktnosti tedy plyne, že i teorie $T$ má model. Říkáme mu \emph{nestandardní model} (označme ho $\A$). Platí v něm tytéž sentence, které platí ve standardním modelu, ale zároveň obsahuje prvek $c^\A$, který je větší než každé $n\in \mathbb N$ (čímž zde myslíme hodnotu termu $\underline n$ v nestandardním modelu $\A$).
    

\section{Hilbertovský kalkulus v predikátové logice}

Na závěr kapitoly si ukážeme, jak lze adaptovat Hilbertův kalkulus, představený v Sekci \ref{section:hilbert-calculus-propositional}, pro použití v predikátové logice. To není těžké, abychom se vypořádali s kvantifikátory, stačí přidat dvě nová schémata logických axiomů a jedno nové inferenční pravidlo. Opět si ukážeme korektnost tohoto dokazovacího systému, a jen zmíníme, že je také úplný.

Důkazy budou sestávat z libovolných formulí, nejen sentencí. Připomeňme, že Hilbertovský kalkulus používá jen spojky $\neg$ a $\limpliessymb$. Budeme mít obdobné logické axiomy, jako ve výrokové logice; v případě jazyka s rovností navíc přidáme \alert{axiomy rovnosti}. 

\begin{definition}[Schémata axiomů v hilbertovském kalkulu v predikátové logice]
Pro libovolné formule $\varphi,\psi,\chi$, term $t$, a proměnnou $x$ jsou následující formule logickými axiomy:
\begin{enumerate}[(i)]
    \item $\varphi \limplies (\psi \limplies \varphi)$
    \item $(\varphi\limplies (\psi \limplies \chi))\limplies ((\varphi \limplies \psi)\limplies(\varphi \limplies \chi))$
    \item $(\neg \varphi \limplies \neg \psi)\limplies(\psi \limplies \varphi)$
    \item $(\forall x)\varphi \limplies \varphi(x/t)$, je-li $t$ substituovatelný za $x$ do $\varphi$
    \item $(\forall x)(\varphi \to \psi) \limplies (\varphi \limplies (\forall x)\psi)$, není-li $x$ volná ve $\varphi$
\end{enumerate}
Je-li jazyk s rovností, potom jsou logickými axiomy také \emph{axiomy rovnosti} pro daný jazyk.       
\end{definition}

Všimněte si, že všechny logické axiomy jsou opravdu tautologie. Jako odvozovací pravidla nám poslouží \emph{modus ponens} a dále \emph{pravidlo generalizace}: 

\begin{definition}[Modus ponens]
\emph{Modus ponens} říká, že pokud jsme již dokázali $\varphi$ a také  $\varphi\limplies\psi$, můžeme odvodit i formuli $\psi$:
$$
\frac{\varphi, \varphi \limplies \psi}{\psi}
$$ 
\end{definition}

\begin{definition}[Pravidlo generalizace]
    \emph{Pravidlo generalizace} říká, že pokud jsme dokázali $\varphi$, lze odvodit i formuli $(\forall x)\varphi$ (pro libovolnou proměnnou $x$):
    $$
    \frac{\varphi}{(\forall x)\varphi}
    $$ 
\end{definition}

Všimněte si, že obě odvozovací pravidla jsou \emph{korektní}, tj. platí-li v nějaké teorii $T\models\varphi$ a $T\models\varphi\limplies\psi$, máme i $T\models\psi$, a podobně platí-li $T\models\varphi$, platí i $T\models(\forall x)\varphi$.

Stejně jako ve výrokové logice, \emph{důkaz} bude konečná posloupnost formulí, ve které je každá nově napsaná formule buď axiomem (logickým, vč. axiomu rovnosti, nebo z teorie, ve které dokazujeme), nebo lze odvodit z předchozích pomocí jednoho z odvozovacích pravidel:

\begin{definition}[Hilbertovský důkaz]
    \emph{Hilbertovský důkaz} formule $\varphi$ z teorie $T$ je \emph{konečná} posloupnost formulí $\varphi_0, \dots, \varphi_n=\varphi$, ve které pro každé $i\le n$ platí:
    \begin{itemize}
    \item $\varphi_i$ je logický axiom (včetně axiomu rovnosti, je-li jazyk s rovností), nebo
    \item $\varphi_i$ je axiom teorie ($\varphi_i \in T$), nebo
    \item $\varphi_i$ lze odvodit z nějakých předchozích formulí $\varphi_j,\varphi_k$ (kde $j,k<i$) pomocí modus ponens, nebo
    \item $\varphi_i$ lze odvodit z nějaké předchozí formule $\varphi_j$ (kde $j<i$) pomocí pravidla generalizace.
    \end{itemize}
    Existuje-li hilbertovský důkaz, je $\varphi$ \emph{(hilbertovsky) dokazatelná}, píšeme $T\proves_H\varphi$.           
\end{definition}

I v predikátové logice je hilbertovský kalkulus korektní a úplný dokazovací systém.

\begin{theorem}[O korektnosti hilbertovského kalkulu]
Pro každou teorii $T$ a formuli $\varphi$ platí: 
$$
T\proves_H \varphi\ \Rightarrow\ T\models\varphi
$$ 
\end{theorem}
\begin{proof}
Indukcí dle indexu $i$ ukážeme, že každá formule $\varphi_i$ z důkazu (tedy i $\varphi_n=\varphi$) platí v $T$.

Je-li $\varphi_i$ logický axiom (včetně axiomu rovnosti), $T \models \varphi_i$ platí protože logické axiomy jsou tautologie. Je-li $\varphi_i\in T$, také jistě platí $T \models \varphi_i$. Zbytek plyne z korektnosti odvozovacích pravidel.
\end{proof}

Pro úplnost ještě vyslovme úplnost, důkaz ale neuvedeme.

\begin{theorem}[O úplnosti hilbertovského kalkulu]
Pro každou teorii $T$ a formuli $\varphi$ platí: 
$$
T\models\varphi\ \Rightarrow\ T\proves_H\varphi
$$
\end{theorem}
