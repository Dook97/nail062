\chapter{Metoda analytického tabla}
\label{chapter:tableau-method-propositional}

V této kapitole představíme \emph{Metodu analytického tabla}. Jde o syntaktickou proceduru, kterou můžeme použít pro zjištění, zda daný výrok platí v dané teorii, aniž bychom se museli zabývat sémantikou (např. hledat všechny modely, což je nepraktické). Dokážeme si její \emph{korektnost} (`dává správné odpovědi') a \emph{úplnost} (`funguje vždy'), a použijeme ji také k důkazu tzv.\ \emph{Věty o kompaktnosti} (`vlastnosti nekonečného objektu stačí ukázat pro jeho konečné části').

\section{Formální dokazovací systémy}

\emph{Formální dokazovací systém} formalizuje `dokazování' (např. v matematice) jako přesně (algoritmicky) danou syntaktickou proceduru. \emph{Důkaz} faktu, že v teorii $T$ platí výrok $\varphi$ (neboli $T\models\varphi$) je konečný syntaktický objekt vycházející z axiomů $T$ a výroku $\varphi$. Pokud důkaz existuje, lze ho nalézt `algoritmicky',\footnote{Zde ale musíme být opatrní v případě nekonečné teorie $T$, jak je zadaná? Algoritmus musí mít efektivní přístup ke všem axiomům.} a algoritmicky jsme také schopni ověřit, že je daný objekt opravdu důkaz.

Existuje-li důkaz, říkáme, že $\varphi$ je (v daném dokazovacím systému) \emph{dokazatelný} z $T$, a píšeme $T\proves\varphi$. Po dokazovacím systému požadujeme dvě vlastnosti:
\begin{itemize}
    \item \emph{korektnost}: je-li výrok dokazatelný z teorie, je v ní pravdivý ($T\proves\varphi\Rightarrow T\models\varphi$)
    \item \emph{úplnost}: je-li výrok pravdivý v teorii, je z ní dokazatelný ($T\models\varphi\Rightarrow T\proves\varphi$)
\end{itemize}
(Přičemž korektnost vyžadujeme vždy, ale efektivní důkazový systém může být praktický, i pokud není úplný, zejména pokud je úplný pro nějakou zajímavou třídu výroků resp. teorií.)

V této kapitole si ukážeme kromě \emph{tablo metody} také \emph{Hilbertovský kalkulus}, a v příští kapitole představíme další dokazovací systém, tzv.\ \emph{rezoluční metodu}.


\section{Úvod do tablo metody}

Po zbytek této kapitoly budeme předpokládat, že máme daný \emph{spočetný} jazyk $\mathbb P$. Z toho plyne, že i každá teorie nad $\mathbb P$ je spočetná. Nejprve se soustředíme na případ, kdy $T=\emptyset$, tedy dokazujeme, že výrok $\varphi$ platí \emph{logicky} (je to \emph{tautologie}). 

\emph{Tablo} je olabelovaný strom představující hledání protipříkladu, tj. modelu, ve kterém $\varphi$ neplatí. Labely na vrcholech, kterým budeme říkat \emph{položky}, sestávají ze symbolu $\mathrm{T}$ resp. $\mathrm{F}$ (`True'/`False') následovaného nějakým výrokem $\psi$ a představují předpoklad (požadavek), že v modelu výrok $\psi$ platí resp. neplatí. Do kořene tabla dáme položku $\mathrm{F}\varphi$, tj. hledáme model, ve kterém \emph{neplatí} $\varphi$. Dále budeme tablo rozvíjet pomocí pravidel pro \emph{redukci} položek. Tato pravidla zajišťují následující invariant: 
\begin{quote}
    Každý model, který se \emph{shoduje} s položkou v kořeni (tj. ve kterém neplatí $\varphi$), se musí \emph{shodovat} i s některou větví tabla (tj. splňovat všechny požadavky vyjádřené položkami na této větvi).
\end{quote}
Pokud na některé větvi dostaneme položky tvaru $\mathrm{T}\psi$ a $\mathrm{F}\psi$, říkáme, že větev \emph{selhala} (je \emph{sporná}) a víme, že žádný model s ní nemůže souhlasit. Pokud selžou všechny větve, víme, že neexistuje žádný model, ve kterém by neplatilo $\varphi$, a máme tedy \emph{důkaz}, že $\varphi$ platí. (Všimněte si, že jde o \emph{důkaz sporem}.)

Pokud nějaká větev neselhala, ale je \emph{dokončená}, tj. všechny položky jsou zredukované, víme, že $\varphi$ neplatí, a budeme z této větve schopni zkonstruovat konkrétní model, ve kterém neplatí.


\begin{example}\label{example:tableau-intro-examples}
Ukažme si celý postup na dvou příkladech, viz Obrázek \ref{figure:tableau-proof-two-examples}.
\begin{itemize}
    \item[(a)] Nejprve sestrojme tablo důkaz výroku $\varphi=((p\limplies q)\limplies p)\limplies p$. Začneme kořenem s položkou $\mathrm{F}\varphi$. Tato položka je tvaru $\mathrm{F}\varphi_1\limplies\varphi_2$ (`neplatí implikace'), pokud se s ní shoduje nějaký model, musí splňovat $\mathrm{T}(p\limplies q)\limplies p$ a $\mathrm{F}p$, připojíme tedy tyto dvě položky. (Ve skutečnosti připojíme \emph{atomické tablo} pro tento případ, viz Tabulka \ref{table:atomic-tableaux}, kořen tohoto atomického tabla ale vynecháme, abychom zbytečně nezopakovali tutéž položku.) Tím jsme \emph{zredukovali} položku v kořeni. 
    
    Pokračujeme položkou $\mathrm{T}(p\limplies q)\limplies p$, ta je tvaru `platí implikace', rozvětvíme na dvě větve: model buď splňuje $\mathrm{F}(p\limplies q)$, nebo $\mathrm{T}p$. Pravá větev \emph{selhala} (je \emph{sporná}), neboť obsahuje položky $\mathrm{T}p$, $\mathrm{F}p$, neshoduje se tedy s žádným modelem, označíme ji symbolem~$\otimes$. V levé větvi ještě zredukujeme položku $\mathrm{F}p\limplies q$ a také dostaneme spornou větev. Všechny větve jsou sporné, neexistuje tedy žádný protipříklad a máme důkaz výroku $\varphi$. Píšeme $\proves\varphi$.
    
    \item[(b)] Nyní sestrojíme tablo s položkou $\mathrm{F}(\neg q\lor p)\limplies p$. Snažíme se tedy najít protipříklad: model, ve kterém neplatí $\neg q\lor p)\limplies p$. Nejprve jsme použili atomické tablo pro `neplatí implikace', a dále redukujeme položku $\mathrm{T}\neg q\lor p$ připojením atomického tabla pro `platí disjunkce'. Pravá větev selhala. V levé větvi ještě zredukujeme $\mathrm{T}\neg q$ na $\mathrm{F}q$ (atomické tablo pro `platí negace') tím dostáváme dokončenou větev, neboť všechny položky už jsme zredukovali. Tato dokončená větev ale není sporná (označíme ji tedy symbolem $\checkmark$). To znamená, že protipříklad existuje: máme položky $\mathrm{F}p$ a $\mathrm{F}q$, kterým odpovídá model $(0,0)$, ve kterém opravdu $(\neg q\lor p)\limplies p$ neplatí.

\end{itemize}

V následující sekci celý postup zformalizujeme a vysvětlíme, co dělat, když chceme dokazovat ne v logice, ale v nějaké teorii $T$ (spoiler alert: při konstrukci připojujeme položky $\mathrm{T}\alpha$ pro axiomy $\alpha\in T$). Také si ukážeme příklad s nekonečnou teorií, kde \emph{dokončená} větev někdy musí být nekonečná.

Ve zbytku této sekce ale nejprve definujeme všechna \emph{atomická tabla} potřebná při konstrukci, a také formalizujeme pojem \emph{stromu}.

\begin{figure}
\begin{minipage}{.49\textwidth}
\centering
\begin{forest}
[$\mathrm{F}((p\limplies q)\limplies p)\limplies p$
    [$\mathrm{T}(p\limplies q)\limplies p$
        [$\mathrm{F}p$
            [$\mathrm{F}p\limplies q$
                [$\mathrm{T}p$ 
                    [$\mathrm{F}q$, tikz={\node[fit to=tree,label=below:$\otimes$] {};}]
                ]                
            ]
            [$\mathrm{T}p$, tikz={\node[fit to=tree,label=below:$\otimes$] {};}
            ]
        ]
    ]
]
\end{forest}
\end{minipage}
\begin{minipage}{.49\textwidth}
\centering
\begin{forest}
[$\mathrm{F}(\neg q\lor p)\limplies p$
    [$\mathrm{T}\neg q\lor p$
        [$\mathrm{F}p$
            [$\mathrm{T}\neg q$
                [$\mathrm{F}q$, tikz={\node[fit to=tree,label=below:$\checkmark$] {};}]
            ]
            [$\mathrm{T}p$, tikz={\node[fit to=tree,label=below:$\otimes$] {};}
            ]
        ]
    ]
]
\end{forest}
\end{minipage}
\label{figure:tableau-proof-two-examples}
\caption{Příklady tabel. (a) Tablo důkaz výroku $((p\limplies q)\limplies p)\limplies p$. (b) Tablo pro výrok $(\neg q\lor p)\limplies p$. Levá větev dává protipříklad, model $(0,0)$ ve kterém výrok neplatí.}
\end{figure}
\end{example}


\subsection{Atomická tabla}
Atomická tabla představují pravidla, pomocí kterých redukujeme položky. Pro každou logickou spojku a každý ze dvou příznaků $\mathrm{T}$/ $\mathrm{F}$ máme jedno atomické tablo, znázorněné v Tabulce \ref{table:atomic-tableaux}.

\begin{table}[htbp]
\centering
\begin{tabular}{@{}c||c|c|c|c|c@{}}
 & $\neg$ & $\land$ & $\lor$ & $\limplies$ & $\liff$  \\ \midrule \midrule
True
&  
\begin{forest}
[$\mathrm{T}\neg\varphi$ [$\mathrm{F}\varphi$]]
\end{forest}
&  
\begin{forest}
[$\mathrm{T}\varphi\land\psi$ [$\mathrm{T}\varphi$ [$\mathrm{T}\psi$]]]
\end{forest}
& 
\begin{forest}
[$\mathrm{T}\varphi\lor\psi$ [$\mathrm{T}\varphi$] [$\mathrm{T}\psi$]]
\end{forest}
&
\begin{forest}
[$\mathrm{T}\varphi\limplies\psi$ [$\mathrm{F}\varphi$] [$\mathrm{T}\psi$]]
\end{forest}
&  
\begin{forest}
[$\mathrm{T}\varphi\liff\psi$ [$\mathrm{T}\varphi$ [$\mathrm{T}\psi$]] [$\mathrm{F}\varphi$ [$\mathrm{F}\psi$]]]
\end{forest}
\\ \midrule
False 
& 
\begin{forest}
[$\mathrm{F}\neg\varphi$ [$\mathrm{T}\varphi$]]
\end{forest}
&
\begin{forest}
[$\mathrm{F}\varphi\land\psi$ [$\mathrm{F}\varphi$] [$\mathrm{F}\psi$]]
\end{forest}
&
\begin{forest}
[$\mathrm{F}\varphi\lor\psi$ [$\mathrm{F}\varphi$ [$\mathrm{F}\psi$]]]
\end{forest}
&
\begin{forest}
[$\mathrm{F}\varphi\limplies\psi$ [$\mathrm{T}\varphi$ [$\mathrm{F}\psi$]]]
\end{forest}
&
\begin{forest}
[$\mathrm{F}\varphi\liff\psi$ [$\mathrm{T}\varphi$ [$\mathrm{F}\psi$]] [$\mathrm{F}\varphi$ [$\mathrm{T}\psi$]]]
\end{forest}
\end{tabular}
\caption{Atomická tabla}
\label{table:atomic-tableaux}
\end{table}

Tabla z Příkladu \ref{example:tableau-intro-examples} jsou zkonstruovaná postupným připojováním atomických tabel, viz Obrázek \ref{figure:tableau-proof-two-examples-construction}. Kořeny atomických tabel jsou označené modře, zavedeme konvenci, že je nebudeme zakreslovat.

\begin{figure}
    \begin{minipage}{.49\textwidth}
    \centering
    \begin{forest}
    [$\mathrm{F}((p\limplies q)\limplies p)\limplies p$
        [\textcolor{blue}{$\mathrm{F}((p\limplies q)\limplies p)\limplies p$}, tikz={\node[fit=()(!1)(!ll),rectangle,draw=blue!20,minimum size=12pt] {};}
            [$\mathrm{T}(p\limplies q)\limplies p$
                [$\mathrm{F}p$
                    [\textcolor{blue}{$\mathrm{T}(p\limplies q)\limplies p$}, tikz={\node[fit=()(!1)(!l),rectangle,draw=blue!20,minimum size=12pt] {};}
                        [$\mathrm{F}p\limplies q$
                            [\textcolor{blue}{$\mathrm{F}p\limplies q$}, tikz={\node[fit=()(!1)(!ll),rectangle,draw=blue!20,minimum size=12pt] {};}
                                [$\mathrm{T}p$ 
                                    [$\mathrm{T}q$, tikz={\node[fit to=tree,label=below:$\otimes$] {};}]
                                ]
                            ]
                        ]
                        [$\mathrm{T}p$, tikz={\node[fit to=tree,label=below:$\otimes$] {};}
                        ]
                    ]
                ]
            ]
        ]
    ]
    \end{forest}
    \end{minipage}
    \begin{minipage}{.49\textwidth}
    \centering
    \begin{forest}
    [$\mathrm{F}(\neg q\lor p)\limplies p$
        [\textcolor{blue}{$\mathrm{F}(\neg q\lor p)\limplies p$}, tikz={\node[fit=()(!1)(!ll),rectangle,draw=blue!20,minimum size=12pt] {};}
            [$\mathrm{T}\neg q\lor p$
                [$\mathrm{F}p$
                    [\textcolor{blue}{$\mathrm{T}(\neg q\lor p)$}, tikz={\node[fit=()(!1)(!l),rectangle,draw=blue!20,minimum size=12pt] {};}
                        [$\mathrm{T}\neg q$
                            [\textcolor{blue}{$\mathrm{T}\neg q$}, tikz={\node[fit=()(!1)(!l),rectangle,draw=blue!20,minimum size=12pt] {};}
                                [$\mathrm{F}q$, tikz={\node[fit to=tree,label=below:$\checkmark$] {};}]
                            ]
                        ]
                        [$\mathrm{T}p$, tikz={\node[fit to=tree,label=below:$\otimes$] {};}
                        ]
                    ]
                ]
            ]
        ]
    ]
    \end{forest}
    \end{minipage}
    \label{figure:tableau-proof-two-examples-construction}
    \caption{Konstrukce tabel z Příkladu \ref{example:tableau-intro-examples}.}
\end{figure}

\begin{exercise}\label{exercise:construct-tableaux}
    Pokuste se zkonstruovat tablo s položkou $\mathrm{F}((\neg p\land\neg q)\lor p)\limplies(\neg p\land\neg q)$ v kořeni a také tablo s položkou $\mathrm{T}(p\limplies q)\liff(p\land\neg q)$. Při konstrukci používejte jen atomická tabla (zkontrolujte, zda vaše konstrukce souhlasí s definicí tabla z následující sekce). Rozmyslete si, co tato tabla říkají o výrocích ve svých kořenech.
\end{exercise}

\begin{exercise}
Ověřte, že všechna atomická tabla splňují invariant: shoduje-li se model s položkou v kořeni, shoduje se s některou z větví.
\end{exercise}

\begin{exercise}
Navrhněte atomická tabla pro logické spojky NAND, NOR, XOR, IFTE.
\end{exercise}

\subsection{O stromech}
Než se pustíme do formální definice a důkazů, specifikujme, co myslíme pojmem strom. V teorii grafů bychom stromem nazvali souvislý graf bez cyklů, naše stromy jsou ale zakořeněné, uspořádané (tzv. pravolevým uspořádáním množiny synů každého vrcholu), a označkované. A mohou, často i budou, nekonečné. Formálně:

\begin{definition}[Strom]
\begin{itemize}
\item \emph{Strom} je neprázdná množina $T$ s částečným uspořádáním $<_T$, které má (jediný) minimální prvek (\emph{kořen}) a ve kterém je množina předků libovolného vrcholu \emph{dobře uspořádaná}.\footnote{Tj. každá její neprázdná podmnožina má nejmenší prvek.}
\item \emph{Větev} stromu $T$ je maximální\footnote{Tj. nelze do ní přidat další vrcholy stromu.} lineárně uspořádaná podmnožina $T$.
\item \emph{Uspořádaný strom} je strom $T$ spolu s lineárním uspořádáním $<_L$ množiny synů každého vrcholu. Uspořádání synů budeme říkat \emph{pravolevé} zatímco uspořádání $<_T$ je \emph{stromové}.
\item \emph{Označkovaný strom} je strom spolu se značkovací funkcí $\mathrm{label}\colon V(T)\to\mathrm{Labels}$.
\end{itemize}
\end{definition}

Budeme používat standardní terminologii o stromech, např. budeme mluvit o \emph{$n$-té úrovni stromu}, nebo o \emph{hloubce} stromu (ta je nekonečná, právě když máme nekonečnou větev). V jedné větě, kterou si níže dokážeme, budeme potřebovat následující slavné tvrzení, které je důsledkem axiomu výběru. 

\begin{lemma}[K\"oenigovo lemma]
Nekonečný, konečně větvící strom má nekonečnou větev.
\end{lemma}
\noindent (Strom je \emph{konečně větvící}, pokud má každý vrchol konečně mnoho synů.)


\section{Tablo důkaz}

Nyní uvedeme formální definici tabla. Do definice přidáme také teorii $T$, jejíž axiomy můžeme při konstrukci připojovat s příznakem $\mathrm{T}$. Připomeňme, že \emph{položka} je nápis $\mathrm{T}\varphi$ nebo $\mathrm{F}\varphi$, kde $\varphi$ je nějaký výrok.

\begin{definition}[Tablo]
\emph{Konečné tablo z teorie $T$} je uspořádaný, položkami označkovaný strom zkonstruovaný aplikací konečně mnoha následujících pravidel:
\begin{itemize}
    \item jednoprvkový strom označkovaný libovolnou položkou je tablo z teorie $T$,
    \item pro libovolnou položkou $P$ na libovolné větvi $V$, můžeme na konec větve $V$ připojit atomické tablo pro položku $P$,
    \item na konec libovolné větve můžeme připojit položku $\mathrm{T}\alpha$ pro libovolný axiom teorie $\alpha\in T$.
\end{itemize}
\emph{Tablo z teorie $T$} je buď konečné, nebo i \emph{nekonečné}: v tom případě vzniklo ve spočetně mnoha krocích. Můžeme ho formálně vyjádřit jako sjednocení $\tau=\bigcup_{i\geq 0}\tau_i$, kde $\tau_i$ jsou konečná tabla z $T$, $\tau_0$ je jednoprvkové tablo, a $\tau_{i+1}$ vzniklo z $\tau_i$ v jednom kroku.\footnote{Sjednocení proto, že v jednotlivých krocích přidáváme do tabla nové vrcholy, $\tau_i$ je tedy podstromem $\tau_{i+1}$.}

Tablo \emph{pro položku $P$} je tablo, které má položku $P$ v kořeni.
\end{definition}

Připomeňme konvenci, že kořen atomického tabla nebudeme zapisovat (neboť vrchol s položkou $P$ už v tablu je). V definici neurčujeme, v jakém pořadí provádět jednotlivé kroky, později ale specifikujeme konkrétní postup konstrukce (algoritmus), kterému budeme říkat \emph{systematické tablo}. 

Abychom získali důkazový systém, zbývá definovat pojem \emph{tablo důkazu} (a související pojmy). Připomeňme ještě jednou, že jde o důkaz sporem, tedy předpokládáme, že výrok neplatí, a najdeme spor(né tablo):

\begin{definition}[Tablo důkaz]
\emph{Tablo důkaz} výroku $\varphi$ z teorie $T$ je \emph{sporné} tablo z teorie $T$ s položkou $\mathrm{F}\varphi$ v kořeni. Pokud existuje, je $\varphi$ \emph{(tablo) dokazatelný} z $T$, píšeme $T\proves\varphi$. (Definujme také \emph{tablo zamítnutí} jako sporné tablo s $\mathrm{T}\varphi$ v kořeni. Pokud existuje, je $\varphi$ \emph{(tablo) zamítnutelný} z $T$, tj. platí $T\proves\neg\varphi$.)  
\begin{itemize}
    \item Tablo je \emph{sporné}, pokud je každá jeho větev sporná.
    \item Větev je \emph{sporná}, pokud obsahuje položky $\mathrm{T}\psi$ a $\mathrm{F}\psi$ pro nějaký výrok $\psi$, jinak je \emph{bezesporná}.
    \item Tablo je \emph{dokončené}, pokud je každá jeho větev dokončená.
    \item Větev je \emph{dokončená}, pokud 
    \begin{itemize}
        \item je sporná, nebo
        \item je každá její položka na této větvi \emph{redukovaná} a zároveň obsahuje položku $\mathrm{T}\alpha$ pro každý axiom $\alpha\in T$.
    \end{itemize}
     
    \item Položka $P$ je \emph{redukovaná} na větvi $V$ procházející touto položkou, pokud 
    \begin{itemize}
        \item je tvaru $\mathrm{T}p$ resp. $\mathrm{F}p$ pro nějakou výrokovou proměnnou $p\in\mathbb P$, nebo
        \item při konstrukci tabla již došlo k jejímu rozvoji na $V$, tj. vyskytuje se na $V$ jako kořen atomického tabla.\footnote{Byť podle konvence tento kořen nezapisujeme.}
    \end{itemize}
\end{itemize}
\end{definition}

\begin{example}\label{example:tableaux-from-theory}
    Ukážeme si dva příklady. Tabla jsou znázorněná na Obrázku \ref{figure:tableaux-from-theory-figure}. 
    \begin{itemize}
        \item[(a)] Tablo důkaz výroku $\psi$ z teorie $T=\{\varphi,\varphi\limplies\psi\}$, tj. $T\proves\psi$ (kde $\varphi,\psi$ jsou nějaké pevně dané výroky). Tomuto faktu se říká \emph{Věta o dedukci}.
        \item[(b)] Dokončené tablo pro výrok $p_0$ z teorie $T=\{p_{n+1}\limplies p_n\mid n\in\mathbb N\}$. Nejlevější větev je bezesporná dokončená. Obsahuje položky $\mathrm{T}p_{i+1}\limplies p_i$ a $\mathrm{F}p_i$ pro všechna $i\in\mathbb N$. Shoduje se tedy s modelem $v=(0,0,\dots)$, tj. $v:\mathbb P\to\{0,1\}$ kde $v(p_i)=0$ pro všechna $i$.
    \end{itemize} 
\end{example}

\begin{figure}
    \begin{minipage}{.49\textwidth}
    \centering
    \begin{forest}
    [$\mathrm{F}\psi$
        [\textcolor{blue}{$\mathrm{T}\varphi\limplies \psi$}
            [$\mathrm{F}\varphi$
                [\textcolor{blue}{$\mathrm{T}\varphi$}, tikz={\node[fit to=tree,label=below:$\otimes$] {};}]
            ]                
            [$\mathrm{T}\psi$, tikz={\node[fit to=tree,label=below:$\otimes$] {};}]
        ]
    ]
    \end{forest}
    \end{minipage}
    \begin{minipage}{.49\textwidth}
    \centering
    \begin{forest}
    [$\mathrm{F}p_0$
        [\textcolor{blue}{$\mathrm{T}p_1\limplies p_0$}
            [$\mathrm{F}p_1$                
                [\textcolor{blue}{$\mathrm{T}p_2\limplies p_1$}
                    [$\mathrm{F}p_2$ [$\vdots$]] 
                    [$\mathrm{T}p_0$, tikz={\node[fit to=tree,label=below:$\otimes$] {};}]                    
                ]                
            ]
            [$\mathrm{T}p_0$, tikz={\node[fit to=tree,label=below:$\otimes$] {};}]
        ]
    ]
    \end{forest}
    \end{minipage}
    \label{figure:tableaux-from-theory-figure}
    \caption{Tabla z Příkladu \ref{example:tableaux-from-theory}. Položky vycházející z axiomů jsou označeny modře.}
\end{figure}


\begin{exercise}
    Vraťme se k tablům z Cvičení \ref{exercise:construct-tableaux}. Jde o tablo důkazy nebo zamítnutí (z teorie $T=\emptyset$)? Které položky na kterých větvích jsou redukované? Které větve jsou sporné, které jsou dokončené?
\end{exercise}

\section{Konečnost a systematičnost důkazů}\label{section:finiteness-and-systematicity-of-proofs}

V této sekci dokážeme, že pokud existuje tablo důkaz, existuje vždy také \emph{konečný} tablo důkaz. Představíme také algoritmus, kterým nějaký tablo důkaz můžeme vždy najít, pro důkaz tohoto faktu ale budeme potřebovat Věty o korektnosti a úplnosti z následující sekce. Prozatím ukážeme, že tento algoritmus nám umožní vždy sestrojit dokončené tablo.

Všimněte si, že při redukci položky přidáváme do tabla pouze položky obsahující kratší výroky. Pokud tedy máme konečnou teorii, a neděláme zbytečné kroky (například nepřidáváme opakovaně tentýž axiom, nebo totéž atomické tablo), je snadné sestrojit dokončené tablo, které bude konečné.

Je-li teorie $T$ nekonečná, musíme ale být opatrnější. Mohli bychom nekonečně dlouho konstruovat tablo, a přitom se nikdy nedostat k redukci určité položky, nebo nikdy nepoužít některý z axiomů. Definujeme tedy konkrétní algoritmus pro konstrukci tabla, výsledku budeme říkat \emph{systematické tablo}. Myšlenka konstrukce je jednoduchá: střídáme krok redukce položky (zároveň na všech bezesporných větvích, které jí procházejí) a krokem použití axiomu. Položky procházíme po úrovních, a v rámci úrovně v pravolevém uspořádání. A axiomy teorie ve zvoleném očíslování.

\begin{definition}
Mějme položku $R$ a (konečnou nebo nekonečnou\footnote{Připomeňme, že $T$ je spočetná, neboť jazyk je (v celé kapitole) spočetný.}) teorii $T=\{\alpha_1,\alpha_2,\dots\}$. \emph{Systematické tablo} z teorie $T$ pro položku $R$ je tablo $\tau=\bigcup_{i\geq 0}\tau_i$, kde $\tau_0$ je jednoprvkové tablo s položkou $R$, a pro každé $i\geq 0$:

\begin{itemize}
    \item Nechť $P$ je nejlevější položka v co nejmenší úrovni, která není redukovaná na nějaké bezesporné větvi procházející $P$. Definujeme nejprve tablo $\tau_i'$ jako tablo vzniklé z $\tau_i$ připojením atomického tabla pro $P$ na každou bezespornou větev procházející $P$. (Pokud taková položka neexistuje, potom  $\tau_i'=\tau_i$.)
    \item Následně, $\tau_{i+1}$ je tablo vzniklé z $\tau_i'$ připojením $\mathrm{T}\alpha_i$ na každou bezespornou větev $\tau_i'$. To v případě, že $i\leq |T|$, jinak (je-li $T$ konečná a už jsme použili všechny axiomy) tento krok přeskočíme a definujeme $\tau_{i+1}=\tau_i'$.
\end{itemize}    
\end{definition}

\begin{lemma}\label{lemma:systematic-is-finished}
    Systematické tablo je dokončené.
\end{lemma}
\begin{proof}
    Ukážeme, že každá větev je dokončená. Sporné větve jsou dokončené. Bezesporné větve obsahují položky $\mathrm{T}\alpha_i$ (ty jsme připojili v $i$-tém kroku) a každá položka na nich je redukovaná. Vskutku, kdyby $P$ byla neredukovaná na bezesporné větvi $V$, přišla by na ni v nějakém kroku řada, neboť v úrovních nad $P$ a vlevo od $P$ existuje jen konečně mnoho položek. (Používáme zjevného faktu, že každý prefix bezesporné větve je také bezesporná větev, tedy během konstrukce $V$ nikdy není sporná.)
\end{proof}

Nyní se vraťme k otázce konečnosti důkazů:

\begin{theorem}[Konečnost sporu]\label{theorem:finiteness-of-contradiction}
    Je-li $\tau=\bigcup_{i\geq 0}\tau_i$ sporné tablo, potom existuje $n\in\mathbb N$ takové, že $\tau_n$ je sporné konečné tablo.
\end{theorem}
\begin{proof}
    Uvažme množinu $S$ všech vrcholů stromu $\tau$, které nad sebou (ve stromovém uspořádání) neobsahují spor, tj. dvojici položek $\mathrm{T}\psi$, $\mathrm{F}\psi$.

    Kdyby množina $S$ byla nekonečná, podle Königova lemmatu použitého na podstrom $\tau$ na množině $S$ bychom měli nekonečnou, bezespornou větev v $S$. To by ale znamenalo, že máme i bezespornou větev v $\tau$, což je ve sporu s tím, že $\tau$ je sporné. (Podrobněji: Větev na $S$ by byla podvětví nějaké větve $V$ v $\tau$, která je sporná, tj. obsahuje nějakou (konkrétní) spornou dvojici položek, která ale existuje už v nějakém konečném prefixu $V$.)

    Množina $S$ je tedy konečná. To znamená, že existuje $d\in\mathbb N$ takové, že celá $S$ leží v hloubce nejvýše $d$. Každý vrchol na úrovni $d+1$ má tedy nad sebou spor. Zvolme $n$ tak, že $\tau_n$ už obsahuje všechny vrcholy $\tau$ z prvních $d+1$ úrovní: každá větev $\tau_n$ je tedy sporná.
\end{proof}

\begin{corollary}\label{corollary:systematic-contradictory-is-finite}
    Pokud při konstrukci tabla nikdy neprodlužujeme sporné větve, např. pro systematické tablo, potom sporné tablo je konečné.
\end{corollary}
\begin{proof}
Použijeme Větu \ref{theorem:finiteness-of-contradiction}, máme $\tau=\tau_n$ neboť sporné tablo už neměníme.    
\end{proof}

\begin{corollary}[Konečnost důkazů]\label{corollary:finiteness-of-proofs}
    Pokud $T\proves\varphi$, potom existuje i \emph{konečný} tablo důkaz $\varphi$ z $T$.
\end{corollary}
\begin{proof}
Snadno plyne z Důsledku \ref{corollary:systematic-contradictory-is-finite}: stačí při konstrukci $\tau$ ignorovat kroky, které by prodloužily spornou větev.
\end{proof}

Vyslovíme zde také následující důsledek. Dokážeme ho ale až v příští sekci.

\begin{corollary}[Systematičnost důkazů]\label{corollary:systematicity-of-proofs}
    Pokud $T\proves\varphi$, potom systematické tablo je (konečným) tablo důkazem $\varphi$ z $T$.
\end{corollary}

K důkazu budeme potřebovat dvě fakta: pokud je $\varphi$ dokazatelná z $T$, potom v $T$ platí (Věta o korektnosti), tj. nemůže existovat protipříklad. A dále pokud by systematické tablo mělo bezespornou větev, znamenalo by to, že existuje protipříklad (to je klíčem k Větě o úplnosti).


\section{Korektnost a úplnost}

V této sekci dokážeme, že je tablo metoda \emph{korektní} a \emph{úplný} důkazový systém, tj. že $T\proves\varphi$ platí právě když $T\models\varphi$.

\subsection{Věta o korektnosti}

Řekneme, model $v$ se \emph{shoduje} s položkou $P$, pokud
$P=\mathrm{T}\varphi$ a $v\models\varphi$, nebo $P=\mathrm{F}\varphi$ a $v\not\models\varphi$. Dále $v$ se shoduje s větví $V$, pokud se shoduje s každou položkou na této větvi.

Jak už jsme zmínili, design atomických tabel zaručuje, že shoduje-li se model s položkou v kořeni tabla, shoduje se s některou větví. Není těžké indukcí podle konstrukce tabla ukázat následující lemma:
\begin{lemma}\label{lemma:agrees-with-branch}
    Shoduje-li se model teorie $T$ s položkou v kořeni tabla z teorie $T$, potom se shoduje s některou větví.
\end{lemma}
\begin{proof}
    Mějme tablo $\tau=\bigcup_{i\geq 0}\tau_i$ z teorie $T$ a model $v\in\M(T)$ shodující se s kořenem $\tau$, tedy s (jednoprvkovou) větví $V_0$ v (jednoprvkovém) $\tau_0$.
    
    Indukcí podle $i$ (podle kroků v při konstrukci tabla) najdeme posloupnost $V_0\subseteq V_1\subseteq\dots$ takovou, že $V_i$ je větev v tablu $\tau_i$ shodující se s modelem $v$, a $V_{i+1}$ je prodloužením $V_i$. Požadovaná větev tabla $\tau$ je potom $V=\bigcup_{i\geq 0}V_i$.
    
    \begin{itemize}
        \item Pokud $\tau_{i+1}$ vzniklo z $\tau_i$ bez prodloužení větve $V_i$, definujeme $V_{i+1}=V_i$.
        \item Pokud $\tau_{i+1}$ vzniklo z $\tau_i$ připojením položky $\mathrm{T}\alpha$ (pro nějaký axiom $\alpha\in T$) na konec větve  $V_i$, definujeme $V_{i+1}$ jako tuto prodlouženou větev. Protože $v$ je model $T$, platí v něm axiom $\alpha$, tedy shoduje se i s novou položkou $\mathrm{T}\alpha$.
        \item Nechť $\tau_{i+1}$ vzniklo z $\tau_i$ připojením atomického tabla pro nějakou položku $P$ na konec větve $V_i$. Protože se model $v$ shoduje s položkou $P$ (která leží na větvi $V_i$), shoduje se i s kořenem připojeného atomického tabla, a proto se shoduje i s některou z jeho větví. (Tuto vlastnost snadno ověříme pro všechna atomická tabla.) Definujeme $V_{i+1}$ jako prodloužení $V_i$ o tuto větev atomického tabla.\footnote{Resp. o libovolnou takovou větev: model $v$ se může shodovat s více větvemi atomického tabla.}
    \end{itemize}
\end{proof}

Nyní už můžeme dokázat Větu o korektnosti. Zkráceně řečeno, pokud by existoval důkaz a zároveň protipříklad, protipříklad by se musel shodovat s některou větví důkazu, ty jsou ale všechny sporné.

\begin{theorem}[O korektnosti]
Je-li výrok $\varphi$ tablo dokazatelný z teorie $T$, potom je $\varphi$ pravdivý v $T$, tj. $T\proves\varphi\ \Rightarrow\ T\models\varphi$.    
\end{theorem}

\begin{proof}
Dokážeme sporem. Předpokládejme, že $\varphi$ v $T$ neplatí, tj. existuje protipříklad: model $v\in\M(T)$, ve kterém $\varphi$ neplatí.

Protože je $\varphi$ dokazatelná z $T$, existuje tablo důkaz $\varphi$ z T, což je sporné tablo z $T$ s položkou $\mathrm{F}\varphi$ v kořeni. Model $v$ se shoduje s položkou $\mathrm{F}\varphi$, tedy podle Lemmatu \ref{lemma:agrees-with-branch} se shoduje s nějakou větví $V$. Všechny větve jsou ale sporné, včetně $V$. Takže $V$ obsahuje položky $\mathrm{T}\psi$ a $\mathrm{F}\psi$ (pro nějaký výrok $\psi$), a model $v$ se s těmito položkami shoduje. Máme tedy $v\models\psi$ a zároveň $v\not\models\psi$, což je spor.

\end{proof}

\subsection{Věta o úplnosti}

Ukážeme, že \emph{bezesporná} větev v \emph{dokončeném} tablo důkazu poskytuje protipříklad: model teorie $T$, který se shoduje s položkou $\mathrm{F}\varphi$ v kořeni tabla, tj. neplatí v něm $\varphi$. Takových modelů může být více, definujeme proto jeden konkrétní:

\begin{definition}[Kanonický model]\label{definition:canonical-model}
Je-li $V$ bezesporná větev dokončeného tabla, potom \emph{kanonický model} pro $V$ je model definovaný předpisem (pro $p\in\mathbb P$):
$$
v(p)=\begin{cases}
    1 \text{ pokud se na $V$ vyskytuje položka $\mathrm{T}p$,}\\
    0 \text{ jinak.}
\end{cases}
$$  
\end{definition}
\begin{lemma}\label{lemma:canonical-model-agrees}
    Kanonický model pro (bezespornou dokončenou) větev $V$ se shoduje s $V$.
\end{lemma}
\begin{proof}
Ukážeme, že kanonický model $v$ se shoduje se všemi položkami $P$ na větvi $V$, a to indukcí podle struktury výroku v položce.\footnote{Připomeňme, že to znamená indukci podle hloubky stromu výroku.} Nejprve základ indukce:
\begin{itemize}
    \item Je-li $P=\mathrm{T}p$ pro nějaký prvovýrok $p\in\mathbb P$, máme podle definice $v(p)=1$; $v$ se s $P$ shoduje.
    \item Je-li $P=\mathrm{F}p$, potom se na větvi $V$ nemůže vyskytovat položka $\mathrm{T}p$, jinak by $V$ byla sporná. Podle definice máme $v(p)=0$ a $v$ se s $P$ opět shoduje.
\end{itemize}
Nyní indukční krok. Rozebereme dva případy, ostatní se dokáží obdobně.
\begin{itemize}
    \item Nechť $P=\mathrm{T}\varphi\land\psi$. Protože je $V$ dokončená větev, je na ní položka $P$ redukovaná. To znamená, že se na $V$ vyskytují i položky $\mathrm{T}\varphi$ a $\mathrm{T}\psi$. Podle indukčního předpokladu se s nimi model $v$ shoduje, tedy $v\models\varphi$ a $v\models\psi$. Takže platí i $v\models\varphi\land\psi$ a $v$ se shoduje s $P$.
    \item Nechť $P=\mathrm{F}\varphi\land\psi$. Protože je $P$ na $V$ redukovaná, vyskytuje se na $V$ položka $\mathrm{F}\varphi$ nebo položka $\mathrm{F}\psi$. Platí tedy $v\not\models\varphi$ nebo $v\not\models\psi$, z čehož plyne $v\not\models\varphi\land\psi$ a $v$ se shoduje s $P$.
\end{itemize}
\end{proof}

\begin{theorem}[O úplnosti]
    Je-li výrok $\varphi$ pravdivý v teorii $T$, potom je tablo dokazatelný z $T$, tj. $T\models\varphi\ \Rightarrow\ T\proves\varphi$.    
\end{theorem}

\begin{proof}
Ukážeme, že libovolné \emph{dokončené} (tedy např. i \emph{systematické}) tablo z $T$ s položkou $\mathrm{F}\varphi$ v kořeni je nutně sporné. Důkaz provedeme sporem: kdyby takové tablo nebylo sporné, existovala by v něm bezesporná (dokončená) větev $V$. Uvažme kanonický model $v$ pro tuto větev. Protože je $V$ dokončená, obsahuje $\mathrm{T}\alpha$ pro všechny axiomy $\alpha\in T$. Model $v$ se podle Lemmatu \ref{lemma:canonical-model-agrees} shoduje se všemi položkami na $V$, splňuje tedy všechny axiomy a máme $v\models T$. Protože se ale $v$ shoduje i s položkou $\mathrm{F}\varphi$ v kořeni, máme $v\not\models\psi$, což znamenám, že $T\not\models\psi$, spor. Tablo tedy muselo být sporné, tj. být tablo důkazem $\varphi$ z $T$.
\end{proof}

\begin{proof}[Důkaz Důsledku \ref{corollary:systematicity-of-proofs}]
Z předchozího důkazu také dostáváme `systematičnost důkazů', tj. že důkaz můžeme vždy hledat konstrukcí systematického tabla: Pokud $T\models\varphi$, tak je i systematické tablo pro položku $\mathrm{F}\varphi$ nutně sporné, a je tedy tablo důkazem $\varphi$ z $T$.
\end{proof}

\begin{exercise}
    Ověřte zbývající případy v důkazu Lemmatu \ref{lemma:canonical-model-agrees}.
\end{exercise}

\begin{exercise}
    Popište, jak vypadají \emph{všechny} modely shodující se s danou bezespornou dokončenou větví.
\end{exercise}

\begin{exercise}
    Navrhněte postup, kterým můžeme za použití tablo metody najít všechny modely dané teorie $T$.
\end{exercise}


\section{Důsledky korektnosti a úplnosti}

Věty o korektnosti a úplnosti dohromady říkají, že \emph{dokazatelnost} je totéž, co \emph{platnost}. To nám umožňuje zformulovat syntaktické analogie sémantických pojmů a vlastností.

Analogií \emph{důsledků} jsou \emph{teorémy} teorie $T$:
$$
\Thm_\mathbb P(T)=\{\varphi\in\VF_\mathbb P\mid T\proves\varphi\}
$$

\begin{corollary}[Dokazatelnost = platnost]\label{corollary:corollary-of-soundness-and-completeness}
    Pro libovolnou teorii $T$ a výroky $\varphi,\psi$ platí:
    \begin{itemize}
        \item $T\proves\varphi$ právě když $T\models\varphi$
        \item $\Thm_\mathbb P(T)=\Conseq_\mathbb P(T)$
    \end{itemize}
\end{corollary}
\begin{proof}
    Plyne okamžitě z Věty o korektnosti a z Věty o úplnosti.
\end{proof}


Ve všech definicích a větách můžeme tedy nahradit pojem `\emph{platnost}' pojmem `\emph{dokazatelnost}' (tj. symbol `$\models$' symbolem `$\proves$') a  pojem `\emph{důsledek}' pojmem `\emph{teorém}'. Například:
\begin{itemize}
    \item Teorie je \emph{sporná}, jestliže je v ní dokazatelný spor (tj. $T\proves\bot$).
    \item Teorie je \emph{kompletní}, jestliže pro každý výrok $\varphi$ je buď $T\proves\varphi$ nebo $T\not\proves\varphi$ (ale ne obojí, jinak by byla sporná).
\end{itemize}

Uveďme ještě jeden snadný důsledek:

\begin{theorem}[O dedukci]
Pro teorii $T$ a výroky $\varphi,\psi$ platí:
$T,\varphi\proves\psi\text{ právě když }T\proves\varphi\to\psi$.
\end{theorem}
\begin{proof}
    Stačí dokázat $T,\varphi\models\psi\Leftrightarrow T\models\varphi\to\psi$, což je snadné.
\end{proof}

\begin{exercise}
Dokažte Větu o dedukci přímo, pomocí transformace tablo důkazů.
\end{exercise}


\section{Věta o kompaktnosti}

Důležitým důsledkem vět o korektnosti a úplnosti je také tzv. \emph{Věta o kompaktnosti}.\footnote{Slovo \emph{kompaktnost} pochází z kompaktních (tj. omezených a uzavřených) množin v Euklidovských prostorech, ve kterých lze z každé posloupnosti vybrat konvergentní podposloupnost. Můžete si představit posloupnost zvětšujících se konečných částí `konvergující' k nekonečnému celku.} Tento princip umožňuje převádět tvrzení o nekonečných objektech/procesech na tvrzení o (všech) jejich konečných částech.

\begin{theorem}[O kompaktnosti]\label{theorem:compactness-theorem}
Teorie má model, právě když každá její konečná část má model.    
\end{theorem}

\begin{proof}
Každý model teorie $T$ je zjevně modelem každé její části. Druhou implikaci dokážeme nepřímým důkazem: Předpokládejme, že $T$ nemá model, tj. je sporná, a najděme konečnou část $T'\subseteq T$, která je také sporná.

Protože je $T$ sporná, platí $T\proves\bot$ (zde potřebujeme Větu o úplnosti). Podle Důsledku \ref{corollary:finiteness-of-proofs} potom existuje \emph{konečný} tablo důkaz $\tau$ výroku $\bot$ z $T$. Konstrukce tohoto důkazu má jen konečně mnoho kroků, použili jsme tedy jen konečně mnoho axiomů z $T$. Definujeme-li $T'=\{\alpha\in T\mid \mathrm{T}\alpha\text{ je položka v tablu $\tau$}\}$, potom $\tau$ je také tablo důkaz sporu z teorie $T'$. Teorie $T'$ je tedy sporná konečná část $T$.
\end{proof}

\subsection{Aplikace kompaktnosti}

Následující jednoduchou aplikaci Věty o kompaktnosti můžete chápat jako šablonu, kterou následuje i mnoho dalších, složitějších aplikací této věty. 

\begin{corollary}\label{corollary:infinite-bipartite-compactness}
Spočetně nekonečný graf je bipartitní, právě když je každý jeho konečný podgraf bipartitní.    
\end{corollary}

\begin{proof}
    Každý podgraf bipartitního grafu je zjevně také bipartitní. Ukažme opačnou implikaci. Graf je bipartitní, právě když je obarvitelný 2 barvami. Označme barvy $0,1$.

    Sestrojíme výrokovou teorii $T$ v jazyce $\mathbb P=\{p_v\mid v\in V(G)\}$, kde hodnota výrokové proměnné $p_v$ reprezentuje barvu vrcholu $v$.
    $$  
        T=\{p_u\limplies\neg p_v\mid \{u,v\}\in E(G)\}
    $$
    Zřejmě platí, že $G$ je bipartitní, právě když $T$ má model. Podle Věty o kompaktnosti stačí ukázat, že každá konečná část $T$ má model. Vezměme tedy konečnou $T'\subseteq T$. Buď $G'$ podgraf $G$ indukovaný na množině vrcholů, o kterých se zmiňuje teorie $T'$, tj. $V(G')=\{v\in V(G)\mid p_v\in\Var(T')\}$. Protože je $T'$ konečná, je $G'$ také konečný, a podle předpokladu je 2-obarvitelný. Libovolné 2-obarvení $V(G')$ ale určuje model teorie $T'$.
\end{proof}

Základem této techniky je popis požadované vlastnosti nekonečného objektu pomocí (nekonečné) výrokové teorie. Dále si všimněte, jak z konečné části teorie sestrojíme konečný podobjekt mající danou vlastnost (v našem případě konečný podgraf, který je bipartitní).

\begin{exercise}
    Zobecněte Důsledek \ref{corollary:infinite-bipartite-compactness} pro více barev, tj. ukažte, že spočetně nekonečný graf je $k$-obarvitelný, právě když je každý jeho konečný podgraf $k$-obarvitelný. (Viz Sekce \ref{subsection:example-graph-coloring}.)
\end{exercise}

\begin{exercise}
    Ukažte, že každé částečné uspořádání na spočetné množině lze rozšířit na lineární uspořádání.
\end{exercise}

\begin{exercise}
    Vyslovte a dokažte `spočetně nekonečnou' analogii Hallovy věty.
\end{exercise}

\section{(draft) Hilbertovský kalkulus}\todo
Na závěr kapitoly o tablo metodě si pro srovnání ukážeme jiný dokazovací systém, tzv. \emph{Hilbertovský deduktivní systém} neboli \emph{Hilbertovský kalkulus}. Jde o nejstarší dokazovací systém, modelovaný podle matematických důkazů. Jak uvidíme na příkladě, dokazování je v něm poměrně pracné, hodí se tedy spíše pro teoretické účely. Jde také o korektní a úplný dokazovací systém (to ale necháme bez důkazu).

% from slides:

\begin{itemize}
    \item základní logické spojky: $\neg$, $\to$ (ostatní z nich odvozené)
    
    \item \mdef{logické axiomy} (\myblue{schémata} logických axiomů):
    \vspace{-2mm}\begin{align*}(i)& &\varphi &\to (\psi \to \varphi) \\
    (ii)& &(\varphi\to (\psi \to \chi))&\to ((\varphi \to \psi)\to(\varphi \to \chi))\qquad\qquad\qquad\qquad\phantom{\ } \\
    (iii)& &(\neg \varphi \to \neg \psi)&\to(\psi \to \varphi)
    \end{align*}
    
    \vspace{-2mm}
    kde $\varphi$, $\psi$, $\chi$ jsou libovolné formule (daného jazyka).
    \item \mdef{odvozovací pravidlo}:
    \vspace{-3mm}
    $$\frac{\varphi,\ \varphi \to \psi}{\psi}\qquad\text{\myblue{(modus ponens)}}$$
    \end{itemize}
    
    \vspace{-1mm}
    \mdef{Důkaz} (\emph{Hilbertova stylu}) formule $\varphi$ v teorii $T$ je \myblue{konečná} posloupnost
    \smallskip
    
    $\varphi_0, \dots, \varphi_n=\varphi$ formulí taková, že pro každé $i\le n$
    \begin{itemize}
    \item $\varphi_i$ je logický axiom nebo $\varphi_i \in T$ (axiom teorie), nebo
    \item $\varphi_i$ lze odvodit z předchozích formulí pomocí odvozovacího pravidla.
    \end{itemize}
    
    %$\varphi$ je \mdef{dokazatelná} v $T$, má-li důkaz z $T$. Značíme $T \vdash_{H} \varphi$, popř. $\vdash_H \varphi$ pro $T=\emptyset$.
    \medskip
    
    {\it \myblue{Poznámka}\ \ Volba axiomů a odvozovacích pravidel se v může v různých dokazovacích systémech Hilbertova stylu lišit.}
    
    \subsubsection*{Příklad a korektnost}
    Formule $\varphi$ je \mdef{dokazatelná} v $T$, má-li důkaz z $T$, značíme $T \vdash_{H} \varphi$.
    \smallskip
    
    Je-li $T=\emptyset$, značíme $\vdash_H \varphi$. Např. pro \mygreen{$T=\{\neg \varphi\}$} je \mygreen{$T \vdash_H \varphi \to \psi$} pro každé $\psi$.
    \mygreen{\vspace{-2mm}\begin{align*}
    1)& &\neg \varphi& & &\text{axiom z $T$} \\
    2)& &\neg \varphi& \to (\neg \psi \to \neg \varphi)& &\text{logický axiom $(i)$}\\
    3)& &\neg \psi &\to \neg \varphi& &\text{modus ponens z 1), 2)}\\
    4)& &(\neg \psi \to \neg \varphi)&\to(\varphi \to \psi)& &\text{logický axiom $(iii)$}\\
    5)& &\varphi &\to \psi&  &\text{modus ponens z 3), 4)}
    \end{align*}}
    
    \vspace{-4mm}
    {\bf \myblue{Věta}}\ \ {\it  Pro každou teorii $T$ a formuli $\varphi$,\ \ $T\vdash_H \varphi\ \Rightarrow\ T\models \varphi$.}
    \smallskip
    
    {\it \myblue{Důkaz}}
    \begin{itemize}
    \item Je-li $\varphi\in T$ nebo logický axiom, je $T \models \varphi$ (logické axiomy jsou tautologie),
    \item jestliže $T \models \varphi$ a $T \models \varphi \to \psi$, pak $T \models \psi$, tj. modus ponens je \myblue{korektní},
    \item tedy každá formule vyskytující se v důkazu z $T$ platí v $T$. \qed
\end{itemize}
    \medskip
    
    {\it \myblue{Poznámka}\ \ Platí i \myblue{úplnost}, tj. $T\models \varphi \Rightarrow T\vdash_H \varphi$ pro každou teorii $T$ a formuli $\varphi$.}

% :from slides


% \subsection*{Hilbert's calculus}
%     The \emph{Hilbert's propositional calculus} is a proof system for propositional logic where 
%     \begin{itemize}
%         \item we only use the logical connectives $\neg,\to$
%         \item we have the following (schemes of) \emph{logical axioms}:
%         \begin{enumerate}[(i)]
%             \item $\varphi \to (\psi \to \varphi)$
%             \item $(\varphi\to (\psi \to \chi)) \to ((\varphi \to \psi)\to(\varphi \to \chi))$
%             \item $(\neg \varphi \to \neg \psi)\to(\psi \to \varphi)$
%         \end{enumerate}
%         \item and the following \emph{rule of inference}:
%         $$\frac{\varphi,\ \varphi \to \psi}{\psi}$$
%         i.e. ``from $\varphi$ and $\varphi\to\psi$ infer $\psi$'' (called ``modus ponens'')
%     \end{itemize}
%     In Hilbert's calculus, a \emph{proof} of a proposition $\varphi$ from a theory $T$ is a finite sequence $\varphi_0,\dots,\varphi_n=\varphi$ of formulas such that for every $i\leq n$,
%     \begin{itemize}
%     \item $\varphi_i$ is a logical axiom, or 
%     \item $\varphi_i \in T$ (an axiom of the theory), or
%     \item $\varphi_i$ can be inferred from a pair of preceding propositions $\varphi_j$, $\varphi_k$ ($j<i,k<i$) by applying the rule of inference.
%     \end{itemize}
%     If such a proof exists, we write $T\ \vdash_H\ \varphi$.