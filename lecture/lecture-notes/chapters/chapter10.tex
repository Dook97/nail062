\chapter{(draft) Nerozhodnutelnost a neúplnost}

V této, závěrečné kapitole se budeme zabývat tím, jak lze s teoriemi pracovat algoritmicky. Zlatým hřebem budou \emph{Gödelovy věty o neúplnosti} z roku 1930, které ukazují limity formálního přístupu, a které zastavily desetiletí trvající program formalizace matematiky. Nemáme zde dostatek prostoru k uvedení formálních definic a úplných důkazů, proto se místy budeme pohybovat na poněkud intuitivní úrovni. Zaměříme se na pochopení smyslu tvrzení a myšlenek důkazů.

Pojem \emph{algoritmu} budeme chápat také jen intuitivně. Pokud bychom ho chtěli formalizovat, potom nejběžnější (ale zdaleka ne jedinou) volbou je koncept \emph{Turingova stroje}.\footnote{Viz přednáška NTIN090 Základy složitosti a vyčíslitelnosti.}

\section{Rekurzivní axiomatizace a rozhodnutelnost}

V důkazových systémech, kterými jsme se zabývali (tablo metoda, rezoluce, hilbertův kalkulus) jsme povolili, aby teorie $T$, ve které dokazujeme, byla nekonečná. Vůbec jsem se ale zatím nezabývali tím, jak je zadaná. Pokud chceme ověřit, že je daný objekt (tablo, rezoluční strom, posloupnost formulí) korektním důkazem, potřebujeme nějaký algoritmický přístup ke všem axiomům $T$. 

Jednou z možností by bylo požadovat \emph{enumerátor} $T$, tj. algoritmus, který vypisuje na výstup axiomy z $T$, a každý axiom někdy vypíše.\footnote{Nutným předpokladem je, aby $T$ byla spočetná. K tomu stačí předpokládat, že jazyk je spočetný.} Potom by bylo snadné potvrdit, že je daný důkaz korektní. Pokud bychom ale dostali důkaz, který použil chybný axiom, který v $T$ není, nikdy bychom se to nedozvěděli: nekonečně dlouho bychom čekali, zda jej enumerátor přeci jen nevypíše. Požadujeme proto silnější vlastnost, která umožňuje rozpoznat i chybné důkazy \emph{rekurzivní axiomatizaci}:\footnote{Slovo \emph{rekurzivní} zde neznamená běžně známou rekurzi, ale odkazuje na formalizaci algoritmu pomocí `rekurzivních funkcí' od Gödela. Rekurzivní funkce zde znamená totéž, co vyčíslitelná nějakým Turingovým strojem, a teorii vyčíslitelnosti (\emph{computability theory}) se někdy také říká \emph{recursion theory}.}

\begin{definition}[Rekurzivní axiomatizace]
    Teorie $T$ je \emph{rekurzivně axiomatizovaná}, pokud existuje algoritmus, který pro každou vstupní formuli $\varphi$ doběhne a odpoví, zda $\varphi\in T$.
\end{definition}

\begin{remark}
    Ve skutečnosti by nám stačil enumerátor pro $T$, pokud by bylo garantováno, že vypisuje axiomy v lexikografickém uspořádání. To už je ekvivalentní rekurzivní axiomatizaci. (Rozmyslete si proč.)
\end{remark}

Zaměříme se na otázku, zda můžeme v dané teorii $T$ `algoritmicky rozhodovat pravdu' (tj. platnost vstupní formule). Pokud ano, říkáme, že je teorie \emph{rozhodnutelná}. To je ale poměrně silná vlastnost, definujeme proto také \emph{částečnou rozhodnutelnost}, která znamená, že pokud formule platí, algoritmus nám to řekne, ale pokud neplatí, nikdy se nemusíme dočkat odpovědi.

\begin{definition}[Rozhodnutelnost]
O teorii $T$ říkáme, že je
\begin{itemize}
    \item \emph{rozhodnutelná}, pokud existuje algoritmus, který pro každou vstupní formuli $\varphi$ doběhne a odpoví, zda $T\models\varphi$,
    \item \emph{částečně rozhodnutelná}, pokud existuje algoritmus, který pro každou vstupní formuli:
    \begin{itemize}
        \item pokud $T\models\varphi$, doběhne a odpoví `ano',
        \item pokud $T\not\models\varphi$, buď nedoběhne, nebo doběhne a odpoví `ne'.
    \end{itemize}
\end{itemize}
\end{definition}
Můžeme jako obvykle předpokládat, že $\varphi$ v definici je sentence. Ukážeme si jednoduché tvrzení:

\begin{proposition}
    Nechť $T$ je rekurzivně axiomatizovaná. Potom:
    \begin{enumerate}[(i)]
        \item $T$ je částečně rozhodnutelná,
        \item je-li $T$ navíc kompletní, potom je rozhodnutelná.
    \end{enumerate}
\end{proposition}
\begin{proof}
Algoritmem ukazujícím částečnou rozhodnutelnost je konstrukce systematického tabla pro $\F\varphi$.\footnote{Zde nám stačí enumerátor axiomů $T$, nebo postupně generujeme všechny sentence (např. v lexikografickém pořadí) a pro každou testujeme, zda je axiomem.} Pokud $\varphi$ v $T$ platí, konstrukce skončí v konečně mnoha krocích a snadno ověříme, že je tablo sporné, jinak ale skončit nemusí.

Je-li $T$ kompletní, víme, že $T\proves\varphi$ právě když $T\not\proves\varphi$. Budeme tedy paralelně konstruovat tablo pro $\F\varphi$ a tablo pro $\T\varphi$ (důkaz a zamítnutí $\varphi$ z $T$): jedna z konstrukcí po konečně mnoha krocích skončí.
\end{proof}


\subsection{Rekurzivně spočetná kompletace}

Požadavek kompletnosti je příliš silný, ukážeme, že stačí pokud jsme schopni efektivně popsat všechny kompletní jednoduché extenze\footnote{Tj. `všechny modely až na elementární ekvivalenci'.}

\todo

\begin{proposition}\label{propositon:efficient-complete-simple}    
    Pokud lze \emph{efektivně (algoritmicky) popsat} všechny kompletní jednoduché extenze\footnote{Představte si algoritmus, který pro daná vstupní $i,j$ odpoví $j$-tý axiom $i$-té kompletní jednoduché extenze (v nějakém pevném očíslování); takový algoritmus ne vždy existuje!} \emph{efektivně dané} teorie $T$,\footnote{$T$ může být nekonečná, ale musí existovat algoritmus, který generuje všechny axiomy $T$.} potom je $T$ \emph{(algoritmicky) rozhodnutelná}.
\end{proposition}
\begin{proof}
Pro danou sentenci $\varphi$ buď $T\proves\varphi$, nebo existuje protipříklad $\A\not\models\varphi$, tedy kompletní jednoduchá extenze $T_i$ teorie $T$ taková, že $T_i\not\proves\varphi$. Z kompletnosti ale plyne, že $T_i\proves\neg\varphi$. Náš algoritmus bude paralelně konstruovat tablo důkaz $\varphi$ z $T$ a tablo důkaz $\neg\varphi$ ze všech kompletních jednoduchých extenzí $T_1,T_2,\dots$ teorie $T$.\footnote{Nevadí, že je jich nekonečně mnoho, můžeme využít tzv. \emph{dovetailing}: Provedeme 1. krok konstrukce 1. tabla, potom 2. krok 1. tabla a 1. krok 2. tabla, 3. krok 1. tabla, 2. krok 2. tabla, 1. krok 3. tabla, atd.} Víme, že alespoň jedno z paralelně konstruovaných tabel je sporné, a můžeme předpokládat, že konečné (neprodlužujeme-li sporné větve tabla), tedy algoritmus ho po konečně mnoha krocích zkonstruuje.
\end{proof}

\begin{corollary}
Následující teorie mají rekurzivně spočetné, jsou tedy rozhodnutelné:
\begin{itemize}    
    \item teorie hustých lineárních uspořádání DeLO* (kompletní jednoduché extenze jsou popsané v Důsledku \ref{corollary:complete-simple-extensions-of-delo})
\end{itemize}
\end{corollary}


    
    %%%%%%%%%%%%%%%%%%%%%%%%%%%%%%%%%%%%%%%%%%%%%%%%%%%%%%
    %%%%%%%%%%%%%%%%%%%%%%%%%%%%%%%%%%%%%%%%%%%%%%%%%%%%%%5
    
    \subsubsection*{Rekurzivně spočetná kompletace}
    {\it Co když efektivně popíšeme všechny jednoduché kompletní extenze?}
    \medskip
    
    \smallskip
    
    Řekneme, že množina všech (až na ekvivalenci) \myblue{jednoduchých kompletních}
    \smallskip
    
    \myblue{extenzí} teorie $T$ je \mdef{rekurzivně spočetná}, existuje-li algoritmus $\alpha(i,j)$, který
    \smallskip
    
    generuje $i$-tý axiom $j$-té extenze (při nějakém očíslování), případně oznámí,
    \smallskip
    
    že (takový axiom či extenze) neexistuje.
    \medskip
    
    \smallskip
    
    {\bf \myblue{Tvrzení}}\ \ {\it Je-li teorie $T$ rekurzivně axiomatizovaná a množina všech
    \smallskip
    
    (až na ekvivalenci) jejích jednoduchých kompletních extenzí je rekurzivně
    \smallskip
    
    spočetná, je $T$ \myblue{rozhodnutelná}.}
    \medskip
    
    \smallskip
    
    {\it \myblue{Důkaz}}\ \ Díky rek. axiomatizaci poskytuje konstrukce systematického tabla z $T$
    \smallskip
    
    s $F\varphi$ v kořeni algoritmus pro rozpoznání $T\vdash \varphi$. Pokud ale $T \not \vdash \varphi$, pak $T'\vdash \neg\varphi$
    \smallskip
    
    v nějaké jednoduché kompletní extenzi $T'$ teorie $T$. To lze rozpoznat \myblue{paralelní}
    \smallskip
    
    \myblue{postupnou} konstrukcí systematických tabel pro $T\varphi$ z jednotlivých extenzí.
    \smallskip
    
    V $i$-tém stupni se sestrojí tabla do $i$ kroků pro prvních $i$ extenzí. $\qed$
    
% :from slides

\subsection{Rozhodnutelné teorie}\todo

% from slides:
\subsubsection*{Příklady rozhodnutelných teorií}
Následující teorie jsou rozhodnutelné, ačkoliv jsou nekompletní.
\smallskip

\begin{itemize}
\item teorie \myblue{čisté rovnosti}; bez axiomů v jazyce $L=\langle \rangle$ s rovností,
\smallskip

\item teorie \myblue{unárního predikátu}; bez axiomů v jazyce $L=\langle U \rangle$ s rovností,
\smallskip

kde $U$ je unární relační symbol,
\smallskip

\item teorie \myblue{hustých lineárních uspořádání} $DeLO^*$,
\smallskip

\item teorie \myblue{algebraicky uzavřených těles} v jazyce $L=\langle +,-,\cdot,0,1\rangle$ s rovností,
\smallskip

s axiomy teorie těles a navíc axiomy pro každé $n\ge 1$,

\mygreen{$$(\forall x_{n-1})\dots(\forall x_0)(\exists y)(y^n+x_{n-1}\cdot y^{n-1}+\dots+x_1\cdot y + x_0 = 0),$$}


kde $y^k$ je zkratka za term $y\cdot y \cdot\ \dotsb\ \cdot y$ ( $\cdot$ aplikováno ($k-1$)-krát).
\smallskip

\item teorie \myblue{komutativních grup},
\smallskip

\item teorie \myblue{Booleových algeber}.
\end{itemize}

% :from slides

\subsection{Rekurzivní axiomatizovatelnost}\todo

% from slides:
\subsubsection*{Rekurzivní axiomatizovatelnost}
    {\it Dají se matematické struktury ``efektivně'' popsat?}
    \smallskip
    
    \begin{itemize}
    \item Třída $K\subseteq M(L)$ je \mdef{rekurzivně axiomatizovatelná}, pokud existuje
    \smallskip
    
    rekurzivně axiomatizovaná teorie $T$ jazyka $L$ s $M(T)=K$.
    \smallskip
    
    \item \myblue{Teorie} $T$ je \myblue{rekurzivně axiomatizovatelná}, pokud $M(T)$ je
    rekurzivně
    \smallskip
    
    axiomatizovatelná.
    \end{itemize}
    %\smallskip
    
    %{\it \myblue{Poznámka}\ \ Obdobně lze zadefinovat r. s. axiomatizovatelnost.}
    \medskip
    
    \smallskip
    
    {\bf \myblue{Tvrzení}}\ \ {\it Pro každou \myblue{konečnou} strukturu $\mathcal{A}$ v konečném jazyce s rovností
    \smallskip
    
    je $\mathrm{Th}(\mathcal{A})$ rekurzivně axiomatizovatelná. Tedy, $\mathrm{Th}(\mathcal{A})$ je \myblue{rozhodnutelná}.}
    \medskip
    
    \smallskip
    
    {\it \myblue{Důkaz}}\ \ Nechť $A=\{a_1,\dots,a_n\}$. Teorii $\mathrm{Th}(\mathcal{A})$ axiomatizujeme jednou sentencí
    \smallskip
    
    (tedy rekurzivně) kompletně popisující $\mathcal{A}$. Bude tvaru \emph{``existuje právě $n$ prvků   
    $a_1,\dots,a_n$ splňujících právě ty \myblue{základní vztahy} o funkčních hodnotách a relacích, které platí ve struktuře $\mathcal{A}$.''} $\qed$
    
    %%%%%%%%%%%%%%%%%%%%%%%%%%%%%%%%%%%%%%%%%%%%%%%%%%%%%%5
    
    \subsubsection*{Příklady rekurzivní axiomatizovatelnosti}
    Následující struktury $\mathcal{A}$ mají \myblue{rekurzivně} axiomatizovatelnou teorii $\mathrm{Th}(\mathcal{A})$.
    \medskip
    
    \begin{itemize}
    \item \mygreen{$\langle \mathbb{Z},\le \rangle$}, teorií \myblue{diskrétních lineárních uspořádání},
    \smallskip
    
    \item \mygreen{$\langle \mathbb{Q},\le \rangle$}, teorií \myblue{hustých lineárních uspořádání bez konců} ($DeLO$),
    \smallskip
    
    \item \mygreen{$\langle \mathbb{N},S,0 \rangle$}, teorií \myblue{následníka s nulou},
    \smallskip
    
    \item \mygreen{$\langle \mathbb{N},S,+,0\rangle$}, tzv. \myblue{Presburgerovou aritmetikou},
    \smallskip
    
    \item \mygreen{$\langle \mathbb{R},+,-,\cdot,0,1 \rangle$}, teorií \myblue{reálně uzavřených těles},
    \smallskip
    
    \item \mygreen{$\langle \mathbb{C},+,-,\cdot,0,1 \rangle$}, teorií \myblue{algebraicky uzavřených těles charakteristiky 0}.
    \end{itemize}
    \medskip
    
    {\bf \myblue{Důsledek}}\ \ {\it Pro uvedené struktury je $\mathrm{Th}(\mathcal{A})$ \myblue{rozhodnutelná}.}
    \medskip
    
    \smallskip
    
    {\it \myblue{Poznámka}\ \ Uvidíme, že ale \mygreen{$\underline{\mathbb{N}}=\langle \mathbb{N},S,+,\cdot,0,\le \rangle$} rekurzivně axiomatizovat
    \smallskip
    
    \mdef{nelze}. (Vyplývá to z první Gödelovy věty o neúplnosti).}
    
% :from slides



\section{Aritmetika}\todo

% from slides:

% :from slides

\subsection{Robinsonova a Peanova aritmetika}\todo

% from slides:
\subsubsection*{Robinsonova aritmetika}
    {\it Jak \myblue{efektivně} a přitom co nejúplněji axiomatizovat $\underline{\mathbb{N}}=\langle \mathbb{N},S,+,\cdot,0,\le\rangle$?}
    \medskip
    
    Jazyk aritmetiky je $L=\langle S,+,\cdot,0,\le \rangle$ s rovností.
    \medskip
    
    \mdef{Robinsonova aritmetika} $Q$ má axiomy (konečně mnoho)

    \mygreen{\begin{align*}
    &S(x)\ne 0& &x\cdot 0=0\\
    &S(x)=S(y)\rightarrow x=y& &x\cdot S(y)=x\cdot y+x\\
    &x+0=x& &x\ne 0 \rightarrow (\exists y)(x=S(y))\\
    &x+S(y)=S(x+y)& &x\le y \leftrightarrow (\exists z)(z+x=y)\qquad
    \end{align*}}
    

    {\it \myblue{Poznámka}\ \ $Q$ je velmi slabá, např. nedokazuje komutativitu či asociativitu
    \smallskip
    
    operací $+$, $\cdot$ ani tranzitivitu $\le$. Nicméně postačuje například k důkazu
    \smallskip
    
    \myblue{existenčních} tvrzení o numerálech, která jsou pravdivá v $\underline{\mathbb{N}}$.}
    \medskip
    
    \mygreen{\it Např. pro $\varphi(x,y)$ tvaru $(\exists z)(x+z=y)$ je
    
    $$Q\vdash \varphi(\underline{1},\underline{2}),\quad\text{kde $\underline{1}=S(0)$ a\ \ $\underline{2}=S(S(0))$}.$$}
    
    
    
    
    %%%%%%%%%%%%%%%%%%%%%%%%%%%%%%%%%%%%%%%%%%%%%%%%%%%%%%5
    
    \subsubsection*{Peanova aritmetika}
    
    \mdef{Peanova aritmetika} $PA$ má axiomy
    
    \begin{enumerate}
    \item[$(a)$] Robinsonovy aritmetiky $Q$,
    \smallskip
    
    \item[$(b)$] \myblue{schéma indukce}, tj. pro každou formuli $\varphi(x,\overline{y})$ jazyka $L$ axiom
    
    \mygreen{$$(\varphi(0,\overline{y}) \mand (\forall x)(\varphi(x,\overline{y})\to \varphi(S(x),\overline{y}))) \to (\forall x)\varphi(x,\overline{y}).\quad$$}
    
    
    \end{enumerate}
    
    {\it \myblue{Poznámka}\ \ $PA$ je poměrně dobrou aproximací $\mathrm{Th}(\underline{\mathbb{N}})$, dokazuje všechny
    \smallskip
    
    základní vlastnosti platné v $\underline{\mathbb{N}}$ (např. komutativitu $+$). Na druhou stranu
    \smallskip
    
    existují sentence pravdivé v $\underline{\mathbb{N}}$ ale nezávislé v $PA$.}
    \bigskip
    
    {\it \myblue{Poznámka}\ \ V jazyce \myblue{2. řádu} lze axiomatizovat $\underline{\mathbb{N}}$ (až na izomorfismus),
    \smallskip
    
    vezmeme-li místo schéma indukce přímo axiom indukce (2. řádu)}
    
    \mygreen{$$(\forall X)\ ((X(0) \mand (\forall x)(X(x) \to X(S(x)))) \to (\forall x)\ X(x)).$$}
    
    
    
% :from slides

\subsection{Hilbertův desátý problém}\todo

% from slides:
\subsubsection*{Hilbertův 10. problém}
    \begin{itemize}
    \item Nechť $p(x_1,\dots,x_n)$ je polynom s celočíselnými koeficienty.
    \smallskip
    
    Má \mdef{Diofantická rovnice} \mygreen{$p(x_1,\dots,x_n)=0$} \myblue{celočíselné} řešení?
    \smallskip
    
    \item \myblue{Hilbert (1900)}\ \ {\it ``Nalezněte algoritmus, který po konečně mnoha krocích
    \smallskip
    
    určí, zda daná Diofantická rovnice s libovolným počtem proměnných a
    \smallskip
    
    celočíselnými koeficienty má \myblue{celočíselné} řešení.''}
    \end{itemize}
    \smallskip
    
    {\it \myblue{Poznámka}\ \ Ekvivalentně lze požadovat algoritmus rozhodující, zda existuje
    \smallskip
    
    řešení v \myblue{přirozených} číslech.}
    \medskip
    
    \smallskip
    
     \myblue{{\bf Věta} (DPRM, 1970)}\ \ {\it Problém existence celočíselného řešení dané
    \smallskip
    
    Diofantické rovnice s celočíselnými koeficienty je alg. \myblue{nerozhodnutelný}.}
    \medskip
    
    \smallskip
    
    {\bf \myblue{Důsledek}}\ \ {\it Neexistuje algoritmus rozhodující pro dané polynomy
    \smallskip
    
    $p(x_1,\dots,x_n)$, $q(x_1,\dots,x_n)$ s \myblue{přirozenými} koeficienty, zda}
    
    \mygreen{$$\underline{\mathbb{N}}\models (\exists x_1)\dots(\exists x_n)(p(x_1,\dots,x_n)=q(x_1,\dots,x_n)).$$}
    
    
    
% :from slides

\section{Nerozhodnutelnost predikátové logiky}\todo

% from slides:
\subsubsection*{Nerozhodutelnost predikátové logiky}
{\it Existuje algoritmus, rozhodující o dané sentenci, zda je \myblue{logicky} pravdivá?} %(Tedy ne v dané teorii, nebo v dané struktuře.)}
\smallskip

\begin{itemize}
\item Víme, že \myblue{Robinsonova aritmetika} $Q$ má konečně axiomů, má za model $\underline{\mathbb{N}}$
\smallskip

a stačí k důkazu \myblue{existenčních} tvrzení o numerálech, která platí v $\underline{\mathbb{N}}$.
\smallskip

\item Přesněji, pro každou existenční formuli $\varphi(x_1,\dots,x_n)$ jazyka aritmetiky

\mygreen{$$Q\vdash \varphi(x_1/\underline{a_1},\dots,x_n/\underline{a_n})\ \ \Leftrightarrow\ \ \underline{\mathbb{N}}\models \varphi[e(x_1/a_1,\dots,x_n/a_n)]$$}


pro každé $a_1,\dots,a_n \in \mathbb{N}$, kde $\underline{a_i}$ značí $a_i$-tý numerál.
\smallskip

\item Speciálně, pro $\varphi$ tvaru \mygreen{$(\exists x_1)\dots(\exists x_n)(p(x_1,\dots,x_n)=q(x_1,\dots,x_n))$,}
\smallskip

kde $p$, $q$ jsou polynomy s přirozenými koeficienty (numerály), platí

\mygreen{$$\underline{\mathbb{N}}\models \varphi\quad\Leftrightarrow\quad Q\vdash \varphi\quad\Leftrightarrow\quad \vdash \psi\to \varphi\quad \Leftrightarrow\quad \models \psi\to \varphi,$$}


kde $\psi$ je konjunkce (uzávěrů) všech axiomů $Q$.
\smallskip

\item Tedy, pokud by existoval algoritmus rozhodující \myblue{logickou pravdivost},
\smallskip

existoval by i algoritmus rozhodující, zda \mygreen{$\underline{\mathbb{N}}\models \varphi$}, což není možné.


\end{itemize}


% :from slides

\section{Gödelovy věty}\todo

% from slides:

% :from slides

\subsection{První věta o neúplnosti}\todo

% from slides:
\subsubsection*{Gödelova 1. věta o neúplnosti}
    \myblue{{\bf Věta} (Gödel)}\ \ {\it Pro každou bezespornou rekurzivně axiomatizovanou extenzi $T$

    
    Robinsonovy aritmetiky existuje sentence \myblue{pravdivá} v $\underline{\mathbb{N}}$ a \myblue{nedokazatelná} v $T$.}
    \medskip
    
    \smallskip
    
    {\it \myblue{Poznámky}
    
    \begin{itemize}
    \item ``Rekurzivně axiomatizovaná'' znamená, že je \emph{``efektivně zadaná''}.
    \smallskip
    
    \item ``Extenze R. aritmetiky'' znamená, že je \emph{``\myblue{základní aritmetické síly}''}.
    \smallskip
    
    \item Je-li navíc $\underline{\mathbb{N}}\models T$, je teorie $T$ \myblue{nekompletní}.
    \smallskip
    
    \item V důkazu sestrojená sentence vyjadřuje \emph{``\myblue{nejsem dokazatelná v $T$}''}.
    \smallskip
    
    \item Důkaz je založen na dvou principech:
    \medskip
    
    \myblue{$(a)$ aritmetizaci syntaxe},
    \medskip
    
    \myblue{$(b)$ self-referenci}.
    \end{itemize}}
    
    

    \subsection*{Aritmetizace dokazatelnosti}
    \subsubsection*{Aritmetizace - predikát dokazatelnosti}
    \begin{itemize}
    \item \myblue{Konečné objekty} syntaxe (symboly jazyka, termy, formule, konečná tabla,
    \smallskip
    
    tablo důkazy) lze vhodně \myblue{zakódovat} přirozenými čísly.
    \smallskip
    
    \item Nechť \mdef{$\lceil \varphi \rceil$} značí kód formule $\varphi$ a nechť \mdef{$\underline{\varphi}$} značí \myblue{numerál} (term jazyka
    \smallskip
    
    aritmetiky) reprezentující $\lceil \varphi \rceil$.
    \smallskip
    
    \item Je-li $T$ rekurzivně axiomatizovaná, je relace $\mathrm{Prf}_T\subseteq \mathbb{N}^2$ \myblue{rekurzivní}.

    \mygreen{$$\mathrm{Prf}_T(x,y)\ \ \Leftrightarrow\ \ \text{\it (tablo) $y$ je důkazem (sentence) $x$ v T.}$$}
    

    \item Je-li $T$ navíc extenze Robinsonovy aritmetiky $Q$, dá se dokázat, že $\mathrm{Prf}_T$ je    
    \myblue{reprezentovatelná} nějakou formulí \mdef{$Prf_T(x,y)$} tak, že pro každé $x,y\in \mathbb{N}$

    \mygreen{\begin{align*}
    Q&\vdash Prf_T(\underline{x},\underline{y}),\ \ \ \ \ \!\text{\it je-li}\ \ \ \ \mathrm{Prf}_T(x,y),\\
    Q&\vdash \neg Prf_T(\underline{x},\underline{y}),\ \ \text{\it jinak}.
    \end{align*}}
    
    \item $Prf_T(x,y)$ vyjadřuje {\it ``\myblue{$y$ je důkaz $x$ v $T$}''}.
    \smallskip
    
    \item \mdef{$(\exists y)Prf_T(x,y)$} vyjadřuje {\it ``\myblue{$x$ je dokazatelná v $T$}''}.
    \smallskip
    
    \item Je-li \mygreen{$T\vdash \varphi$}, pak \mygreen{$\underline{\mathbb{N}}\models (\exists y)Prf_T(\underline{\varphi},y)$} a navíc \mygreen{$T\vdash (\exists y)Prf_T(\underline{\varphi},y)$}.
    \end{itemize}
    
    
    %%%%%%%%%%%%%%%%%%%%%%%%%%%%%%%%%%%%%%%%%%%%%%%%%%%%%%5
    
    \subsection*{Self-reference}
    \subsubsection*{Princip self-reference}
    
    \begin{itemize}
    \item \mygreen{\it Tato věta má 16 písmen.}
    \medskip
    
    \myblue{Self-reference} ve formálních systémech většinou není přímo k dispozici.
    \medskip
    
    \item \mygreen{\it Následující věta má 24 písmen ``Následující věta má 24 písmen''.}
    \medskip
    
    \myblue{Přímá reference} obvykle je k dispozici, stačí, když umíme ``mluvit''
    \smallskip
    
    o posloupnostech symbolů. Uvedená věta ale není self-referenční.
    \medskip
    
    \item \mygreen{\it Následující věta zapsaná jednou a ještě jednou v uvozovkách má 116}
    \smallskip
    
    \mygreen{\it písmen ``Následující věta zapsaná jednou a ještě jednou v uvozovkách}
    \smallskip
    
    \mygreen{\it má 116 písmen''.}
    \medskip
    
    Pomocí přímé reference lze dosáhnout self-reference. Namísto
    \smallskip
    
    \emph{``má $x$ písmen''} může být jiná vlastnost.
    \medskip
    
    \item \texttt{main()\{char *c="main()\{char *c=\%c\%s\%c; printf(c,34,}
    \smallskip
    
    \texttt{c,34);\}"; printf(c,34,c,34);\}}
    \end{itemize}
    
    
    
    %%%%%%%%%%%%%%%%%%%%%%%%%%%%%%%%%%%%%%%%%%%%%%%%%%%%%%5
    
    \subsubsection*{Věta o pevném bodě}
    
    {\bf \myblue{Věta}}\ \ {\it Nechť $T$ je bezesporné rozšíření Robinsonovy aritmetiky. Pro každou
    \smallskip
    
    formuli $\varphi(x)$ jazyka teorie $T$ existuje sentence $\psi$ taková, že \mygreen{$T\vdash \psi \leftrightarrow \varphi(\underline{\psi})$}.}
    \medskip
    
    {\it \myblue{Poznámka}\ \ Sentence $\psi$ je self-referenční, říká \emph{``\myblue{splňuji podmínku $\varphi$}''.}}
    \medskip
    
    \myblue{{\it Důkaz} (idea)}\ \ Uvažme \myblue{\emph{zdvojující}} funkci $d$ takovou, že pro každou formuli $\chi(x)$
    
    \mygreen{$$d(\lceil \chi(x)\rceil)=\lceil\chi(\underline{\chi(x)}) \rceil$$}
    
    
    \begin{itemize}
    \item Platí, že $d$ je \myblue{reprezentovatelná} v $T$. Předpokládejme \emph{(pro jednoduchost)},
    \smallskip
    
    že nějakým termem, který si označme $d$, stejně jako funkci $d$.
    \smallskip
    
    \item Pak pro každou formuli $\chi(x)$ jazyka teorie $T$ platí
    
    \begin{equation}\label{eq:a}\mygreen{T\vdash d(\underline{\chi(x)})=\underline{\chi(\underline{\chi(x)})}}\end{equation}
    
    
    \item Za $\psi$ vezměme sentenci \mdef{$\varphi(d(\underline{\varphi(d(x))}))$}. Stačí ověřit \mygreen{$T \vdash d(\underline{\varphi(d(x))})=\underline{\psi}$}.
    \smallskip
    
    \item To plyne z \eqref{eq:a} pro $\chi(x)$ tvaru $\varphi(d(x))$, neboť v tom případě
    
    \mygreen{$$T\vdash d(\underline{\varphi(d(x))})=\underline{\varphi(d(\underline{\varphi(d(x))}))} \qed$$}
    
    
    \end{itemize}
    
    
    %%%%%%%%%%%%%%%%%%%%%%%%%%%%%%%%%%%%%%%%%%%%%%%%%%%%%%5
    \subsection*{Nedefinovatelnost pravdy}
    \subsubsection*{Nedefinovatelnost pravdy}
    
    Řekneme, že formule $\tau(x)$ \mdef{definuje pravdu} v aritmetické teorii $T$, pokud
    \smallskip
    
    pro každou sentenci $\varphi$ platí \mygreen{$T \vdash \varphi \leftrightarrow \tau(\underline{\varphi})$}.
    \medskip
    
    \smallskip
    
    {\bf \myblue{Věta}}\ \ {\it V žádném bezesporném rozšíření Robinsonovy aritmetiky neexistuje
    \smallskip
    
    definice pravdy.}
    \medskip
    
    \myblue{{\it Důkaz}}\ \ Dle věty o pevném bodě pro $\neg\tau(x)$ existuje sentence $\varphi$ taková, že
    
    \mygreen{$$T\vdash \varphi \leftrightarrow \neg \tau(\underline{\varphi}).$$}
    
    
    Kdyby formule $\tau(x)$ definovala pravdu v $T$, bylo by
    
    \mygreen{$$T\vdash \varphi \leftrightarrow \neg \varphi,$$}
    
    
    což v bezesporné teorii není možné. $\qed$
    \medskip
    
    \smallskip
    
    {\it \myblue{Poznámka}\ \ Důkaz je založen na paradoxu lháře, sentence $\varphi$ by vyjadřovala
    \smallskip
    
    ``\myblue{nejsem pravdivá v $T$}''.}
    
    
    %%%%%%%%%%%%%%%%%%%%%%%%%%%%%%%%%%%%%%%%%%%%%%%%%%%%%%5
    
    \subsection*{1. věta o neúplnosti}
    \subsubsection*{Důkaz 1. věty o neúplnosti}
    \myblue{{\bf Věta} (Gödel)}\ \ {\it Pro každou bezespornou rekurzivně axiomatizovanou extenzi $T$
    
    
    Robinsonovy aritmetiky existuje sentence \myblue{pravdivá} v $\underline{\mathbb{N}}$ a \myblue{nedokazatelná} v $T$.}
    \medskip
    
    \smallskip
    
    \myblue{{\it Důkaz}}\ \ Nechť $\varphi(x)$ je \mygreen{$\neg(\exists y)Prf_T(x,y)$}, vyjadřuje {\it ``$x$ není dokazatelná v $T$''}.
    
    \begin{itemize}
    \item Dle věty o pevném bodě pro $\varphi(x)$ existuje sentence $\psi_T$ taková, že
    
    \begin{equation}\label{eq:g1}
    \mygreen{T\vdash \psi_T \leftrightarrow \neg (\exists y)Prf_T(\underline{\psi_T},y)}.
    \end{equation}
    
    
    $\psi_T$ říká {\it ``\myblue{nejsem dokazatelná v $T$}''}. Přesněji,  $\psi_T$ je ekvivalentní sentenci
    \smallskip
    
    vyjadřující, že $\psi_T$ není dokazatelná v $T$. (Ekvivalence platí v $\underline{\mathbb{N}}$ i v $T$).
    \smallskip
    
    \item Nejprve ukážeme, že {\it $\psi_T$ není dokazatelná v $T$}. Kdyby \mdef{$T \vdash \psi_T$}, tj. $\psi_T$ je
    \smallskip
    
    lživá v $\underline{\mathbb{N}}$, pak \mygreen{$\underline{\mathbb{N}}\models (\exists y)Prf_T(\underline{\psi_T},y)$} a navíc
    \mygreen{$T\vdash (\exists y)Prf_T(\underline{\psi_T},y)$}. Tedy
    \smallskip
    
    z \eqref{eq:g1} plyne \mdef{$T \vdash \neg \psi_T$}, což ale není možné, neboť $T$ je bezesporná.
    \smallskip
    
    \item Zbývá dokázat, že $\psi_T$ je pravdivá v $\underline{\mathbb{N}}$. Kdyby ne, tj. \mygreen{$\underline{\mathbb{N}}\models \neg \psi_T$}, pak
    \smallskip
    
    \mygreen{$\underline{\mathbb{N}}\models(\exists y)Prf_T(\underline{\psi_T},y)$}. Tedy $T \vdash \psi_T$, což jsme již dokázali, že neplatí. $\qed$
    \end{itemize}
    


% :from slides

\subsection{Důsledky první věty}\todo

% from slides:
\subsubsection*{Důsledky a zesílení 1. věty}
    \myblue{{\bf Důsledek}}\ \ {\it Je-li navíc $\underline{\mathbb{N}}\models T$, je teorie $T$ nekompletní.}
    \medskip
    
    \myblue{{\it Důkaz}}\ \ Kdyby byla $T$ kompletní, pak $T\vdash \neg\psi_T$ a tedy $\underline{\mathbb{N}}\models \neg\psi_T$, což je
    \smallskip
    
    ve sporu s $\underline{\mathbb{N}}\models \psi_T$. $\qed$
    \bigskip
    
    \myblue{{\bf Důsledek}}\ \ {\it $\mathrm{Th}(\underline{\mathbb{N}})$ není rekurzivně axiomatizovatelná.}
    \medskip
    
    \myblue{{\it Důkaz}}\ \ $\mathrm{Th}(\underline{\mathbb{N}})$ je bezesporná extenze Robinsonovy aritmetiky a má model $\underline{\mathbb{N}}$.
    \smallskip
    
    Kdyby byla rekurzivně axiomatizovatelná, dle předchozího důsledku by byla
    \smallskip
    
    nekompletní, ale $\mathrm{Th}(\underline{\mathbb{N}})$ je kompletní. $\qed$
    \bigskip
    
    {\it Gödelovu 1. větu o neúplnosti lze následovně zesílit.}
    \medskip
    
    \myblue{{\bf Věta} (Rosser)}\ \ {\it Pro každou bezespornou rekurzivně axiomatizovanou extenzi
    \smallskip
    
    $T$ Robinsonovy aritmetiky existuje \myblue{nezávislá} sentence. Tedy $T$ je nekompletní.}
    
    
    {\it \myblue{Poznámka}\ \ Tedy předpoklad, že $\underline{\mathbb{N}}\models T$, je v prvním důsledku nadbytečný.}
    
% :from slides

\subsection{Druhá věta o neúplnosti}\todo

% from slides:
\subsubsection*{Gödelova 2. věta o neúplnosti}
    
    Označme \mdef{$Con_T$} sentenci \mygreen{$\neg(\exists y)Prf_T(\underline{0=1},y)$}.
    \smallskip
    Platí $\underline{\mathbb{N}}\models Con_T \Leftrightarrow T\not\vdash 0=\underline{1}$. Tedy $Con_T$ vyjadřuje, že {\it ``\myblue{$T$ je bezesporná}''.}
    \medskip
    
    \smallskip
    
    \myblue{{\bf Věta} (Gödel)}\ \ {\it Pro každou bezespornou rekurzivně axiomatizovanou extenzi $T$
    
    \myblue{Peanovy aritmetiky} platí, že $Con_T$ není dokazatelná v $T$.}
    \medskip
    
    \smallskip
    
    \myblue{{\it Důkaz} (náznak)}\ \ Nechť $\psi_T$ je Gödelova sentence {\it ``nejsem dokazatelná v $T$''}.
    \begin{itemize}
    \item V první části důkazu 1. věty o neúplnosti jsme ukázali, že

    \mygreen{
    \begin{equation}\label{eq:g2a}\text{\it ``Je-li $T$ bezesporná, pak $\psi_T$ není dokazatelná v $T$.''}
    \end{equation}}
    

    Jinak vyjádřeno, platí $Con_T\to \psi_T$.

    
    \item Je-li $T$ extenze Peanovy aritmetiky, důkaz tvrzení \eqref{eq:g2a} lze \myblue{formalizovat}

    
    v rámci $T$. Tedy \mygreen{$T \vdash Con_T \to \psi_T$}.

    
    \item Jelikož $T$ je bezesporná dle předpokladu věty, podle \eqref{eq:g2a} je
    \mygreen{$T \not\vdash \psi_T$}.

    
    \item Z předchozích dvou bodů vyplývá, že \mygreen{$T \not\vdash Con_T$}. $\qed$
    \end{itemize}
    \smallskip
    
    {\it \myblue{Poznámka} Taková teorie $T$ tedy neumí dokázat vlastní bezespornost.}
    
    
% :from slides


\subsection{Důsledky druhé věty}\todo

% from slides:
\subsubsection*{Důsledky 2. věty}
    \myblue{{\bf Důsledek}}\ \ {\it Existuje model $\mathcal{A}$ Peanovy aritmetiky t.ž. \mygreen{$\mathcal{A}\models (\exists y)Prf_{PA}(\underline{0=1},y)$}.}
    \medskip
    
    {\it \myblue{Poznámka}\ \ $\mathcal{A}$ musí být nestandardní model $PA$, svědkem musí být
    \smallskip
    
    nestandardní prvek (jiný než hodnoty numerálů).}
    \bigskip
    
    %{\it $Con_T$ nemusí být nezávislá v $T$.}
    %\medskip
    
    \myblue{{\bf Důsledek}}\ \ {\it Existuje bezesporná rekurzivně axiomatizovaná extenze $T$
    \smallskip
    
    Peanovy aritmetiky taková, že $T\vdash \neg Con_T$.}
    \medskip
    
    \smallskip
    
    \myblue{{\it Důkaz}}\ \ Nechť $T=PA \cup \{\neg Con_{PA}\}$. Pak $T$ je bezesporná, neboť $PA \not \vdash Con_{PA}$.
    \smallskip
    
    Navíc $T \vdash \neg Con_{PA}$, tj. $T$ dokazuje spornost $PA\subseteq T$, tedy i $T\vdash \neg Con_T$. $\qed$
    
    \medskip
    {\it \myblue{Poznámka}\ \ $\underline{\mathbb{N}}$ nemůže být modelem teorie $T$.}
    
    \bigskip
    
    \myblue{{\bf Důsledek}}\ \ {\it Je-li teorie množin $ZFC$ bezesporná, není $Con_{ZFC}$ dokazatelná
    \smallskip
    
    v $ZFC$.}
    
% :from slides



