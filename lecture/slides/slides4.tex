\documentclass{beamer}

%% slide-specific

\usetheme[progressbar=frametitle]{metropolis}
%\usecolortheme{spruce}
%\metroset{block=fill}

% define Metropolis colors    
\definecolor{mAlert}{HTML}{EB811B}
\definecolor{mExample}{HTML}{14B03D}
\definecolor{mBlock}{HTML}{23373b}

% my blocks
\setlength\fboxsep{0pt}%

\newcommand{\myblock}[1]{\colorbox{mBlock!8}{\begin{minipage}{\linewidth}#1\end{minipage}}}
\newcommand{\myblockmath}[1]{\colorbox{mBlock!8}{\begin{minipage}{\linewidth}\vspace{-6pt}#1\end{minipage}}}
\newcommand{\myblockinline}[1]{\colorbox{mBlock!8}{#1}}
\newcommand{\myexample}[1]{\colorbox{mExample!8}{\begin{minipage}{\linewidth}#1\end{minipage}}}
\newcommand{\myexamplemath}[1]{\colorbox{mExample!8}{\begin{minipage}{\linewidth}\vspace{-6pt}#1\end{minipage}}}
\newcommand{\myexampleinline}[1]{\colorbox{mExample!8}{#1}}
\newcommand{\myalert}[1]{\colorbox{mAlert!8}{\begin{minipage}{\linewidth}#1\end{minipage}}}
\newcommand{\myalertmath}[1]{\colorbox{mAlert!8}{\begin{minipage}{\linewidth}\vspace{-6pt}#1\end{minipage}}}
\newcommand{\myalertinline}[1]{\colorbox{mAlert!8}{#1}}

%% other
\newcommand{\mystructure}[1]{\mathcal{#1}}




%% packages
\usepackage{amsmath,amssymb,amsthm}
\usepackage{booktabs}
\usepackage[czech]{babel}
\usepackage{enumerate}
\usepackage{forest}
\usepackage{multicol}
% \usepackage{tcolorbox}
\usepackage{tikz}
    \usetikzlibrary{arrows.meta}
%\usepackage[unicode]{hyperref}
\usepackage[utf8x]{inputenc}
\usepackage{xfrac}

% %% theorems
% \theoremstyle{plain}
%     \newtheorem{theorem}{Věta}[section]
%     \newtheorem*{theorem-unnumbered}{Věta}
%     \newtheorem{proposition}[theorem]{Tvrzení}
%     \newtheorem{corollary}[theorem]{Důsledek}
%     \newtheorem{lemma}[theorem]{Lemma}
%     \newtheorem{observation}[theorem]{Pozorování}
% \theoremstyle{definition}
%     \newtheorem{definition}[theorem]{Definice}
%     \newtheorem*{algorithm}{Algoritmus}
% \theoremstyle{remark}
%     \newtheorem{remark}[theorem]{Poznámka}
%     \newtheorem{example}[theorem]{Příklad}
%     \newtheorem{exercise}{Cvičení}[chapter]
%     \newtheorem*{solution}{Řešení}

%% macros and definitions
\DeclareMathOperator{\Aut}{Aut}
\DeclareMathOperator{\Conseq}{Csq}
\DeclareMathOperator{\DeLO}{DeLO}
\DeclareMathOperator{\dom}{dom}
\DeclareMathOperator{\Fm}{Fm}
\DeclareMathOperator{\M}{M}
%\DeclareMathOperator{\Proof}{Proof}
\DeclareMathOperator{\rng}{rng}
\DeclareMathOperator{\Term}{Term}
\DeclareMathOperator{\Th}{Th}
\DeclareMathOperator{\Thm}{Thm}
\DeclareMathOperator{\Tree}{Tree}
\DeclareMathOperator{\Var}{Var}
\DeclareMathOperator{\VF}{VF}

\newcommand{\A}{\structure{A}}
\newcommand{\B}{\structure{B}}
\newcommand{\Con}{\mathit{Con}}
\newcommand{\disjointunion}{\mathbin{\dot{\sqcup}}}
\newcommand{\F}{\ensuremath{\mathrm{F}}}
\newcommand{\landsymb}{{\land}}
\newcommand{\lbin}{\mathbin{\square}}
\newcommand{\lbinsymb}{{\lbin}}
\newcommand{\liff}{\mathbin{\leftrightarrow}}
\newcommand{\liffsymb}{{\liff}}
\newcommand{\limplies}{\mathbin{\rightarrow}}
\newcommand{\limpliessymb}{{\limplies}}
\newcommand{\lorsymb}{{\lor}}
\newcommand{\Prf}{\mathit{Prf}}
\newcommand{\proves}{\vdash}
%\newcommand{\structure}[1]{\mathcal{#1}}
\newcommand{\todo}{[TODO]}
\newcommand{\T}{\ensuremath{\mathrm{T}}}
\newcommand{\union}{\mathbin{\cup}}


\title{Čtvrtá přednáška}
\subtitle{NAIL062 Výroková a predikátová logika}
\author{Jakub Bulín (KTIML MFF UK)}
% \institute{KTIML MFF UK}
\date{Zimní semestr 2023}


\begin{document}


\frame{\titlepage}


\begin{frame}{Čtvrtá přednáška}

    \textbf{Program}
        \begin{itemize}
            \item úvod do tablo metody
            \item tablo důkaz
            \item korektnost a úplnost
        \end{itemize}

    \textbf{Materiály}

        \href{https://github.com/jbulin-mff-uk/nail062/raw/main/lecture/lecture-notes/lecture-notes.pdf}{\alert{\textbf{Zápisky z přednášky}}}, Sekce 4.1-4.6 z Kapitoly 4

\end{frame}


\section{\sc Kapitola 4: Metoda analytického tabla}


\section{4.1 Formální dokazovací systémy}


\begin{frame}{Formální dokazovací systém}

    chceme zjistit, zda výrok platí [\alert{$T\models\varphi$}], a to čistě syntakticky, aniž bychom se zabývali sémantikou: najít \alert{(formální) důkaz} [\alert{$T\proves\varphi$}]

    \alert{důkaz} je konečný syntaktický objekt vycházející z $\varphi$ a axiomů $T$

    dokazování lze dělat \alert{algoritmicky} (pokud máme algoritmický přístup k axiomům $T$, která může být nekonečná), a lze rychle algoritmicky \alert{ověřit}, zda je daný objekt opravdu korektní důkaz

    \begin{itemize}
        \item \alert{korektnost}: ``co dokážu, platí'' \hfill \myalertinline{%
            $T\proves\varphi\ \Rightarrow\ T\models\varphi$}
        \item \alert{úplnost}: ``dokážu vše, co platí'' \hfill \myalertinline{%
            $T\models\varphi\ \Rightarrow\ T\proves\varphi$}
    \end{itemize}
    (korektnost je nutná, úplnost ne: rychlý dokazovací systém může být praktický i když není úplný)

    ukážeme si: \emph{tablo metodu}, \emph{hilbertovský kalkulus}, \emph{rezoluční metodu}
    
    \myblock{nutný předpoklad: \alert{jazyk musí být spočetný} (potom i $T$ je spočetná)}

\end{frame}


\section{4.2 Úvod do tablo metody}


\begin{frame}{Tablo metoda neformálně}

    nejprve případ $T=\emptyset$, tedy dokazujeme, že $\varphi$ platí \emph{v logice}

    \alert{tablo} je strom představující \alert{hledání protipříkladu} (modelu $v\not\models\varphi$), když všechny větve \alert{selžou}, máme důkaz (sporem)

    labely: \alert{položky} $\mathrm{T}\psi,\mathrm{F}\psi$ (určují, zda na dané větvi platí výrok $\psi$)

    kořen \alert{$\mathrm{F}\varphi$}, dále rozvíjíme \alert{redukcí} položek (podle struktury výroků v nich), aby platil \alert{invariant}:

    \myalert{
        Každý model, který se \emph{shoduje} s položkou v kořeni (tj. ve kterém neplatí $\varphi$), se musí \emph{shodovat} i s některou větví tabla (tj. splňovat všechny požadavky vyjádřené položkami na této větvi).
    }

    je-li na větvi \alert{$\mathrm{T}\psi$} a zároveň \alert{$\mathrm{F}\psi$}, potom \alert{selhala} (je \alert{sporná}), pokud všechny větve selhaly, je tablo \alert{sporné}, je to \alert{důkaz} $T\proves\varphi$

    pokud nějaká větev neselhala a je \alert{dokončená} (vše na ní zredukované), lze z ní zkonstruovat model, ve kterém $\varphi$ neplatí

\end{frame}


\end{document}


