\documentclass{beamer}

%% slide-specific

\usetheme[progressbar=frametitle]{metropolis}
%\usecolortheme{spruce}
%\metroset{block=fill}

% define Metropolis colors    
\definecolor{mAlert}{HTML}{EB811B}
\definecolor{mExample}{HTML}{14B03D}
\definecolor{mBlock}{HTML}{23373b}

% my blocks
\setlength\fboxsep{0pt}%

\newcommand{\myblock}[1]{\colorbox{mBlock!8}{\begin{minipage}{\linewidth}#1\end{minipage}}}
\newcommand{\myblockmath}[1]{\colorbox{mBlock!8}{\begin{minipage}{\linewidth}\vspace{-6pt}#1\end{minipage}}}
\newcommand{\myblockinline}[1]{\colorbox{mBlock!8}{#1}}
\newcommand{\myexample}[1]{\colorbox{mExample!8}{\begin{minipage}{\linewidth}#1\end{minipage}}}
\newcommand{\myexamplemath}[1]{\colorbox{mExample!8}{\begin{minipage}{\linewidth}\vspace{-6pt}#1\end{minipage}}}
\newcommand{\myexampleinline}[1]{\colorbox{mExample!8}{#1}}
\newcommand{\myalert}[1]{\colorbox{mAlert!8}{\begin{minipage}{\linewidth}#1\end{minipage}}}
\newcommand{\myalertmath}[1]{\colorbox{mAlert!8}{\begin{minipage}{\linewidth}\vspace{-6pt}#1\end{minipage}}}
\newcommand{\myalertinline}[1]{\colorbox{mAlert!8}{#1}}

%% other
\newcommand{\mystructure}[1]{\mathcal{#1}}




%% packages
\usepackage{amsmath,amssymb,amsthm}
\usepackage{booktabs}
\usepackage[czech]{babel}
\usepackage{enumerate}
\usepackage{forest}
\usepackage{multicol}
% \usepackage{tcolorbox}
\usepackage{tikz}
    \usetikzlibrary{arrows.meta}
%\usepackage[unicode]{hyperref}
\usepackage[utf8x]{inputenc}
\usepackage{xfrac}

% %% theorems
% \theoremstyle{plain}
%     \newtheorem{theorem}{Věta}[section]
%     \newtheorem*{theorem-unnumbered}{Věta}
%     \newtheorem{proposition}[theorem]{Tvrzení}
%     \newtheorem{corollary}[theorem]{Důsledek}
%     \newtheorem{lemma}[theorem]{Lemma}
%     \newtheorem{observation}[theorem]{Pozorování}
% \theoremstyle{definition}
%     \newtheorem{definition}[theorem]{Definice}
%     \newtheorem*{algorithm}{Algoritmus}
% \theoremstyle{remark}
%     \newtheorem{remark}[theorem]{Poznámka}
%     \newtheorem{example}[theorem]{Příklad}
%     \newtheorem{exercise}{Cvičení}[chapter]
%     \newtheorem*{solution}{Řešení}

%% macros and definitions
\DeclareMathOperator{\Aut}{Aut}
\DeclareMathOperator{\Conseq}{Csq}
\DeclareMathOperator{\DeLO}{DeLO}
\DeclareMathOperator{\dom}{dom}
\DeclareMathOperator{\Fm}{Fm}
\DeclareMathOperator{\M}{M}
%\DeclareMathOperator{\Proof}{Proof}
\DeclareMathOperator{\rng}{rng}
\DeclareMathOperator{\Term}{Term}
\DeclareMathOperator{\Th}{Th}
\DeclareMathOperator{\Thm}{Thm}
\DeclareMathOperator{\Tree}{Tree}
\DeclareMathOperator{\Var}{Var}
\DeclareMathOperator{\VF}{VF}

\newcommand{\A}{\structure{A}}
\newcommand{\B}{\structure{B}}
\newcommand{\Con}{\mathit{Con}}
\newcommand{\disjointunion}{\mathbin{\dot{\sqcup}}}
\newcommand{\F}{\ensuremath{\mathrm{F}}}
\newcommand{\landsymb}{{\land}}
\newcommand{\lbin}{\mathbin{\square}}
\newcommand{\lbinsymb}{{\lbin}}
\newcommand{\liff}{\mathbin{\leftrightarrow}}
\newcommand{\liffsymb}{{\liff}}
\newcommand{\limplies}{\mathbin{\rightarrow}}
\newcommand{\limpliessymb}{{\limplies}}
\newcommand{\lorsymb}{{\lor}}
\newcommand{\Prf}{\mathit{Prf}}
\newcommand{\proves}{\vdash}
%\newcommand{\structure}[1]{\mathcal{#1}}
\newcommand{\todo}{[TODO]}
\newcommand{\T}{\ensuremath{\mathrm{T}}}
\newcommand{\union}{\mathbin{\cup}}


\title{Čtvrtá přednáška}
\subtitle{NAIL062 Výroková a predikátová logika}
\author{Jakub Bulín (KTIML MFF UK)}
% \institute{KTIML MFF UK}
\date{Zimní semestr 2024}


\begin{document}


\maketitle


\begin{frame}{Čtvrtá přednáška}

    \textbf{Program}
        \begin{itemize}            
            \item tablo důkaz
            \item korektnost a úplnost
            \item věta o kompaktnosti
            %\item hilbertovský kalkulus
        \end{itemize}

    \textbf{Materiály}

        \href{https://github.com/jbulin-mff-uk/nail062/raw/main/lecture/lecture-notes/lecture-notes.pdf}{\alert{\textbf{Zápisky z přednášky}}}, Sekce 4.3-4.7 z Kapitoly 4 (Sekci 4.8 zatím přeskočíme)

\end{frame}


\section{4.3 Tablo důkaz}


\begin{frame}{Formální definice tabla}

    \begin{itemize}[<+->]
        \item \alert{položka} je nápis $\mathrm{T}\varphi$ nebo $\mathrm{F}\varphi$, kde $\varphi$ je nějaký výrok
        \item \alert{konečné tablo z teorie $T$} je uspořádaný, položkami označkovaný strom zkonstruovaný aplikací konečně mnoha následujících pravidel:
        \begin{itemize}[<+->]
            \item jednoprvkový strom s libovolnou položkou je tablo z teorie $T$
            \item pro libovolnou položku $P$ na libovolné větvi $V$ můžeme na konec větve $V$ připojit atomické tablo pro položku $P$
            \item na konec libovolné větve můžeme připojit položku $\mathrm{T}\alpha$ pro libovolný axiom $\alpha\in T$
        \end{itemize}
        \item \alert{tablo z teorie $T$} je buď konečné, nebo i nekonečné: v tom případě je spočetné a definujeme ho jako $\tau=\bigcup_{i\geq 0}\tau_i$, kde:
        \begin{itemize}[<+->]
            \item $\tau_i$ jsou konečná tabla z $T$
            \item $\tau_0$ je jednoprvkové tablo
            \item $\tau_{i+1}$ vzniklo z $\tau_i$ v jednom kroku
        \end{itemize}
        \item \alert{tablo pro položku $P$} je tablo, které má položku $P$ v kořeni
    \end{itemize}

\end{frame}


\begin{frame}{Dokončené a sporné tablo}

    \begin{itemize}[<+->]
        \item Tablo je \alert{sporné}, pokud je každá jeho větev sporná.
        \item Větev je \alert{sporná}, pokud obsahuje položky $\mathrm{T}\psi$ a $\mathrm{F}\psi$ pro nějaký výrok $\psi$, jinak je \alert{bezesporná}.
        \item Tablo je \alert{dokončené}, pokud je každá jeho větev dokončená.
        \item Větev je \alert{dokončená}, pokud je sporná, nebo
        \begin{itemize}[<+->]
            \item každá její položka je na této větvi \alert{redukovaná},
            \item a zároveň obsahuje položku $\mathrm{T}\alpha$ pro každý axiom $\alpha\in T$.
        \end{itemize}
         
        \item Položka $P$ je \alert{redukovaná} na větvi $V$ procházející touto položkou, pokud 
        \begin{itemize}[<+->]
            \item je tvaru $\mathrm{T}p$ resp. $\mathrm{F}p$ pro nějaký prvovýrok $p\in\mathbb P$, 
            \item nebo se vyskytuje na $V$ jako kořen atomického tabla (byť ho podle konvence nezakreslujeme), tj., typicky, při konstrukci tabla již došlo k jejímu rozvoji na $V$.
        \end{itemize}
    \end{itemize}

\end{frame}


\begin{frame}{Tablo důkaz a tablo zamítnutí}

    \begin{itemize}[<+->]
        \item \alert{tablo důkaz} výroku $\varphi$ z teorie $T$ je \alert{sporné} tablo z teorie $T$ s položkou $\mathrm{F}\varphi$ v kořeni
        \item pokud existuje, je $\varphi$ \alert{(tablo) dokazatelný} z $T$, píšeme \alert{$T\proves\varphi$}
        \item podobně, \alert{tablo zamítnutí} je sporné tablo s $\mathrm{T}\varphi$ v kořeni
        \item existuje-li, je $\varphi$ \alert{(tablo) zamítnutelný} z $T$, tj. platí \alert{$T\proves\neg\varphi$}
    \end{itemize}

\end{frame}


\begin{frame}{Příklad: tablo důkaz}

    \begin{columns}
    
        \column{0.32\textwidth}

        \centering
        \begin{forest}
            [$\mathrm{F}\psi$
                [\textcolor{blue}{$\mathrm{T}\varphi\limplies \psi$}
                    [$\mathrm{F}\varphi$
                        [\textcolor{blue}{$\mathrm{T}\varphi$}, tikz={\node[fit to=tree,label=below:$\otimes$] {};}]
                    ]                
                    [$\mathrm{T}\psi$, tikz={\node[fit to=tree,label=below:$\otimes$] {};}]
                ]
            ]
        \end{forest}

        \column{0.68\textwidth}

        \begin{itemize}[<+->]
            \item tablo důkaz výroku \alert{$\psi$} z \alert{$T=\{\varphi,\varphi\limplies\psi\}$}
            \item položky vycházející z axiomů jsou \textcolor{blue}{modře}
            \item ukázali jsme tedy \alert{$T\proves\psi$}
            \item $\varphi,\psi$ jsou libovolné pevně dané výroky \item tím jsme dokázali tzv. \alert{větu o dedukci}
        \end{itemize}
    
    \end{columns}

\end{frame}


\begin{frame}{Příklad: dokončené tablo, které není sporné}

    \begin{columns}
    
        \column{0.32\textwidth}

        \centering
        \begin{forest}
            [$\mathrm{F}p_0$
                [\textcolor{blue}{$\mathrm{T}p_1\limplies p_0$}
                    [$\mathrm{F}p_1$                
                        [\textcolor{blue}{$\mathrm{T}p_2\limplies p_1$}
                            [$\mathrm{F}p_2$ [$\vdots$]] 
                            [$\mathrm{T}p_0$, tikz={\node[fit to=tree,label=below:$\otimes$] {};}]                    
                        ]                
                    ]
                    [$\mathrm{T}p_0$, tikz={\node[fit to=tree,label=below:$\otimes$] {};}]
                ]
            ]
        \end{forest}   
        
        \column{0.68\textwidth}

        \begin{itemize}[<+->]
            \item dokončené tablo pro výrok \alert{$p_0$} z teorie \alert{$T=\{p_{n+1}\limplies p_n\mid n\in\mathbb N\}$}. \item nejlevější větev je \alert{dokončená} a \alert{bezesporná}
            \item sestává z položek \alert{$\mathrm{T}p_{i+1}\limplies p_i$} a \alert{$\mathrm{F}p_i$} pro všechna $i\in\mathbb N$
            \item shoduje se s modelem \alert{$v=(0,0,\dots)$}, tj. $v:\mathbb P\to\{0,1\}$ kde $v(p_i)=0$ pro vš. $i$
            \item máme protipříklad ukazující, že \alert{$T\not\models p_0$}
        \end{itemize}

    \end{columns}

\end{frame}


\section{4.4 Konečnost a systematičnost důkazů}


\begin{frame}{Konečnost a systematičnost důkazů}

    Dokážeme:
    \begin{itemize}[<+->]
        \item existuje-li tablo důkaz, existuje i \alert{konečný} tablo důkaz
        \item existuje algoritmus, který umí vždy zkonstruovat dokončené tablo, tzv. \alert{systematické tablo}
        \item tento algoritmus tedy \alert{zkonstruuje tablo důkaz}, pokud existuje 
        
        \medskip
        
        \onslide<+->{(zde potřebujeme \emph{korektnost} a \emph{úplnost}, ty dokážeme později)}    
        
        \medskip
        \onslide<+->{(pokud tablo důkaz neexistuje, algoritmus se nemusí zastavit)}
    \end{itemize}
    

\end{frame}


\begin{frame}{Dokončení tabla: v čem je problém?}

    \pause
    Pro konečnou $T$ je snadné zkonstruovat dokončené tablo:\pause
    \begin{itemize}
        \item na začátku použijeme všechny axiomy\pause
        \item při redukci položek se výroky v nich zkracují\pause
        \item stačí nedělat zbytečné kroky\pause
    \end{itemize}

    Pro \alert{nekonečnou} $T$ bychom ale mohli zkonstruovat nekonečné tablo, a přitom:\pause
    \begin{itemize}
        \item nikdy nepoužít některý axiom, nebo\pause
        \item nikdy se nedostat k redukci některé položky
    \end{itemize}
    
    \pause
    \textbf{Myšlenka} \alert{systematického tabla}: na všechny se dostane, střídáme:\pause
    \begin{itemize}
        \item \alert{redukce následující položky} (po úrovních, zleva doprava) na všech bezesporných větvích, které jí procházejí \pause
        \item \alert{přidání následujícího axiomu} na všechny bezesporné větve ($T$~je spočetná, axiomy libovolně očíslujeme)         
    \end{itemize}

\end{frame}


\begin{frame}{Definice systematického tabla}
 
    \alert{Systematické tablo} z teorie $T=\{\alpha_1,\alpha_2,\dots\}$ pro položku $R$ je tablo $\tau=\bigcup_{i\geq 0}\tau_i$, kde $\tau_0$ je jednoprvkové tablo s položkou $R$, a pro každé $i\geq 0$:\pause

    \begin{itemize}[<+->]
        \item buď $P$ nejlevější položka v co nejmenší úrovni, která není redukovaná na nějaké bezesporné větvi procházející $P$
        \item nejprve definujeme $\tau_i'$ jako tablo vzniklé z $\tau_i$ připojením atomického tabla pro $P$ na každou bezespornou větev procházející $P$
        \item pokud taková položka $P$ neexistuje, potom $\tau_i'=\tau_i$
        \item tablo $\tau_{i+1}$ vznikne z $\tau_i'$ připojením $\mathrm{T}\alpha_{i+1}$ na každou bezespornou větev
        \item to v případě, že $i<|T|$, jinak (je-li $T$ konečná a už jsme použili všechny axiomy) definujeme $\tau_{i+1}=\tau_i'$
    \end{itemize}  

\end{frame}


\begin{frame}{Dokončenost systematického tabla}

    \myblock{
    \textbf{Lemma:} Systematické tablo je dokončené.
    }

    \pause
    \textbf{Důkaz:} 
    Jsou všechny větve dokončené?\pause

    \begin{itemize}
        \item Sporné větve jsou dokončené z definice.\pause
        \item Bezesporná větev:\pause
        \begin{itemize}
            \item \alert{obsahuje $\mathrm{T}\alpha_i$} pro všechna $i$ (připojeno v $i$-tém kroku)\pause
            \item každá položka je na ní \alert{zredukovaná} (leží-li v hloubce $d$, dostali jsme se k ní nejdéle v kroku $i=2^{d+1}-1$)\pause
        \end{itemize}  
        \item Tedy i všechny bezesporné větve jsou dokončené.\hfill\qedsymbol        
    \end{itemize}  

\end{frame}


\begin{frame}{Konečnost sporu}

    \myblock{
    \textbf{Věta (Konečnost sporu):} Je-li $\tau=\bigcup_{i\geq 0}\tau_i$ sporné tablo, potom existuje $n\in\mathbb N$ takové, že $\tau_n$ je sporné konečné tablo.
    }

    \pause
    \textbf{Důkaz:} 
    Buď $S$ množina všech vrcholů, nad kterými (ve stromovém uspořádání) není spor, tj. dvojice položek $\mathrm{T}\psi$, $\mathrm{F}\psi$.\pause

    \begin{itemize}[<+->]
        \item \alert{Kdyby byla $S$ nekonečná:} Podle Königova lemmatu pro podstrom $\tau$ na množině $S$ máme nekonečnou, bezespornou větev v $S$. To ale dává i \alert{bezespornou větev v $\tau$}, což je spor. 
        %(Podrobněji: kdyby nekonečná bezesporná větev $V_S$ v $S$ byla podvětví nějaké sporné větve $V$ v $\tau$, spor na $V$ leží už v nějakém konečném prefixu $V$, a tedy nekonečně mnoho vrcholů z $V_S\subseteq S$ musí ležet pod tím to sporem).
    
        \item \alert{Množina $S$ je tedy konečná,} celá leží v hloubce $\leq d$ pro nějaké $d\in\mathbb N$. Každý vrchol \alert{na úrovni $d+1$ už má nad sebou spor}. 
        
        \item Zvolme $n$ tak, že $\tau_n$ už obsahuje všechny vrcholy $\tau$ z prvních $d+1$ úrovní. Potom každá větev tabla $\tau_n$ je sporná.\hfill\qedsymbol
    \end{itemize}    

\end{frame}


\begin{frame}{Důsledky konečnosti sporu}
    
    Tedy: Pokud neprodlužujeme už sporné větve (např. pro systematické tablo), potom sporné tablo je konečné.
    \medskip

    \pause
    \myblock{
    \textbf{Důsledek (Konečnost důkazů):} Pokud $T\proves\varphi$, potom existuje i~\alert{konečný} tablo důkaz $\varphi$ z $T$.
    }

    \pause
    \textbf{Důkaz:} 
    Platí $\tau=\tau_n$, neboť sporné tablo už neměníme.\hfill\qedsymbol   

    \bigskip

    \pause
    \myblock{
    \textbf{Důsledek (Systematičnost důkazů):} Pokud $T\proves\varphi$, potom systematické tablo je (konečným) tablo důkazem $\varphi$ z $T$.
    }

    \pause
    Důkaz bude až v příští sekci, chybí nám dvě fakta:\pause
    \begin{itemize}
        \item je-li $\varphi$ dokazatelná z $T$, potom v $T$ platí (Věta o korektnosti)\pause
        \item pokud by systematické tablo mělo bezespornou větev, šel by z ní vyrobit protipříklad (to je klíč k důkazu Věty o úplnosti)1
    \end{itemize}

\end{frame}


\section{4.5 Korektnost a úplnost}


\begin{frame}{Korektnost a úplnost}
    
    Nyní ukážeme, že \alert{dokazatelnost} je totéž, co \alert{platnost}, tj. pro každou teorii $T$ a výrok $\varphi$:
    \alert{
    $$
    T\proves\varphi\ \Leftrightarrow\ T\models\varphi
    $$
    }

    \pause
    Rozdělíme na dvě implikace:\pause
    
    \begin{itemize}
        \item \alert{$T\proves\varphi\ \Rightarrow\T\models\varphi$} \hspace{0.5cm} (korektnost) \hfill {\it``co jsme dokázali, platí''}\pause
        \item\alert{$T\models\varphi\ \Rightarrow\T\proves\varphi$}  \hspace{0.5cm} (úplnost) \hfill {\it ``co platí, lze dokázat''}
    \end{itemize} 
 
\end{frame}


\begin{frame}{Korektnost: pomocné lemma}


    Model $v$ se \alert{shoduje}\pause
    \begin{itemize}
        \item \alert{s položkou $P$}, pokud
        $P=\mathrm{T}\varphi$ a $v\models\varphi$, nebo $P=\mathrm{F}\varphi$ a $v\not\models\varphi$\pause
        \item \alert{s větví $V$}, pokud se shoduje s každou položkou na této větvi
    \end{itemize}
    
    \medskip

    \pause
    \myblock{
    \textbf{Lemma:}
        Shoduje-li se model teorie $T$ s položkou v kořeni tabla z teorie $T$, potom se shoduje s některou větví.
    }

    \smallskip
    
    \pause
    \textbf{Důkaz:}
    Indukcí podle kroků $i$ při konstrukci tabla $\tau=\bigcup_{i\geq 0}\tau_i$ najdeme posloupnost větví $V_0\subseteq V_1\subseteq\dots$ takovou, že:\pause
     \begin{itemize}
        \item $V_i$ je větev v tablu $\tau_i$ shodující se s modelem $v$\pause
        \item $V_{i+1}$ je prodloužením $V_i$\pause
     \end{itemize}

    Hledaná větev v $\tau$ je potom $V=\bigcup_{i\geq 0}V_i$.\pause
    
    \alert{Báze indukce:} Model $v$ se shoduje s kořenem $\tau$, tj. s (jednoprvkovou) větví $V_0$ v $\tau_0$.

\end{frame}


\begin{frame}{Pokračování důkazu pomocného lemmatu}

    \alert{Indukční krok:}
    
    \pause
    Pokud $\tau_{i+1}$ vzniklo z $\tau_i$ bez prodloužení $V_i$, definujeme $V_{i+1}=V_i$.

    \medskip

    \pause
    Pokud $\tau_{i+1}$ vzniklo připojením $\mathrm{T}\alpha$ (pro axiom $\alpha\in T$) na konec $V_i$, definujeme $V_{i+1}$ jako tuto prodlouženou větev. \pause
    Protože $v\models T$, máme i $v\models\alpha$, tedy $v$ se shoduje i s novou položkou.
    
    \medskip

    \pause
    Nechť $\tau_{i+1}$ vzniklo připojením atomického tabla pro položku $P$ na konec $V_i$. \pause
    Protože se $v$ shoduje s $P$ (která leží na $V_i$), shoduje se i s kořenem připojeného atomického tabla, a proto se shoduje i s některou z jeho větví. \pause
    (Ověříme si pro všechna atomická tabla.) \pause
    Definujeme $V_{i+1}$ jako prodloužení $V_i$ o tuto větev atomického tabla.\hfill\qedsymbol

\end{frame}


\begin{frame}{Věta o korektnosti [tablo metody ve výrokové logice]}

    \myblock{
    \textbf{Věta (O korektnosti):}
    Je-li výrok $\varphi$ tablo dokazatelný z teorie $T$, potom je $\varphi$ pravdivý v~$T$, tj. \alert{$T\proves\varphi\ \Rightarrow\T\models\varphi$}.
    }

    \medskip

    \pause
    \textbf{Myšlenka důkazu:} Protipříklad by se shodoval s některou z větví tablo důkazu, ty jsou ale všechny sporné.

    \medskip

    \pause
    \textbf{Důkaz:} Sporem, nechť $T\not\models\varphi$, tj. existuje $v\in\M(T)$, že $v\not\models\varphi$.
     
    \pause
    Protože je $T\proves\varphi$, existuje tablo důkaz $\varphi$ z $T$, což je sporné tablo z $T$ s položkou $\mathrm{F}\varphi$ v kořeni. 
    
    \pause
    Model $v$ se shoduje s kořenem $\mathrm{F}\varphi$, tedy podle Lemmatu se shoduje s nějakou větví $V$. \pause
    Všechny větve jsou ale sporné. \pause
     Takže na $V$ jsou $\mathrm{T}\psi$ a $\mathrm{F}\psi$ (pro nějaký výrok $\psi$), a model $v$ se s těmito položkami shoduje. \pause
     Máme $v\models\psi$ a zároveň $v\not\models\psi$, což je spor.\hfill\qedsymbol

\end{frame}


\begin{frame}{Úplnost: pomocné lemma}

    Selže-li dokazování, dostaneme \alert{bezespornou, dokončenou} větev v tablu z $T$ s $\mathrm{F}\varphi$ v kořeni; ukážeme, že dává protipříklad:

    \pause
    \alert{Kanonický model} pro bezespornou, dokončenou větev $V$ je model
    $$
    v(p)=\begin{cases}
        1 \text{ pokud se na $V$ vyskytuje položka $\mathrm{T}p$}\\
        0 \text{ jinak}
    \end{cases}
    $$  
    
    \medskip

    \pause
    \myblock{
    \textbf{Lemma:}
    Kanonický model pro (bezespornou, dokončenou) větev $V$ se shoduje s $V$.
    }

    \bigskip
    
    \pause
    (tento model tedy musí splňovat všechny axiomy $T$, ale protože se shoduje s položkou $\mathrm{F}\varphi$ v kořeni, neplatí v něm výrok $\varphi$)

\end{frame}


\begin{frame}{Důkaz pomocného lemmatu}

    \textbf{Důkaz:}
    Indukcí podle struktury výroků v položkách. \pause \alert{Báze indukce:} \pause
    \begin{itemize}
        \item je-li \alert{$P=\mathrm{T}p$} pro prvovýrok $p$, máme $v(p)=1$, shoduje se\pause
        \item je-li \alert{$P=\mathrm{F}p$}, potom na $V$ nemůže být $\mathrm{T}p$ (byla by sporná), máme tedy $v(p)=0$, shoduje se\pause
    \end{itemize}

    \alert{Indukční krok:} \pause rozebereme dva případy, ostatní jsou obdobné \pause

    \begin{itemize}
        \item \alert{$P=\mathrm{T}\varphi\land\psi$}. Protože je $V$ dokončená, je na ní $P$ redukovaná. To znamená, že se na $V$ vyskytují i položky $\mathrm{T}\varphi$ a $\mathrm{T}\psi$. Podle indukčního předpokladu se s nimi $v$ shoduje: $v\models\varphi$ a $v\models\psi$. Takže platí i $v\models\varphi\land\psi$ a $v$ se shoduje s $P$.\pause
    
        \item \alert{$P=\mathrm{F}\varphi\land\psi$}. Protože je $P$ na $V$ redukovaná, vyskytuje se na $V$ položka $\mathrm{F}\varphi$ nebo položka $\mathrm{F}\psi$. Platí tedy $v\not\models\varphi$ nebo $v\not\models\psi$, z čehož plyne $v\not\models\varphi\land\psi$ a $v$ se shoduje s $P$.\hfill\qedsymbol
    \end{itemize}

\end{frame}


\begin{frame}{Věta o úplnosti (+ důkaz systematičnosti)}

    \myblock{
    \textbf{Věta (O úplnosti):} Je-li výrok $\varphi$ pravdivý v teorii $T$, potom je tablo dokazatelný z $T$, tj. \alert{$T\models\varphi\ \Rightarrow\T\proves\varphi$}.
    }
    
    \pause
    \textbf{Důkaz:}
    Ukážeme, že libovolné dokončené (např. \alert{systematické}) tablo z $T$ s $\mathrm{F}\varphi$ v kořeni je nutně sporné, tedy je tablo důkazem. 
    
    \pause
    Sporem: \alert{Není-li sporné}, má bezespornou (dokončenou) větev $V$, a dle Lemmatu se s ní kanonický model pro $V$ shoduje. 
    
    \pause
    Protože je $V$ dokončená, obsahuje $\mathrm{T}\alpha$ pro všechny axiomy $T$. Model $v$ tedy splňuje všechny axiomy a máme $v\models T$. 
    
    \pause
    Protože se ale $v$ shoduje i s položkou $\mathrm{F}\varphi$ v kořeni, máme $v\not\models\varphi$, což dává protipříklad, a máme $T\not\models\varphi$, spor.\hfill\qedsymbol

    \pause
    \textbf{Dokázali jsme i Důsledek o systematičnosti důkazů:}  Z důkazu vidíme, že i systematické tablo pro položku $\mathrm{F}\varphi$ je nutně sporné, a je tedy tablo důkazem.\hfill\qedsymbol

\end{frame}


\section{4.6 Důsledky korektnosti a úplnosti}


\begin{frame}{$\proves\ =\ \models$}

    \pause
    Syntaktickou analogií \alert{důsledků} jsou \alert{teorémy}:
    $$\Thm_\mathbb P(T)=\{\varphi\in\VF_\mathbb P\mid T\proves\varphi\}$$
    
    \pause
    Z korektnosti a úplnosti okamžitě dostáváme:
        \begin{itemize}
            \item $T\proves\varphi$ právě když $T\models\varphi$
            \item $\Thm_\mathbb P(T)=\Conseq_\mathbb P(T)$
        \end{itemize}
    
        \pause
    Všude můžeme nahradit `\alert{platnost}' pojmem `\alert{dokazatelnost}'.  Např:\pause
    \begin{itemize}
        \item $T$ je \alert{sporná}, je-li v ní dokazatelný spor (tj. \alert{$T\proves\bot$})\pause
        \item $T$ je \alert{kompletní}, je-li pro každý výrok buď $T\proves\varphi$ nebo $T\proves\neg\varphi$, ale ne obojí (jinak by byla sporná)
    \end{itemize}

    \pause
    \myblock{
        \textbf{Věta (O dedukci):} $T,\varphi\proves\psi$ právě když  $T\proves\varphi\to\psi$.
    }

    \pause
    \textbf{Důkaz:} Stačí dokázat: $T,\varphi\models\psi\Leftrightarrow T\models\varphi\to\psi$. To je snadné.\hfill\qedsymbol

\end{frame}


\section{4.7 Věta o kompaktnosti}


\begin{frame}{Kompaktnost}

    \myblock{
        \textbf{Věta (O kompaktnosti):} Teorie má model, právě když každá její konečná část má model.
    }
    \smallskip

    \pause
    \textbf{Důkaz:} \pause \alert{$\Rightarrow$ Snadné:} Model $T$ je zjevně modelem každé její části.
    
    \pause
    \alert{$\Leftarrow$ Nepřímo:} buď $T$ sporná, najdeme spornou konečnou $T'\subseteq T$.

    \pause
    Z \alert{úplnosti} víme, že $T\proves\bot$, tedy existuje i \alert{konečný} tablo důkaz $\tau$ výroku $\bot$ z $T$. \pause Konstrukce $\tau$ má konečně mnoho kroků, použili jsme tedy jen konečně mnoho axiomů z $T$. \pause Definujme:
    
    \pause
    \myalertmath{
    $$
    T'=\{\alpha\in T\mid \mathrm{T}\alpha\text{ je položka v tablu $\tau$}\}
    $$
    }   

    \pause
    Tedy $\tau$ je tablo jen z teorie $T'$, máme tablo důkaz $T'\proves\bot$, dle \alert{korektnosti} je $T'$ sporná.\hfill\qedsymbol

\end{frame}


\begin{frame}{Aplikace kompaktnosti}
     
    \begin{center}
        \alert{vlastnost nekonečného objektu~$\mathcal O$}
    
        $\Updownarrow$

        \alert{vlastnost všech konečných podobjektů~$\mathcal O'$}
    \end{center}
    
    \pause
    \begin{itemize}
        \item vlastnost popíšeme pomocí (nekonečné) teorie $T$\pause
        \item ke každé konečné $T'\subseteq T$ sestrojíme konečný podobjekt $\mathcal O'$\pause
        \item $\mathcal O'$ splňuje danou vlastnost\pause
        \item to nám dává model $T'$\pause
        \item dle Věty o kompaktnosti má i $T$ model\pause
        \item což ukazuje, že i nekonečný objekt $\mathcal O$ splňuje vlastnost
    \end{itemize}
     
    \pause
    Věta o kompaktnosti má mnoho aplikací (několik z nich uvidíme později), následující příklad chápejte jako `šablonu'.

\end{frame}


\begin{frame}{Aplikace kompaktnosti: příklad}

    \myexample{
        \textbf{Důsledek:} Spočetně nekonečný graf je bipartitní, právě když je každý jeho konečný podgraf bipartitní.
    }

    \pause
    \textbf{Důkaz:} \alert{$\Rightarrow$} Každý podgraf bipartitního grafu je bipartitní. 
    
    \pause
    \alert{$\Leftarrow$} $G$ je bipartitní, právě když je obarvitelný 2 barvami. \pause Mějme jazyk $\mathbb P=\{p_v\mid v\in V(G)\}$ (kde $p_v$ je barva $v$) a uvažme teorii
    
    \myalertmath{
    $$  
        T=\{p_u\liff\neg p_v\mid \{u,v\}\in E(G)\}
    $$
    }

    \pause
    Zřejmě $G$ je bipartitní, právě když $T$ má model. Dle Věty o kompaktnosti stačí ukázat, že každá konečná $T'\subseteq T$ má model.
    
    \pause
    Buď $G'$ podgraf $G$ indukovaný na vrcholech, o kterých $T'$ mluví:
    $$
    V(G')=\{v\in V(G)\mid p_v\in\Var(T')\}
    $$
    \pause
    Protože je $T'$ konečná, je $G'$ také konečný, tedy je dle předpokladu 2-obarvitelný. Libovolné 2-obarvení $V(G')$ ale určuje model $T'$.\hfill\qedsymbol

\end{frame}


\end{document} %TODO


\section{Zatím přeskočíme: 4.8 Hilbertovský kalkulus}


\begin{frame}{Hilbertovský deduktivní systém}

    \begin{itemize}
        \item jiný, původní dokazovací systém 
        \item používá jen logické spojky $\neg$, $\limplies$
        \item \alert{schémata logických axiomů} ($\varphi,\psi,\chi$ jsou libovolné výroky)
        \begin{enumerate}[(i)]
            \item $\varphi \limplies (\psi \limplies \varphi)$
            \item $(\varphi\limplies (\psi \limplies \chi))\limplies ((\varphi \limplies \psi)\limplies(\varphi \limplies \chi))$
            \item $(\neg \varphi \limplies \neg \psi)\limplies(\psi \limplies \varphi)$
        \end{enumerate}
        \item \alert{odvozovací pravidlo}: tzv. \alert{modus ponens}
                $$\frac{\varphi, \varphi \limplies \psi}{\psi}$$       
        \item \alert{hilbertovský důkaz} výroku $\varphi$ z teorie $T$ je \alert{konečná} posloupnost výroků $\varphi_0, \dots, \varphi_n=\varphi$, ve které pro každé $i\le n$:
        \begin{itemize}
        \item $\varphi_i$ je \alert{logický axiom}, nebo
        \item $\varphi_i$ je \alert{axiom teorie} ($\varphi_i \in T$), nebo
        \item $\varphi_i$ lze odvodit z předchozích pomocí \alert{odvozovacího pravidla}
        \end{itemize}
        \item existuje-li hilbertovský důkaz, píšeme: \alert{$T\proves_H\varphi$}
    \end{itemize}

\end{frame}


\begin{frame}{Příklad hilbertovského důkazu}

    Ukažme, že pro teorii $T=\{\neg\varphi\}$ a pro libovolný výrok $\psi$ platí:  
    \myexamplemath{  
    $$
    T\proves_H\varphi\limplies\psi
    $$
    }

    Hilbertovským důkazem je následující posloupnost výroků:
    
    \begin{enumerate}\it
        \item $\neg\varphi$\hfill axiom teorie
        \item $\neg \varphi \limplies (\neg \psi \limplies \neg \varphi)$\hfill logický axiom (i)
        \item $\neg \psi \limplies \neg \varphi$\hfill modus ponens na 1. a 2.
        \item $(\neg \psi \limplies \neg \varphi)\limplies(\varphi \limplies \psi)$ \hfill logický axiom (iii)
        \item $\varphi \limplies \psi$ \hfill modus ponens na 3. a 4.
    \end{enumerate}    

\end{frame}


\begin{frame}{Korektnost a úplnost}

    \myblock{
    \textbf{Věta (o korektnosti hilbertovského kalkulu):}
    $T\proves_H\varphi \Rightarrow T\models\varphi$
    }

    \medskip

    \textbf{Důkaz:} Indukcí dle délky důkazu ukážeme, že každý výrok $\varphi_i$ z důkazu (tedy i $\varphi_n=\varphi$) platí v $T$.
    \begin{itemize}
        \item Je-li $\varphi_i$ logický axiom, $T \models \varphi_i$ platí protože logické axiomy jsou tautologie.
        \item Je-li $\varphi_i\in T$, jistě platí $T \models \varphi_i$.
        \item Získáme-li $\varphi_i$ pomocí modus ponens z $\varphi_j$ a $\varphi_k=\varphi_j\limplies\varphi_i$ (pro nějaká $j,k<i$), víme z indukčního předpokladu, že platí $T \models \varphi_j$ a $T \models \varphi_j\limplies\varphi_i$. Potom ale platí i $T \models \varphi_i$. (Modus ponens je \alert{korektní} odvozovací pravidlo)\hfill\qedsymbol
    \end{itemize}

    \myblock{
    \textbf{Věta (o úplnosti hilbertovského kalkulu):}
    $T\models\varphi\ \Rightarrow\ T\proves_H\varphi$
    }

    Důkaz vynecháme.
    
\end{frame}


\end{document}



