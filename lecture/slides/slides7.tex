\documentclass{beamer}

%% slide-specific

\usetheme[progressbar=frametitle]{metropolis}
%\usecolortheme{spruce}
%\metroset{block=fill}

% define Metropolis colors    
\definecolor{mAlert}{HTML}{EB811B}
\definecolor{mExample}{HTML}{14B03D}
\definecolor{mBlock}{HTML}{23373b}

% my blocks
\setlength\fboxsep{0pt}%

\newcommand{\myblock}[1]{\colorbox{mBlock!8}{\begin{minipage}{\linewidth}#1\end{minipage}}}
\newcommand{\myblockmath}[1]{\colorbox{mBlock!8}{\begin{minipage}{\linewidth}\vspace{-6pt}#1\end{minipage}}}
\newcommand{\myblockinline}[1]{\colorbox{mBlock!8}{#1}}
\newcommand{\myexample}[1]{\colorbox{mExample!8}{\begin{minipage}{\linewidth}#1\end{minipage}}}
\newcommand{\myexamplemath}[1]{\colorbox{mExample!8}{\begin{minipage}{\linewidth}\vspace{-6pt}#1\end{minipage}}}
\newcommand{\myexampleinline}[1]{\colorbox{mExample!8}{#1}}
\newcommand{\myalert}[1]{\colorbox{mAlert!8}{\begin{minipage}{\linewidth}#1\end{minipage}}}
\newcommand{\myalertmath}[1]{\colorbox{mAlert!8}{\begin{minipage}{\linewidth}\vspace{-6pt}#1\end{minipage}}}
\newcommand{\myalertinline}[1]{\colorbox{mAlert!8}{#1}}

%% other
\newcommand{\mystructure}[1]{\mathcal{#1}}




%% packages
\usepackage{amsmath,amssymb,amsthm}
\usepackage{booktabs}
\usepackage[czech]{babel}
\usepackage{enumerate}
\usepackage{forest}
\usepackage{multicol}
% \usepackage{tcolorbox}
\usepackage{tikz}
    \usetikzlibrary{arrows.meta}
%\usepackage[unicode]{hyperref}
\usepackage[utf8x]{inputenc}
\usepackage{xfrac}

% %% theorems
% \theoremstyle{plain}
%     \newtheorem{theorem}{Věta}[section]
%     \newtheorem*{theorem-unnumbered}{Věta}
%     \newtheorem{proposition}[theorem]{Tvrzení}
%     \newtheorem{corollary}[theorem]{Důsledek}
%     \newtheorem{lemma}[theorem]{Lemma}
%     \newtheorem{observation}[theorem]{Pozorování}
% \theoremstyle{definition}
%     \newtheorem{definition}[theorem]{Definice}
%     \newtheorem*{algorithm}{Algoritmus}
% \theoremstyle{remark}
%     \newtheorem{remark}[theorem]{Poznámka}
%     \newtheorem{example}[theorem]{Příklad}
%     \newtheorem{exercise}{Cvičení}[chapter]
%     \newtheorem*{solution}{Řešení}

%% macros and definitions
\DeclareMathOperator{\Aut}{Aut}
\DeclareMathOperator{\Conseq}{Csq}
\DeclareMathOperator{\DeLO}{DeLO}
\DeclareMathOperator{\dom}{dom}
\DeclareMathOperator{\Fm}{Fm}
\DeclareMathOperator{\M}{M}
%\DeclareMathOperator{\Proof}{Proof}
\DeclareMathOperator{\rng}{rng}
\DeclareMathOperator{\Term}{Term}
\DeclareMathOperator{\Th}{Th}
\DeclareMathOperator{\Thm}{Thm}
\DeclareMathOperator{\Tree}{Tree}
\DeclareMathOperator{\Var}{Var}
\DeclareMathOperator{\VF}{VF}

\newcommand{\A}{\structure{A}}
\newcommand{\B}{\structure{B}}
\newcommand{\Con}{\mathit{Con}}
\newcommand{\disjointunion}{\mathbin{\dot{\sqcup}}}
\newcommand{\F}{\ensuremath{\mathrm{F}}}
\newcommand{\landsymb}{{\land}}
\newcommand{\lbin}{\mathbin{\square}}
\newcommand{\lbinsymb}{{\lbin}}
\newcommand{\liff}{\mathbin{\leftrightarrow}}
\newcommand{\liffsymb}{{\liff}}
\newcommand{\limplies}{\mathbin{\rightarrow}}
\newcommand{\limpliessymb}{{\limplies}}
\newcommand{\lorsymb}{{\lor}}
\newcommand{\Prf}{\mathit{Prf}}
\newcommand{\proves}{\vdash}
%\newcommand{\structure}[1]{\mathcal{#1}}
\newcommand{\todo}{[TODO]}
\newcommand{\T}{\ensuremath{\mathrm{T}}}
\newcommand{\union}{\mathbin{\cup}}


\title{Sedmá přednáška}
\subtitle{NAIL062 Výroková a predikátová logika}
\author{Jakub Bulín (KTIML MFF UK)}
% \institute{KTIML MFF UK}
\date{Zimní semestr 2023}


\begin{document}


\maketitle


\begin{frame}{Sedmá přednáška}

    \textbf{Program}
        \begin{itemize}
            \item podstruktury, expanze, redukty           
            \item extenze teorií, extenze o definice
            \item definovatelnost a databázové dotazy
            \item vztah výrokové a predikátové logiky
        \end{itemize}

    \textbf{Materiály}

        \href{https://github.com/jbulin-mff-uk/nail062/raw/main/lecture/lecture-notes/lecture-notes.pdf}{\alert{\textbf{Zápisky z přednášky}}}, Sekce 6.6-6.9 z Kapitoly 6

\end{frame}


\section{6.6 Podstruktura, expanze, redukt}


\begin{frame}{Podstruktura}

    \begin{itemize}
        \item \alert{podstruktura} zobecňuje podgrupu, podprostor vektorového prostoru, (indukovaný) podgraf: na podmnožině $B$ univerza vytvoříme strukturu, která \myalertinline{``zdědí'' relace, funkce a konstanty}
        \item $B$ musí být \alert{uzavřená} na všechny funkce (vč. konstant)
    \end{itemize}

    \myblock{
    Struktura $\B=\langle B,\mathcal R^\mathcal B,\mathcal F^\mathcal B\rangle$ je \alert{(indukovaná) podstruktura} struktury $\A=\langle A,\mathcal R^\mathcal A,\mathcal F^\mathcal A\rangle$ (v též signatuře $\langle\mathcal R,\mathcal F\rangle$), značíme \alert{$\B\subseteq\A$}, jestliže:

    \begin{itemize}
        \item $\emptyset\neq B\subseteq A$
        \item $R^\B=R^\A\cap B^{\mathrm{ar(R)}}$ pro každý relační symbol $R\in \mathcal R$
        \item $f^\B=f^\A\cap (B^{\mathrm{ar(f)}}\times B)$ pro každý funkční symbol $f\in \mathcal F$, tj. $f^\B$ je restrikce $f^\A$ na množinu $B$, a výstupy jsou všechny z $B$
    \end{itemize}
    }

    speciálně, pro konstantní symbol $c\in\mathcal F$ máme $c^\B=c^\A\in B$
    


\end{frame}


\begin{frame}{Restrikce na podmnožinu, příklady}
    
    Množina $C\subseteq A$ je \alert{uzavřená} na funkci $f:A^n\to A$, pokud $f(x_1,\dots,x_n)\in C$ pro všechna $x_i\in C$.

    \medskip

    \myblock{
    \textbf{Pozorování:} Množina $\emptyset\neq C\subseteq A$ je univerzem podstruktury, právě když je uzavřená na všechny funkce struktury $\A$ (včetně konstant). V tom případě je to \alert{restrikce} $\A$ na množinu $C$, značíme \alert{$\A\restriction C$}.
    }

    \medskip
    
    \begin{itemize}
        \item \myexampleinline{
            $\underline{\mathbb Z}=\langle \mathbb Z,+,\cdot,0\rangle$
         } je podstrukturou \myexampleinline{
            $\underline{\mathbb Q}=\langle \mathbb Q,+,\cdot,0\rangle$
            }, můžeme psát: \alert{$\underline{\mathbb Z}=\underline{\mathbb Q}\restriction\mathbb Z$}
        \item \myexampleinline{
            $\underline{\mathbb N}=\langle \mathbb N,+,\cdot,0\rangle$
         } je podstrukturou obou těchto struktur, platí: \alert{$\underline{\mathbb N}=\underline{\mathbb Q}\restriction\mathbb N=\underline{\mathbb Z}\restriction\mathbb N$}
        \item Množina \myexampleinline{
            $\{k\in\mathbb Z\mid k\leq 0\}$
         } není univerzem podstruktury $\underline{\mathbb Z}$ ani~$\underline{\mathbb Q}$, není uzavřená na násobení.
    \end{itemize}
\end{frame}


\begin{frame}{Platnost v podstruktuře (pro otevřené formule je zachována)}
    
    
    
    \myblock{
    \textbf{Pozorování:}
    Je-li $\B\subseteq\A$, $\varphi$ \alert{otevřená} formule, a \alert{$e\colon\Var\to B$}, potom platí: $\B\models\varphi[e]$ právě když $\A\models\varphi[e]$.
    }

    \textbf{Důkaz:}
    Snadno indukcí dle struktury $\varphi$, pro atomickou zřejmé.\hfill\qedsymbol
    
    \medskip

    \myblock{
    \textbf{Důsledek:}
    \alert{Otevřená} formule platí ve struktuře $\A$, právě když platí v každé podstruktuře $\B\subseteq\A$.
    }

    \medskip

    Teorie $T$ je \alert{otevřená}, jsou-li všechny její axiomy otevřené formule.

    \myblock{
    \textbf{Důsledek:}
    Modely otevřené teorie jsou uzavřené na podstruktury, tj. každá podstruktura modelu této teorie je také její model.
    }

    \medskip

    \begin{itemize}
        \item \myexampleinline{Teorie grafů} je otevřená. Podstruktura grafu je také graf: (indukovaný) \alert{podgraf}. Stejně podgrupy, Booleovy podalgebry.
        \item \myexampleinline{Teorie těles} není otevřená. Později ukážeme, že ani \alert{otevřeně axiomatizovatelná} (kvantifikátoru v axiomu o existenci inverzního prvku se nezbavíme). Podstruktura tělesa $\mathbb Q$ na množině $\mathbb Z$, $\mathbb Q\restriction\mathbb Z$, není těleso. (Je to tzv. \alert{okruh}.)
    \end{itemize}
    
\end{frame}


\begin{frame}{Generovaná podstruktura (zobecníme lineární obal vektorů)}
    
    Co když podmnožina univerza \alert{není} uzavřená? Vezmeme její \alert{uzávěr}. %(zobecňuje \alert{lineární obal} množiny vektorů).

    \medskip

    \myblock{
    Mějme $\A=\langle A,\mathcal R^\mathcal A,\mathcal F^\mathcal A\rangle$ a $\emptyset\neq X\subseteq A$. Buď $B\subseteq A$ nejmenší podmnožina, která obsahuje $X$ a je uzavřená na všechny funkce $\A$ (tj. obsahuje i všechny konstanty). Potom podstruktura $\A\restriction B$ je \alert{generovaná} $X$, značíme ji \alert{$\A\langle X\rangle$}.
    }

    \medskip

    Např. pro \myexampleinline{
        $\underline{\mathbb Q}=\langle\mathbb Q,+,\cdot,0\rangle$
    }, \myexampleinline{
        $\underline{\mathbb Z}=\langle\mathbb Z,+,\cdot,0\rangle$
    }, \myexampleinline{
        $\underline{\mathbb N}=\langle\mathbb N,+,\cdot,0\rangle$
    }: 
    \begin{itemize}
        \item $\underline{\mathbb Q}\langle\{1\}\rangle=\underline{\mathbb N}$
        \item $\underline{\mathbb Q}\langle\{-1\}\rangle=\underline{\mathbb Z}$
        \item $\underline{\mathbb Q}\langle\{2\}\rangle$ je podstruktura $\underline{\mathbb N}$ na množině všech sudých čísel
    \end{itemize}

    \medskip
    
    Pokud $\A$ nemá žádné funkce (ani konstanty), např. graf či uspořádání, potom není čím generovat, a $\A\langle X\rangle=\A\restriction X$.
    
\end{frame}


\begin{frame}{Expanze a redukt}
 
    \myblock{
    Mějme $L\subseteq L'$,  $L$-strukturu $\A$ a $L'$-strukturu $\A'$ na stejné doméně. Je-li interpretace každého symbolu z $L$ stejná v $\A$ i v $\A'$, potom:
    \begin{itemize}
        \item $\A'$ je \alert{expanze} $\A$ do $L'$ (\alert{$L'$-expanze} struktury $\A$)
        \item $\A$ je \alert{redukt} $\A'$ na $L$ (\alert{$L$-redukt} struktury $\A'$) 
    \end{itemize} 
    }  

    \medskip
    Například:
    \begin{itemize}
        \item Mějme \myexampleinline{grupu celých čísel $\langle\mathbb Z,+,-,0\rangle$}. Potom: 
        \begin{itemize}
            \item struktura $\langle \mathbb Z,+\rangle$ je její redukt
            \item struktura $\langle\mathbb Z,+,-,0,\cdot,1\rangle$ (\alert{okruh} celých čísel) je její expanze
        \end{itemize}
    
        \item Mějme \myexampleinline{graf $\mathcal G=\langle G, E^\mathcal G\rangle$}. Potom \alert{expanze $\mathcal G$ o jména prvků} (z množiny G) je struktura $\langle G, E^G,c_v^\mathcal G\rangle_{v\in G}$ v jazyce $\langle E,c_v\rangle_{v\in G}$, kde $c_v^\mathcal G=v$ pro všechny vrcholy $v\in G$.
    \end{itemize}

\end{frame}


\begin{frame}{Věta o konstantách}

    \begin{itemize}
        \item splnit formuli s volnou proměnnou $x$ je totéž, co splnit formuli, ve které je $x$ nahrazena \alert{novým} konstantním symbolem $c$
        \item proč: nový symbol lze v modelu interpretovat každým prvkem
        \item podobný trik využijeme v tablo metodě
    \end{itemize}

    \myblock{
    \textbf{Věta (O konstantách):}
    Mějme $L$-formuli $\varphi$ s volnými proměnnými $x_1,\dots,x_n$. Označme jako $L'$ rozšíření $L$ o nové konstantní symboly $c_1,\dots,c_n$ a buď $T'$ stejná teorie jako $T$, ale v jazyce $L'$. Potom:
    $$
    T\models\varphi\ \text{ právě když }\ T'\models\varphi(x_1/c_1,\dots,x_n/c_n)
    $$
    \vspace{-12pt}
    }

    \textbf{Důkaz:} stačí ukázat pro jednu volnou proměnnou, rozšířit indukcí

    \alert{\Large \bf $\Rightarrow$} \textbf{Víme:} $\varphi$ platí v každém modelu $T$. \textbf{Chceme:} $\varphi(x/c)$ platí v každém modelu $T'$. Mějme model $\A'\models T'$ a ohodnocení $e\colon\Var\to A'$ a ukažme, že \alert{$\A'\models\varphi(x/c)[e]$}.

\end{frame}


\begin{frame}{Pokračování důkazu}

    Buď $\A$ redukt $\A'$ na $L$ (`zapomeneme' konstantu $c^{\A'}$). Všimněte si, že \alert{$\A$ je model $T$} (axiomy $T=T'$ neobsahují \alert{nový} symbol $c$). Dle předpokladu $\A\models\varphi$, tj. $\A\models\varphi[e']$ pro \alert{libovolné} ohodnocení~$e'$, speciálně pro $e(x/c^{\A'})$ kde $x$ ohodnotíme  interpretací $c$ v $\A'$.
    
    Máme $\A\models\varphi[e(x/c^{\A'})]$, což ale znamená $\A'\models\varphi(x/c)[e]$.
    
    \alert{\Large \bf $\Leftarrow$} \textbf{Víme:} $\varphi(x/c)$ platí v každém modelu $T'$. \textbf{Chceme:} $\varphi$ platí v každém modelu $T$. Zvolme $\A\models T$ a ohodnocení $e\colon\Var\to A$ a ukažme, že \alert{$\A\models\varphi[e]$}.

    Buď $\A'$ expanze $\A$ do $L'$, kde $c$ interpretujeme jako $c^{\A'}=e(x)$. Dle předpokladu platí $\A'\models\varphi(x/c)[e']$ pro všechna ohodnocení $e'$. Tedy $\A'\models\varphi(x/c)[e]$, což znamená \alert{$\A'\models\varphi[e]$} (\myalertinline{
        $e=e(x/c^{\A'})$
        }, z toho plyne \myalertinline{
            $\A'\models\varphi(x/c)[e]\Leftrightarrow\A'\models\varphi[e(x/c^{\A'})]\Leftrightarrow\A'\models\varphi[e]$
            }).
    
    Formule $\varphi$ neobsahuje $c$ (je \alert{nový}), máme tedy i $\A\models\varphi[e]$. \hfill\qedsymbol
    
\end{frame}


\section{6.7 Extenze teorií}


\begin{frame}{Extenze teorie}

    Stejně jako ve výrokové logice, je-li $T$ teorie v jazyce $L$:

    \myblock{
        \begin{itemize}
            \item \alert{extenze:} $T'$ v jazyce $L'\supseteq L$ splňující $\Conseq_L(T)\subseteq\Conseq_{L'}(T')$
            \item \alert{jednoduchá:} $L'=L$
            \item \alert{konzervativní:} $\Conseq_L(T)=\Conseq_L(T')=\Conseq_{L'}(T')\cap \Fm_L$
            \item \alert{ekvivalentní:} $T'$ extenzí $T$ a $T$ extenzí $T'$ (obě v témž jazyce)
        \end{itemize}
    }
    
    Jsou-li $T,T'$ ve stejném jazyce $L$:
    \begin{itemize}
        \item $T'$ je extenze $T$, právě když $\M_L(T')\subseteq\M_L(T)$
        \item $T'$ je ekvivalentní s $T$, právě když $\M_L(T')=\M_L(T)$
    \end{itemize}
       
    Zvětšíme-li jazyk:
    \begin{itemize}
        \item \alert{ve výrokové logice:} přidáváme/zapomínáme hodnoty pro nové prvovýroky
        \item \alert{v predikátové logice:} expandujeme/redukujeme modely (přidáváme/zapomínáme nové relace, funkce, konstanty)
    \end{itemize}

\end{frame}


\begin{frame}{Extenze teorie: sémantický popis}

    \myblock{
        Mějme jazyky $L\subseteq L'$, $L$-teorii $T$ a $L'$-teorii $T'$:
        \begin{enumerate}[(i)]
            \item $T'$ je \alert{extenzí} $T$ $\Leftrightarrow$ $L$-redukt každého modelu $T'$ je model $T$
            \item $T'$ je \alert{konzervativní extenzí} $T$ $\Leftrightarrow$ $T'$ je extenzí $T$, a každý model $T$ lze expandovat do $L'$ na nějaký model $T'$
        \end{enumerate}        
    }
    
    {\small \textbf{Poznámka:} Důkaz \alert{(ii) \Large $\Rightarrow$} vynecháme (technický problém: model, který

    \vspace{-11pt} nelze expandovat $\rightsquigarrow$ $L$-sentence platná v $T$ ale ne v $T'$)}

    \vspace{-3pt}
    \textbf{Důkaz:}
    \alert{(i) \Large $\Rightarrow$} Buď $\A'$ model $T'$, $\A$ jeho $L$-redukt. Protože $T'$ je extenzí, platí v ní, tedy i v $\A'$, každý axiom $\varphi\in T$. Ten ale obsahuje jen symboly z $L$, tedy platí i v $\A$.
    
    \vspace{-3pt}
    \alert{(i) \Large $\Leftarrow$} \textbf{Mějme:} $L$-sentenci $\varphi$,  $T\models\varphi$. \textbf{Chceme:} $T'\models\varphi$. Pro lib. model $\A'\in\M_{L'}(T')$ víme, že jeho $L$-redukt $\A$ je modelem $T$, tedy $\A\models\varphi$. Z toho plyne i $\A'\models\varphi$ (opět $\varphi$ je v $L$).
    
    \vspace{-3pt}
    \alert{(ii) \Large $\Leftarrow$} \textbf{Mějme:} $L$-sentenci $\varphi$,  $T'\models\varphi$. \textbf{Chceme:} $T\models\varphi$. Každý $\A\in\M_L(T)$ lze expandovat na nějaký $\A'\in\M_{L'}(T')$. Víme, že $\A'\models\varphi$, takže i $\A\models\varphi$. Tím jsme dokázali $T\models\varphi$.    
    \hfill\qedsymbol
    
\end{frame}


\begin{frame}{Extenze o definice (neformálně)}

    \begin{itemize}
        %\item speciální případ extenze: \alert{extenze o definici}
        \item přidáme nový symbol, jehož význam je jednoznačně daný \alert{definující formulí} (jako procedura/funkce v programování)
        \item pro relační symboly jednoduché, pro funkční symboly musíme navíc zaručit \alert{existenci} a \alert{jednoznačnost} funkční hodnoty
    \end{itemize}

    Ukážeme:
    \begin{itemize}
        \item je to konzervativní extenze, dokonce každý model původní teorie lze \alert{jednoznačně} expandovat na model nové teorie
        \item každou formuli používající nové symboly lze přepsat na  formuli v původním jazyce (tak, že jsou v extenzi ekvivalentní)
    \end{itemize}

\end{frame}



\begin{frame}{Definice relačního symbolu}
    
    nový $n$-ární relační symbol $R$ lze definovat lib. formulí $\psi(x_1,\dots,x_n)$

    \begin{itemize}
        \item teorii v jazyce s rovností lze \myexampleinline{rozšířit o symbol $\neq$} \alert{definovaný} formulí \alert{$\neg x_1=x_2$}; tj. požadujeme, aby:\myalertinline{
            $x_1\neq x_2\liff\neg x_1=x_2$
            }
        \item teorii uspořádání lze \myexampleinline{rozšířit o $<$} definovaný formulí \alert{$x_1\leq x_2\land \neg x_1=x_2$}; tj. platí:\myalertinline{
            $x_1<x_2 \liff x_1\leq x_2\land \neg x_1=x_2$
        }
        \item v aritmetice lze \myexampleinline{zavést $\leq$} takto:\myalertinline{
            $x_1\leq x_2\liff(\exists y)(x_1+y=x_2)$
        }
        \item v uspořádaném stromu lze zavést unární predikát \myexampleinline{
            $\mathrm{Leaf}(x)$
        }:
        \myalertinline{
            $\mathrm{Leaf}(x)\liff\neg(\exists y)(x<_Ty)$
        }
    \end{itemize}

    \medskip

    \myblock{
    Mějme teorii $T$ a formuli $\psi(x_1,\dots,x_n)$ v jazyce $L$. Označme jako $L'$ rozšíření jazyka $L$ o nový $n$-ární relační symbol $R$. \alert{Extenze teorie $T$ o definici $R$ formulí $\psi$} je $L'$-teorie:
    $$
    T'=T\cup\{R(x_1,\dots,x_n)\ \liff\ \psi(x_1,\dots,x_n)\}
    $$
    }

\end{frame}


\begin{frame}{Definice relačního symbolu: vlastnosti}
    
    \myblock{
        \textbf{Tvrzení:}
        \begin{enumerate}[(i)]
            \item $T'$ je konzervativní extenze $T$.
            \item Pro každou $L'$-formuli $\varphi'$ existuje $L$-formule $\varphi$ taková, že $T'\models\varphi'\liff\varphi$.
        \end{enumerate}
        
    }

    \textbf{Důkaz:} \alert{(i)} ihned ze sémantického popisu extenzí, neboť zřejmě každý model $T$ lze \alert{jednoznačně} expandovat na model $T'$

    \alert{(ii)} atomickou podformuli s novým symbolem $R$, tj. tvaru \alert{$R(t_1,\dots,t_n)$}, nahradíme formulí
    $$
    \psi'(x_1/t_1,\dots,x_n/t_n)
    $$
    kde $\psi'$ je \alert{varianta $\psi$ zaručující substituovatelnost} všech termů (např. přejmenujeme všechny vázané proměnné $\psi$ na zcela nové)\hfill\qedsymbol

\end{frame}


\begin{frame}{Definice funkčního symbolu: příklady}
    
    \myblock{
    vztah $f(x_1,\dots,x_n)=y$ definujeme formulí $\psi(x_1,\dots,x_n,y)$; pro každý vstup $(x_1,\dots,x_n)$ musí \alert{existovat jednoznačný} výstup $y$
    }

    \bigskip

    1. \myexampleinline{Teorie grup:} binární funkční symbol \alert{$-_b$} pomocí $+$ a unárního $-$
        
        \myalertmath{
        $$
        x_1 -_b x_2 = y\ \liff\ x_1 + (-x_2) = y
        $$
        }

    \begin{itemize}
        \item zřejmě pro každá $x,y$ \alert{existuje} \alert{jednoznačné} $z$ splňující definici  
    \end{itemize}
        
    \bigskip
    
    2. \myexampleinline{Teorie \alert{lineárních uspořádání}:} binární funkční symbol \alert{$\min$}
        
        \myalertmath{
        \vspace{-12pt}
        $$
        \min(x_1,x_2)=y\ \liff\ y\leq x_1\land y\leq x_2\land (\forall z)(z\leq x_1\land z\leq x_2\limplies z\leq y)
        $$
        }

    \begin{itemize}
        \item existence a jednoznačnost platí díky linearitě ($x\leq y\lor y\leq x$)
        \item pouze v teorii uspořádání by nešlo o dobrou definici: $\min^\A(a_1,a_2)$ nemusí existovat
    \end{itemize}        
    
\end{frame}


\begin{frame}{Definice funkčního symbolu: definice}


    \myblock{
    Mějme teorii $T$ a formuli $\psi(x_1,\dots,x_n,y)$ v jazyce $L$. Označme  $L'$ rozšíření $L$ o nový $n$-ární funkční symbol $f$. Nechť platí:
    \begin{itemize}
        \item $T\models(\exists y)\psi(x_1,\dots,x_n,y)$ \hfill \alert{\footnotesize (existence)}
        \item $T\models\psi(x_1,\dots,x_n,y)\land \psi(x_1,\dots,x_n,z)\limplies y=z$ \hfill \alert{\footnotesize (jednoznačnost)}
    \end{itemize}
    Potom \alert{extenze teorie $T$ o definici $f$ formulí $\psi$} je $L'$-teorie:

    \vspace{-12pt}
    $$
    T'=T\cup\{f(x_1,\dots,x_n)=y\ \liff\ \psi(x_1,\dots,x_n,y)\}
    $$
    }

    \begin{itemize}
        \item $\psi$ definuje v modelu $(n+1)$-ární relaci, ta \alert{musí být funkcí}        
        \item je-li $\psi$ tvaru $t(x_1,\dots,x_n)=y$ pro term $t$, vždy to platí
    \end{itemize}

    \myblock{
        \textbf{Tvrzení:}
        \begin{enumerate}[(i)]
            \item $T'$ je konzervativní extenze $T$.
            \item Pro každou $L'$-formuli $\varphi'$ existuje $L$-formule $\varphi$ taková, že $T'\models\varphi'\liff\varphi$.
        \end{enumerate}        
    }

    \textbf{Důkaz:} \alert{(i)} modely $T$ lze \alert{jednoznačně} expandovat na modely $T'$

\end{frame}


\begin{frame}{Pokračování důkazu}

    \alert{(ii)} stačí pro jediný výskyt symbolu $f$, jinak induktivně (je-li více vnořených výskytů $f(\dots f(\dots)\dots)$, potom od vnitřních k vnějším)

    \begin{enumerate}
    \item  nahradíme term $f(t_1,\dots,t_n)$ \alert{novou} proměnnou $z$: \alert{výsledek $\varphi^*$}     
    \item $\varphi$ zkonstruujeme takto: \myalertinline{
     $(\exists z)(\varphi^*\land \psi'(x_1/t_1,\dots,x_n/t_n,y/z))$
     }
     (kde $\psi'$ je varianta $\psi$ zaručující substituovatelnost)
    \end{enumerate}

    Ukážeme, že pro libovolný model $\A\models T'$ a ohodnocení $e$ platí:
    $$
    \A\models\varphi'[e]\ \text{ právě když }\ \A\models\varphi[e]
    $$
      
    Označme \alert{$a=(f(t_1,\dots,t_n))^\A[e]$}. Díky existenci a jednoznačnosti:
    $$
    \A\models\psi'(x_1/t_1,\dots,x_n/t_n,y/z)[e]\ \text{ právě když }\ e(z)=a 
    $$
    Máme tedy: $\A\models\varphi'[e]$ $\Leftrightarrow$ $\A\models\varphi^*[e(z/a)]$ $\Leftrightarrow$ $\A\models\varphi[e]$ \hfill\qedsymbol    

\end{frame}


\begin{frame}{Definice konstantního symbolu}
    
    \begin{itemize}
        \item \alert{speciální případ:} funkční symbol arity $0$
        \item extenze o definici konstantního symbolu $c$ formulí $\psi(y)$:
        $$
        \alert{T'=T\cup\{c=y\liff \psi(y)\}}
        $$
        \item musí platit \myalertinline{
            $T\models (\exists y)\psi(y)$
        } a \myalertinline{
            $T\models\psi(y)\land\psi(z)\limplies y=z$
        }
        \item platí stejná tvrzení
    \end{itemize}

    \myexampleinline{1. teorie v jazyce aritmetiky}, rozšíříme o definici symbolu $1$ formulí $\psi(y)$ tvaru \alert{$y=S(0)$}, přidáme tedy axiom \myalertinline{
            $1=y\ \liff\ y=S(0)$
            }

    \myexampleinline{2. teorie těles}, nový symbol $\frac{1}{2}$, definice formulí \alert{$y\cdot (1+1)=1$}, tj. přidáním 
        \myalertinline{
            $\frac{1}{2}=y\ \liff\ y\cdot (1+1)=1$
        }?
        \begin{itemize}
            \item \myalertinline{není extenze o definici!} neplatí existence: v tělese \alert{charakteristiky 2}, např. $\mathbb Z_2$, nemá rovnice $y\cdot (1+1)=1$ řešení
            \item ale v teorii těles \alert{charakteristiky různé od 2}, tj. přidáme-li axiom $\neg (1+1=0)$, už ano; např. v $\mathbb Z_3$ máme $\frac{1}{2}^{\mathbb Z_3}=2$
        \end{itemize}        

\end{frame}


\begin{frame}{Extenze o definice}

    \myblock{
        $L'$-teorie $T'$ je \alert{extenzí}  $L$-teorie $T$ \alert{o definice}, pokud vznikla postupnou extenzí o definice relačních a funkčních (vč. konstantních) symbolů.
    } 

    \medskip
    
    \myblock{
    \textbf{Tvrzení:} (snadno indukcí)
        \begin{itemize}
            \item Každý model $T$ lze jednoznačně expandovat na model $T'$.
            \item $T'$ je konzervativní extenze $T$.
            \item Pro $L'$-formuli $\varphi'$ existuje $L$-formule $\varphi$, že $T'\models\varphi'\liff\varphi$.
        \end{itemize}
    }

    \bigskip

    \myexampleinline{
        Příklad: $T=\{(\exists y)(x+y=0),(x+y=0)\land(x+z=0)\limplies y=z\}$
    } 
    $L=\langle +,0,\leq\rangle$ s rovností, zavedeme \alert{\large $<$} a unární \alert{\large $-$} přidáním axiomů:
    \begin{align*}
        T'= T\cup\{ -x=y\ &\liff\ x+y=0,\\
        x<y\ &\liff\ x\leq y\land\neg(x=y)\}
    \end{align*}
    Formule \alert{$-x<y$} v jazyce $L'=\langle +,-,0,\leq,<\rangle$ s rovností je v $T'$ ekvivalentní formuli:
    \myalertinline{
        $(\exists z)((z\leq y\land\neg(z=y))\land x+z=0)$
    }

\end{frame}


\section{6.8 Definovatelnost ve struktuře}


\begin{frame}{Definovatelné množiny}

    \begin{itemize}
        \item formule $\varphi$ s jednou volnou proměnnou $x$ \dots ``vlastnost'' prvků
        \item ve struktuře \alert{definuje} množinu prvků, které vlastnost splňují (tj. prvků $a$ takových, že $\varphi$ platí při ohodnocení kde $e(x)=a$)
        \item $\varphi(x,y)$ definuje binární relaci, atp.
    \end{itemize}
    
    \myblock{
    Množina \alert{definovaná} $\varphi(x_1,\dots,x_n)$ \alert{ve struktuře} $\A$ (v témž jazyce):
    \vspace{-6pt}
    $$
    \alert{\varphi^\A(x_1,\dots,x_n)}=\{(a_1,\dots,a_n)\in A^n\mid\A\models\varphi[e(x_1/a_1,\dots,x_n/a_n)]\}
    $$
    \vspace{-16pt}
    }

    Zkráceně píšeme: \myalertinline{
        $\varphi^\A(\bar x)=\{\bar a\in A^n\mid\A\models\varphi[e(\bar x/\bar a)]\}$
    }

    \begin{itemize}
        \item formule \myexampleinline{
            $\neg(\exists y)E(x,y)$
         } definuje \myexampleinline{ v daném grafu } množinu všech \alert{izolovaných} vrcholů
        \item \myexampleinline{
            $(\exists y)(y\cdot y=x)\land\neg (x=0)$
        } definuje \myexampleinline{
            v tělese $\underline{\mathbb R}$
        } množinu všech kladných reálných čísel
        \item \myexampleinline{
            $x\leq y\land \neg (x=y)$
         } definuje v \myexampleinline{
            uspořádané množině $\langle S,\leq^S\rangle$
          } relaci \alert{ostrého uspořádání} $<^S$
    \end{itemize}

\end{frame}


\begin{frame}{Definovatelnost s parametry}

    \begin{itemize}
        \item vlastnosti prvků relativně k jiným prvkům? nelze čistě syntakticky, ale můžeme dosadit prvky jako \alert{parametry}
        \item zápis \alert{$\varphi(\bar x,\bar y)$}: volné proměnné $x_1,\dots,x_n,y_1,\dots,y_k$
    \end{itemize}

	\myblock{
    Mějme $\varphi(\bar x,\bar y)$ (kde $|\bar x|=n$, $|\bar y|=k$), 
    strukturu $\A$ (v témž jazyce), $\bar b\in A^k$. Množina \alert{definovaná} $\varphi(\bar x,\bar y)$ \alert{s parametry} $\bar b$ \alert{ve struktuře} $\A$:
    $$
    \varphi^{\A,\bar b}(\bar x,\bar y)=\{\bar a\in A^n\mid\A\models\varphi[e(\bar x/\bar a,\bar y/\bar b)]\}
    $$
    
    Pro $B\subseteq A$ označíme $\mathrm{Df}^n(\A,B)$ množinu všech množin definovatelných v $\A$ s parametry pocházejícími z $B$.
    }

    \textbf{Pozorování:} $\mathrm{Df}^n(\A,B)$ je uzavřená na doplněk, průnik, sjednocení, a obsahuje $\emptyset$ a $A^n$: je to \alert{podalgebra potenční algebry} $\mathcal P(A^n)$.

    \medskip
    Např. pro \myexampleinline{
        $\varphi(x,y)=E(x,y)$
     } a \myexampleinline{
        vrchol $v\in V(\mathcal G)$
      } je $\varphi^{\mathcal G,v}(x,y)$ množina všech sousedů vrcholu $v$.
    
\end{frame}


\begin{frame}{Aplikace: databázové dotazy}
    
    \begin{itemize}
        \item \alert{relační databáze}: jedna nebo více \alert{tabulek}, také \alert{relace}
        \item řádky tabulky jsou \alert{záznamy (records)}, také \alert{tice (tuples)}
        \item struktura v čistě relačním jazyce
    \end{itemize}

    \medskip

    {\ttfamily\scriptsize

        \hspace{2cm}{\bf\small Movies}
        
        \hspace{2cm}\begin{tabular}{lll}
            title          & director    & actor \\ \hline
            Forrest Gump   & R. Zemeckis & T. Hanks      \\
            Philadelphia   & J. Demme    & T. Hanks      \\
            Batman Returns & T. Burton   & M. Keaton     \\
            \vdots         & \vdots      & \vdots
        \end{tabular} 
 
        \hspace{2cm}{\bf\small Program}

        \hspace{2cm}\begin{tabular}{lll}
            cinema         & title          & time   \\ \hline
            Atlas          & Forrest Gump   & 20:00  \\
            Lucerna        & Forrest Gump   & 21:00  \\
            Lucerna        & Philadelphia   & 18:30  \\
            \vdots         & \vdots         & \vdots
        \end{tabular}

    }

\end{frame}


\begin{frame}{Příklad SQL dotazu}

    \begin{itemize}
        \item  SQL dotaz v nejjednodušší formě je formule (pomineme např. \alert{agregační funkce})
        \item výsledek je množina definovaná touto formulí (s parametry) 
       \end{itemize}
    
    \begin{center}
        \myexampleinline{``Kdy a kde můžeme vidět film s Tomem Hanksem?''}
    \end{center}

    \vspace{-6pt}
        
    {\tt\footnotesize
    \alert{select} Program.cinema, Program.time \alert{from} Program, Movies \alert{where} Program.title = Movies.title  \alert{and} Movies.actor = `T. Hanks' 
    }  
    
    \begin{itemize}
        \item výsledek je množina \myalertinline{
            $\varphi^{\text{Database},\text{`T. Hanks'}} (x_\mathrm{cinema},x_\mathrm{time},y_\mathrm{actor})$
        }
        \item definovaná ve struktuře \alert{$\text{Database}=\langle D, \mathrm{Program}, \mathrm{Movies}\rangle$}
        \item jejíž doména je {\small $D=\{\text{\tt`Atlas'},\text{\tt`Lucerna'},\dots,\text{\tt`M. Keaton'}\}$}
        \item s parametrem \alert{\small\tt`T. Hanks'}, 
        \item definující formule \alert{\small $\varphi(x_\mathrm{cinema},x_\mathrm{time},y_\mathrm{actor})$}:
        
        \medskip
        \myexampleamsmath{
        \begin{align*}
            (\exists y_\mathrm{title})(\exists y_\mathrm{director})(&\mathrm{Program}(x_\mathrm{cinema},y_\mathrm{title},x_\mathrm{time}) \land \\ &\mathrm{Movies}(y_\mathrm{title},y_\mathrm{director},y_\mathrm{actor}))          
        \end{align*}        
        }  
    \end{itemize}

\end{frame}


\section{6.9 Vztah výrokové a predikátové logiky}


\begin{frame}{Teorie Booleových algeber \hfill $L=\langle -,\landsymb,\lorsymb,\bot,\top\rangle$ s rovností}

    \myblock{\vspace{-9pt}
    \begin{columns}
        
        \footnotesize

        \column{0.5\textwidth}
        \begin{itemize}
            \item \alert{asociativita} $\land$ a $\lor$:
            \vspace{-9pt}\begin{align*}
                x\land(y\land z) &=(x\land y)\land z\\
                x\lor(y\lor z) &=(x\lor y)\lor z
            \end{align*}\vspace{-24pt}
            \item \alert{komutativita} $\land$ a $\lor$:
            \vspace{-9pt}\begin{align*}
                x\land y &= y\land x\\
                x\lor y &= y\lor x
            \end{align*}\vspace{-24pt}
            \item \alert{distributivita} $\land$ vůči $\lor$, $\lor$ vůči $\land$:
            \vspace{-9pt}\begin{align*}
                x\land(y\lor z) &= (x\land y)\lor (x\land z)\\
                x\lor(y\land z) &= (x\lor y)\land (x\lor z)
            \end{align*}\vspace{-24pt}
        \end{itemize}

        \column{0.5\textwidth}
        \begin{itemize}
            \item \alert{absorpce}:
            \vspace{-9pt}\begin{align*}
                x\land(x\lor y) &= x\\
                x\lor(x\land y) &= x
            \end{align*}\vspace{-24pt}
            \item \alert{komplementace}:
            \vspace{-9pt}\begin{align*}
                x\land(-x) &= \bot \\
                x\lor(-x) &= \top
            \end{align*}\vspace{-24pt}
            \item \alert{netrivialita}:
            \vspace{-9pt}\begin{align*}
                \neg (\bot &= \top)
            \end{align*}
        \end{itemize}
                

    \end{columns}
    \vspace{9pt}
    }

    \begin{itemize}
        \item dualita: záměnou $\land$ s $\lor$ a $\bot$ s $\top$ získáme tytéž axiomy    
        \item nejmenší model: \alert{2-prvková B. algebra} $\langle \{0,1\},f_\neg,f_\land,f_\lor,0,1\rangle$
        \item konečné modely, až na \alert{izomorfismus} ($f^n$ je $f$ po složkách):
        \myalertmath{
        $$
        \langle \{0,1\}^n,f_\neg^n,f_\land^n,f_\lor^n,(0,\dots,0),(1,\dots,1)\rangle
        $$
        }
        
        \item jsou izomorfní \alert{potenčním algebrám} $\mathcal P(\{1,\dots,n\})$ pomocí bijekce mezi podmnožinami a  charakteristickými vektory
    \end{itemize}
    
\end{frame}


\begin{frame}{Vztah výrokové a predikátové logiky}
   
    \begin{itemize}
        \item výrokovou logiku lze `simulovat' v predikátové logice v teorii Booleových algeber
        \item výroky jsou \alert{Booleovské termy}, konstanty $\bot,\top$ představují pravdu a lež
        \item pravdivostní hodnota výroku (při daném pravdivostním ohodnocení) je hodnota termu v 2-prvkové Booleově algebře
        \item kromě toho, \alert{algebra výroků} daného výrokového jazyka nebo teorie je Booleovou algebrou (i pro nekonečné jazyky)        
    \end{itemize}
    
\end{frame}

\begin{frame}{Na druhou stranu\dots}

    \begin{itemize}
        \item máme-li \alert{otevřenou} formuli $\varphi$ (bez rovnosti), můžeme reprezentovat atomické formule pomocí prvovýroků, a získat tak výrok, který platí, právě když platí $\varphi$
        \item viz Kapitola 8: Rezoluce v predikátové logice, kde se nejprve zbavíme kvantifikátorů pomocí tzv. \alert{Skolemizace}
        \item výrokovou logiku lze také zavést jako fragment logiky predikátové, pokud povolíme \alert{nulární relace}
        \item $A^0=\{\emptyset\}$, tedy na libovolné množině jsou právě dvě nulární relace $R^A\subseteq A^0$: $R^A=\emptyset=0$ a $R^A=\{\emptyset\}=\{0\}=1$
    \end{itemize}

    
        
\end{frame}


\end{document}


