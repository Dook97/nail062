\documentclass{beamer}

%% slide-specific

\usetheme[progressbar=frametitle]{metropolis}
%\usecolortheme{spruce}
%\metroset{block=fill}

% define Metropolis colors    
\definecolor{mAlert}{HTML}{EB811B}
\definecolor{mExample}{HTML}{14B03D}
\definecolor{mBlock}{HTML}{23373b}

% my blocks
\setlength\fboxsep{0pt}%

\newcommand{\myblock}[1]{\colorbox{mBlock!8}{\begin{minipage}{\linewidth}#1\end{minipage}}}
\newcommand{\myblockmath}[1]{\colorbox{mBlock!8}{\begin{minipage}{\linewidth}\vspace{-6pt}#1\end{minipage}}}
\newcommand{\myblockinline}[1]{\colorbox{mBlock!8}{#1}}
\newcommand{\myexample}[1]{\colorbox{mExample!8}{\begin{minipage}{\linewidth}#1\end{minipage}}}
\newcommand{\myexamplemath}[1]{\colorbox{mExample!8}{\begin{minipage}{\linewidth}\vspace{-6pt}#1\end{minipage}}}
\newcommand{\myexampleinline}[1]{\colorbox{mExample!8}{#1}}
\newcommand{\myalert}[1]{\colorbox{mAlert!8}{\begin{minipage}{\linewidth}#1\end{minipage}}}
\newcommand{\myalertmath}[1]{\colorbox{mAlert!8}{\begin{minipage}{\linewidth}\vspace{-6pt}#1\end{minipage}}}
\newcommand{\myalertinline}[1]{\colorbox{mAlert!8}{#1}}

%% other
\newcommand{\mystructure}[1]{\mathcal{#1}}




%% packages
\usepackage{amsmath,amssymb,amsthm}
\usepackage{booktabs}
\usepackage[czech]{babel}
\usepackage{enumerate}
\usepackage{forest}
\usepackage{multicol}
% \usepackage{tcolorbox}
\usepackage{tikz}
    \usetikzlibrary{arrows.meta}
%\usepackage[unicode]{hyperref}
\usepackage[utf8x]{inputenc}
\usepackage{xfrac}

% %% theorems
% \theoremstyle{plain}
%     \newtheorem{theorem}{Věta}[section]
%     \newtheorem*{theorem-unnumbered}{Věta}
%     \newtheorem{proposition}[theorem]{Tvrzení}
%     \newtheorem{corollary}[theorem]{Důsledek}
%     \newtheorem{lemma}[theorem]{Lemma}
%     \newtheorem{observation}[theorem]{Pozorování}
% \theoremstyle{definition}
%     \newtheorem{definition}[theorem]{Definice}
%     \newtheorem*{algorithm}{Algoritmus}
% \theoremstyle{remark}
%     \newtheorem{remark}[theorem]{Poznámka}
%     \newtheorem{example}[theorem]{Příklad}
%     \newtheorem{exercise}{Cvičení}[chapter]
%     \newtheorem*{solution}{Řešení}

%% macros and definitions
\DeclareMathOperator{\Aut}{Aut}
\DeclareMathOperator{\Conseq}{Csq}
\DeclareMathOperator{\DeLO}{DeLO}
\DeclareMathOperator{\dom}{dom}
\DeclareMathOperator{\Fm}{Fm}
\DeclareMathOperator{\M}{M}
%\DeclareMathOperator{\Proof}{Proof}
\DeclareMathOperator{\rng}{rng}
\DeclareMathOperator{\Term}{Term}
\DeclareMathOperator{\Th}{Th}
\DeclareMathOperator{\Thm}{Thm}
\DeclareMathOperator{\Tree}{Tree}
\DeclareMathOperator{\Var}{Var}
\DeclareMathOperator{\VF}{VF}

\newcommand{\A}{\structure{A}}
\newcommand{\B}{\structure{B}}
\newcommand{\Con}{\mathit{Con}}
\newcommand{\disjointunion}{\mathbin{\dot{\sqcup}}}
\newcommand{\F}{\ensuremath{\mathrm{F}}}
\newcommand{\landsymb}{{\land}}
\newcommand{\lbin}{\mathbin{\square}}
\newcommand{\lbinsymb}{{\lbin}}
\newcommand{\liff}{\mathbin{\leftrightarrow}}
\newcommand{\liffsymb}{{\liff}}
\newcommand{\limplies}{\mathbin{\rightarrow}}
\newcommand{\limpliessymb}{{\limplies}}
\newcommand{\lorsymb}{{\lor}}
\newcommand{\Prf}{\mathit{Prf}}
\newcommand{\proves}{\vdash}
%\newcommand{\structure}[1]{\mathcal{#1}}
\newcommand{\todo}{[TODO]}
\newcommand{\T}{\ensuremath{\mathrm{T}}}
\newcommand{\union}{\mathbin{\cup}}


\title{Sedmá přednáška}
\subtitle{NAIL062 Výroková a predikátová logika}
\author{Jakub Bulín (KTIML MFF UK)}
% \institute{KTIML MFF UK}
\date{Zimní semestr 2023}


\begin{document}


\frame{\titlepage}


\begin{frame}{Sedmá přednáška}

    \textbf{Program}
        \begin{itemize}
            \item podstruktury, expanze, redukty           
            \item extenze teorií, extenze o definice
            \item definovatelnost a databázové dotazy
            \item vztah výrokové a predikátové logiky
        \end{itemize}

    \textbf{Materiály}

        \href{https://github.com/jbulin-mff-uk/nail062/raw/main/lecture/lecture-notes/lecture-notes.pdf}{\alert{\textbf{Zápisky z přednášky}}}, Sekce 6.6-6.9 z Kapitoly 6

\end{frame}


\section{6.6 Podstruktura, expanze, redukt}


\begin{frame}{Podstruktura}

    \begin{itemize}
        \item \alert{podstruktura} zobecňuje podgrupu, podprostor vektorového prostoru, (indukovaný) podgraf: na podmnožině $B$ univerza vytvoříme strukturu, která \myalertinline{``zdědí'' relace, funkce a konstanty}
        \item $B$ musí být \alert{uzavřená} na všechny funkce (vč. konstant)
    \end{itemize}

    \myblock{
    Struktura $\B=\langle B,\mathcal R^\mathcal B,\mathcal F^\mathcal B\rangle$ je \alert{(indukovaná) podstruktura} struktury $\A=\langle A,\mathcal R^\mathcal A,\mathcal F^\mathcal A\rangle$ (v též signatuře $\langle\mathcal R,\mathcal F\rangle$), značíme \alert{$\B\subseteq\A$}, jestliže:

    \begin{itemize}
        \item $\emptyset\neq B\subseteq A$
        \item $R^\B=R^\A\cap B^{\mathrm{ar(R)}}$ pro každý relační symbol $R\in \mathcal R$
        \item $f^\B=f^\A\cap (B^{\mathrm{ar(f)}}\times B)$ pro každý funkční symbol $f\in \mathcal F$, tj. $f^\B$ je restrikce $f^\A$ na množinu $B$, a výstupy jsou všechny z $B$
    \end{itemize}
    }

    speciálně, pro konstantní symbol $c\in\mathcal F$ máme $c^\B=c^\A\in B$
    


\end{frame}


\begin{frame}{Restrikce na podmnožinu, příklady}
    
    Množina $C\subseteq A$ je \alert{uzavřená} na funkci $f:A^n\to A$, pokud $f(x_1,\dots,x_n)\in C$ pro všechna $x_i\in C$.

    \medskip

    \myblock{
    \textbf{Pozorování:} Množina $\emptyset\neq C\subseteq A$ je univerzem podstruktury, právě když je uzavřená na všechny funkce struktury $\A$ (včetně konstant). V tom případě je to \alert{restrikce} $\A$ na množinu $C$, značíme \alert{$\A\restriction C$}.
    }

    \medskip
    
    \begin{itemize}
        \item \myexampleinline{
            $\underline{\mathbb Z}=\langle \mathbb Z,+,\cdot,0\rangle$
         } je podstrukturou \myexampleinline{
            $\underline{\mathbb Q}=\langle \mathbb Q,+,\cdot,0\rangle$
            }, můžeme psát: \alert{$\underline{\mathbb Z}=\underline{\mathbb Q}\restriction\mathbb Z$}
        \item \myexampleinline{
            $\underline{\mathbb N}=\langle \mathbb N,+,\cdot,0\rangle$
         } je podstrukturou obou těchto struktur, platí: \alert{$\underline{\mathbb N}=\underline{\mathbb Q}\restriction\mathbb N=\underline{\mathbb Z}\restriction\mathbb N$}
        \item Množina \myexampleinline{
            $\{k\in\mathbb Z\mid k\leq 0\}$
         } není univerzem podstruktury $\underline{\mathbb Z}$ ani~$\underline{\mathbb Q}$, není uzavřená na násobení.
    \end{itemize}
\end{frame}


\begin{frame}{Platnost v podstruktuře (pro otevřené formule je zachována)}
    
    
    
    \myblock{
    \textbf{Pozorování:}
    Je-li $\B\subseteq\A$, $\varphi$ \alert{otevřená} formule, a \alert{$e\colon\Var\to B$}, potom platí: $\B\models\varphi[e]$ právě když $\A\models\varphi[e]$.
    }

    \textbf{Důkaz:}
    Snadno indukcí dle struktury $\varphi$, pro atomickou zřejmé.\hfill\qedsymbol
    
    \medskip

    \myblock{
    \textbf{Důsledek:}
    \alert{Otevřená} formule platí ve struktuře $\A$, právě když platí v každé podstruktuře $\B\subseteq\A$.
    }

    \medskip

    Teorie $T$ je \alert{otevřená}, jsou-li všechny její axiomy otevřené formule.

    \myblock{
    \textbf{Důsledek:}
    Modely otevřené teorie jsou uzavřené na podstruktury, tj. každá podstruktura modelu této teorie je také její model.
    }

    \medskip

    \begin{itemize}
        \item \myexampleinline{Teorie grafů} je otevřená. Podstruktura grafu je také graf: (indukovaný) \alert{podgraf}. Stejně podgrupy, Booleovy podalgebry.
        \item \myexampleinline{Teorie těles} není otevřená. Později ukážeme, že ani \alert{otevřeně axiomatizovatelná} (kvantifikátoru v axiomu o existenci inverzního prvku se nezbavíme). Podstruktura tělesa $\mathbb Q$ na množině $\mathbb Z$, $\mathbb Q\restriction\mathbb Z$, není těleso. (Je to tzv. \alert{okruh}.)
    \end{itemize}
    
\end{frame}


\begin{frame}{Generovaná podstruktura (zobecníme lineární obal vektorů)}
    
    Co když podmnožina univerza \alert{není} uzavřená? Vezmeme její \alert{uzávěr}. %(zobecňuje \alert{lineární obal} množiny vektorů).

    \medskip

    \myblock{
    Mějme $\A=\langle A,\mathcal R^\mathcal A,\mathcal F^\mathcal A\rangle$ a $\emptyset\neq X\subseteq A$. Buď $B\subseteq A$ nejmenší podmnožina, která obsahuje $X$ a je uzavřená na všechny funkce $\A$ (tj. obsahuje i všechny konstanty). Potom podstruktura $\A\restriction B$ je \alert{generovaná} $X$, značíme ji \alert{$\A\langle X\rangle$}.
    }

    \medskip

    Např. pro \myexampleinline{
        $\underline{\mathbb Q}=\langle\mathbb Q,+,\cdot,0\rangle$
    }, \myexampleinline{
        $\underline{\mathbb Z}=\langle\mathbb Z,+,\cdot,0\rangle$
    }, \myexampleinline{
        $\underline{\mathbb N}=\langle\mathbb N,+,\cdot,0\rangle$
    }: 
    \begin{itemize}
        \item $\underline{\mathbb Q}\langle\{1\}\rangle=\underline{\mathbb N}$
        \item $\underline{\mathbb Q}\langle\{-1\}\rangle=\underline{\mathbb Z}$
        \item $\underline{\mathbb Q}\langle\{2\}\rangle$ je podstruktura $\underline{\mathbb N}$ na množině všech sudých čísel
    \end{itemize}

    \medskip
    
    Pokud $\A$ nemá žádné funkce (ani konstanty), např. graf či uspořádání, potom není čím generovat, a $\A\langle X\rangle=\A\restriction X$.
    
\end{frame}


\begin{frame}{Expanze a redukt}
 
    \myblock{
    Mějme $L\subseteq L'$,  $L$-strukturu $\A$ a $L'$-strukturu $\A'$ na stejné doméně. Je-li interpretace každého symbolu z $L$ stejná v $\A$ i v $\A'$, potom:
    \begin{itemize}
        \item $\A'$ je \alert{expanze} $\A$ do $L'$ (\alert{$L'$-expanze} struktury $\A$)
        \item $\A$ je \alert{redukt} $\A'$ na $L$ (\alert{$L$-redukt} struktury $\A'$) 
    \end{itemize} 
    }  

    \medskip
    Například:
    \begin{itemize}
        \item Mějme \myexampleinline{grupu celých čísel $\langle\mathbb Z,+,-,0\rangle$}. Potom: 
        \begin{itemize}
            \item struktura $\langle \mathbb Z,+\rangle$ je její redukt
            \item struktura $\langle\mathbb Z,+,-,0,\cdot,1\rangle$ (\alert{okruh} celých čísel) je její expanze
        \end{itemize}
    
        \item Mějme \myexampleinline{graf $\mathcal G=\langle G, E^\mathcal G\rangle$}. Potom \alert{expanze $\mathcal G$ o jména prvků} (z množiny G) je struktura $\langle G, E^G,c_v^\mathcal G\rangle_{v\in G}$ v jazyce $\langle E,c_v\rangle_{v\in G}$, kde $c_v^\mathcal G=v$ pro všechny vrcholy $v\in G$.
    \end{itemize}

\end{frame}


\begin{frame}{Věta o konstantách}

    \begin{itemize}
        \item splnit formuli s volnou proměnnou $x$ je totéž, co splnit formuli, ve které je $x$ nahrazena \alert{novým} konstantním symbolem $c$
        \item proč: nový symbol lze v modelu interpretovat každým prvkem
        \item podobný trik využijeme v tablo metodě
    \end{itemize}

    \myblock{
    \textbf{Věta (O konstantách):}
    Mějme $L$-formuli $\varphi$ s volnými proměnnými $x_1,\dots,x_n$. Označme jako $L'$ rozšíření $L$ o nové konstantní symboly $c_1,\dots,c_n$ a buď $T'$ stejná teorie jako $T$, ale v jazyce $L'$. Potom:
    $$
    T\models\varphi\ \text{ právě když }\ T'\models\varphi(x_1/c_1,\dots,x_n/c_n)
    $$
    \vspace{-12pt}
    }

    \textbf{Důkaz:} stačí ukázat pro jednu volnou proměnnou, rozšířit indukcí

    \alert{\Large \bf $\Rightarrow$} \textbf{Víme:} $\varphi$ platí v každém modelu $T$. \textbf{Chceme:} $\varphi(x/c)$ platí v každém modelu $T'$. Mějme model $\A'\models T'$ a ohodnocení $e\colon\Var\to A'$ a ukažme, že \alert{$\A'\models\varphi(x/c)[e]$}.

\end{frame}


\begin{frame}{Pokračování důkazu}

    Buď $\A$ redukt $\A'$ na $L$ (`zapomeneme' konstantu $c^{\A'}$). Všimněte si, že \alert{$\A$ je model $T$} (axiomy $T=T'$ neobsahují \alert{nový} symbol $c$). Dle předpokladu $\A\models\varphi$, tj. $\A\models\varphi[e']$ pro \alert{libovolné} ohodnocení~$e'$, speciálně pro $e(x/c^{\A'})$ kde $x$ ohodnotíme  interpretací $c$ v $\A'$.
    
    Máme $\A\models\varphi[e(x/c^{\A'})]$, což ale znamená $\A'\models\varphi(x/c)[e]$.
    
    \alert{\Large \bf $\Leftarrow$} \textbf{Víme:} $\varphi(x/c)$ platí v každém modelu $T'$. \textbf{Chceme:} $\varphi$ platí v každém modelu $T$. Zvolme $\A\models T$ a ohodnocení $e\colon\Var\to A$ a ukažme, že \alert{$\A\models\varphi[e]$}.

    Buď $\A'$ expanze $\A$ do $L'$, kde $c$ interpretujeme jako $c^{\A'}=e(x)$. Dle předpokladu platí $\A'\models\varphi(x/c)[e']$ pro všechna ohodnocení $e'$. Tedy $\A'\models\varphi(x/c)[e]$, což znamená \alert{$\A'\models\varphi[e]$} (\myalertinline{
        $e=e(x/c^{\A'})$
        }, z toho plyne \myalertinline{
            $\A'\models\varphi(x/c)[e]\Leftrightarrow\A'\models\varphi[e(x/c^{\A'})]\Leftrightarrow\A'\models\varphi[e]$
            }).
    
    Formule $\varphi$ neobsahuje $c$ (je \alert{nový}), máme tedy i $\A\models\varphi[e]$. \hfill\qedsymbol
    
\end{frame}


\section{6.7 Extenze teorií}


\begin{frame}{Extenze teorie}

    Stejně jako ve výrokové logice, je-li $T$ teorie v jazyce $L$:

    \myblock{
        \begin{itemize}
            \item \alert{extenze:} $T'$ v jazyce $L'\supseteq L$ splňující $\Conseq_L(T)\subseteq\Conseq_{L'}(T')$
            \item \alert{jednoduchá:} $L'=L$
            \item \alert{konzervativní:} $\Conseq_L(T)=\Conseq_L(T')=\Conseq_{L'}(T')\cap \Fm_L$
            \item \alert{ekvivalentní:} $T'$ extenzí $T$ a $T$ extenzí $T'$ (obě v témž jazyce)
        \end{itemize}
    }
    
    Jsou-li $T,T'$ ve stejném jazyce $L$:
    \begin{itemize}
        \item $T'$ je extenze $T$, právě když $\M_L(T')\subseteq\M_L(T)$
        \item $T'$ je ekvivalentní s $T$, právě když $\M_L(T')=\M_L(T)$
    \end{itemize}
       
    Zvětšíme-li jazyk:
    \begin{itemize}
        \item \alert{ve výrokové logice:} přidáváme/zapomínáme hodnoty pro nové prvovýroky
        \item \alert{v predikátové logice:} expandujeme/redukujeme modely (přidáváme/zapomínáme nové relace, funkce, konstanty)
    \end{itemize}

\end{frame}


\begin{frame}{Extenze teorie: sémantický popis}

    \myblock{
        Mějme jazyky $L\subseteq L'$, $L$-teorii $T$ a $L'$-teorii $T'$:
        \begin{enumerate}[(i)]
            \item $T'$ je \alert{extenzí} $T$ $\Leftrightarrow$ $L$-redukt každého modelu $T'$ je model $T$
            \item $T'$ je \alert{konzervativní extenzí} $T$ $\Leftrightarrow$ $T'$ je extenzí $T$, a každý model $T$ lze expandovat do $L'$ na nějaký model $T'$
        \end{enumerate}
    }
    \textbf{Poznámka:} Důkaz \alert{(ii) \Large $\Leftarrow$} vynecháme (problém je jak získat z modelu $T$ který nelze expandovat na model $T'$ $L$-sentenci, která platí v $T$ ale ne v $T'$)
    % V části (ii) platí i opačná implikace, důkaz ale není tak jednoduchý, jako ve výrokové logice, a proto ho neuvedeme. (Problémem je jak získat z modelu $T$ který nelze expandovat na model $T'$ $L$-sentenci, která platí v $T$ ale ne v $T'$.)
     
    %\todo ukázat i opačnou implikaci?
    %$T'$ je konzervativní extenzí teorie $T$, právě když je extenzí, a každý model $T$ lze expandovat do jazyka $L'$ na nějaký model teorie $T'$.
    \textbf{Důkaz:}
    \alert{(i) \Large $\Rightarrow$} Buď $\A'$ model $T'$, $\A$ jeho $L$-redukt. Protože $T'$ je extenzí, platí v ní, tedy i v $\A'$, každý axiom $\varphi\in T$. Ten ale obsahuje jen symboly z $L$, tedy platí i v $\A$.
    
    \alert{(i) \Large $\Leftarrow$} \textbf{Mějme:} $L$-sentenci $\varphi$,  $T\models\varphi$. \textbf{Chceme:} $T'\models\varphi$. Pro lib. model $\A'\in\M_{L'}(T')$ víme, že jeho $L$-redukt $\A$ je modelem $T$, tedy $\A\models\varphi$. Z toho plyne i $\A'\models\varphi$ (opět $\varphi$ je v $L$).
    
    \alert{(ii) \Large $\Leftarrow$} \textbf{Mějme:} $L$-sentenci $\varphi$,  $T'\models\varphi$. \textbf{Chceme:} $T\models\varphi$. Každý $\A\in\M_L(T)$ lze expandovat na nějaký $\A'\in\M_{L'}(T')$. Víme, že $\A'\models\varphi$, takže i $\A\models\varphi$. Tím jsme dokázali $T\models\varphi$.
    
    \hfill\qedsymbol

    %TODO!
    
\end{frame}


\begin{frame}{Definice relačního symbolu}
    

\end{frame}


\begin{frame}{Definice funkčního symbolu}
    

\end{frame}


\begin{frame}{Definice konstantního symbolu}

   

\end{frame}


\begin{frame}{Extenze o definice}

    

\end{frame}


\section{6.8 Definovatelnost ve struktuře}


\begin{frame}{Definovatelné množiny}

    
\end{frame}


\begin{frame}{Definovatelnost s parametry}

    
\end{frame}


\begin{frame}{Aplikace: databázové dotazy}
    

\end{frame}


\section{6.9 Vztah výrokové a predikátové logiky}


\begin{frame}{Booleovy algebry}

    
\end{frame}


\begin{frame}{Vztah výrokové a predikátové logiky}
   

\end{frame}


\end{document}


