
\section{Axiomatizovatelnost}\label{section:axiomatizability}

Na závěr této kapitoly se podíváme, za jakých okolností lze `popsat' (\alert{axiomatizovat}) třídu modelů respektive teorii. Zajímat nás bude také kdy si vystačíme s konečně mnoha axiomy, a kdy to lze pomocí otevřených axiomů (kterých může být i nekonečně mnoho). Srovnejte s Tvrzením \ref{proposition:axiomatize-in-DNF-CNF} z výrokové logiky. 

\begin{definition}[Axiomatizovatelnost]
Mějme třídu struktur $K\subseteq\M_L$ v nějakém jazyce $L$. Říkáme, že $K$ je
\begin{itemize}
    \item \alert{axiomatizovatelná}, pokud existuje $L$-teorie $T$ taková, že $\M_L(T)=K$,
    \item \alert{konečně axiomatizovatelná}, pokud je axiomatizovatelná konečnou teorií, a
    \item \alert{otevřeně axiomatizovatelná}, pokud je axiomatizovatelná otevřenou teorií.
\end{itemize}
O $L$-teorii $T'$ říkáme, že je \alert{konečně} resp. \alert{otevřeně axiomatizovatelná}, pokud to platí o třídě modelů $K=\M_L(T')$.
\end{definition}

\begin{example}
    Uveďme několik příkladů:
    \begin{itemize}
        \item grafy nebo částečná uspořádání jsou konečně i otevřeně axiomatizovatelné,
        \item tělesa jsou konečně, ale ne otevřeně axiomatizovatelná,
        \item nekonečné grupy jsou axiomatizovatelné, ale ne konečně,
        \item konečné grafy nejsou axiomatizovatelné.
    \end{itemize}
    Proč tomu tak je ukážeme níže.
\end{example}

Začněme jednoduchým faktem:

\begin{observation}
    Je-li $K$ axiomatizovatelná, musí být uzavřená na elementární ekvivalenci.  
\end{observation}

Z věty o kompaktnosti snadno získáme následující tvrzení, pomocí kterého lze ukázat neaxiomatizovatelnost např. konečných grafů, konečných grup, konečných těles.

\begin{theorem}
    Pokud má teorie libovolně velké konečné modely, potom má i nekonečný model. V tom případě není třída všech jejích konečných modelů axiomatizovatelná.
\end{theorem}
\begin{proof}
    Je-li jazyk bez rovnosti, stačí vzít kanonický model pro některou bezespornou větev v tablu z $T$ pro položku $\F\bot$ ($T$ je bezesporná, neboť má model(y), tedy tablo není sporné).     
    
    Mějme jazyk s rovností a označme jako $T'$ následující extenzi teorie $T'$ do jazyka rozšířeného o spočetně mnoho nových konstantních symbolů $c_i$:
    $$
    T'=T \cup \{\neg c_i = c_j \mid i\neq j\in\mathbb N\}
    $$
    Každá konečná část teorie $T'$ má model: nechť $k$ je největší takové, že symbol $c_k$ se vyskytuje v $T'$. Potom stačí vzít libovolný alespoň $(k+1)$-prvkový model $T$ a interpretovat konstanty $c_0,\dots,c_k$ jako navzájem různé prvky tohoto modelu.

    Dle věty o kompaktnosti má potom i $T'$ model. Ten je nutně nekonečný. Jeho redukt na původní jazyk (zapomenutí konstant $c_i^\A$) je nekonečným modelem $T$.
\end{proof}

\begin{remark}
    Třída všech \alert{nekonečných} modelů teorie ale je vždy axiomatizovatelná, máme-li jazyk s rovností: stačí k teorii přidat pro každé $n\in\mathbb N$ axiom vyjadřující `existuje alespoň $n$ prvků'.
\end{remark}


\subsection{Konečná axiomatizovatelnost}

Ukážeme následující kritérium konečné axiomatizovatelnosti: jak třída struktur $K$ tak i $\overline{K}$ musí být axiomatizovatelné.

\begin{theorem}[O konečné axiomatizovatelnosti]\label{theorem:finite-axiomatizability}
    Mějme třídu struktur $K\subseteq \M_L$ a uvažme také její doplněk $\overline{K}=\M_L\setminus K$. Potom $K$ je konečně axiomatizovatelná, právě když $K$ i $\overline{K}$ jsou axiomatizovatelné.   
\end{theorem}
\begin{proof}
Je-li $K$ konečně axiomatizovatelná, potom je axiomatizovatelná i konečně mnoha sentencemi $\varphi_1,\dots,\varphi_n$ (nahradíme formule jejich generálními uzávěry). Jako axiomatizaci $\overline{K}$ stačí vzít sentenci $\psi=\neg(\varphi_1\land\varphi_2\land\dots\land\varphi_n)$. Zřejmě platí $\M(\psi)=\overline{K}$.

Naopak, nechť $T$ a $S$ jsou teorie takové, že $\M(T)=K$ a $\M(S)=\overline{K}$. Uvažme teorii $T\cup S$. Tato teorie je sporná, neboť:
$$
\M(T\cup S)=\M(T)\cap \M(S)=K\cap\overline{K}=\emptyset
$$
Podle věty o kompaktnosti\footnote{Vidíte, jak je užitečná!} existují konečné podteorie $T'\subseteq T$ a $S'\subseteq S$ takové, že:
$$
\emptyset = \M(T'\cup S')=\M(T')\cap \M(S')
$$
Nyní si všimněme, že platí
$$
\M(T)\subseteq \M(T')\subseteq \overline{\M(S')}\subseteq \overline{\M(S)}=\M(T)
$$
tím jsme dokázali, že $M(T)=M(T')$, tj. teorie $T'$ je hledanou konečnou axiomatizací $K$.
\end{proof}

Jako aplikaci si dokážeme, že tělesa charakteristiky 0 nejsou konečně axiomatizovatelná.

\subsubsection*{Příklad: tělesa charakteristiky 0}
 
Nechť $T$ je teorie těles. Charakteristika tělesa je nejmenší počet jedniček, které je třeba sečíst, abychom dostali nulu (v tom případě musí být charakteristika prvočíslo---dokažte si!), nebo, pokud nikdy nedostaneme sčítáním jedniček nulu, říkáme že je charakteristika 0. Trochu formálněji:

\begin{definition}[Charakteristika tělesa]
Říkáme, že těleso $\A=\langle A,+,-,0,\cdot,1 \rangle$ je
\begin{itemize}
    \item \alert{charakteristiky $p$}, je-li $p$ nejmenší prvočíslo takové, že $\A\models p1=0$, kde $p1$ označuje term $1+1+\dots+1$ s $p$ jedničkami, nebo
    \item \alert{charakteristiky 0}, pokud není charakteristiky $p$ pro žádné prvočíslo $p$.
\end{itemize}
\end{definition}
Nechť $T$ je teorie těles. Potom třída těles charakteristiky $p$ je konečně axiomatizována teorií $T\cup \{p1=0\}$. Třída těles charakteristiky 0 je axiomatizována následující (nekonečnou) teorií:
$$
T'=T\cup \{\neg\, p1=0\mid p\text{ je prvočíslo}\}
$$
Konečná axiomatizace ale neexistuje.

\begin{proposition}
Třída $K$ těles charakteristiky $0$ není konečně axiomatizovatelná.   
\end{proposition}
\begin{proof}
Díky Větě \ref{theorem:finite-axiomatizability} stačí ukázat, že $\overline{K}$ (sestávající z těles nenulové charakteristiky a struktur, které nejsou tělesa) není axiomatizovatelná, což dokážeme sporem. Nechť existuje teorie $S$ taková, že $\M(S)=\overline{K}$. Potom teorie 
$S'=S\cup T'$ má model, neboť každá její konečná část má model: stačí vzít těleso prvočíselné charakteristiky větší než jakékoliv $p$ z axiomu $T'$ tvaru $\neg\, p1=0$. Nechť $\A$ je model $S'$. Potom je i $\A\in\M(S)=\overline{K}$. Zároveň je ale $\A\in\M(T')=K$, což je spor.
\end{proof}

\subsection{Otevřená axiomatizovatelnost}

Pro otevřenou axiomatizovatelnost existuje jednoduché sémantické kritérium: třída jejích modelů musí být uzavřená na podstruktury. Platí dokonce ekvivalence, dokážeme ale jen jednu implikaci (důkaz druhé je obtížnější).

\begin{theorem}\label{theorem:open-axiomatizability}
Pokud je teorie $T$ otevřeně axiomatizovatelná, potom je každá podstruktura modelu $T$ také modelem $T$.   
\end{theorem}

\begin{remark}
    Platí i obrácená implikace: Je-li každá podstruktura modelu $T$ také modelem, potom je $T$ otevřeně axiomatizovatelná. Důkaz zde ale neuvedeme.
\end{remark}

\begin{proof}
Nechť $T'$ je otevřená axiomatizace $T$. Mějme model $\A\models T'$  a podstrukturu $\B\subseteq\A$. Pro každou formuli $\varphi\in T'$ platí $\B\models\varphi$ (neboť $\varphi$ je otevřená), tedy i $\B\models T'$.  
\end{proof}

\begin{example}
    Uveďme několik příkladů:
    \begin{itemize}
        \item Teorie DeLO není otevřeně axiomatizovatelná, například žádná konečná podstruktura modelu DeLO nemůže být hustá.
        \item Teorie těles není otevřeně axiomatizovatelná, podstruktura $\mathbb Z\subseteq\mathbb Q$ tělesa celých čísel není tělesem, v $\mathbb Z$ neexistuje inverzní prvek vůči násobení k číslu $2$.
        \item Pro dané $n\in\mathbb N$ jsou nejvýše $n$-prvkové grupy otevřeně axiomatizovatelné(podgrupy jsou jistě také nejvýše $n$-prvkové). Jako otevřenou axiomatizaci lze vzít následující extenzi (otevřené) teorie grup $T$:
        $$
        T\cup \{\bigvee_{1\leq i<j\leq n+1}x_i=x_j\}
        $$
    \end{itemize}
\end{example}
