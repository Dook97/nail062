\documentclass{beamer}

%% slide-specific

\usetheme[progressbar=frametitle]{metropolis}
%\usecolortheme{spruce}
%\metroset{block=fill}

% define Metropolis colors    
\definecolor{mAlert}{HTML}{EB811B}
\definecolor{mExample}{HTML}{14B03D}
\definecolor{mBlock}{HTML}{23373b}

% my blocks
\setlength\fboxsep{0pt}%

\newcommand{\myblock}[1]{\colorbox{mBlock!8}{\begin{minipage}{\linewidth}#1\end{minipage}}}
\newcommand{\myblockmath}[1]{\colorbox{mBlock!8}{\begin{minipage}{\linewidth}\vspace{-6pt}#1\end{minipage}}}
\newcommand{\myblockinline}[1]{\colorbox{mBlock!8}{#1}}
\newcommand{\myexample}[1]{\colorbox{mExample!8}{\begin{minipage}{\linewidth}#1\end{minipage}}}
\newcommand{\myexamplemath}[1]{\colorbox{mExample!8}{\begin{minipage}{\linewidth}\vspace{-6pt}#1\end{minipage}}}
\newcommand{\myexampleinline}[1]{\colorbox{mExample!8}{#1}}
\newcommand{\myalert}[1]{\colorbox{mAlert!8}{\begin{minipage}{\linewidth}#1\end{minipage}}}
\newcommand{\myalertmath}[1]{\colorbox{mAlert!8}{\begin{minipage}{\linewidth}\vspace{-6pt}#1\end{minipage}}}
\newcommand{\myalertinline}[1]{\colorbox{mAlert!8}{#1}}

%% other
\newcommand{\mystructure}[1]{\mathcal{#1}}




%% packages
\usepackage{amsmath,amssymb,amsthm}
\usepackage{booktabs}
\usepackage[czech]{babel}
\usepackage{enumerate}
\usepackage{forest}
\usepackage{multicol}
% \usepackage{tcolorbox}
\usepackage{tikz}
    \usetikzlibrary{arrows.meta}
%\usepackage[unicode]{hyperref}
\usepackage[utf8x]{inputenc}
\usepackage{xfrac}

% %% theorems
% \theoremstyle{plain}
%     \newtheorem{theorem}{Věta}[section]
%     \newtheorem*{theorem-unnumbered}{Věta}
%     \newtheorem{proposition}[theorem]{Tvrzení}
%     \newtheorem{corollary}[theorem]{Důsledek}
%     \newtheorem{lemma}[theorem]{Lemma}
%     \newtheorem{observation}[theorem]{Pozorování}
% \theoremstyle{definition}
%     \newtheorem{definition}[theorem]{Definice}
%     \newtheorem*{algorithm}{Algoritmus}
% \theoremstyle{remark}
%     \newtheorem{remark}[theorem]{Poznámka}
%     \newtheorem{example}[theorem]{Příklad}
%     \newtheorem{exercise}{Cvičení}[chapter]
%     \newtheorem*{solution}{Řešení}

%% macros and definitions
\DeclareMathOperator{\Aut}{Aut}
\DeclareMathOperator{\Conseq}{Csq}
\DeclareMathOperator{\DeLO}{DeLO}
\DeclareMathOperator{\dom}{dom}
\DeclareMathOperator{\Fm}{Fm}
\DeclareMathOperator{\M}{M}
%\DeclareMathOperator{\Proof}{Proof}
\DeclareMathOperator{\rng}{rng}
\DeclareMathOperator{\Term}{Term}
\DeclareMathOperator{\Th}{Th}
\DeclareMathOperator{\Thm}{Thm}
\DeclareMathOperator{\Tree}{Tree}
\DeclareMathOperator{\Var}{Var}
\DeclareMathOperator{\VF}{VF}

\newcommand{\A}{\structure{A}}
\newcommand{\B}{\structure{B}}
\newcommand{\Con}{\mathit{Con}}
\newcommand{\disjointunion}{\mathbin{\dot{\sqcup}}}
\newcommand{\F}{\ensuremath{\mathrm{F}}}
\newcommand{\landsymb}{{\land}}
\newcommand{\lbin}{\mathbin{\square}}
\newcommand{\lbinsymb}{{\lbin}}
\newcommand{\liff}{\mathbin{\leftrightarrow}}
\newcommand{\liffsymb}{{\liff}}
\newcommand{\limplies}{\mathbin{\rightarrow}}
\newcommand{\limpliessymb}{{\limplies}}
\newcommand{\lorsymb}{{\lor}}
\newcommand{\Prf}{\mathit{Prf}}
\newcommand{\proves}{\vdash}
%\newcommand{\structure}[1]{\mathcal{#1}}
\newcommand{\todo}{[TODO]}
\newcommand{\T}{\ensuremath{\mathrm{T}}}
\newcommand{\union}{\mathbin{\cup}}


\title{Šestá přednáška}
\subtitle{NAIL062 Výroková a predikátová logika}
\author{Jakub Bulín (KTIML MFF UK)}
% \institute{KTIML MFF UK}
\date{Zimní semestr 2023}


\begin{document}


\frame{\titlepage}


\begin{frame}{Šestá přednáška}

    \textbf{Program}
        \begin{itemize}
            \item úvod do predikátové logiky
            \item syntaxe a sémantika predikátové logiky
            \item vlastnosti teorií
        \end{itemize}

    \textbf{Materiály}

        \href{https://github.com/jbulin-mff-uk/nail062/raw/main/lecture/lecture-notes/lecture-notes.pdf}{\alert{\textbf{Zápisky z přednášky}}}, Sekce 6.1-6.5 z Kapitoly 6

\end{frame}


\section{ČÁST II -- PREDIKÁTOVÁ LOGIKA}


\section{\sc Kapitola 6: Syntaxe a sémantika predikátové logiky}


\section{6.1 Úvod}


\begin{frame}{Predikátová logika neformálně}

    \textbf{Výroková logika:} popis světa pomocí \alert{výroků} složených z \alert{prvovýroků} (\alert{výrokových proměnných}) -- bitů informace
    
    \textbf{Predikátová logika [prvního řádu]:}
    \begin{itemize}
        \item základní stavební kámen jsou \alert{proměnné} reprezentující \alert{individua} -- nedělitelné objekty z nějaké množiny (např. přirozená čísla, vrcholy grafu, stavy mikroprocesoru)
        \item tato individua mají určité vlastnosti a vzájemné vztahy (\alert{relace}), kterým říkáme \alert{predikáty}
        \begin{itemize}
            \item \myexampleinline{
                $\mathrm{Leaf}(x)$
                }nebo \myexampleinline{
                    $\mathrm{Edge}(x,y)$
                    } mluvíme-li o grafu
            \item \myexampleinline{
                $x\leq y$
                } v přirozených číslech
        \end{itemize}
        \item a mohou vstupovat do \alert{funkcí}
        \begin{itemize}
            \item \myexampleinline{
                $\mathrm{lowest\_common\_ancestor}(x,y)$
                } v zakořeněném stromu
            \item \myexampleinline{
                $\mathrm{succ}(x)$
                } nebo \myexampleinline{
                    $x+y$
                    } v přirozených číslech
        \end{itemize}
        \item a mohou být \alert{konstantami} se speciálním významem, např. \myexampleinline{
            $\mathrm{root}$
            } v zakořeněném stromu, 
            \myexampleinline{
                $0$
            } v tělese. 
    \end{itemize}    

\end{frame}


\begin{frame}{Syntaxe neformálně}

    \begin{itemize}
        \item \alert{atomické formule}: predikát (včetně \alert{rovnosti} $=$) o proměnných nebo o \alert{termech} (`výrazy' složené z funkcí popř. konstant)
        \item \alert{formule} jsou složené z atomických formulí pomocí logických spojek, a dvou \alert{kvantifikátorů}:
    \end{itemize}  

    \medskip
    \myalert{
        $\forall x$ ``pro všechna individua (reprezentovaná proměnnou $x$)''
            
        $\exists x$ ``existuje individuum (reprezentované proměnnou $x$)''
    }

    \bigskip

    Např. \myexampleinline{\textit{``Každý, kdo má dítě, je rodič.''}} lze formalizovat takto:
    \myexamplemath{
    $$
    (\forall x)((\exists y)\mathrm{child\_of}(y,x)\limplies\mathrm{is\_parent}(x))
    $$
    }

    \begin{itemize}
        \item \alert{$\mathrm{child\_of}(y,x)$} je binární predikát vyjadřující, že individuum reprezentované proměnnou $y$ je dítětem individua reprezentovaného proměnnou $x$
        \item \alert{$\mathrm{is\_parent}(x)$} je unární predikát vyjadřující, že individuum reprezentované $x$ je rodič
    \end{itemize}
    
\end{frame}


\begin{frame}{Sémantika neformálně}

    \myexamplemath{
    $$
    (\forall x)((\exists y)\mathrm{child\_of}(y,x)\limplies\mathrm{is\_parent}(x))
    $$
    }

    Platnost? Záleží na \alert{modelu} světa/systému, který nás zajímá:
    
    \alert{Model} je\dots
    \begin{itemize}
        \item (neprázdná) množina individuí, spolu
        \item s binární relací \alert{interpretující} binární relační symbol \alert{$\mathrm{child\_of}$}, a
        \item s unární relací (tj. podmnožinou) interpretující unární relační symbol \alert{$\mathrm{is\_parent}$}        
    \end{itemize}
    Obecně mohou být relace jakékoliv, snadno sestrojíme model, ve kterém formule neplatí, např. 
    $$
    \mystructure{A}=\langle\{0,1\},\{(0,0),(0,1),(1,0),(1,1)\},\emptyset\rangle
    $$

\end{frame}


\begin{frame}{Příklad s funkcemi a konstantami}

    ``Je-li $x_1\leq y_1$ a $x_2\leq y_2$, potom platí $(y_1 \cdot y_2)-(x_1\cdot x_2)\geq 0$.''

    \myexampleamsmath{
        $$
        \varphi=(x_1\leq y_1)\land (x_2\leq y_2)\limplies ((y_1 \cdot y_2)+(-(x_1\cdot x_2))\geq 0)
        $$
    }

    \begin{itemize}
        \item dva binární relační symboly ($\leq,\geq$), binární funkční symbol $+$, unární funkční symbol $-$, a konstantní symbol $0$
        \item \alert{model, ve kterém $\varphi$ platí:} $\mathbb N$ s binárními relacemi $\leq^\mathbb N,\geq^\mathbb N$, bin. funkcemi $+^\mathbb N,\cdot^\mathbb N$, unární funkcí $-^\mathbb N$, a konstantou $0^\mathbb N=0$ 
        \item vezmeme-li ale podobně množinu $\mathbb Z$, $\varphi$ už platit nebude
    \end{itemize}
    Poznámky:
    \begin{itemize}
        \item mohli bychom chápat `$-$' jako binární, obvykle ale bývá unární
        \item pro \alert{konstantní symbol} $0$ používáme (jak je zvykem) stejný symbol, jako pro přirozené číslo 0. Ale pozor, v našem modelu může být \alert{symbol} $0$ interpretován jako \alert{jiné číslo}, nebo náš model vůbec nemusí sestávat z čísel!
    \end{itemize}

\end{frame}


\begin{frame}{Ještě o syntaxi}

    \myexampleamsmath{
        $$
        \varphi=(x_1\leq y_1)\land (x_2\leq y_2)\limplies ((y_1 \cdot y_2)+(-(x_1\cdot x_2))\geq 0)
        $$
    }

    \begin{itemize}
        \item $\varphi$ nemá žádné kvantifikátory, tj. je \alert{otevřená}
        \item $x_1,x_2,y_1,y_2$ jsou \alert{volné proměnné} této formule (nejsou \alert{vázané} žádným kvantifikátorem), píšeme $\varphi(x_1,x_2,y_1,y_2)$
        \item sémantiku $\varphi$ chápeme stejně jako $(\forall x_1)(\forall x_2)(\forall y_1)(\forall y_2)\varphi$
        \item používáme \alert{konvence} (infixový zápis, vynechání závorek), jinak:
        $$
        \varphi=((\leq (x_1,y_1) \land \leq(x_2,y_2))\limplies \leq(+(\cdot (y_1,y_2),-(\cdot(x_1,x_2))),0))
        $$
        \item cvičení: definujte \alert{strom formule}, nakreslete ho pro $\varphi$
    \end{itemize}
   
\end{frame}


\begin{frame}{Termy vs. atomické formule}
    
    \myexampleamsmath{
        $$
        \varphi=(x_1\leq y_1)\land (x_2\leq y_2)\limplies ((y_1 \cdot y_2)+(-(x_1\cdot x_2))\geq 0)
        $$
    }
    
    \begin{itemize}
        \item výraz \alert{$(y_1 \cdot y_2)+(-(x_1\cdot x_2))$} je \alert{term}
        \item výrazy \alert{$(x_1\leq y_1)$}, \alert{$(x_2\leq y_2)$} a \alert{$((y_1 \cdot y_2)+(-(x_1\cdot x_2))\geq 0)$} jsou (všechny) \alert{atomické (pod)formule} $\varphi$ 
    \end{itemize}
   
    V čem je rozdíl? Máme-li konkrétní model, a konkrétní \alert{ohodnocení proměnných} individui (prvky) tohoto modelu:
    
    \begin{itemize}
        \item výsledkem termu (při daném ohodnocení proměnných) je konkrétní \alert{individuum z modelu}, zatímco
        \item atomickým formulí lze přiřadit \alert{pravdivostní hodnotu} (a tedy kombinovat je logickými spojkami)
    \end{itemize}   

\end{frame}


\section{6.2 Struktury}


\begin{frame}{Signatura}

    \begin{itemize}
        \item specifikuje jakého \alert{typu} bude daná struktura, tj. jaké  má relace, funkce (jakých arit) a konstanty, a symboly pro ně 
        \item \alert{konstanty} lze chápat jako funkce arity 0, tj. funkce bez vstupů
    \end{itemize}

    \myblock{
        \alert{Signatura} je dvojice $\langle\mathcal R,\mathcal F\rangle$, kde $\mathcal R,\mathcal F$ jsou disjunktní množiny symbolů (\alert{relační} a \alert{funkční}, ty zahrnují \alert{konstantní}) spolu s danými aritami (tj. danými funkcí $\mathrm{ar}\colon \mathcal R\cup\mathcal F\to\mathbb N$) a neobsahující symbol `$=$' (ten je rezervovaný pro \alert{rovnost}). 
    }

    \bigskip

    \begin{itemize}
        \item často zapíšeme jen výčtem symbolů, jsou-li arity a zda jsou relační nebo funkční zřejmé
        \item kromě běžně používaných symbolů typicky používáme:
        \begin{itemize}
            \item pro relační symboly $P,Q,R,\dots$
            \item pro funkční (nekonstantní) symboly $f,g,h,\dots$
            \item pro konstantní symboly $c,d,a,b,\dots$
        \end{itemize}
    \end{itemize}
    

\end{frame}


\begin{frame}{Příklady signatur}

    \begin{itemize}
        \item \alert{$\langle E \rangle$} signatura \alert{grafů}: $E$ je binární relační symbol (struktury jsou uspořádané grafy)
        \item \alert{$\langle \leq \rangle$} signatura \alert{částečných uspořádání}: stejná jako signatura grafů, jen jiný symbol (ne každá struktura v této signatuře je částečné uspořádání! k tomu musí splňovat příslušné \alert{axiomy})
        \item \alert{$\langle +, -, 0\rangle$} signatura \alert{grup}: $+$ je binární funkční, $-$ unární funkční, $0$ konstantní symbol
        \item \alert{$\langle +, -, 0,\cdot,1\rangle$} signatura \alert{těles}: $\cdot$ je binární funkční, $1$ konstantní symbol
        \item \alert{$\langle +, -, 0,\cdot,1,\leq\rangle$} signatura \alert{uspořádaných těles}: $\leq$ je binární relační symbol
        \item \alert{$\langle -,\landsymb,\lorsymb,\bot,\top\rangle$} signatura \alert{Booleových algeber}: $\landsymb,\lorsymb$ jsou binární funkční, $\bot,\top$ jsou konstantní symboly
        \item \alert{$\langle S,+,\cdot,0,\leq\rangle$} signatura \alert{aritmetiky}: $S$ je unární funkční symbol
    \end{itemize}
    
\end{frame}


\begin{frame}{Struktury}

    \alert{Strukturu} dané signatury získáme tak, že:
    \begin{itemize}
        \item zvolíme neprázdnou \alert{doménu}, a na ní
        \item zvolíme \alert{realizace} (také říkáme \alert{interpretace}) všech relačních a funkčních symbolů (a konstant)
        \item to znamená \alert{konkrétní} relace resp. funkce příslušných arit
        \item realizací konstantního symbolu je zvolený prvek z domény
        \item na tom, jaké konkrétní symboly jsou v signatuře nezáleží (např. $+$ neznamená, že realizace musí souviset se sčítáním)
    \end{itemize}

\end{frame}


\begin{frame}{Příklady struktur: čistě relační}
    
    \begin{itemize}
        \item Struktura \alert{v prázdné signatuře} $\langle\ \rangle$ je libovolná neprázdná množina. (Nemusí být konečná, ani spočetná! Formálně to bude trojice $\langle A,\emptyset,\emptyset\rangle$, ale rozdíl zanedbáme.)            

        \item Struktura \alert{v signatuře grafů} je $\mathcal G=\langle V,E\rangle$, kde $V\neq\emptyset$ a $E\subseteq V^2$, říkáme jí \alert{orientovaný graf}. 
        \begin{itemize}
            \item je-li $E$ ireflexivní a symetrická, je to \alert{jednoduchý} graf
            \item je-li $E$ reflexivní, tranzitivní, a antisymetrická, jde o \alert{částečné uspořádání}
            \item je-li $E$ reflexivní, tranzitivní, a symetrická, je to \alert{ekvivalence}
        \end{itemize}
        \item Struktury \alert{v signatuře částečných uspořádání} jsou tytéž, jako v signatuře grafů, signatury se liší jen symbolem. (Ne každá struktura v signatuře částečných uspořádání je č. uspořádání!)
    \end{itemize}

\end{frame}


\begin{frame}{Příklady struktur: čistě funkční}
    
    Struktury \alert{v signatuře grup} jsou například následující \alert{grupy}:
        \begin{itemize}
            \item $\underline{\mathbb Z_n}=\langle\mathbb Z_n,+,-,0\rangle$, \alert{aditivní grupa celých čísel modulo $n$} (operace jsou modulo $n$). 
            
            \smallskip
            
            \textbf{Poznámka:} $\underline{\mathbb Z_n}$ znamená strukturu, zatímco $\mathbb Z_n$ jen její doménu. Často se to ale nerozlišuje a $\mathbb Z_n$ se používá i pro strukturu. Podobně $+,-,0$ jsou jak symboly, tak interpretace.

            \smallskip

            \item $\mathcal S_n=\langle \mathrm{Sym}_n,\circ,{}^{-1},\mathrm{id}\rangle$ je \alert{symetrická grupa} (grupa všech permutací) na $n$ prvcích.
            \item $\underline{\mathbb Q}^*=\langle \mathbb Q\setminus\{0\},\cdot,{}^{-1},1\rangle$ je \alert{multiplikativní grupa (nenulových) racionálních čísel}. \myalertinline{(Interpretací \alert{symbolu} $0$ je \alert{číslo} $1$!)}
        \end{itemize}
        Všechny tyto struktury \alert{splňují axiomy teorie grup}, snadno ale najdeme jiné, které axiomy nesplňují, nejsou tedy grupami.
    

\end{frame}


\begin{frame}{Příklady struktur: relace i funkce}
    
    \begin{itemize}
        \item Struktury $\underline{\mathbb Q}=\langle \mathbb Q, +, -, 0,\cdot,1,\leq\rangle$ a $\underline{\mathbb Z}=\langle \mathbb Z, +, -, 0,\cdot,1,\leq\rangle$ (se standardními operacemi a uspořádáním) jsou \alert{v signatuře uspořádaných těles} (ale jen první z nich je uspořádané těleso).
        \item $\underline{\mathcal P(X)}=\langle \mathcal P(X),\bar{},\cap,\cup,\emptyset,X\rangle$, tzv. \alert{potenční algebra} nad množinou $X$, je struktura \alert{v signatuře Booleových algeber}. (\alert{Booleova algebra} je to pokud $X\neq\emptyset$.)
        \item $\underline{\mathbb N}=\langle \mathbb N,S,+,\cdot,0,\leq\rangle$, kde $S(x)=x+1$, a ostatní symboly jsou interpretovány standardně, je \alert{standardní model aritmetiky}.
    \end{itemize}

\end{frame}



\begin{frame}{Definice struktury}
    
    \myblock{
    \alert{Struktura v signatuře} $\langle\mathcal R,\mathcal F\rangle$ je trojice $\A=\langle A, \mathcal R^\A,\mathcal F^\A \rangle$, kde
    \begin{itemize}
    \item  $A$ je neprázdná množina, říkáme jí \alert{doména} (také \alert{univerzum}),
    \item $\mathcal R^\A=\{R^\A\mid R\in\mathcal R\}$ kde $R^\A\subseteq A^{\mathrm{ar}(R)}$ je \alert{interpretace} relačního symbolu $R$,
    \item $\mathcal F^\A=\{f^\A\mid f\in\mathcal F\}$ kde $f^\A\colon A^{\mathrm{ar}(f)}\to A$ je \alert{interpretace} funkčního symbolu $f$ (speciálně pro konstantní symbol $c\in\mathcal F$ máme $c^\A\in A$).
    \end{itemize}
    }

    \medskip

    \myexampleinline{Příklad:} rozmyslete si, jak vypadají struktury v \alert{signatuře $n$ konstant} $\langle c_1,c_2,\dots,c_n\rangle$? Popište všechny 5-prvkové v signatuře 3 konstant. 

\end{frame}




\section{6.3 Syntaxe}


\section{6.4 Sémantika}


\section{6.5 Vlastnosti teorií}


\end{document}


