\documentclass{beamer}

%% slide-specific

\usetheme[progressbar=frametitle]{metropolis}
%\usecolortheme{spruce}
%\metroset{block=fill}

% define Metropolis colors    
\definecolor{mAlert}{HTML}{EB811B}
\definecolor{mExample}{HTML}{14B03D}
\definecolor{mBlock}{HTML}{23373b}

% my blocks
\setlength\fboxsep{0pt}%

\newcommand{\myblock}[1]{\colorbox{mBlock!8}{\begin{minipage}{\linewidth}#1\end{minipage}}}
\newcommand{\myblockmath}[1]{\colorbox{mBlock!8}{\begin{minipage}{\linewidth}\vspace{-6pt}#1\end{minipage}}}
\newcommand{\myblockinline}[1]{\colorbox{mBlock!8}{#1}}
\newcommand{\myexample}[1]{\colorbox{mExample!8}{\begin{minipage}{\linewidth}#1\end{minipage}}}
\newcommand{\myexamplemath}[1]{\colorbox{mExample!8}{\begin{minipage}{\linewidth}\vspace{-6pt}#1\end{minipage}}}
\newcommand{\myexampleinline}[1]{\colorbox{mExample!8}{#1}}
\newcommand{\myalert}[1]{\colorbox{mAlert!8}{\begin{minipage}{\linewidth}#1\end{minipage}}}
\newcommand{\myalertmath}[1]{\colorbox{mAlert!8}{\begin{minipage}{\linewidth}\vspace{-6pt}#1\end{minipage}}}
\newcommand{\myalertinline}[1]{\colorbox{mAlert!8}{#1}}

%% other
\newcommand{\mystructure}[1]{\mathcal{#1}}




%% packages
\usepackage{amsmath,amssymb,amsthm}
\usepackage{booktabs}
\usepackage[czech]{babel}
\usepackage{enumerate}
\usepackage{forest}
\usepackage{multicol}
% \usepackage{tcolorbox}
\usepackage{tikz}
    \usetikzlibrary{arrows.meta}
%\usepackage[unicode]{hyperref}
\usepackage[utf8x]{inputenc}
\usepackage{xfrac}

% %% theorems
% \theoremstyle{plain}
%     \newtheorem{theorem}{Věta}[section]
%     \newtheorem*{theorem-unnumbered}{Věta}
%     \newtheorem{proposition}[theorem]{Tvrzení}
%     \newtheorem{corollary}[theorem]{Důsledek}
%     \newtheorem{lemma}[theorem]{Lemma}
%     \newtheorem{observation}[theorem]{Pozorování}
% \theoremstyle{definition}
%     \newtheorem{definition}[theorem]{Definice}
%     \newtheorem*{algorithm}{Algoritmus}
% \theoremstyle{remark}
%     \newtheorem{remark}[theorem]{Poznámka}
%     \newtheorem{example}[theorem]{Příklad}
%     \newtheorem{exercise}{Cvičení}[chapter]
%     \newtheorem*{solution}{Řešení}

%% macros and definitions
\DeclareMathOperator{\Aut}{Aut}
\DeclareMathOperator{\Conseq}{Csq}
\DeclareMathOperator{\DeLO}{DeLO}
\DeclareMathOperator{\dom}{dom}
\DeclareMathOperator{\Fm}{Fm}
\DeclareMathOperator{\M}{M}
%\DeclareMathOperator{\Proof}{Proof}
\DeclareMathOperator{\rng}{rng}
\DeclareMathOperator{\Term}{Term}
\DeclareMathOperator{\Th}{Th}
\DeclareMathOperator{\Thm}{Thm}
\DeclareMathOperator{\Tree}{Tree}
\DeclareMathOperator{\Var}{Var}
\DeclareMathOperator{\VF}{VF}

\newcommand{\A}{\structure{A}}
\newcommand{\B}{\structure{B}}
\newcommand{\Con}{\mathit{Con}}
\newcommand{\disjointunion}{\mathbin{\dot{\sqcup}}}
\newcommand{\F}{\ensuremath{\mathrm{F}}}
\newcommand{\landsymb}{{\land}}
\newcommand{\lbin}{\mathbin{\square}}
\newcommand{\lbinsymb}{{\lbin}}
\newcommand{\liff}{\mathbin{\leftrightarrow}}
\newcommand{\liffsymb}{{\liff}}
\newcommand{\limplies}{\mathbin{\rightarrow}}
\newcommand{\limpliessymb}{{\limplies}}
\newcommand{\lorsymb}{{\lor}}
\newcommand{\Prf}{\mathit{Prf}}
\newcommand{\proves}{\vdash}
%\newcommand{\structure}[1]{\mathcal{#1}}
\newcommand{\todo}{[TODO]}
\newcommand{\T}{\ensuremath{\mathrm{T}}}
\newcommand{\union}{\mathbin{\cup}}


\title{Šestá přednáška}
\subtitle{NAIL062 Výroková a predikátová logika}
\author{Jakub Bulín (KTIML MFF UK)}
% \institute{KTIML MFF UK}
\date{Zimní semestr 2023}


\begin{document}


\frame{\titlepage}


\begin{frame}{Šestá přednáška}

    \textbf{Program}
        \begin{itemize}
            \item úvod do predikátové logiky
            \item syntaxe a sémantika predikátové logiky
            \item vlastnosti teorií
        \end{itemize}

    \textbf{Materiály}

        \href{https://github.com/jbulin-mff-uk/nail062/raw/main/lecture/lecture-notes/lecture-notes.pdf}{\alert{\textbf{Zápisky z přednášky}}}, Sekce 6.1-6.5 z Kapitoly 6

\end{frame}


\section{ČÁST II -- PREDIKÁTOVÁ LOGIKA}


\section{\sc Kapitola 6: Syntaxe a sémantika predikátové logiky}


\section{6.1 Úvod}


\begin{frame}{Predikátová logika neformálně}

    \textbf{Výroková logika:} popis světa pomocí \alert{výroků} složených z \alert{prvovýroků} (\alert{výrokových proměnných}) -- bitů informace
    
    \textbf{Predikátová logika [prvního řádu]:}
    \begin{itemize}
        \item základní stavební kámen jsou \alert{proměnné} reprezentující \alert{individua} -- nedělitelné objekty z nějaké množiny (např. přirozená čísla, vrcholy grafu, stavy mikroprocesoru)
        \item tato individua mají určité vlastnosti a vzájemné vztahy (\alert{relace}), kterým říkáme \alert{predikáty}
        \begin{itemize}
            \item \myexampleinline{
                $\mathrm{Leaf}(x)$
                }nebo \myexampleinline{
                    $\mathrm{Edge}(x,y)$
                    } mluvíme-li o grafu
            \item \myexampleinline{
                $x\leq y$
                } v přirozených číslech
        \end{itemize}
        \item a mohou vstupovat do \alert{funkcí}
        \begin{itemize}
            \item \myexampleinline{
                $\mathrm{lowest\_common\_ancestor}(x,y)$
                } v zakořeněném stromu
            \item \myexampleinline{
                $\mathrm{succ}(x)$
                } nebo \myexampleinline{
                    $x+y$
                    } v přirozených číslech
        \end{itemize}
        \item a mohou být \alert{konstantami} se speciálním významem, např. \myexampleinline{
            $\mathrm{root}$
            } v zakořeněném stromu, 
            \myexampleinline{
                $0$
            } v tělese. 
    \end{itemize}    

\end{frame}


\begin{frame}{Syntaxe neformálně}

    \begin{itemize}
        \item \alert{atomické formule}: predikát (včetně \alert{rovnosti} $=$) o proměnných nebo o \alert{termech} (`výrazy' složené z funkcí popř. konstant)
        \item \alert{formule} jsou složené z atomických formulí pomocí logických spojek, a dvou \alert{kvantifikátorů}:
    \end{itemize}  

    \medskip
    \myalert{
        $\forall x$ ``pro všechna individua (reprezentovaná proměnnou $x$)''
            
        $\exists x$ ``existuje individuum (reprezentované proměnnou $x$)''
    }

    \bigskip

    Např. \myexampleinline{\textit{``Každý, kdo má dítě, je rodič.''}} lze formalizovat takto:
    \myexamplemath{
    $$
    (\forall x)((\exists y)\mathrm{child\_of}(y,x)\limplies\mathrm{is\_parent}(x))
    $$
    }

    \begin{itemize}
        \item \alert{$\mathrm{child\_of}(y,x)$} je binární predikát vyjadřující, že individuum reprezentované proměnnou $y$ je dítětem individua reprezentovaného proměnnou $x$
        \item \alert{$\mathrm{is\_parent}(x)$} je unární predikát vyjadřující, že individuum reprezentované $x$ je rodič
    \end{itemize}
    
\end{frame}


\begin{frame}{Sémantika neformálně}

    \myexamplemath{
    $$
    (\forall x)((\exists y)\mathrm{child\_of}(y,x)\limplies\mathrm{is\_parent}(x))
    $$
    }

    Platnost? Záleží na \alert{modelu} světa/systému, který nás zajímá:
    
    \alert{Model} je\dots
    \begin{itemize}
        \item (neprázdná) množina individuí, spolu
        \item s binární relací \alert{interpretující} binární relační symbol \alert{$\mathrm{child\_of}$}, a
        \item s unární relací (tj. podmnožinou) interpretující unární relační symbol \alert{$\mathrm{is\_parent}$}        
    \end{itemize}
    Obecně mohou být relace jakékoliv, snadno sestrojíme model, ve kterém formule neplatí, např. 
    $$
    \mystructure{A}=\langle\{0,1\},\{(0,0),(0,1),(1,0),(1,1)\},\emptyset\rangle
    $$

\end{frame}


\begin{frame}{Příklad s funkcemi a konstantami}

    ``Je-li $x_1\leq y_1$ a $x_2\leq y_2$, potom platí $(y_1 \cdot y_2)-(x_1\cdot x_2)\geq 0$.''

    \myexampleamsmath{
        $$
        \varphi=(x_1\leq y_1)\land (x_2\leq y_2)\limplies ((y_1 \cdot y_2)+(-(x_1\cdot x_2))\geq 0)
        $$
    }

    \begin{itemize}
        \item dva binární relační symboly ($\leq,\geq$), binární funkční symbol $+$, unární funkční symbol $-$, a konstantní symbol $0$
        \item \alert{model, ve kterém $\varphi$ platí:} $\mathbb N$ s binárními relacemi $\leq^\mathbb N,\geq^\mathbb N$, bin. funkcemi $+^\mathbb N,\cdot^\mathbb N$, unární funkcí $-^\mathbb N$, a konstantou $0^\mathbb N=0$ 
        \item vezmeme-li ale podobně množinu $\mathbb Z$, $\varphi$ už platit nebude
    \end{itemize}
    Poznámky:
    \begin{itemize}
        \item mohli bychom chápat `$-$' jako binární, obvykle ale bývá unární
        \item pro \alert{konstantní symbol} $0$ používáme (jak je zvykem) stejný symbol, jako pro přirozené číslo 0. Ale pozor, v našem modelu může být \alert{symbol} $0$ interpretován jako \alert{jiné číslo}, nebo náš model vůbec nemusí sestávat z čísel!
    \end{itemize}

\end{frame}


\begin{frame}{Ještě o syntaxi}

    \myexampleamsmath{
        $$
        \varphi=(x_1\leq y_1)\land (x_2\leq y_2)\limplies ((y_1 \cdot y_2)+(-(x_1\cdot x_2))\geq 0)
        $$
    }

    \begin{itemize}
        \item $\varphi$ nemá žádné kvantifikátory, tj. je \alert{otevřená}
        \item $x_1,x_2,y_1,y_2$ jsou \alert{volné proměnné} této formule (nejsou \alert{vázané} žádným kvantifikátorem), píšeme $\varphi(x_1,x_2,y_1,y_2)$
        \item sémantiku $\varphi$ chápeme stejně jako $(\forall x_1)(\forall x_2)(\forall y_1)(\forall y_2)\varphi$
        \item používáme \alert{konvence} (infixový zápis, vynechání závorek), jinak:
        $$
        \varphi=((\leq (x_1,y_1) \land \leq(x_2,y_2))\limplies \leq(+(\cdot (y_1,y_2),-(\cdot(x_1,x_2))),0))
        $$
        \item cvičení: definujte \alert{strom formule}, nakreslete ho pro $\varphi$
    \end{itemize}
   
\end{frame}


\begin{frame}{Termy vs. atomické formule}
    
    \myexampleamsmath{
        $$
        \varphi=(x_1\leq y_1)\land (x_2\leq y_2)\limplies ((y_1 \cdot y_2)+(-(x_1\cdot x_2))\geq 0)
        $$
    }
    
    \begin{itemize}
        \item výraz \alert{$(y_1 \cdot y_2)+(-(x_1\cdot x_2))$} je \alert{term}
        \item výrazy \alert{$(x_1\leq y_1)$}, \alert{$(x_2\leq y_2)$} a \alert{$((y_1 \cdot y_2)+(-(x_1\cdot x_2))\geq 0)$} jsou (všechny) \alert{atomické (pod)formule} $\varphi$ 
    \end{itemize}
   
    V čem je rozdíl? Máme-li konkrétní model, a konkrétní \alert{ohodnocení proměnných} individui (prvky) tohoto modelu:
    
    \begin{itemize}
        \item výsledkem termu (při daném ohodnocení proměnných) je konkrétní \alert{individuum z modelu}, zatímco
        \item atomickým formulí lze přiřadit \alert{pravdivostní hodnotu} (a tedy kombinovat je logickými spojkami)
    \end{itemize}   

\end{frame}


\section{6.2 Struktury}


\begin{frame}{Signatura}

    \begin{itemize}
        \item specifikuje jakého \alert{typu} bude daná struktura, tj. jaké  má relace, funkce (jakých arit) a konstanty, a symboly pro ně 
        \item \alert{konstanty} lze chápat jako funkce arity 0, tj. funkce bez vstupů
    \end{itemize}

    \myblock{
        \alert{Signatura} je dvojice $\langle\mathcal R,\mathcal F\rangle$, kde $\mathcal R,\mathcal F$ jsou disjunktní množiny symbolů (\alert{relační} a \alert{funkční}, ty zahrnují \alert{konstantní}) spolu s danými aritami (tj. danými funkcí $\mathrm{ar}\colon \mathcal R\cup\mathcal F\to\mathbb N$) a neobsahující symbol `$=$' (ten je rezervovaný pro \alert{rovnost}). 
    }

    \bigskip

    \begin{itemize}
        \item často zapíšeme jen výčtem symbolů, jsou-li arity a zda jsou relační nebo funkční zřejmé
        \item kromě běžně používaných symbolů typicky používáme:
        \begin{itemize}
            \item pro relační symboly $P,Q,R,\dots$
            \item pro funkční (nekonstantní) symboly $f,g,h,\dots$
            \item pro konstantní symboly $c,d,a,b,\dots$
        \end{itemize}
    \end{itemize}
    

\end{frame}


\begin{frame}{Příklady signatur}

    \begin{itemize}
        \item \myexampleinline{
        $\langle E \rangle$
        } signatura \alert{grafů}: $E$ je binární relační symbol (struktury jsou uspořádané grafy)
        \item \myexampleinline{
        $\langle \leq \rangle$
        } signatura \alert{částečných uspořádání}: stejná jako signatura grafů, jen jiný symbol (ne každá struktura v této signatuře je částečné uspořádání! k tomu musí splňovat příslušné \alert{axiomy})
        \item \myexampleinline{
        $\langle +, -, 0\rangle$
        } signatura \alert{grup}: $+$ je binární funkční, $-$ unární funkční, $0$ konstantní symbol
        \item \myexampleinline{
        $\langle +, -, 0,\cdot,1\rangle$
        } signatura \alert{těles}: $\cdot$ je binární funkční, $1$ konstantní symbol
        \item \myexampleinline{
        $\langle +, -, 0,\cdot,1,\leq\rangle$
        } signatura \alert{uspořádaných těles}: $\leq$ je binární relační symbol
        \item \myexampleinline{
        $\langle -,\landsymb,\lorsymb,\bot,\top\rangle$
        } signatura \alert{Booleových algeber}: $\landsymb,\lorsymb$ jsou binární funkční, $\bot,\top$ jsou konstantní symboly
        \item \myexampleinline{
        $\langle S,+,\cdot,0,\leq\rangle$
        } signatura \alert{aritmetiky}: $S$ je unární funkční symbol
    \end{itemize}
    
\end{frame}


\begin{frame}{Struktury}

    \alert{Strukturu} dané signatury získáme tak, že:
    \begin{itemize}
        \item zvolíme neprázdnou \alert{doménu}, a na ní
        \item zvolíme \alert{realizace} (také říkáme \alert{interpretace}) všech relačních a funkčních symbolů (včetně konstantních)
        \item to znamená \alert{konkrétní} relace resp. funkce příslušných arit
        \item realizací konstantního symbolu je zvolený prvek z domény
        \item na tom, jaké konkrétní symboly jsou v signatuře nezáleží (např. $+$ neznamená, že realizace musí souviset se sčítáním)
    \end{itemize}

\end{frame}


\begin{frame}{Příklady struktur 1/3}
    
    \begin{itemize}
        \item Struktura \alert{v prázdné signatuře} $\langle\ \rangle$ je libovolná neprázdná množina. (Nemusí být konečná, ani spočetná! Formálně to bude trojice $\langle A,\emptyset,\emptyset\rangle$, ale rozdíl zanedbáme.)            

        \item Struktura \alert{v signatuře grafů} je $\mathcal G=\langle V,E\rangle$, kde $V\neq\emptyset$ a $E\subseteq V^2$, říkáme jí \alert{orientovaný graf}. 
        \begin{itemize}
            \item je-li $E$ ireflexivní a symetrická, je to \alert{jednoduchý} graf
            \item je-li $E$ reflexivní, tranzitivní, a antisymetrická, jde o \alert{částečné uspořádání}
            \item je-li $E$ reflexivní, tranzitivní, a symetrická, je to \alert{ekvivalence}
        \end{itemize}
        \item Struktury \alert{v signatuře částečných uspořádání} jsou tytéž, jako v signatuře grafů, signatury se liší jen symbolem. (Ne každá struktura v signatuře částečných uspořádání je č. uspořádání!)
    \end{itemize}

\end{frame}


\begin{frame}{Příklady struktur 2/3}
    
    Struktury \alert{v signatuře grup} jsou například následující \alert{grupy}:
        \begin{itemize}
            \item \myexampleinline{
                $\underline{\mathbb Z_n}=\langle\mathbb Z_n,+,-,0\rangle$
            }, \alert{aditivní grupa celých čísel modulo $n$} (operace jsou modulo $n$). 
            
            \smallskip
            
            \textbf{Poznámka:} $\underline{\mathbb Z_n}$ znamená strukturu, zatímco $\mathbb Z_n$ jen její doménu. Často se to ale nerozlišuje a $\mathbb Z_n$ se používá i pro strukturu. Podobně $+,-,0$ jsou jak symboly, tak interpretace.

            \smallskip

            \item \myexampleinline{
                $\mathcal S_n=\langle \mathrm{Sym}_n,\circ,{}^{-1},\mathrm{id}\rangle$
             } je \alert{symetrická grupa} (grupa všech permutací) na $n$ prvcích.
            \item \myexampleinline{
                $\underline{\mathbb Q}^*=\langle \mathbb Q\setminus\{0\},\cdot,{}^{-1},1\rangle$
                } je \alert{multiplikativní grupa (nenulových) racionálních čísel}. \myalertinline{(Interpretací \alert{symbolu} $0$ je \alert{číslo} $1$!)}
        \end{itemize}
        Všechny tyto struktury \alert{splňují axiomy teorie grup}, snadno ale najdeme jiné, které axiomy nesplňují, nejsou tedy grupami.
    

\end{frame}


\begin{frame}{Příklady struktur 3/3}
    
    \begin{itemize}
        \item Struktury \myexampleinline{
            $\underline{\mathbb Q}=\langle \mathbb Q, +, -, 0,\cdot,1,\leq\rangle$
         }a\myexampleinline{
            $\underline{\mathbb Z}=\langle \mathbb Z, +, -, 0,\cdot,1,\leq\rangle$
          } (se standardními operacemi a uspořádáním) jsou \alert{v signatuře uspořádaných těles} (ale jen první z nich je uspořádané těleso).
        \item \myexampleinline{
            $\underline{\mathcal P(X)}=\langle \mathcal P(X),\bar{},\cap,\cup,\emptyset,X\rangle$
        }, tzv. \alert{potenční algebra} nad množinou $X$, je struktura \alert{v signatuře Booleových algeber}. (\alert{Booleova algebra} je to pokud $X\neq\emptyset$.)
        \item \myexampleinline{
            $\underline{\mathbb N}=\langle \mathbb N,S,+,\cdot,0,\leq\rangle$
        }, kde $S(x)=x+1$, a ostatní symboly jsou realizovány standardně, je \alert{standardní model aritmetiky}.
    \end{itemize}

\end{frame}


\begin{frame}{Definice struktury}
    
    \myblock{
    \alert{Struktura v signatuře} $\langle\mathcal R,\mathcal F\rangle$ je trojice $\A=\langle A, \mathcal R^\A,\mathcal F^\A \rangle$, kde
    \begin{itemize}
    \item  $A$ je neprázdná množina, říkáme jí \alert{doména} (také \alert{univerzum}),
    \item $\mathcal R^\A=\{R^\A\mid R\in\mathcal R\}$ kde $R^\A\subseteq A^{\mathrm{ar}(R)}$ je \alert{interpretace} relačního symbolu $R$,
    \item $\mathcal F^\A=\{f^\A\mid f\in\mathcal F\}$ kde $f^\A\colon A^{\mathrm{ar}(f)}\to A$ je \alert{interpretace} funkčního symbolu $f$ (speciálně pro konstantní symbol $c\in\mathcal F$ máme $c^\A\in A$).
    \end{itemize}
    }

    \medskip

    \myexampleinline{Příklad:} rozmyslete si, jak vypadají struktury v \alert{signatuře $n$ konstant} $\langle c_1,c_2,\dots,c_n\rangle$? Popište všechny 5-prvkové v signatuře 3 konstant. 

\end{frame}


\section{6.3 Syntaxe}


\begin{frame}{Jazyk}

    \myblock{
    Jazyk je daný \alert{signaturou} a informací, zda je \alert{s rovností} nebo ne. 
    }    

    Tj. specifikujeme `typ' modelů a zda můžeme používat symbol `\alert{$=$}' interpretovaný jako \alert{identita} prvků z domény; většinou to dovolíme. (Je-li jazyk bez  rovnosti, musí mít signatura relační symbol. Proč?)
    
    Do jazyka patří:
    \begin{itemize}
        \item spočetně mnoho \alert{proměnných} $x_0,x_1,x_2,\dots$ (píšeme také $x,y,z,\dots$; množinu všech proměnných označíme \alert{$\Var$})
        \item \alert{relační}, \alert{funkční} a \alert{konstantní symboly} ze signatury, symbol \alert{$=$} jde-li o jazyk s rovností (to jsou `\alert{mimologické}' symboly)
        \item \alert{univerzální} a \alert{existenční} \alert{kvantifikátory} $(\forall x),(\exists x)$ pro každou proměnnou $x\in\Var$ (kvantifikátor `$(\forall x)$' chápeme jako jediný symbol, tj. \alert{neobsahuje} proměnnou $x$)
        \item symboly pro log. spojky \alert{\( \neg,\landsymb,\lorsymb, \limpliessymb, \liffsymb \)}, závorky \alert{\( (,) \)}, a čárka `\alert{$,$}'
    \end{itemize}

\end{frame}


\begin{frame}{Jazyk: příklady}

    
    \begin{itemize}
        \item Jazyk \alert{$L=\langle\rangle$ s rovností} je jazyk \alert{čisté rovnosti}
        \item jazyk $L=\langle c_0,c_1,c_2,\dots\rangle$ s rovností je jazyk \alert{spočetně mnoha konstant}
        \item jazyk \alert{uspořádání} je $\langle \leq \rangle$ s rovností
        \item jazyk \alert{teorie grafů} je $\langle E \rangle$ s rovností
        \item jazyky \alert{teorie grup, teorie těles, teorie uspořádaných těles, Booleových algeber, aritmetiky} jsou jazyky \alert{s rovností} odpovídající daným signaturám
    \end{itemize}
    
\end{frame}


\begin{frame}{Termy}

    \vspace{-6pt}
    \alert{čistě syntaktické} `výrazy' z proměnných, konstantních symbolů, funkčních symbolů, závorek a čárek
    
    \myblock{
        \alert{Termy} jazyka $L$ jsou konečné nápisy definované induktivně:
        \begin{itemize}
            \item každá proměnná a každý konstantní symbol z $L$ je term,
            \item je-li $f$ funkční symbol z $L$ arity $n$ a jsou-li $t_1,\dots,t_n$ termy, potom nápis $f(t_1,t_2,\dots,t_n)$ je také term.
        \end{itemize}
        Množinu všech \alert{termů} jazyka $L$ označíme $\Term_L$. 
    }

    \begin{itemize}
        \item \alert{podterm} je podřetězec, který je sám termem
        \item term bez proměnných je \alert{konstantní (ground)}, např. \myexampleinline{$((S(0)+S(0))\cdot S(S(0)))$} v jazyce aritmetiky
        \item \myalertinline{termy nesmí obsahovat prvky struktury, jen symboly z jazyka}
        \item \myexampleinline{$(1+1)\cdot 2$} \alert{není} term, ledaže rozšíříme jazyk o \alert{symboly} $1$ a $2$
        \item jako lidé můžeme použít \alert{infixový} zápis, např. $(t_1+t_2)$ místo $+(t_1,t_2)$, vynechat závorky je-li struktura termu zřejmá
    \end{itemize}

\end{frame}


\begin{frame}{Strom termu}
    
    \vspace{-3pt}
    \alert{Strom termu} $t$, \alert{$\Tree(t)$}: v listech proměnné nebo konst. symboly, ve vnitřních vrcholech funkční symboly (arita je rovna počtu synů)

    \begin{center}

    \scalebox{0.8}{
        \tikzset{every label/.style = {text=red}}
        \begin{minipage}{.49\textwidth}
            \centering
            \begin{forest}
                for tree={math content,circle,draw=blue!20,fill=blue!10,minimum size=22pt}
                [\cdot 
                    [+ 
                        [S
                            [0]                    
                        ] 
                        [x]
                    ]
                    [y]
                ]
            \end{forest}
    
            (a) $(S(0) + x) \cdot y$ v jazyce aritmetiky
        \end{minipage}
        \begin{minipage}{.49\textwidth}
            \centering
            \begin{forest}
                for tree={math content,circle,draw=blue!20,fill=blue!10,minimum size=22pt}
                [\lor 
                    [\neg 
                        [\land
                            [x]
                            [y]                    
                        ]
                    ]
                    [\bot]
                ]
            \end{forest}
            
            (b) $\neg (x\land y)\lor \bot$ v jazyce Booleových algeber 
        \end{minipage}
    }
    \end{center}

    \begin{itemize}
        \item symboly $\neg,\land,\lor$ nejsou logické, ale mimologické ze signatury
        \item \alert{sémantika}: proměnné ohodnotíme prvky, konst. a funkční symboly nahradíme interpretacemi, výsledek je prvek z domény
    \end{itemize}
    
\end{frame}


\begin{frame}{Atomické formule}
    Termům nelze přiřadit \alert{pravdivostní hodnotu}, potřebujeme \alert{predikát} (relační symbol nebo $=$), který mluví \alert{o `vztahu' termů}: v dané struktuře při ohodnocení proměnných prvky je buď splněn, nebo ne.

    \medskip

    \alert{Formule} (`tvrzení o strukturách') skládáme z \alert{atomických formulí} pomocí  logických spojek a kvantifikátorů:  

    \medskip

    \myblock{
        \alert{Atomická formule} jazyka $L$ je nápis $R(t_1,\dots,t_n)$, kde $R$ je $n$-ární relační symbol z $L$ (včetně $=$ jde-li o jazyk s rovností) a $t_i\in\Term_L$. %Množinu všech \alert{atomických formulí} jazyka $L$ označíme $\AFm_L$. 
    }

    \medskip

    \begin{itemize}
        \item \myexampleinline{
            $R(f(f(x)),c, f(d))$
            } kde $R$ je ternární relační, $f$ unární funkční, $c,d$ konstantní symboly
            \item \myalertinline{ infixový zápis } $\leq(x,y)$, $=(t_1,t_2)$ píšeme jako $x\leq y$, $t_1=t_2$
        \item \myexampleinline{
            $(x\cdot x)+(y\cdot y)\leq (x+y)\cdot(x+y)$
            } v jazyce uspořád. těles
        \item \myexampleinline{
            $x\cdot y\leq (S(0)+x)\cdot y$
            } v jazyce aritmetiky
        \item \myexampleinline{
            $\neg(x\land y)\lor\bot=\bot$
            } v jazyce Booleových algeber        
    \end{itemize} 
    
\end{frame}


\begin{frame}{Formule}
    
    \myblock{
        \alert{Formule} jazyka $L$ jsou konečné nápisy definované induktivně: 
        \begin{itemize}
            \item každá \alert{atomická formule} jazyka $L$ je formule,
            \item je-li $\varphi$ formule, potom \alert{$(\neg\varphi)$} je také formule
            \item jsou-li $\varphi,\psi$ formule, potom \alert{$(\varphi\lbin\psi)$ pro $\lbinsymb\in\{\landsymb,\lorsymb,\limpliessymb,\liffsymb\}$} jsou také formule
            \item je-li $\varphi$ formule a $x$ proměnná, potom \alert{$((Q x)\varphi)$ pro $Q\in\{\forall,\exists\}$} jsou také formule
            \end{itemize}    
        %Množinu všech \alert{formulí} jazyka $L$ označíme $\Fm_L$.
    }
    \begin{itemize}
        \item \alert{podformule} je podřetězec, který je sám formulí
        \item při zápisu formulí jako lidé používáme obvyklé konvence
        \item kvantifikátory mají stejnou prioritu jako $\neg$, vyšší než ostatní logické spojky! místo $((\forall x)\varphi$) píšeme $(\forall x)\varphi$
        \item \myalertinline{pozor, 
        $(\forall x)\varphi\land\psi$ neznamená totéž, co $(\forall x)(\varphi\land\psi)$!}
        \item někde uvidíte $\forall x\varphi$ nebo $\forall_ x\varphi$, my ale budeme psát jen $(\forall x)\varphi$
    
    \end{itemize}
    

\end{frame}


\begin{frame}{Strom formule}

    \begin{columns}
        
        \column{0.65\textwidth}

        Příklad: \myexampleinline{
        $(\forall x)(x\cdot y\leq (S(0)+x)\cdot y)$
        }

        \bigskip

        \alert{Strom formule}, \alert{$\Tree(\varphi)$}:
        
        \begin{itemize}
            \item strom atomické formule $R(t_1,\dots,t_n)$: v kořeni $R$, připojíme stromy $\Tree(t_i)$
            \item pro složené formule podobně jako ve výrokové logice
            \item kvantifikátory mají jediného syna
        \end{itemize}

        \vspace{1.5cm}

        \column{0.3\textwidth}

        \begin{center}
            \scalebox{0.9}{        
                \begin{forest}
                    for tree={math content,circle,draw=blue!20,fill=blue!10,minimum size=22pt}
                    [\forall x
                        [\leq 
                            [\cdot [x] [y]] 
                            [\cdot [+ [S [0]] [x]] [y]]
                        ]
                    ]
                \end{forest}        
            }
        \end{center}

    \end{columns}

\end{frame}


\begin{frame}{Volné a vázané proměnné}

    Význam formule (\alert{pravdivostní hodnota}) může/nemusí záviset na proměnných~v~ní:\myexampleinline{
            $x\leq 0$
        }vs.\myexampleinline{
            $(\exists x)(x\leq 0)$
         }vs.\myexampleinline{
            $x\leq 0 \lor (\exists x)(x\leq 0)$
         }

    \begin{itemize}
        \item \alert{výskyt $x$ ve $\varphi$:} list $\Tree(\varphi)$ označený $x$ [v $(Qx)$ nemá výskyt!]
        \item \alert{vázaný}: součástí podformule začínající $(Qx)$, jinak \alert{volný}
        \item $x$ je \alert{volná} ve $\varphi$ má-li volný výskyt, \alert{vázaná} má-li vázaný výskyt
        \item zápis \alert{$\varphi(x_1,\dots,x_n)$} znamená, že mezi $x_1,\dots,x_n$ jsou všechny volné proměnné ve formuli $\varphi$
    \end{itemize}

    Proměnná může být \alert{volná i vázaná}, např.:
    \myexamplemath{
    $$
    \varphi=(\forall x)(\exists y)(x\leq y)\lor x\leq z
    $$
    }

    \begin{itemize}
        \item první výskyt $x$ je vázaný a druhý volný (nakreslete si strom!) 
        \item $y$ je vázaná a $z$ je volná, můžeme tedy psát $\varphi(x,z)$
    \end{itemize}

\end{frame}


\begin{frame}{Otevřené a uzavřené formule}

    \alert{otevřená formule:} nemá žádný kvantifikátor\\
    \alert{uzavřená formule (sentence):} nemá žádnou volnou proměnnou  

     \begin{itemize}
        \item \myexampleinline{
            $x+y\leq 0$
        } je otevřená formule
        \item \myexampleinline{
            $(\forall x)(\forall y)(x+y\leq 0)$
        } je uzavřená formule  neboli sentence
        \item \myexampleinline{
            $(\forall x)(x+y\leq 0)$
         } není ani otevřená, ani uzavřená
        \item \myexampleinline{
            $(0+1=1)\land (1+1=0)$
         } je otevřená i uzavřená
        \item atomické formule je otevřená, otevřené formule jsou kombinace atomických pomocí logických spojek
        \item je-li formule otevřená i uzavřená potom nemá žádné proměnné (všechny termy v ní jsou konstantní)
        \item formule bez vázané proměnné není nutně otevřená!\myexampleinline{
            $(\forall x)0=1$
        }
    \end{itemize}

    Uvidíme, že \alert{pravdivostní hodnota} závisí jen na ohodnocení volných proměnných; \alert{sentence} mají ve struktuře pravdiv. hodnotu 0 nebo 1
    
\end{frame}


\begin{frame}{Instance a varianty: neformálně}

    \begin{itemize}
        \item proměnná může hrát různé \myalertinline{`role'} (\myalertinline{`lokální'} vs. \myalertinline{`globální'})
        \item \alert{instance}: \myalertinline{`dosazení'} do \myalertinline{`globální'} proměnné (lépe \myalertinline{`nahrazení'} proměnné nějakým termem, který ji počítá, čistě syntaktické!)
        \item \alert{varianta}: \myalertinline{`přejmenování'} \myalertinline{`lokální'} proměnné
    \end{itemize}
    
    \myexamplemath{
    $$
    P(x)\land (\forall x)(Q(x) \land (\exists x)R(x))
    $$
    }

    \begin{itemize}
        \item první výskyt $x$ je volný, 2. je vázaný $(\forall x)$, 3. je vázaný  $(\exists x)$
        \item pokud \alert{substituujeme} za proměnnou $x$ term $t=1+1$, dostáváme \alert{instanci} formule $\varphi$, kterou označíme \alert{$\varphi(x/t)$}:
        \myexamplemath{
        $$
        P(1+1)\land (\forall x)(Q(x) \land (\exists x)R(x))
        $$
        }
        \item přejmenujeme-li kvantifikátory, získáme \alert{variantu} formule $\varphi$:
        \myexamplemath{
        $$
        P(x)\land (\forall y)(Q(y) \land (\exists z)R(z))
        $$
        }
    \end{itemize}

    \myalert{Kdy a jak to lze, aby instance byla \alert{důsledek} a varianta \alert{ekvivalentní}?}    

\end{frame}


\begin{frame}{Instance}
    
    Substituujeme-li do $\varphi$ za $x$ term $t$, chceme aby výsledná formule \myalertinline{`říkala o $t$ totéž, co $\varphi$ o $x$'}. Např. \myexampleinline{
        $\varphi(x)=(\exists y)(x+y=1)$
        }
    \begin{itemize}
        \item říká o $x$, že \myalertinline{`existuje $x-1$'}
        \item term $t=1$ lze: \myexampleinline{
            $\varphi(x/t)=(\exists y)(1+y=1)$} říká \myalertinline{`existuje 1-1'}
        \item term $t=y$ nelze: \myexampleinline{
            $(\exists y)(y+y=1)$} říká \myalertinline{`1 je dělitelné 2'}\\
        \textbf{problém:} obsahuje $y$, po nahrazení bude nově vázané $(\exists y)$
    \end{itemize}
    \myblock{
    Term $t$ je \alert{substituovatelný} za proměnnou $x$ ve formuli $\varphi$, pokud po simultánním nahrazení všech volných výskytů $x$ za $t$ nevznikne žádný vázaný výskyt proměnné z $t$. Potom je vzniklá formule \alert{instance} $\varphi$ vzniklá substitucí $t$ za $x$, \alert{$\varphi(x/t)$}.
    }
    \begin{itemize}
        \item $t$ \alert{není} substituovatelný za $x$ do $\varphi$, právě když $x$ má volný výskyt v nějaké podformuli $\varphi$ tvaru $(Qy)\psi$ a $y$ se vyskytuje v $t$
        \item speciálně: konstantní termy jsou vždy substituovatelné
    \end{itemize}

\end{frame}


\begin{frame}{Varianta}
    
    Substituovat $t$ můžeme vždy do \alert{varianty} $\varphi$, ve které přejmenujeme všechny kvantifikované proměnné na nové (které nejsou v $t$ ani $\varphi$)

    \smallskip
    
    \myblock{
    Má-li formule $\varphi$ podformuli tvaru $(Qx)\psi$ a je-li $y$ proměnná, že
    \begin{enumerate}[(i)]
        \item $y$ je substituovatelná za $x$ do $\psi$, a
        \item $y$ nemá volný výskyt v $\psi$.
    \end{enumerate}    
    \alert{Varianta} $\varphi$ vznikne nahrazením $(Qx)\psi$ formulí $(Qy)\psi(x/y)$,  říkáme tak i výsledku postupné variace ve více podformulích.
    }

    \medskip

    Mějme \myexampleinline{
        $\varphi=(\exists x)(\forall y)(x\leq y)$}:
    \begin{itemize}
        \item \myexampleinline{
            $(\exists u)(\forall v)(u\leq v)$} je varianta $\varphi$
        \item \myexampleinline{
            $(\exists y)(\forall y)(y\leq y)$} není varianta kvůli (i): $y$ není substituovatelná za $x$ do $\psi=(\forall y)(y\leq y)$
        \item \myexampleinline{
            $(\exists x)(\forall x)(x\leq x)$} není varianta kvůli (ii): $x$ má volný výskyt v~$\psi=(x\leq y)$        
    \end{itemize}   

\end{frame}


\section{6.4 Sémantika}


\begin{frame}{Sémantika neformálně}

    \begin{itemize}
        \item \alert{modely jsou struktury} dané signatury,
        \item formule \alert{platí} ve struktuře, pokud platí při každém ohodnocení volných proměnných prvky z domény,
        \item \alert{hodnoty termů} (jsou to prvky z domény) se vyhodnocují podle jejich stromů, kde symboly nahradíme jejich interpretacemi (relacemi, funkcemi, a konstantami z domény),
        \item z hodnot termů získáme \alert{pravdivostní hodnoty atomických formulí}: je výsledná $n$-tice v relaci?
        \item hodnoty složených formulí vyhodnocujeme také podle jejich stromu, přičemž \alert{$(\forall x)$ hraje roli `konjunkce přes všechny prvky'} a  \alert{$(\exists y)$ hraje roli `disjunkce přes všechny prvky'} z domény struktury
    \end{itemize}    

\end{frame}


\begin{frame}{Modely jazyka}

    \medskip

    \myblock{
    \alert{Model jazyka $L$}, nebo také \alert{$L$-struktura}, je libovolná struktura v signatuře jazyka $L$. \alert{Třídu} všech modelů jazyka označíme \alert{$\M_L$}.
    }

    \begin{itemize}
        \item zda je jazyk s rovností nebo bez nehraje roli
        \item proč \alert{třída} a ne \alert{množina} všech modelů $\M_L$? doména je libovolná neprázdná množina, `množina všech množin' neexistuje; třída je \alert{`soubor'} všech množin splňujících danou vlastnost (popsatelnou v \alert{jazyce teorie množin})
    \end{itemize}
    
    Mezi \myexampleinline{modely jazyka uspořádání} $L=\langle \leq \rangle$ patří: \begin{itemize}
        \item částečně uspořádané množiny $\langle \mathbb N,\leq\rangle$, $\langle \mathbb Q, > \rangle$, $\langle\mathcal P(X),\subseteq\rangle$
        \item libovolný orientovaný graf $G=\langle V,E\rangle$, typicky není částečné uspořádání, tj. nesplňuje axiomy \alert{teorie uspořádání}
        \item $\langle \mathbb C,R^\mathbb C\rangle$ kde $(z_1,z_2)\in R^\mathbb C$ právě když $|z_1|=|z_2|$ (není č. usp.)
        %\item $\langle \{0,1\},\emptyset\rangle$
    \end{itemize}

\end{frame}


\begin{frame}{Hodnota termu}

    Mějme term $t$ jazyka $L=\langle \mathcal R,\mathcal F\rangle$ a $L$-strukturu $\A=\langle A,\mathcal R^\A,F^\A\rangle$. \alert{Ohodnocení proměnných} v množině $A$ je lib. funkce $e:\Var\to A$.
    
    \myblock{
        \alert{Hodnota termu} $t$ \alert{ve struktuře} $\A$ \alert{při ohodnocení} $e$, značíme \alert{$t^\A[e]$}, je definovaná induktivně:
        \begin{itemize}
            \item $x^\A[e]=e(x)$ pro proměnnou $x\in\Var$,
            \item $c^\A[e]=c^\A$ pro konstantní symbol $c\in\mathcal F$, a
            \item je-li $t=f(t_1,\dots,t_n)$ složený term, kde $f\in\mathcal F$, potom:
            $$
            t^\A[e]=f^\A(t_1^\A[e],\dots,t_n^\A[e])
            $$
        \end{itemize}
    }

    \begin{itemize}
        \item závisí pouze na ohodnocení proměnných vyskytujících se v $t$
        \item obecně, term $t$ reprezentuje \alert{termovou funkci} $f_t^\A\colon A^k\to A$, kde $k$ je počet proměnných v $t$
        \item speciálně, hodnota konstantního termu na ohodnocení nezávisí, konstantní termy reprezentují konstantní funkce
    \end{itemize}
    
\end{frame}


\begin{frame}{Hodnota termu: příklady}
    
    1. Hodnota termu \myexampleinline{
            $t=-(x\lor \bot)\land y$
         } v Booleově algebře $\A=\underline{\mathcal P(\{0,1,2\})}$ při ohodnocení $e$ ve kterém:
        \begin{itemize}
            \item $e(x)=\{0,1\}$
            \item $e(y)=\{1,2\}$
        \end{itemize}
  
        \myalertmath{
            $$t^\A[e]=\{2\}$$
        }
    
    \bigskip\bigskip

    2. Hodnota termu \myexampleinline{
            $x+1$
         } ve struktuře $\mathcal N=\langle\mathbb N,\cdot,3\rangle$ jazyka $L=\langle +,1\rangle$ při ohodnocení $e$ ve kterém $e(x)=2$ 

        \myalertmath{
            $$(x+1)^\mathcal N[e]=6$$
        }
    
\end{frame}


\begin{frame}{Pravdivostní hodnota formule}

    \myblock{\smallskip
    Buď $\varphi$ v jazyce $L$, $\A\in\M_L$, $e:\Var\to A$ ohodnocení proměnných. \alert{Pravdivostní hodnota} $\varphi$ v $\A$ při ohodnocení $e$, \alert{$\mathrm{PH}^\A(\varphi)[e]$}:
    \begin{itemize}
        \item pro atomickou formuli $R(t_1,\dots,t_n)$:
        $$
        \mathrm{PH}^\A(R(t_1,\dots,t_n))[e]=
        \begin{cases}
            1 & \text{pokud }(t_1^\A[e],\dots,t_n^\A[e])\in R^\A\\
            0 & \text{jinak}    
        \end{cases}
        $$        
        \item pro formuli tvaru $(\neg\varphi)$:
        $$
        \mathrm{PH}^\A(\neg \varphi)[e]=f_\neg(\mathrm{PH}^\A(\varphi)[e])=1-\mathrm{PH}^\A(\varphi)[e]
        $$
        \item pro formuli tvaru $(\varphi\lbin\psi)$ kde $\lbinsymb\in\{\landsymb,\lorsymb,\limpliessymb,\liffsymb\}$:
        $$
        \mathrm{PH}^\A(\varphi\lbin\psi)[e]=f_\lbinsymb(\mathrm{PH}^\A(\varphi)[e],\mathrm{PH}^\A(\psi)[e])
        $$
    \end{itemize}
    \smallskip
    }
    
\end{frame}


\begin{frame}{Pravdivostní hodnota formule: zbytek definice a poznámky}
    
    \myblock{
    \smallskip
    \begin{itemize}
        \item pro formuli tvaru $(Qx)\varphi$ kde $Q\in\{\forall,\exists\}$:    
        \begin{align*}
            \mathrm{PH}^\A((\forall x)\varphi)[e]&=\min_{a\in A}(\mathrm{PH}^\A(\varphi)[e(x/a)])\\ 
            \mathrm{PH}^\A((\exists x)\varphi)[e]&=\max_{a\in A}(\mathrm{PH}^\A(\varphi)[e(x/a)])
        \end{align*}
        kde \alert{$e(x/a)$} je ohodnocení získané z $e$ změnou $e(x)$ na $a$
    \end{itemize}
    \smallskip
    }

    \textbf{Pozorování:} Závisí pouze na ohodnocení volných proměnných. Speciálně, pro sentenci nezávisí na ohodnocení.

    \begin{itemize}
        \item tedy v ohodnocení $e$ nastavíme hodnotu proměnné $x$ postupně na všechny prvky $a\in A$ a požadujeme, aby PH byla jedna vždy (v případě $\forall$) nebo alespoň jednou (v případě $\exists$)
        \item speciálně, $\mathrm{PH}^\A(t_1=t_2)[e]=1$ $\Leftrightarrow$ $(t_1^\A[e],t_2^\A[e])\in {=^\A}$ (\alert{identita} na $A$), tj. $t_1^\A[e]=t_2^\A[e]$ (je to stejný prvek $A$)
    \end{itemize}
    
\end{frame}


\begin{frame}{Příklady}
    
    Vezměme si uspořádané těleso $\underline{\mathbb Q}$. Potom:
    \begin{itemize}
        \item $\mathrm{PH}^{\underline{\mathbb Q}}(x\leq 1 \land \neg (x\leq 0))[e]=1$ právě když $e(x)\in (0,1]$
        \item $\mathrm{PH}^{\underline{\mathbb Q}}((\forall x)(x\cdot y = y))[e]=1$ právě když $e(y)=0$
        \item $\mathrm{PH}^{\underline{\mathbb Q}}((\exists x)(x \leq 0 \land \neg x=0))[e]=1$ pro každé ohodnocení $e$ (je to sentence)
    \end{itemize}   
    Ale pro strukturu $\A=\langle \mathbb N,+,-,0,\cdot,1,\leq\rangle$ máme: 
    \begin{itemize}
        \item $\mathrm{PH}^{\A}((\exists x)(x \leq 0 \land \neg x=0))[e]=0$ 
    \end{itemize}     

\end{frame}


\begin{frame}{Platnost ve struktuře}
    
    \myblock{
        Mějme formuli $\varphi$, strukturu $\A$ (ve stejném jazyce), a ohodnocení $e$.

        \begin{itemize}
            \item je-li $\mathrm{PH}^\A(\varphi)[e]=1$, $\varphi$ \alert{platí} v $\A$ \alert{při ohodnocení} $e$, \alert{$\A\models\varphi[e]$}
            \item je-li $\mathrm{PH}^\A(\varphi)[e]=0$, $\varphi$ \alert{neplatí} v $\A$ \alert{při ohodnoc.} $e$, \alert{$\A\not\models\varphi[e]$} 
            \item $\varphi$ je \alert{pravdivá} (\alert{platí}) v $\A$, \alert{$\A\models\varphi$}, pokud platí při každém ohodnocení $e:\Var\to A$
            \item $\varphi$ je \alert{lživá} v $\A$, pokud neplatí při žádném ohodnocení (v tom případě $\A\models\neg\varphi$)                       
        \end{itemize} 
    }
    
    \begin{itemize}
        \item pozor, \alert{lživá} není totéž, co \alert{není pravdivá} (\alert{neplatí})! \\(je to pravda jen pro sentence)
        \item \alert{platnost} je klíčový pojem sémantiky a celé logiky
    \end{itemize}
    
\end{frame}


\begin{frame}{Zřejmé vlastnosti platnosti ve struktuře při ohodnocení}

    \begin{itemize}
        \item $\A\models\neg\varphi[e]$ právě když $\A\not\models\varphi[e]$
        \item $\A\models(\varphi\land\psi)[e]$ právě když $\A\models\varphi[e]$ a $\A\models\psi[e]$
        \item $\A\models(\varphi\lor\psi)[e]$ právě když $\A\models\varphi[e]$ nebo $\A\models\psi[e]$
        \item $\A\models(\varphi\limplies\psi)[e]$ právě když platí: jestliže $\A\models\varphi[e]$ potom $\A \models\psi[e]$
        \item $\A\models(\varphi\liff\psi)[e]$ právě když platí: $\A\models\varphi[e]$ právě když $\A\models\psi[e]
        $
        \item \myalertinline{
            $\A\models(\forall x)\varphi[e]$ právě když $\A\models\varphi[e(x/a)]$ pro každé $a\in A$
            }
        \item \myalertinline{
            $\A\models(\exists x)\varphi[e]$ právě když $\A\models\varphi[e(x/a)]$ pro nějaké $a\in A$
        }
        \item je-li term $t$ substituovatelný za proměnnou $x$ do $\varphi$, potom:
        \myalertamsmath{
        $$
        \A\models\varphi(x/t)[e]\text{ právě když }\A\models\varphi[e(x/a)]\text{ pro }a=t^\A[e]
        $$
        }
        \item je-li $\psi$ varianta $\varphi$, potom $\A\models\varphi[e]$ právě když $\A\models\psi[e]$
    \end{itemize}

    (dokažte si snadno z definic, najděte protipříklady)  

\end{frame}


\begin{frame}{Zřejmé vlastnosti platnosti ve struktuře}

    \begin{itemize}
        \item pokud $\A\models\varphi$, potom $\A\not\models\neg\varphi$; je-li $\varphi$ sentence, platí i opačná implikace
        \item $\A\models\varphi\land\psi$ právě když $\A\models\varphi$ a $\A\models\psi$
        \item pokud $\A\models\varphi$ nebo $\A\models\psi$, potom $\A\models\varphi\lor\psi$; je-li $\varphi$ sentence,  platí i opačná implikace.
        \item $\A\models\varphi$ právě když $\A\models
        (\forall x)\varphi$
        \item speciálně, $\varphi(x_1,\dots,x_n)$ platí ve struktuře $\A$, právě když v $\A$ platí její \alert{generální uzávěr}, tj. sentence $(\forall x_1)\cdots(\forall x_n)\varphi$ 
    \end{itemize}

    (dokažte si snadno z definic, najděte protipříklady)

\end{frame}


\section{6.5 Vlastnosti teorií}


\begin{frame}{Platnost v teorii}

    \begin{itemize}
        \item \alert{teorie} jazyka $L$ je množina $L$-formulí, její prvky jsou \alert{axiomy}
        \item \alert{model} teorie $T$ je $L$-struktura, ve které platí všechny axiomy $T$, tj. $\A\models\varphi$ pro všechna $\varphi\in T$, značíme $\A\models T$
        \item \alert{třída modelů} teorie $T$ je:
        $$
        \alert{\M_L(T)}=\{\A\in\M_L\mid\A\models T\}
        $$
    \end{itemize}
    
    Je-li $T$ teorie v jazyce $L$ a $\varphi$ $L$-formule, potom $\varphi$ je:
    \begin{itemize}
        \item \alert{pravdivá (platí) v $T$}, značíme \alert{$T\models\varphi$}, pokud $\A\models\varphi$ pro všechna $\A\in\M(T)$ (neboli: \myalertinline{
            $\M(T)\subseteq\M(\varphi)$
            })
        \item \alert{lživá v $T$}, pokud $T\models\neg\varphi$, tj. pokud je lživá v každém modelu $T$ (neboli: \myalertinline{$\M(T)\cap\M(\varphi)=\emptyset$})
        \item \alert{nezávislá v $T$}, pokud není pravdivá v $T$ ani lživá v $T$
        \item je-li $T=\emptyset$ (tj. $\M(T)=\M_L$), píšeme jen \alert{$\models\varphi$}, a říkáme, že $\varphi$ \alert{je pravdivá (v logice), (logicky) platí, je tautologie}, apod.
    \end{itemize}
    
\end{frame}


\begin{frame}{Další sémantické pojmy o teorii}
    
    \begin{itemize}
        \item $T$ je \alert{sporná}, pokud v ní platí \alert{spor} $\bot$ (definujeme jako $R(x_1,\dots,x_n)\land \neg R(x_1,\dots,x_n)$, kde $R$ je lib. relační symbol)
        \item $T$ je sporná, právě když v ní platí každá formule (ekvivalentně, nemá žádný model), jinak je \alert{bezesporná} (neplatí-li v ní spor, má-li alespoň jeden model)
        \item \alert{důsledky} $T$ jsou \alert{sentence} pravdivé v $T$, množina všech důsledků $T$ v jazyce $L$ je
        $$
        \alert{\Conseq_L(T)}=\{\varphi\mid\text{$\varphi$ je sentence a }T\models \varphi\}
        $$
    \end{itemize}

\end{frame}


\begin{frame}{Kompletnost v predikátové logice}

    %Připomeňme, že \alert{výroková} teorie je kompletní, je-li bezesporná a každý výrok v ní buď platí, nebo platí jeho negace. Ekvivalentně, má právě jeden model.

    
    \begin{itemize}
        \item $T$ je \alert{kompletní}, je-li bezesporná a každá \alert{sentence} je v ní buď pravdivá, nebo lživá. \myalertinline{Pozor: neplatí, že má jediný model!}
        \item máme-li jeden model, máme i nekonečně mnoho \alert{izomorfních} modelů (liší se jen pojmenováním prvků, definujeme později)
        \item uvažovat jediný model \alert{až na izomorfismus} ale také \alert{nestačí}!
    \end{itemize}
    
    \myblock{ 
        Struktury $\A,\B$ (v témž jazyce) jsou \alert{elementárně ekvivalentní}, píšeme \alert{$\A\equiv\B$}, pokud v nich platí tytéž sentence.
    }
 
    \textbf{Pozorování:} Teorie je kompletní, právě když má právě jeden model \alert{až na elementární ekvivalenci}.   

    Příklad: uspořádané množiny \myexampleinline{
        $\A=\langle\mathbb Q,\leq\rangle$
     } a \myexampleinline{
        $\B=\langle\mathbb R,\leq\rangle$
     }.
     \begin{itemize}
        \item \alert{nejsou izomorfní}, $\mathbb Q$ je spočetná a $\mathbb R$ nespočetná množina, neexistuje dokonce žádná \alert{bijekce} mezi doménami
        \item \alert{ale $\A\equiv\B$}: indukcí dle struktury sentence $\varphi$ lze ukázat $\A\models\varphi\Leftrightarrow\B\models\varphi$; netriviální případ je $\exists$, klíčová je \alert{hustota} uspořádání $(x\leq y\land \neg x=y)\limplies(\exists z)(x\leq z\land z\leq y\land \neg x=z\land\neg y=z)$
     \end{itemize}     

\end{frame}


\begin{frame}{Platnost pomocí nesplnitelnosti}

    Otázku platnosti v teorii lze převést na problém existence modelu:

    \medskip

    \myblock{
    \textbf{Tvrzení (O nesplnitelnosti a pravdivosti):}
        Je-li $T$ teorie a $\varphi$ \alert{sentence} (v témž jazyce), potom: $T\models\varphi$ $\Leftrightarrow$ $T\cup\{\neg\varphi\}$ nemá model.
    }

    \medskip

    \textbf{Důkaz:}
        Platí následující ekvivalence: 
        \begin{itemize}
            \item $T\cup\{\neg\varphi\}$ nemá model,
            \item právě když $\neg\varphi$ neplatí v žádném modelu $T$,
            \item právě když $\varphi$ platí v každém modelu $T$  ($\varphi$ je sentence!). \hfill\qedsymbol
        \end{itemize}


    NB: Předpoklad, že $\varphi$ je sentence, je nutný: pro $T=\{P(c)\}$ a formuli $\varphi=P(x)$ je $P(c)\not\models P(x)$ ale $\{P(c),\neg P(x)\}$ nemá model.

\end{frame}


\begin{frame}{Příklady teorií: Teorie grafů}
    
    \myexampleinline{Teorie grafů}: $L=\langle E\rangle$ s rovností, axiomy \alert{ireflexivity} a \alert{symetrie}
    
    \myalertmath{
    $$
    T_\text{graph}=\{\neg E(x, x),E(x,y)\limplies E(y,x)\}
    $$
    }

    \textbf{Modely:} $\mathcal G=\langle G,E^\mathcal G\rangle$, kde $E^\mathcal G$ je symetrická ireflexivní relace, tj. \alert{jednoduché} grafy, hranu $\{x,y\}$ reprezentuje dvojice $(x,y),(y,x)$
    \begin{itemize}
        \item Formule \myexampleinline{
            $\neg x=y\limplies E(x,y)$
         } platí v grafu, právě když je \alert{úplný}. Je tedy nezávislá v $T_\text{graph}$.
        \item Formule \myexampleinline{
            $(\exists y_1)(\exists y_2)(\neg y_1=y_2\land E(x,y_1)\land E(x,y_2)\land$
         } \myexampleinline{
            $(\forall z)(E(x,z)\limplies z=y_1\lor z=y_2)$
        } vyjadřuje, že každý vrchol má stupeň právě 2. Platí tedy právě v grafech, které jsou disjunktní sjednocení kružnic, a je nezávislá v teorii $T_\text{graph}$.
    \end{itemize}

\end{frame}


\begin{frame}{Příklady teorií: Teorie uspořádání}

    \myexampleinline{Teorie uspořádání}: v jazyce uspořádání $L=\langle\leq\rangle$ s rovností, axiomy \alert{reflexivity}, \alert{antisymetrie}, a \alert{tranzitivity}

    \myalertamsmath{
    $$
    T=\{ x\leq x,\ x\leq y\land y\leq x\limplies x=y,\ x\leq y\land y\leq z\limplies x\leq z\}
    $$
    }

    \textbf{Modely:} $\langle S,\leq^S\rangle$, kde $\leq^S$ je \alert{částečné uspořádání}. 
    
    Příklad: \myexampleinline{
        $\A=\langle\mathbb N,\leq\rangle$
    }, \myexampleinline{
        $\B=\langle\mathcal P(X),\subseteq\rangle$
     } pro $X=\{0,1,2\}$.
    \begin{itemize}
        \item Formule \myexampleinline{
            $x\leq y\lor y\leq x$
         } (\alert{linearita}) platí v $\A$, ale neplatí v $\B$: neplatí např. při ohodnocení kde $e(x)=\{0\}$, $e(y)=\{1\}$ (píšeme $\B\not\models\varphi[e]$). Je tedy nezávislá v $T$.
        \item Sentence \myexampleinline{
            $(\exists x)(\forall y)(y\leq x)$
         } (označme $\psi$) je pravdivá v $\B$ a lživá v $\A$, píšeme $\B\models\psi$, $\A\models\neg\psi$. Je také nezávislá v $T$.
        \item Formule \myexampleinline{
            $(x\leq y\land y\leq z\land z\leq x)\limplies (x=y\land y=z)$
         } (označme  $\chi$) je pravdivá v $T$, píšeme $T\models\chi$. Totéž platí pro její \alert{generální uzávěr} $(\forall x)(\forall y)(\forall z)\chi$.
    \end{itemize}

\end{frame}


\begin{frame}{Příklady teorií: Algebraické teorie 1/2}

    \myexampleinline{Teorie grup}: $L=\langle +,-,0\rangle$ s rovností, axiomy 
    \alert{asociativita $+$}, \alert{neutralita $0$ vůči $+$}, a \alert{$-x$ je inverzní prvek k $x$ (vůči $+$ a $0$)}
    {\small
    \begin{align*}
        T_1=\{& x + (y + z) = (x + y) + z,\\
            & 0 + x = x,\ x + 0 = 0,\\
            & x + (-x) = 0,\ (-x) + x = 0\}\\
    \end{align*}
    }
   
    \vspace{-24pt}

    \myexampleinline{Teorie komutativních grup}: navíc \alert{komutativita $+$} 
    {\small
    $$
    T_2=T_1\cup\{x+y=y+x\}
    $$
    }

    \vspace{-6pt}

    \myexampleinline{Teorie okruhů}: $L=\langle +,-,0,\cdot,1\rangle$ s rovností, navíc \alert{neutralita $1$ vůči~$\cdot$}, \alert{asociativita~$\cdot$}, a \alert{(levá i pravá) distributivita $\cdot$ vůči $+$}
    {\small
    \begin{align*}
        T_3=T_2\cup\{   & 1 \cdot x = x \cdot 1,\\
        & x \cdot (y \cdot z) = (x \cdot y) \cdot z,\\
        & x \cdot (y + z) = x \cdot y + x \cdot z,\\
        & (x + y) \cdot z = x \cdot z + y \cdot z\}
    \end{align*}
    }
\end{frame}
    

\begin{frame}{Příklady teorií: Algebraické teorie 2/2}
    
    \myexampleinline{Teorie komutativních okruhů}: navíc axiom \alert{komutativity $\cdot$}:
    {\small
    $$
    T_4 = T_3 \cup \{x \cdot y = y \cdot x\}
    $$
    }
    
    \myexampleinline{Teorie těles} je ve stejném jazyce, ale má navíc axiomy \alert{existence inverzního prvku k $\cdot$} a \alert{netriviality}:
    {\small
    $$
    T_5 = T_4 \cup \{\neg\,x=0 \limplies (\exists y)(x\cdot y = 1),\ \neg\,0=1\}
    $$
    }

    \myexampleinline{Teorie uspořádaných těles} je v jazyce $\langle +, -, 0,\cdot,1,\leq\rangle$ s rovností, sestává z axiomů teorie těles, teorie uspořádání spolu s axiomem linearity, a z následujících axiomů \alert{kompatibility uspořádání}: 

    \begin{itemize}
        \item $x\leq y\limplies (x+z\leq y+z)$
        \item $(0\leq x\land 0\leq y)\limplies 0\leq x\cdot y$
    \end{itemize}

    Modely jsou tělesa s \alert{lineárním (totálním)} uspořádáním, které je kompatibilní s tělesovými operacemi.

\end{frame}


\end{document}
