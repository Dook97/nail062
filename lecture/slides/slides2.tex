\documentclass{beamer}


\usepackage{a4wide}
\usepackage{amsmath}
\usepackage{amssymb}
\usepackage{amsthm}
\usepackage[czech]{babel}
\usepackage{bookmark}
\usepackage{enumerate}
\usepackage[T1]{fontenc}
\usepackage{forest}
\usepackage{hyperref}
\usepackage[utf8]{inputenc}
\usepackage{lmodern}
\usepackage{multicol}
\usepackage{tikz}

\theoremstyle{definition}
    \newtheorem{problem}{Příklad}

% \theoremstyle{remark}
%     \newtheorem*{steps}{Postup řešení}

\theoremstyle{plain}
    \newtheorem*{solution}{Řešení}
    

\DeclareRobustCommand\proves{\mathrel{|}\joinrel\mkern-.5mu\mathrel{-}}
\DeclareMathOperator{\Conseq}{Csq}
\DeclareMathOperator{\M}{M}

% hide solutions
\newif\ifhidesolutions
    \hidesolutionstrue
    % \hidesolutionsfalse

\ifhidesolutions
    \usepackage{environ}
    \NewEnviron{hide}{}
    \let\solution\hide
    \let\endsolution\endhide
\fi








\title{Druhá přednáška}
\subtitle{NAIL062 Výroková a predikátová logika}
\author{Jakub Bulín (KTIML MFF UK)}
% \institute{KTIML MFF UK}
\date{Zimní semestr 2023}


\begin{document}


\frame{\titlepage}


\begin{frame}{Druhá přednáška}

    \textbf{Program}
        \begin{itemize}
            \item sémantika výrokové logiky
            \item normální formy
            \item vlastnosti a důsledky teorií
        \end{itemize}        
    

    \textbf{Materiály}

        \href{https://github.com/jbulin-mff-uk/nail062/raw/main/lecture/lecture-notes/lecture-notes.pdf}{\alert{\textbf{Zápisky z přednášky}}}, Sekce 2.2-2.4 z Kapitoly 2
    

\end{frame}


\section{2.2 Sémantika výrokové logiky}


\begin{frame}{Pravdivostní hodnota: příklad}

    \myalert{
        pravdivostní ohodnocení \alert{výrokových proměnných} jednoznačně určuje pravdivostní hodnotu výroku (vyhodnoť od listů ke kořeni)
    }

    \myexamplemath{
    $$\varphi = ((p\lor (\neg q)) \liff (r\limplies (p \land q)))$$
    }

    \begin{columns}        
        \column{0.5\textwidth}\centering
        (a) $\varphi$ \alert{platí} při ohodnocení \\$p=0$, $q=0$, $r=0$

        \medskip
        \scalebox{0.8}{
            \tikzset{every label/.style = {text=red}}
            \begin{forest}                
                for tree={circle,draw=blue!20,fill=blue!10,minimum size=24pt},        
                [$\liffsymb$, label=1
                    [$\lorsymb$, label={above left:1}
                        [$p$, label={[text=blue]below:0}] 
                        [$\neg$, label={above:1}  
                            [$q$, label={[text=blue]below:0}]
                        ]
                    ] 
                    [$\limpliessymb$, label={above right:1} 
                        [$r$, label={[text=blue]below:0}] 
                        [$\landsymb$, label={above right:0} 
                            [$p$, label={[text=blue]below:0}] 
                            [$q$, label={[text=blue]below:0}]
                        ]
                    ]
                ]
            \end{forest} 
        }

        \column{0.5\textwidth}\centering
        (b) $\varphi$ \alert{neplatí} při ohodnocení \\$p=1$, $q=0$, $r=1$

        \medskip
        \scalebox{0.8}{
            \tikzset{every label/.style = {text=red}}
            \begin{forest}
                for tree={circle,draw=blue!20,fill=blue!10,minimum size=24pt},        
                [$\liffsymb$, label=0
                    [$\lorsymb$, label={above left:1}
                        [$p$, label={[text=blue]below:1}] 
                        [$\neg$, label={above:1}  
                            [$q$, label={[text=blue]below:0}]
                        ]
                    ] 
                    [$\limpliessymb$, label={above right:0} 
                        [$r$, label={[text=blue]below:1}] 
                        [$\landsymb$, label={above right:0} 
                            [$p$, label={[text=blue]below:1}] 
                            [$q$, label={[text=blue]below:0}]
                        ]
                    ]
                ]
            \end{forest}
        }
    \end{columns}

\end{frame}


\begin{frame}{Sémantika logických spojek}
    
    \begin{table}[htbp]
    \centering
    \begin{tabular}{@{}cc|ccccc@{}}
        \toprule
        $p$ & $q$ & $\neg p$ & $p\land q$ & $p\lor q$ & $p\limplies q$ & $p\liff q$ \\ \midrule
        0   & 0   & 1        & 0          & 0         & \alert{1}          & 1          \\
        0   & 1   & 1        & 0          & 1         & \alert{1}          & 0          \\
        1   & 0   & 0        & 0          & 1         & 0          & 0          \\
        1   & 1   & 0        & 1          & \alert{1}         & 1          & 1          \\ \bottomrule
    \end{tabular}    
    \end{table}

    %nebo jako \alert{booleovské funkce} ($f\colon\{0,1\}^n\to\{0,1\}$):
    
    \begin{center}\small
    \begin{tabular}{rl}

        \scalebox{0.5}{
            \begin{tabular}{cc}
            0 & 1  \\ \hline
            1 & 0
            \end{tabular}
        }
        &
        $f_\neg(x)=1-x$
        \medskip
        \\

        \scalebox{0.5}{
            \begin{tabular}{c|cc}
            & 0 & 1  \\ \hline
            0 & 0 & 0  \\
            1 & 0 & 1 
            \end{tabular}
        }
        &
        $f_\landsymb(x,y)=\min(x,y)$        
        \medskip
        \\

        \scalebox{0.5}{
                \begin{tabular}{c|cc}
                & 0 & 1  \\ \hline
                0 & 1 & 1  \\
                1 & 1 & 0 
                \end{tabular}
        }
        &        
        $f_\lorsymb(x,y)=\max(x,y)$
        \medskip
        \\
        
        \scalebox{0.5}{
            \begin{tabular}{c|cc}
            & 0 & 1  \\ \hline
            0 & 1 & 1  \\
            1 & 0 & 1 
            \end{tabular}
        }
        &
        $f_\limpliessymb(x,y)$
        \medskip
        \\

        \scalebox{0.5}{
                \begin{tabular}{c|cc}
                & 0 & 1  \\ \hline
                0 & 1 & 0  \\
                1 & 0 & 1 
                \end{tabular}
            }
        &
        $f_\liffsymb(x,y)$            
    \end{tabular}
    \end{center}

\end{frame}


\begin{frame}{Výroky a booleovské funkce}

    sémantika logických spojek je daná booleovskými funkcemi, každý
    výrok určuje \emph{složenou} booleovskou funkci, tzv. \alert{pravdivostní funkci}

    např. \myexampleinline{$\varphi = ((p\lor (\neg q)) \liff (r\limplies (p \land q)))$}\alert{v jazyce~$\mathbb P'=\{p,q,r,s\}$}
  
    \myexamplemath{
        $$        
        f_{\varphi,\mathbb P'}(x_0,x_1,x_2,x_3)=f_\liffsymb(f_\lorsymb(x_0,f_\neg(x_1)),f_\limpliessymb(x_2,f_\landsymb(x_0,x_1)))    
        $$
    }
    
    \medskip

    \begin{columns}        
        \begin{column}{0.33\textwidth}
            \scalebox{0.8}{
            \tikzset{every label/.style = {text=red}}
            \begin{forest}
                for tree={circle,draw=blue!20,fill=blue!10,minimum size=24pt},        
                [$f_\liffsymb$, label=0
                    [$f_\lorsymb$, label={above left:1}
                        [$x_0$, label={[text=blue]below:1}] 
                        [$f_\neg$, label={above:1}  
                            [$x_1$, label={[text=blue]below:0}]
                        ]
                    ] 
                    [$f_\limpliessymb$, label={above right:0} 
                        [$x_2$, label={[text=blue]below:1}] 
                        [$f_\landsymb$, label={above right:0} 
                            [$x_0$, label={[text=blue]below:1}] 
                            [$x_1$, label={[text=blue]below:0}]
                        ]
                    ]
                ]
            \end{forest}
        }
        \end{column}
        \begin{column}{0.66\textwidth}
            \alert{pravdivostní hodnota} $\varphi$ při ohodnocení $p=1$, $q=0$, $r=1$, $s=1$:
            \begin{align*}
                f_{\varphi,\mathbb P'}(1,0,1,1)
                    &=f_\liffsymb(f_\lorsymb(1,f_\neg(0)),f_\limpliessymb(1,f_\landsymb(1,0))) \\
                    &=f_\liffsymb(f_\lorsymb(1,1),f_\limpliessymb(1,0))\\
                    &=f_\liffsymb(1,0)\\
                    &=0
                \end{align*}            
        \end{column}
    \end{columns}

\end{frame}


\begin{frame}{Pravdivostní funkce formálně}

\myblock{
    \alert{Pravdivostní funkce} výroku $\varphi$ v \emph{konečném} jazyce $\mathbb P$ je funkce  \alert{$f_{\varphi,\mathbb P}\colon\{0,1\}^{|\mathbb P|}\to\{0,1\}$} definovaná induktivně:
\begin{itemize}\setlength{\leftmargini}{-0.5cm}
    \item je-li $\varphi$ $i$-tý prvovýrok z $\mathbb P$: \alert{$f_{\varphi,\mathbb P}(x_0,\dots,x_{n-1})=x_i$}
    \item je-li $\varphi=(\neg\varphi')$: \alert{ 
    $f_{\varphi,\mathbb P}(x_0,\dots,x_{n-1})=f_\neg(f_{\varphi',\mathbb P}(x_0,\dots,x_{n-1}))$}
    \item je-li $(\varphi'\lbin\varphi'')$ kde $\lbinsymb\in\{\landsymb,\lorsymb,\limpliessymb,\liffsymb\}$:\alert{
    $f_{\varphi,\mathbb P}(x_0,\dots,x_{n-1})=f_\lbinsymb(f_{\varphi',\mathbb P}(x_0,\dots,x_{n-1}), f_{\varphi'',\mathbb P}(x_0,\dots,x_{n-1}))$
    }
\end{itemize}
}

\textbf{Poznámka:} Pravdivostní funkce $f_{\varphi,\mathbb P}$ závisí pouze na proměnných odpovídajících prvovýrokům z $\Var(\varphi)\subseteq\mathbb P$. 

Je-li výrok v \emph{nekonečném} jazyce $\mathbb P$, můžeme se omezit na jazyk $\Var(\varphi)$ (který je konečný) a uvažovat pravdivostní funkci nad ním.

\end{frame}


\begin{frame}{Modely}

Pravdivostní ohodnocení reprezentuje `reálný svět' (systém) v námi zvoleném `formálním světě', proto mu také říkáme \alert{model}

\myblock{
    \alert{Model jazyka} $\mathbb P$: libovolné pravdivostní ohodnocení $v\colon \mathbb P\to \{0,1\}$ Množina všech modelů:
    $
    \alert{\M_\mathbb P}=\left\{v\mid v\colon \mathbb P\to \{0,1\}\right\}=\{0,1\}^\mathbb P
    $
}

\myexample{
    $\mathbb P=\{p,q,r\}$, ohodnocení $p$ je pravda, $q$ nepravda, a $r$ pravda:
    
    formálně \alert{$v=\{(p,1),(q,0),(r,1)\}$} ale píšeme{\footnotemark} jen \alert{$v=(1,0,1)$}
\footnotesize
$$
\M_\mathbb P=\{(0,0,0),(0,0,1),(0,1,0),(0,1,1),(1,0,0),(1,0,1),(1,1,0),(1,1,1)\}
$$
}

\footnotetext{Formálně ztotožňujeme $\{0,1\}^\mathbb P$ s $\{0,1\}^{|\mathbb P|}$, množina $\mathbb P$ je uspořádaná.}

\end{frame}


\begin{frame}{Platnost}
    výrok platí v modelu, pokud je jeho pravdivostní hodnota rovna 1

    \myblock{
        Výrok $\varphi$ v jazyce $\mathbb P$, model $v\in\M_\mathbb P$. Pokud $f_{\varphi,\mathbb P}(v)=1$, potom říkáme, že $\varphi$ \alert{platí} v modelu~$v$, $v$ je \alert{modelem}~$\varphi$, a píšeme \alert{$v\models\varphi$}.
    }
    
    Množina všech modelů resp. \emph{nemodelů} $\varphi$:
    \begin{align*}
        \alert{\M_\mathbb P(\varphi)}&=\{v\in\M_\mathbb P\mid v\models \varphi\}=f_{\varphi,\mathbb P}^{-1}[1]\\
        \overline{\M_\mathbb P(\varphi)}=M_\mathbb P\setminus M_\mathbb P(\varphi)&=\{v\in\M_\mathbb P\mid v\not\models \varphi\}=f_{\varphi,\mathbb P}^{-1}[0]
    \end{align*}
    

    Je-li jazyk zřejmý z kontextu, můžeme vynechat, ale jinak ne!
    \myexample{\vspace{-12pt} \small
    \begin{align*}
        \M_{\{p,q\}}(p\limplies q)&=\{(0,0),(0,1),(1,1)\} \\
        \M_{\{p,q,r\}}(p\limplies q)&=\{(0,0,0),(0,0,1),(0,1,0),(0,1,1),(1,1,0),(1,1,1)\}
    \end{align*}
    }   
    
\end{frame}


\begin{frame}{Platnost teorie, model teorie}

    \myblock{
    Teorie $T$ \alert{platí} v modelu $v$, pokud každý axiom $\varphi\in T$ platí ve $v$. 
    }
    Podobně jako pro výrok: $v$ je \alert{modelem} $T$, \alert{$v\models T$}, \alert{$v\in\M_\mathbb P(T)$}.

    Někdy píšeme $\M_\mathbb P(T,\varphi)$ místo $\M_\mathbb P(T\cup\{\varphi\})$, $\M_\mathbb P(\varphi_1,\varphi_2,\dots,\varphi_n)$ místo $\M_\mathbb P(\{\varphi_1,\varphi_2,\dots,\varphi_n\})$.

    Všimněte si:
    \begin{itemize}
        \item $\M_\mathbb P(T,\varphi)=\M_\mathbb P(T)\cap\M_\mathbb P(\varphi)$
        \item $\M_\mathbb P(T)=\bigcap_{\varphi\in T}\M_\mathbb P(\varphi)$
        \item $
        \M_\mathbb P(\varphi_1)\supseteq \M_\mathbb P(\varphi_1,\varphi_2)\supseteq\dots\supseteq\M_\mathbb P(\varphi_1,\varphi_2,\dots,\varphi_n)
        $
    \end{itemize}

    \myexample{
    Najděme modely $T=\{p\lor q\lor r, q\limplies r, \neg r\}$ (v jazyce $\mathbb P=\{p,q,r\}$):\vspace{-6pt}
    \begin{align*}
        &\M_\mathbb P(r)=\{(0,0,0),(0,1,0),(1,0,0),(1,1,0)\}\\
        &\M_\mathbb P(r,q\limplies r)=\{(0,0,0),(1,0,0)\}\\
        &\M_\mathbb P(T)=\{(1,0,0)\}    
    \end{align*}
    }

\end{frame}


\begin{frame}{Další sémantické pojmy}

\begin{itemize}
    \item výrok $\varphi$ (nad $\mathbb P$) je \alert{pravdivý}, \alert{tautologie}, \alert{platí (v~logice)}, \alert{$\models \varphi$}, pokud platí v každém modelu, $\M_\mathbb P(\varphi)=\M_\mathbb P$
    \item \alert{lživý}, \alert{sporný}, pokud nemá žádný model, $\M_\mathbb P(\varphi)=\emptyset$
    \myalertinline{\it (Být \alert{lživý} není totéž, co nebýt \alert{pravdivý}!)}
    \item \alert{nezávislý}, pokud platí v nějakém modelu a neplatí v nějakém jiném modelu, tj.\ není pravdivý ani lživý, $\emptyset\subsetneq\M_\mathbb P(\varphi)\subsetneq\M_\mathbb P$
    \item \alert{splnitelný}, pokud má nějaký model, tj.\ není lživý, $\M_\mathbb P(\varphi)\neq\emptyset$
\end{itemize}
výroky $\varphi,\psi$ (ve stejném jazyce) jsou \alert{(logicky) ekvivalentní}, \alert{$\varphi\sim\psi$}, pokud mají stejné modely, tj.\
\myalertinline{
$
\varphi\sim\psi\ \Leftrightarrow\ \M_\mathbb P(\varphi)=\M_\mathbb P(\psi)
$
}

\myexample{
\begin{itemize}
    \item pravdivé jsou např.: $\top$, $p\lor q\liff q\lor p$
    \item lživé: $\bot$, $(p\lor q)\land (p\lor \neg q)\land \neg p$
    \item nezávislé a také splnitelné: $p$, $p\land q$
    \item ekvivalentní: $p\sim p\lor p$, $p\limplies q\sim \neg p\lor q$, $\neg p \limplies (p\limplies q) \sim \top $
\end{itemize}
}
    
\end{frame}


\begin{frame}{Sémantické pojmy vzhledem k teorii}

relativně k dané teorii $T$ (omezíme se na její modely)
\begin{itemize}
    \item \alert{pravdivý/platí v $T$}, \alert{důsledek $T$}, \alert{$T \models \varphi$} je-li \myalertinline{$\M_\mathbb P(T)\subseteq \M_\mathbb P(\varphi)$}
    \item \alert{lživý/sporný v $T$} pokud $\M_\mathbb P(\varphi)\cap\M_\mathbb P(T)=\M_\mathbb P(T,\varphi)=\emptyset$.
    \item \alert{nezávislý v $T$} pokud $\emptyset\subsetneq\M_\mathbb P(T,\varphi)\subsetneq\M_\mathbb P(T)$,
    \item \alert{splnitelný v $T$}, \alert{konzistentní s $T$} pokud $\M_\mathbb P(T,\varphi)\neq\emptyset$ (platí v alespoň jednom modelu $T$)
    \item $\varphi$ a $\psi$ jsou \alert{ekvivalentní v $T$}, \alert{$T$-ekvivalentní}, \alert{$\varphi\sim_T\psi$} platí-li v~týchž modelech $T$, tj. \myalertinline{
$
\varphi\sim_T\psi\ \Leftrightarrow\ \M_\mathbb P(T,\varphi)=\M_\mathbb P(T,\psi)
$}
\end{itemize}

\myexample{např. pro $T=\{p\lor q,\neg r\}$:
    \begin{itemize}
        \item výroky $q\lor p$, $\neg p\lor\neg q\lor \neg r$ jsou pravdivé v $T$
        \item výrok $\neg p\lor\neg q\lor r$ je v $T$ lživý
        \item výroky $p\liff q, p\land q$ jsou v $T$ nezávislé, a také splnitelné
        \item platí $p\sim_T p\lor r$ (ale $p\not\sim p\lor r$)
    \end{itemize}      
}

\end{frame}


\begin{frame}{Univerzálnost logických spojek}

    množina logických spojek je \alert{univerzální}, pokud:
    \begin{itemize}        
        \item každá booleovská funkce je pravdivostní funkcí nějakého výroku vybudovaného z těchto spojek
        \item ekvivalentně: každá množina modelů nad konečným jazykem je množinou modelů nějakého výroku
    \end{itemize}

    \mytheorem{Tvrzení}{
    $\{\neg, \landsymb,\lorsymb\}$ a $\{\neg, \limpliessymb\}$ jsou univerzální.
    }
    {\footnotesize [Důkaz na příštím slidu.]}


Další zajímavé logické spojky:
\begin{itemize}
    \item \alert{Shefferova spojka} (NAND, $\uparrow$)\hfill $p\uparrow q \sim \neg (p\land q)$,
    \item \alert{Pierceova spojka} (NOR, $\downarrow$)\hfill $p\downarrow q \sim \neg (p\lor q)$,
    \item \alert{Exclusive-OR} (XOR, $\oplus$)\hfill $p\oplus q \sim (p\lor q)\land\neg(p\land q)$
\end{itemize}
\myexampleinline{
    např. $\{\uparrow\}$ je univerzální, $\{\land,\lor\}$ není
}

\end{frame}


\begin{frame}{Důkaz, že $\{\neg, \landsymb,\lorsymb\}$ a $\{\neg, \limpliessymb\}$ jsou univerzální}  

    Mějme $f\colon \{0,1\}^n\to \{0,1\}$, resp. $M=f^{-1}[1]\subseteq \{0,1\}^n$
 
    \textbf{Pro jediný model:} \myalertinline{
        $\varphi_v=\text{`musím být model $v$'}$
    }

    \begin{itemize}
        \item příklad: \myexampleinline{$v=(1,0,1,0)\ \rightsquigarrow\ \varphi_v=p_1\land \neg p_2 \land p_3\land \neg p_4$}
        \item obecně: $v=(v_1,\dots,v_n)$, použijeme značení $p^1=p$, $p^0=\neg p$
        $$
        \varphi_v = p_1^{v_1}\land p_2^{v_2}\land \dots\land p_n^{v_n}=\bigwedge_{i=1}^n p_i^{v(p_i)}=\bigwedge_{p\in\mathbb P}p^{v(p)}
        $$    
    \end{itemize}
    
    \textbf{Pro více modelů:} \myalertinline{`musím být alespoň jeden z modelů z $M$'}
    $$
    \varphi_M = \bigvee_{v\in M}\varphi_v=\bigvee_{v\in M}\bigwedge_{p\in\mathbb P}p^{v(p)}
    $$

    Zřejmě $\M(\varphi_M)=M$ neboli $f_{\varphi_M,\mathbb P}=f$, a $\varphi_M$ používá jen $\{\neg, \landsymb,\lorsymb\}$. Protože $p\land q\sim \neg (p\limplies \neg q)$ a $p\lor q\sim \neg p\limplies q$, mohli bychom $\varphi_M$ ekvivalentně vyjádřit i pomocí  $\{\neg, \limpliessymb\}$. \hfill\qedsymbol   

\end{frame}


\section{2.3 Normální formy}


\begin{frame}{CNF a DNF}

\begin{itemize}
    \item \emph{Literál} $\ell$ je buď prvovýrok $p$ nebo negace prvovýroku $\neg p$. Pro prvovýrok $p$ označme $p^0=\neg p$ a $p^1=p$. Je-li $\ell$ literál, potom $\bar \ell$ označuje \emph{opačný literál} k $\ell$. Je-li $\ell=p$ (\emph{pozitivní literál}), potom $\bar \ell=\neg p$, je-li $\ell=\neg p$ (\emph{negativní literál}), potom $\bar \ell=p$
    \item \emph{Klauzule (clause)} je disjunkce literálů $C=\ell_1\lor\ell_2\lor\dots\lor\ell_n$. \emph{Jednotková klauzule (unit clause)} je samotný literál ($n=1$) a \emph{prázdnou klauzulí} ($n=0$) myslíme $\bot$.
    \item Výrok je \emph{v konjunktivní normální formě (v CNF)} pokud je konjunkcí klauzulí. \emph{Prázdný výrok v CNF} je $\top$.
    \item \emph{Elementární konjunkce} je konjunkce literálů $E=\ell_1\land\ell_2\land\dots\land\ell_n$. \emph{Jednotková elementární konjunkce} je samotný literál ($n=1$). \emph{Prázdná elementární konjunkce} ($n=0$) je $\top$.
    \item Výrok je \emph{v disjunktivní normální formě (v DNF)} pokud je disjunkcí elementárních konjunkcí. \emph{Prázdný výrok v DNF} je $\bot$.
\end{itemize}

\begin{example}
    Výrok ${{p\lor q}\lor\neg r}$ je v CNF (je to jediná klauzule) a zároveň v DNF (je to disjunkce jednotkových elementárních konjunkcí). Výrok $(p\lor q)\land (p\lor \neg q)\land \neg p$ je v CNF, výrok $\neg p\lor (p\land q)$ je v DNF.
\end{example}

\begin{example}
    Výrok $\varphi_v$ z důkazu Tvrzení~\ref{proposition:not-and-or-is-universal} je v CNF (je to konjunkce jednotkových klauzulí, tj.\ literálů) a také v DNF (je to jediná elementární konjunkce). Výrok $\varphi_M$ je v DNF.
\end{example}


Všimněte si, že výrok v CNF je pravdivý, právě když každá jeho klauzule obsahuje dvojici opačných literálů. Podobně, výrok v DNF je splnitelný, pokud ne každá elementární konjunkce obsahuje dvojici opačných literálů.

\end{frame}


\section{2.4 Vlastnosti a důsledky teorií}


\end{document}




% \section{Normální formy}


% \begin{itemize}
%     \item \emph{Literál} $\ell$ je buď prvovýrok $p$ nebo negace prvovýroku $\neg p$. Pro prvovýrok $p$ označme $p^0=\neg p$ a $p^1=p$. Je-li $\ell$ literál, potom $\bar \ell$ označuje \emph{opačný literál} k $\ell$. Je-li $\ell=p$ (\emph{pozitivní literál}), potom $\bar \ell=\neg p$, je-li $\ell=\neg p$ (\emph{negativní literál}), potom $\bar \ell=p$
%     \item \emph{Klauzule (clause)} je disjunkce literálů $C=\ell_1\lor\ell_2\lor\dots\lor\ell_n$. \emph{Jednotková klauzule (unit clause)} je samotný literál ($n=1$) a \emph{prázdnou klauzulí} ($n=0$) myslíme $\bot$.
%     \item Výrok je \emph{v konjunktivní normální formě (v CNF)} pokud je konjunkcí klauzulí. \emph{Prázdný výrok v CNF} je $\top$.
%     \item \emph{Elementární konjunkce} je konjunkce literálů $E=\ell_1\land\ell_2\land\dots\land\ell_n$. \emph{Jednotková elementární konjunkce} je samotný literál ($n=1$). \emph{Prázdná elementární konjunkce} ($n=0$) je $\top$.
%     \item Výrok je \emph{v disjunktivní normální formě (v DNF)} pokud je disjunkcí elementárních konjunkcí. \emph{Prázdný výrok v DNF} je $\bot$.
% \end{itemize}

% \begin{example}
%     Výrok ${{p\lor q}\lor\neg r}$ je v CNF (je to jediná klauzule) a zároveň v DNF (je to disjunkce jednotkových elementárních konjunkcí). Výrok $(p\lor q)\land (p\lor \neg q)\land \neg p$ je v CNF, výrok $\neg p\lor (p\land q)$ je v DNF.
% \end{example}

% \begin{example}
%     Výrok $\varphi_v$ z důkazu Tvrzení~\ref{proposition:not-and-or-is-universal} je v CNF (je to konjunkce jednotkových klauzulí, tj.\ literálů) a také v DNF (je to jediná elementární konjunkce). Výrok $\varphi_M$ je v DNF.
% \end{example}

% \begin{observation}
% Všimněte si, že výrok v CNF je pravdivý, právě když každá jeho klauzule obsahuje dvojici opačných literálů. Podobně, výrok v DNF je splnitelný, pokud ne každá elementární konjunkce obsahuje dvojici opačných literálů.
% \end{observation}

% \subsection{O dualitě}

% Všimněte si, že pokud ve výrokové logice zaměníme hodnoty pro pravdu a nepravdu, tj.\ 0 a 1, pravdivostní tabulka negace zůstává stejná, z konjunkce se stává disjunkce, a naopak. Tomuto konceptu se říká \emph{dualita}; v logice uvidíme mnoho příkladů. 

% Platí $\neg(p\land q)\sim (\neg p\lor \neg q)$ a \emph{z duality} víme také $\neg(\neg p\lor \neg q)\sim (\neg \neg p\land \neg \neg q)$, z čehož snadno odvodíme $\neg(p\lor q)\sim (\neg p\land \neg q)$.\footnote{Neboť $p,q$ jsou proměnné, mohou za ně být dosazeny obě hodnoty 0 i 1, tedy je můžeme zaměnit za k nim opačné literály.} Obecněji, $n$-ární booleovské funkce $f,g$ jsou navzájem \emph{duální}, pokud platí pokud $f(\neg x)=\neg g(x)$. Máme-li výrok $\varphi$ vybudovaný z $\{\neg,\landsymb,\lorsymb\}$ a zaměníme-li v něm $\landsymb$ a $\lorsymb$, a znegujeme-li výrokové proměnné (resp.\ zaměníme-li literály za opačné literály), dostáváme výrok $\psi\sim\neg\varphi$ (tj.\ modely $\varphi$ jsou nemodely $\psi$ a naopak), a funkce $f_{\varphi,\mathbb P},f_{\psi,\mathbb P}$ jsou navzájem duální.

% Pojem DNF je duální k pojmu CNF, `je pravdivý' je duální k `není splnitelný', předchozí pozorování tedy můžeme chápat jako příklad duality. Ke každému tvrzení ve výrokové logice získáváme `zdarma' tvrzení \emph{duální}, vzniklé záměnou $\land$ a $\lor$, pravdy a nepravdy.


% \subsection{Převod do normální formy}\label{subsection:convert-to-normal-form}

% Disjunktivní normální formu jsme již potkali, v důkazu Tvrzení~\ref{proposition:not-and-or-is-universal}. Klíčovou část důkazu bychom mohli zformulovat takto: `Je-li jazyk konečný, lze každou množinu modelů \emph{axiomatizovat} výrokem v DNF'. Z duality dostáváme také axiomatizaci v CNF, neboť doplněk množiny modelů je také množina modelů:

% \begin{proposition} \label{proposition:axiomatize-in-DNF-CNF}
%     Mějme konečný jazyk $\mathbb P$ a libovolnou množinu modelů $M\subseteq\M_\mathbb P$. Potom existuje výrok $\varphi_{\mathrm{DNF}}$ v DNF a výrok $\varphi_{\mathrm{CNF}}$ v CNF takový, že $M=\M_\mathbb P(\varphi_{\mathrm{DNF}})=\M_\mathbb P(\varphi_{\mathrm{CNF}})$. Konkrétně:
% \begin{align*}
%     \varphi_{\mathrm{\mathrm{DNF}}} &= \bigvee_{v\in M}\bigwedge_{p\in\mathbb P}p^{v(p)}\\
%     \varphi_{\mathrm{CNF}} &= \bigwedge_{v\in \overline{M}}\bigvee_{p\in\mathbb P}\overline{p^{v(p)}}=\bigwedge_{v\notin M}\bigvee_{p\in\mathbb P}p^{1-v(p)}
% \end{align*}
% \end{proposition}


% \begin{proof}
%     Pro výrok $\varphi_{\mathrm{DNF}}$ viz důkaz Tvrzení~\ref{proposition:not-and-or-is-universal}, každá elementární konjunkce popisuje jeden model. Výrok $\varphi_{\mathrm{CNF}}$ je duální k výroku $\varphi'_{\mathrm{DNF}}$ sestrojenému pro doplněk $M'=\overline{M}$. Nebo můžeme dokázat přímo: modely klauzule $C_v=\bigvee_{p\in\mathbb P}p^{1-v(p)}$ jsou všechny modely kromě $v$, $\M_C=\M_P\setminus\{v\}$, tedy každá klauzule v konjunkci zakazuje jeden nemodel.
% \end{proof}

% Tvrzení~\ref{proposition:axiomatize-in-DNF-CNF} dává návod, jak převádět výrok do disjunktivní nebo do konjunktivní normální formy:

% \begin{example}
%     Uvažme výrok $\varphi=p\liff (q\lor \neg r)$. Nejprve najdeme množinu modelů: $M=\M_\varphi=\{(0,0,1),(1,0,0),(1,1,0),(1,1,1)\}$. Nyní najdeme výroky $\varphi_{\mathrm{DNF}},\varphi_{\mathrm{CNF}}$ podle Tvrzení \ref{proposition:axiomatize-in-DNF-CNF}, ty mají stejnou množinu modelů jako $\varphi$, jsou tedy ekvivalentní.

%     Výrok $\varphi_{\mathrm{DNF}}$ najdeme tak, že pro každý model sestrojíme elementární konjunkci vynucující právě tento model:
%     $$
%     \varphi_{\mathrm{DNF}}=(\neg p\land\neg q\land r)\lor (p\land\neg q\land\neg r) \lor (p\land q\land\neg r)\lor (p\land q\land r)
%     $$
%     Při konstrukci $\varphi_{\mathrm{CNF}}$ budeme potřebovat \emph{nemodely} $\varphi$, $\overline{M}=\{(0,0,0),(0,1,0),(0,1,1),(1,0,1)\}$. Každá klauzule zakáže jeden nemodel:
%     $$
%     \varphi_{\mathrm{CNF}}=(p\lor q\lor r)\land (p\lor\neg q\lor r) \land (p\lor \neg q\lor\neg r)\land (\neg p\lor q\lor\neg r)
%     $$   
% \end{example}

% \begin{corollary}
%     Každý výrok (v libovolném, i nekonečném jazyce $\mathbb P$) je ekvivalentní nějakému výroku v CNF a nějakému výroku v DNF.
% \end{corollary}
% \begin{proof}
% I když je jazyk $\mathbb P$ nekonečný, výrok $\varphi$ obsahuje jen konečně mnoho výrokových proměnných, můžeme tedy použít Tvrzení \ref{proposition:axiomatize-in-DNF-CNF} pro jazyk $\mathbb P'=\Var(\varphi)$, a množinu modelů $M=\M_{\mathbb P'}(\varphi)$. Protože $M=\M_{\mathbb P'}(\varphi_{\mathrm{DNF}})=\M_{\mathbb P'}(\varphi_{\mathrm{CNF}})=M$, máme $\varphi\sim\varphi_{\mathrm{DNF}}\sim\varphi_{\mathrm{CNF}}$.
% \end{proof}

% \begin{exercise}
% Rozmyslete si, jak lze z DNF výroku snadno vygenerovat jeho modely, a z CNF výroku jeho nemodely.
% \end{exercise}

% \begin{remark}
%     Kdy lze axiomatizovat \emph{teorii} výrokem v DNF nebo výrokem v CNF? Mějme jazyk $\mathbb P'=\Var(T)$ (tj. všechny výrokové proměnné vyskytující se v axiomech $T$). Má-li $T$ v jazyce $\mathbb P'$ konečně mnoho modelů (tj.\ je-li $\M_{\mathbb P'}(T)$ konečná), můžeme sestrojit výrok v DNF, a má-li konečně mnoho \emph{nemodelů}, můžeme sestrojit výrok v CNF. Obecně ale ne každou teorii lze axiomatizovat \emph{jediným} výrokem v CNF nebo v DNF. Vždy můžeme převést jednotlivé axiomy do CNF (nebo DNF), a můžeme také axiomatizovat teorii jen pomocí (potenciálně nekonečně mnoha) klauzulí.
% \end{remark}

% Tento způsob převodu do CNF resp. do DNF vyžaduje znalost množiny modelů výroku, je tedy poměrně neefektivní. A také výsledná normální forma může být velmi dlouhá. Ukážeme si ještě jeden postup.

% \subsubsection{Převod pomocí ekvivalentních úprav}

% Využijeme následujícího pozorování: Nahradíme-li nějaký podvýrok $\psi$ výroku $\varphi$ ekvivalentním výrokem $\psi'$, výsledný výrok $\varphi'$ bude také ekvivalentní $\varphi$. Nejprve si ukážeme postup na příkladě:

% \begin{example}
%     Převedeme opět výrok $\varphi=p\liff (q\lor \neg r)$. Nejprve se zbavíme ekvivalence, vyjádříme ji jako konjunkci dvou implikací. V dalším kroku odstraníme implikace, pomocí pravidla $\varphi\limplies\psi\sim\neg\varphi\lor\psi$:
%     \begin{align*}
%         p\liff (q\lor \neg r) &\sim (p\limplies (q\lor \neg r)) \land ((q\lor \neg r) \limplies p)\\
%         &\sim (\neg p\lor q\lor \neg r) \land (\neg (q\lor \neg r) \lor p)
%     \end{align*}
%     Nyní si představme strom výroku, v dalším kroku chceme dostat negace na co nejnižší úroveň stromu, bezprostředně nad listy: využijeme toho, že $\neg (q\lor \neg r)\sim \neg q\land \neg\neg r$ a zbavíme se dvojité negace $\neg\neg r\sim r$. Dostáváme výrok 
%     $$
%     (\neg p\lor q\lor \neg r) \land ( (\neg q\land r) \lor p)
%     $$
%     Nyní již necháme literály nedotčené, a použijeme distributivitu $\landsymb$ vůči $\lorsymb$, nebo naopak, podle toho, zda chceme DNF nebo CNF. Pro převod do CNF použijeme úpravu $(\neg q\land r) \lor p\sim (\neg q\lor p)\land (r \lor p) $, kterou jsme dostali symbol $\lorsymb$ na nižší úroveň stromu. (Nakreslete si!) Tím už dostáváme výrok v CNF, pro přehlednost ještě seřadíme literály v klauzulích:
%     $$
%     (\neg p\lor q\lor \neg r) \land (p\lor \neg q) \land (p \lor r) 
%     $$
%     Při převodu do DNF bychom postupovali obdobně, opakovanou aplikací distributivity. Zde vyjdeme z CNF formy a zkombinujeme každý literál z první klauzule s každým literálem z druhé a s každým literálem z třetí klauzule. Všimneme si, že stejný literál nemusíme v elementární konjunkci opakovat dvakrát, a že obsahuje-li elementární klauzule dvojici opačných literálů, je sporná, a můžeme ji tedy v DNF vynechat. Také můžeme vynechat elementární konjunkci $E$, pokud máme jinou elementární konjunkci $E'$ takovou, že $E'$ obsahuje všechny literály obsažené v $E$, např. $E=p\land \neg r$ a $E'=(p\land q \land \neg r)$. (Rozmyslete si proč, a zformulujte duální zjednodušení při převodu do CNF.) Výsledný výrok v DNF je:
%     $$
%     (\neg p \land \neg q\land r) \lor (p\land q \land r) \lor (p\land \neg r)
%     $$
% \end{example}

% Nyní vypíšeme všechny potřebné ekvivalentní úpravy. Důkaz, že každý výrok lze převést do DNF a do CNF lze snadno provést indukcí podle struktury výroku (podle hloubky stromu výroku).

% \begin{tcolorbox}
% \begin{multicols}{2}
% \begin{itemize}
%     \item Implikace a ekvivalence:
%     \begin{itemize}
%         \item[] $\varphi\limplies\psi\sim\neg\varphi\lor\psi$
%         \item[] $\varphi\liff\psi\sim(\neg\varphi\lor\psi)\land(\neg\psi\lor\varphi)$
%     \end{itemize}
%     \item Negace:
%     \begin{itemize}
%         \item[] $\neg(\varphi\land\psi)\sim\neg\varphi\lor\neg\psi$
%         \item[] $\neg(\varphi\lor\psi)\sim\neg\varphi\land\neg\psi$
%         \item[] $\neg\neg\varphi\sim\varphi$
%     \end{itemize}
%     \item Konjunkce (převod do DNF):
%     \begin{itemize}
%         \item[] $\varphi \land (\psi\lor\chi) \sim (\varphi\land\psi)\lor (\varphi\land\chi)$
%         \item[] $(\varphi \lor \psi)\land\chi \sim (\varphi\land\chi)\lor (\psi\land\chi)$
%     \end{itemize}
%     \item Disjunkce (převod do CNF):
%     \begin{itemize}
%         \item[] $\varphi \lor (\psi\land\chi) \sim (\varphi\lor\psi)\land (\varphi\lor\chi)$
%         \item[] $(\varphi \land \psi)\lor\chi \sim (\varphi\lor\chi)\land (\psi\lor\chi)$
%     \end{itemize}
% \end{itemize}
% \end{multicols}
% \end{tcolorbox}


% Jak uvidíme v příští kapitole, CNF je v praxi mnohem důležitější než DNF (byť jde o duální pojmy). Při popisu reálného systému je přirozenější vyjádření pomocí konjunkce mnoha jednodušších vlastností, než jako jednu velmi dlouhou disjunkci. Existuje mnoho dalších forem reprezentace booleovských funkcí. Podobně jako datové struktury, vhodnou formu reprezentace volíme podle toho, jaké operace potřebujeme s funkcí dělat.\footnote{Viz například přednáška \href{https://is.cuni.cz/studium/predmety/index.php?do=predmet&kod=NAIL031}{NAIL031 Reprezentace booleovských funkcí}.}


% \section{Vlastnosti a důsledky teorií}

% Podívejme se nyní hlouběji na vlastnosti teorií. Podobně jako pro výroky řekneme, že dvě teorie $T,T'$ v jazyce $\mathbb P$ jsou \emph{ekvivalentní}, pokud mají stejnou množinu modelů:
% $$
% T\sim T' \text{ právě když } \M_\mathbb P(T)=\M_\mathbb P(T')
% $$
% Jde tedy o teorie vyjadřující tytéž vlastnosti, jen jinak vyjádřené (\emph{axiomatizované}). Zajímat nás budou vlastnosti nezávislé na konkrétní \emph{axiomatizaci}.

% \begin{example}
%     Například teorie $T=\{p\limplies q,p\liff r\}$ je ekvivalentní teorii $T'=\{(\neg p\lor q)\land(\neg p\lor r)\land(p\lor\neg r)\}$.
% \end{example}

% \begin{definition}[Vlastnosti teorií]
% Řekneme, že teorie $T$ v jazyce $\mathbb P$ je
% \begin{itemize}
%     \item \emph{sporná}, jestliže v ní platí $\bot$ (spor), ekvivalentně, jestliže nemá žádný model, ekvivalentně, jestliže v ní platí všechny výroky,
%     \item \emph{bezesporná} (\emph{splnitelná}), pokud není sporná, tj. má nějaký model,
%     \item \emph{kompletní}, jestliže není sporná a každý výrok je v ní pravdivý nebo lživý (tj. nemá žádné nezávislé výroky), ekvivalentně, pokud má právě jeden model.
% \end{itemize}    
% \end{definition}

% Rozmysleme si, proč platí ekvivalence vlastností v definici. Uvědomme si, že ve sporné teorii platí skutečně platí všechny výroky! Vskutku, výrok platí v $T$, pokud platí v každém modelu $T$, ty ale žádné nejsou. Naopak, pokud teorie má alespoň jeden model, v tomto modelu nemůže platit $\bot=p\land\neg p$.

% A je-li teorie kompletní, nemůže mít dva různé modely $v\neq v'$. Výrok $\varphi_{v}=\bigwedge_{p\in\mathbb P}p^{v(p)}$ (který jsme potkali v důkazu Tvrzení \ref{proposition:not-and-or-is-universal}) by totiž byl nezávislý v $T$, protože platí v modelu $v$ ale ne v modelu $v'$. Naopak, má-li $T$ jediný model $v$, potom každý výrok buď platí ve $v$, a tedy platí v $T$, nebo neplatí ve $v$ a potom je lživý v $T$.

% \begin{example} 
%     Příkladem sporné teorie je třeba $T_1=\{p,p\limplies q,\neg q\}$. Teorie $T_2=\{p\lor q,r\}$ je bezesporná, ale není kompletní, například výrok $p\land q$ v ní není pravdivý (neplatí v modelu $(1,0,1)$) ale ani lživý (platí v modelu $(1,1,1)$). Teorie $T_2\cup\{\neg p\}$ je kompletní, jejím jediným modelem je $(0,1,1)$.
% \end{example}

% \subsection{Důsledky teorií}

% Připomeňme, že důsledek teorie $T$ je každý výrok, který v $T$ platí (tj. platí v každém modelu $T$) a označme si \emph{množinu všech důsledků} teorie $T$ v jazyce $\mathbb P$ jako
% $$
% \Conseq_\mathbb P(T)=\{\varphi\in\VF_\mathbb P\mid T\models \varphi\}
% $$
% Pokud je teorie $T$ v jazyce $\mathbb P$, můžeme psát: 
% $$
% \Conseq_\mathbb P(T)=\{\varphi\in\VF_\mathbb P\mid \M_\mathbb P(T)\subseteq \M_\mathbb P(\varphi)\}
% $$
% (Dává ale smysl mluvit i o důsledcích teorie v nějakém menším jazyce, který je podmnožinou jazyka $T$). 

% Ukážeme si několik jednoduchých vlastností důsledků:
% \begin{proposition}\label{proposition:properties-of-consequences}
%     Mějme teorie $T,T'$ a výroky $\varphi,\varphi_1,\dots,\varphi_n$ v jazyce $\mathbb P$. Potom platí:
%     \begin{enumerate}[(i)]       
%         \item $T\subseteq\Conseq_\mathbb P(T)$,
%         \item $\Conseq_\mathbb P(T)=\Conseq_\mathbb P(\Conseq_\mathbb P(T))$,
%         \item pokud $T\subseteq T'$, potom $\Conseq_\mathbb P(T)\subseteq\Conseq_\mathbb P(T')$,
%         \item $\varphi\in\Conseq_\mathbb P(\{\varphi_1,\dots,\varphi_n\})$ právě když je výrok $(\varphi_1\land \dots\land\varphi_n)\to\varphi$ tautologie.
%     \end{enumerate}
% \end{proposition}

% \begin{proof}
%     Důkaz je snadný, použijeme-li, že $\varphi$ je důsledek $T$ právě když $\M_\mathbb P(T)\subseteq \M_\mathbb P(\varphi)$, a uvědomíme-li si následující vztahy:
%     \begin{itemize}
%         \item $\M(\Conseq(T))=\M(T)$,
%         \item je-li $T\subseteq T'$ potom $\M(T)\supseteq\M(T')$,\footnote{Čím více vlastností předepíšeme, tím méně objektů je bude všechny splňovat.}
%         \item $\psi\limplies\varphi$ je tautologie, právě když platí $\M(\psi)\subseteq\M(\varphi)$,
%         \item $\M(\varphi_1\land \dots\land\varphi_n)=\M(\varphi_1,\dots,\varphi_n)$.
%     \end{itemize}
% \end{proof}

% \begin{exercise}
%     Dokažte podrobně Tvrzení~\ref{proposition:properties-of-consequences}.
% \end{exercise}

% \subsection{Extenze teorií}

% Neformálně řečeno, rozšířením, neboli \emph{extenzí} teorie $T$ myslíme jakoukoliv teorii $T'$, která splňuje vše, co platí v teorii $T$ (a něco navíc, nejde-li o triviální případ). Modeluje-li $T$ nějaký systém, lze ji rozšířit dvěma způsoby: přidáním dodatečných požadavků o systému (tomu budeme říkat \emph{jednoduchá extenze}) nebo i rozšířením systému o nějaké nové části. Pokud ve druhém případě nemáme dodatečné požadavky na původní část systému, tedy platí-li o původní části totéž, co předtím, říkáme, že je extenze \emph{konzervativní}.

% \begin{example}
%     Vraťme se k úvodnímu příkladu o barvení grafů, Příklad \ref{example:graph-coloring-intro}. Teorie $T_3$ (úplná obarvení grafu zachovávající hranovou podmínku) je jednoduchou extenzí teorie $T_1$ (částečná obarvení množiny vrcholů bez ohledu na hrany). Teorie $T_3'$ z Sekce \ref{subsection:disadvantages-of-propositional-logic} (přidání nového vrcholu do grafu) je konzervativní, ale ne jednoduchou extenzí $T_3$. A jde o extenzi $T_1$, která není ani jednoduchá ani konzervativní.
% \end{example}

% Uveďme nyní konečně formální definice:

% \begin{definition}[Extenze teorie]
%     Mějme teorii $T$ v jazyce $\mathbb P$.
%     \begin{itemize}
%         \item \emph{Extenze} teorie $T$ je libovolná teorie $T'$ v jazyce $\mathbb P'\supseteq\mathbb P$ splňující $\Conseq_\mathbb P(T)\subseteq\Conseq_{\mathbb P'}(T')$,
%         \item je to \emph{jednoduchá extenze}, pokud $\mathbb P'=\mathbb P$,
%         \item je to \emph{konzervativní extenze}, pokud $\Conseq_\mathbb P(T)=\Conseq_\mathbb P(T')=\Conseq_{\mathbb P'}(T')\cap \VF_\mathbb P$.
%     \end{itemize}
% \end{definition}
% Extenze tedy znamená, že splňuje všechny důsledky původní teorie. Extenze je jednoduchá, pokud do jazyka nepřidáváme žádné nové výrokové proměnné, a konzervativní, pokud neměníme platnost tvrzení vyjádřitelných v původním jazyce, každý nový důsledek tedy musí obsahovat nějakou nově přidanou výrokovou proměnnou. 

% Co tyto pojmy znamenají \emph{sémanticky}, v řeči modelů? Zformulujme nejprve obecné pozorování, které ihned poté ilustrujeme na příkladě:
% \begin{observation}\label{observation:extensions-semantic-description-propositional}
%     Je-li $T$ teorie v jazyce $\mathbb P$ a $T'$ teorie v jazyce $\mathbb P'$ obsahujícím jazyk $P$. Potom platí:
%     \begin{itemize}
%         \item $T'$ je jednoduchou extenzí $T$, právě když $\mathbb P'=\mathbb P$ a $\M_\mathbb P(T')\subseteq\M_\mathbb P(T)$,
%         \item $T'$ je extenzí $T$, právě když $\M_{\mathbb P'}(T')\subseteq\M_{\mathbb P'}(T)$. Uvažujeme tedy modely teorie $T$ nad rozšířeným jazykem $\mathbb P'$.\footnote{Pozor, nemůžeme psát $\M_\mathbb P(T')$, protože modely $T'$ musí být ohodnoceními většího jazyka $\mathbb P'$, hodnoty jen pro proměnné z $\mathbb P$ nestačí k určení pravdivostní hodnoty. A nelze psát ani $\M_{\mathbb P'}(T')\subseteq\M_{\mathbb P}(T)$, jde o množiny vektorů jiné dimenze.} Jinými slovy, \emph{restrikce}\footnote{\emph{Restrikce} znamená zapomenutí hodnot pro nové výrokové proměnné, resp. smazání příslušných souřadnic při reprezentaci modelu vektorem.} libovolného modelu $v\in\M_{\mathbb P'}(T')$ na původní jazyk $\mathbb P$ musí být modelem $T$, mohli bychom psát $v{\restriction_\mathbb P}\in\M_\mathbb P(T)$ nebo:
%         $$
%         \{v{\restriction_\mathbb P}\mid v\in\M_{\mathbb P'}(T')\}\subseteq\M_\mathbb P(T)
%         $$
%         \item $T'$ je konzervativní extenzí $T$, pokud je extenzí a navíc platí, že každý model $T$ (v jazyce $\mathbb P$) lze nějak \emph{expandovat} (rozšířit)\footnote{Přidáním hodnot pro nové výrokové proměnné, resp. přidáním odpovídajících souřadnic ve vektorové reprezentaci} na model $T'$ (v jazyce $\mathbb P'$), neboli \emph{každý} model $T$ (v jazyce $\mathbb P$) získáme restrikcí \emph{nějakého} modelu $T'$ na jazyk $\mathbb P$. Mohli bychom psát:
%         $$
%         \{v{\restriction_\mathbb P}\mid v\in\M_      {\mathbb P'}(T')\}=\M_\mathbb P(T)
%         $$
%         \item $T'$ je extenzí $T$ a zároveň $T$ je extenzí $T'$, právě když $\mathbb P'=\mathbb P$ a $\M_\mathbb P(T')=\M_\mathbb P(T)$, neboli $T'\sim T$.        
%         \item Kompletní jednoduché extenze $T$ jednoznačně až na ekvivalenci odpovídají modelům $T$.
%     \end{itemize}
% \end{observation}

% \begin{example}
% Mějme teorii $T=\{p\limplies q\}$ v jazyce $\mathbb P=\{p,q\}$. Teorie $T_1=\{p\land q\}$ v jazyce $\mathbb P$ je jednoduchou extenzí $T$, máme $\M_\mathbb P(T_1)=\{(1,1)\}\subseteq\{(0,0),(0,1),(1,1)\}=\M_\mathbb P(T)$. Je to kompletní teorie, další kompletní jednoduché extenze teorie $T$ jsou např.\ $T_2=\{\neg p,q\}$ a $T_3=\{\neg p,\neg q\}$. Každá kompletní jednoduchá extenze teorie $T$ je ekvivalentní s $T_1$, $T_2$, nebo $T_3$.

% Uvažme nyní teorii $T'=\{p\liff (q\land r)\}$ v jazyce $\mathbb P'=\{p,q,r\}$. Je extenzí $T$, neboť $\mathbb P=\{p,q\}\subseteq\{p,q,r\}=\mathbb P'$ a platí:
% \begin{align*}
%     \M_{\mathbb P'}(T')&=\{(0,0,0),(0,0,1),(0,1,0),(1,1,1)\}\\ 
%     &\subseteq\{(0,0,0),(0,0,1),(0,1,0),(0,1,1),(1,1,0),(1,1,1)\}=\M_{\mathbb P'}(T)     
% \end{align*}
% Jinými slovy, zúžením modelů $T'$ na jazyk $\mathbb P$ dostáváme $\{(0,0),(0,1),(1,1)\}$ což je podmnožina $\M_\mathbb P(T)$. 

% Protože platí dokonce $\{(0,0),(0,1),(1,1)\}=\M_\mathbb P(T)$, jinými slovy, každý model $v\in\M_\mathbb P(T)$ lze rozšířit na model $v'\in\M_{\mathbb P'}(T')$ (např. $(0,1)$ lze rozšířit dodefinováním $v'(r)=0$ na model $(0,1,0)$), je $T'$ dokonce konzervativní extenzí $T$. To znamená, že každý výrok v jazyce $\mathbb P$ platí v $T$, právě když platí v $T'$. Ale výrok $p\limplies r$ (který je v jazyce $\mathbb P'$, ale ne v jazyce $\mathbb P$) je novým důsledkem: platí v $T'$ ale ne v $T$ (viz model $(1,1,0)$).

% Teorie $T''=\{\neg p\lor q,\neg q\lor r,\neg r\lor p\}$ v jazyce $\mathbb P'$ je extenzí $T$, ale ne konzervativní extenzí, neboť v ní platí $p\liff q$, což neplatí v $T$. Nebo také proto, že model $(0, 1)$ teorie $T$ nelze rozšířit na model teorie $T''$: $(0,1,0)$ ani $(0,1,1)$ nesplňují axiomy $T''$.

% Teorie $T$ je (jednoduchou) extenzí teorie $\{\neg p\lor q\}$ v jazyce $\mathbb P$ a naopak, $T\sim\{\neg p\lor q\}$. Je také, jako každá teorie, jednoduchou konzervativní extenzí sebe sama.
% \end{example}

% \begin{exercise}
%     Ukažte (podrobně), že má-li teorie $T$ kompletní konzervativní extenzi, potom je sama nutně kompletní.
% \end{exercise}
