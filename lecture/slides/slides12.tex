\documentclass{beamer}

%% slide-specific

\usetheme[progressbar=frametitle]{metropolis}
%\usecolortheme{spruce}
%\metroset{block=fill}

% define Metropolis colors    
\definecolor{mAlert}{HTML}{EB811B}
\definecolor{mExample}{HTML}{14B03D}
\definecolor{mBlock}{HTML}{23373b}

% my blocks
\setlength\fboxsep{0pt}%

\newcommand{\myblock}[1]{\colorbox{mBlock!8}{\begin{minipage}{\linewidth}#1\end{minipage}}}
\newcommand{\myblockmath}[1]{\colorbox{mBlock!8}{\begin{minipage}{\linewidth}\vspace{-6pt}#1\end{minipage}}}
\newcommand{\myblockinline}[1]{\colorbox{mBlock!8}{#1}}
\newcommand{\myexample}[1]{\colorbox{mExample!8}{\begin{minipage}{\linewidth}#1\end{minipage}}}
\newcommand{\myexamplemath}[1]{\colorbox{mExample!8}{\begin{minipage}{\linewidth}\vspace{-6pt}#1\end{minipage}}}
\newcommand{\myexampleinline}[1]{\colorbox{mExample!8}{#1}}
\newcommand{\myalert}[1]{\colorbox{mAlert!8}{\begin{minipage}{\linewidth}#1\end{minipage}}}
\newcommand{\myalertmath}[1]{\colorbox{mAlert!8}{\begin{minipage}{\linewidth}\vspace{-6pt}#1\end{minipage}}}
\newcommand{\myalertinline}[1]{\colorbox{mAlert!8}{#1}}

%% other
\newcommand{\mystructure}[1]{\mathcal{#1}}




%% packages
\usepackage{amsmath,amssymb,amsthm}
\usepackage{booktabs}
\usepackage[czech]{babel}
\usepackage{enumerate}
\usepackage{forest}
\usepackage{multicol}
% \usepackage{tcolorbox}
\usepackage{tikz}
    \usetikzlibrary{arrows.meta}
%\usepackage[unicode]{hyperref}
\usepackage[utf8x]{inputenc}
\usepackage{xfrac}

% %% theorems
% \theoremstyle{plain}
%     \newtheorem{theorem}{Věta}[section]
%     \newtheorem*{theorem-unnumbered}{Věta}
%     \newtheorem{proposition}[theorem]{Tvrzení}
%     \newtheorem{corollary}[theorem]{Důsledek}
%     \newtheorem{lemma}[theorem]{Lemma}
%     \newtheorem{observation}[theorem]{Pozorování}
% \theoremstyle{definition}
%     \newtheorem{definition}[theorem]{Definice}
%     \newtheorem*{algorithm}{Algoritmus}
% \theoremstyle{remark}
%     \newtheorem{remark}[theorem]{Poznámka}
%     \newtheorem{example}[theorem]{Příklad}
%     \newtheorem{exercise}{Cvičení}[chapter]
%     \newtheorem*{solution}{Řešení}

%% macros and definitions
\DeclareMathOperator{\Aut}{Aut}
\DeclareMathOperator{\Conseq}{Csq}
\DeclareMathOperator{\DeLO}{DeLO}
\DeclareMathOperator{\dom}{dom}
\DeclareMathOperator{\Fm}{Fm}
\DeclareMathOperator{\M}{M}
%\DeclareMathOperator{\Proof}{Proof}
\DeclareMathOperator{\rng}{rng}
\DeclareMathOperator{\Term}{Term}
\DeclareMathOperator{\Th}{Th}
\DeclareMathOperator{\Thm}{Thm}
\DeclareMathOperator{\Tree}{Tree}
\DeclareMathOperator{\Var}{Var}
\DeclareMathOperator{\VF}{VF}

\newcommand{\A}{\structure{A}}
\newcommand{\B}{\structure{B}}
\newcommand{\Con}{\mathit{Con}}
\newcommand{\disjointunion}{\mathbin{\dot{\sqcup}}}
\newcommand{\F}{\ensuremath{\mathrm{F}}}
\newcommand{\landsymb}{{\land}}
\newcommand{\lbin}{\mathbin{\square}}
\newcommand{\lbinsymb}{{\lbin}}
\newcommand{\liff}{\mathbin{\leftrightarrow}}
\newcommand{\liffsymb}{{\liff}}
\newcommand{\limplies}{\mathbin{\rightarrow}}
\newcommand{\limpliessymb}{{\limplies}}
\newcommand{\lorsymb}{{\lor}}
\newcommand{\Prf}{\mathit{Prf}}
\newcommand{\proves}{\vdash}
%\newcommand{\structure}[1]{\mathcal{#1}}
\newcommand{\todo}{[TODO]}
\newcommand{\T}{\ensuremath{\mathrm{T}}}
\newcommand{\union}{\mathbin{\cup}}


\title{Dvanáctá přednáška}
\subtitle{NAIL062 Výroková a predikátová logika}
\author{Jakub Bulín (KTIML MFF UK)}
% \institute{KTIML MFF UK}
\date{Zimní semestr 2023}


\begin{document}


\frame{\titlepage}


\begin{frame}{Dvanáctá přednáška}

    \textbf{Program}
        \begin{itemize}
            \item izomorfismus a konečné modely
            \item definovatelnost a automorfismy
            \item $\omega$-kategoricita a úplnost
            \item axiomatizovatelnost
            \item rekurzivní axiomatizace a rozhodnutelnost
        \end{itemize}

    \textbf{Materiály}

        \href{https://github.com/jbulin-mff-uk/nail062/raw/main/lecture/lecture-notes/lecture-notes.pdf}{\alert{\textbf{Zápisky z přednášky}}}, Sekce 9.2-9.4 z Kapitoly 9, Sekce 10.1 z Kapitoly 10

\end{frame}


\section{9.2 Izomorfismus struktur}


\begin{frame}{Definice izomorfismu}
    
    \myblock{
        \alert{Izomorfismus} $\A$ a $\B$ (v $L=\langle\mathcal R,\mathcal F\rangle$) je bijekce $h\colon A\to B$ splňující:
    \begin{itemize}
        \item pro každý ($n$-ární) $f\in\mathcal F$ a pro všechna $a_i\in A$:
        \vspace{-6pt}
        $$
        h(f^\A(a_1,\dots,a_n))=f^\B(h(a_1),\dots,h(a_n))
        $$
        \vspace{-24pt}
        \item speciálně, je-li $c\in\mathcal F$ konstantní: $h(c^\A)=c^\B$
        \item pro každý ($n$-ární) $R\in\mathcal R$ a pro všechna $a_i\in A$:
        \vspace{-6pt}
        $$
        R^\A(a_1,\dots,a_n)\ \text{ právě když }\ R^\B(h(a_1),\dots,h(a_n))
        $$
        \vspace{-24pt}
    \end{itemize}
    Existuje-li, jsou \alert{izomorfní} (`\alert{via $h$}'), \alert{$\A\simeq\B$} (nebo \alert{$\A\simeq_h\B$}).\\
    \alert{Automorfismus} $\A$ je izomorfismus $\A$ a $\A$.
    }
    \vspace{-6pt}
    \begin{itemize}
        \item tj. liší se jen `pojmenováním prvků'
        \item relace `být izomorfní' je ekvivalence
        \item např. potenční algebra \myexampleinline{\small
            $\underline{\mathcal P(X)}=\langle \mathcal P(X),-,\cap,\cup,\emptyset,X\rangle$
         }, $|X|=n$, je izomorfní s {\small $\underline{2^n}=\langle \{0,1\}^n,-_n,\land_n,\lor_n,(0,\dots,0),(1,\dots,1)\rangle$} (operace po složkách) via \alert{$h(A)=\chi_A$} (charakt. vektor $A\subseteq X$)
    \end{itemize}

\end{frame}


\begin{frame}{Izomorfismus zachovává sémantiku \& vztah $\simeq$ a $\equiv$}
        
    \myblock{
        \textbf{Tvrzení:} Bijekce $h\colon A\to B$ je izomorfismus $\A$ a $\B$, právě když:
        \begin{enumerate}[(i)]
            \item pro každý term $t$ a $e:\Var\to A$:\hfill
            \alert{$h(t^\A[e])=t^\B[e\circ h]$}
            \item pro každou $\varphi$ a $e:\Var\to A$:\hfill
            \alert{$\A\models\varphi[e]\Leftrightarrow\B\models\varphi[e\circ h]$}
        \end{enumerate}
    }

    \textbf{Důkaz:} \alert{\Large$\Rightarrow$} snadno indukcí podle struktury termu resp. formule

    \alert{\Large$\Leftarrow$} je-li $h$ bijekce splňující (i)\&(ii), dosazení $t=f(x_1,\dots,x_n)$ resp. $\varphi=R(x_1,\dots,x_n)$ dává vlastnosti z definice izomorfismu \hfill\qedsymbol

    \myblock{
        \textbf{Důsledek:} 
        $\A\simeq\B\ \Rightarrow\ \A\equiv\B$.
    }
    \textbf{Důkaz:} pro každou sentenci $\varphi$ máme z (ii) $\A\models\varphi\Leftrightarrow\B\models\varphi$\hfill\qedsymbol
    
    Naopak obecně ne, \myexampleinline{
        $\langle\mathbb Q,\leq \rangle\equiv\langle\mathbb R,\leq\rangle$, $\langle\mathbb Q,\leq \rangle\not\simeq \langle \mathbb R,\leq\rangle$
    } Platí ale:

    \myblock{\textbf{Tvrzení:}
        Jsou-li $\A,\B$ konečné v jazyce s rovností, potom
        \vspace{-9pt}
        $$
        \A\simeq\B\ \Leftrightarrow\ \A\equiv\B
        $$
        \vspace{-20pt}        
    }

    \myblock{
        \textbf{Důsledek}
        Pokud má kompletní teorie v jazyce s rovností konečný model, potom jsou všechny její modely izomorfní.
    }   
    
\end{frame}


\begin{frame}{Důkaz $\equiv\ \Rightarrow\ \simeq$ pro konečné struktury s rovností}
    
    \vspace{-6pt}
    Díky $=$ vyjádříme ``existuje právě $n$ prvků'', z toho plyne \alert{$|A|=|B|$}.    
    Buď $\A'$ expanze $\A$ o jména prvků, v jazyce $L'=L\cup\{c_a\mid a\in A\}$. Ukážeme, že $\B$ \alert{lze expandovat} na $L'$-strukturu $\B$ \alert{tak, že $\A'\equiv \B'$}.     
    Potom je \alert{$h(a)= c_a^{\B'}$} izomorfismus $\A'$ a $\B'$, i pro $L$-redukty $\A\simeq\B$.
    
    Stačí ukázat, že \alert{pro $c_a^{\A'}=a\in A$ existuje $b\in B$} tak, že expanze o interpretaci konstantního symbolu $c_a$ splňují \alert{$\langle \A,a\rangle\equiv\langle\B,b\rangle$}. 
    
    Buď $\Omega$ množina `\alert{vlastností prvku $a$}', tj. formulí $\varphi(x)$ splňujících $\langle \A,a\rangle \models \varphi(x/c_a)$, neboli $\A\models \varphi[e(x/a)]$. Protože je $A$ konečná, existuje \alert{konečně mnoho $\varphi_1(x),\dots,\varphi_m(x)$} tak, že pro každou $\varphi \in \Omega$ existuje $i$ takové, že $\A\models \varphi\liff\varphi_i$. Potom i $\B\models\varphi\liff\varphi_i$. %(neboť $\A\equiv\B$, stačí vzít generální uzávěr, což je sentence).
        
        Protože v $\A$ platí sentence \alert{$(\exists x)\bigwedge_{i=1}^m\varphi_i$} (je splněna díky $a\in A$) a $\B\equiv\A$, máme i $\B\models (\exists x)\bigwedge_{i=1}^m\varphi_i$. Neboli existuje $b\in B$ takové, že $\B\models \bigwedge_{i=1}^m\varphi_i[e(x/b)]$. Tedy pro každou $\varphi\in \Omega$ platí $\B\models \varphi[e(x/b)]$, tj. $\langle\mathcal{B},b\rangle\models \varphi(x/c_a)$, z toho $\langle \A,a\rangle\equiv\langle\B,b\rangle$.\hfill\qedsymbol

\end{frame}


\begin{frame}{Definovatelnost a automorfismy}

    definovatelné množiny jsou \alert{invariantní} na automorfismy (např. automorfismus grafu musí zobrazit trojúhelník na trojúhelník):

    \medskip

    \myblock{
        \textbf{Tvrzení:}
        Je-li $D\subseteq A^n$ definovatelná v $\A$, potom pro každý automorfismus $h\in\Aut(\A)$ platí $h[D]=D$ (kde $h[D]$ značí $\{(h(\overline{a})\mid\overline{a}\in D\}$).
    
        Je-li definovatelná s parametry $\overline{b}$, platí to pro automorfismy identické na $\overline{b}$ (tj. $h(\overline{b})=\overline{b}$ neboli $h(b_i)=b_i$ pro všechna $i$).
    }

    \textbf{Důkaz:}
    Ukážeme jen verzi s parametry. Nechť $D=\varphi^{\A,\overline{b}}(\overline{x},\overline{y})$. Potom pro každé $\overline{a}\in A^n$ platí následující ekvivalence:
    \begin{align*}
    \overline{a}\in D\ 
    &\Leftrightarrow\ \A\models \varphi[e(\overline{x}/\overline{a},\overline{y}/\overline{b})]\\
    &\Leftrightarrow\  \A\models \varphi[(e\circ h)(\overline{x}/\overline{a},\overline{y}/\overline{b})]\\
    &\Leftrightarrow\ \A\models \varphi[e(\overline{x}/h(\overline{a}),\overline{y}/h(\overline{b}))]\\
    &\Leftrightarrow\ \A\models \varphi[e(\overline{x}/h(\overline{a}),\overline{y}/\overline{b})]\\
    &\Leftrightarrow\ h(\overline{a})\in D.
    \end{align*}

    \vspace{-0.85cm}
    \hfill\qedsymbol    

\end{frame}


\begin{frame}{Příklad}

    \vspace{6pt}

    \begin{columns}

        \begin{column}{0.8\textwidth}            
            Množiny definovatelné s parametrem $0$, $\mathrm{Df}^1(\mathcal G,\{0\})$? Jediný netriviální automorfismus zachovávající $0$: \alert{$h(i)=(5-i) \bmod 5$}, orbity $\{0\}$, $\{1,4\}$, a $\{2,3\}$. Tyto množiny jsou definovatelné:            
        \end{column} 

        \begin{column}{0.2\textwidth}            
            \begin{tikzpicture}[every node/.style={circle,fill=blue!10,draw,minimum size=0.4cm,node distance=1cm}]
                \node (1) {$1$};
                \node[below left of=1](0) {$0$};
                \node[below right of=0] (4) {$4$};
                \node[right of=4] (3) {$2$};
                \node[right of=1] (2) {$3$};
                \path[draw] (0) -- (1) -- (2) -- (3) -- (4) -- (0);
            \end{tikzpicture}                        
        \end{column}

    \end{columns}

    \begin{itemize}
        \item $\{0\}$ formulí $x=y$, tj. $(x=y)^{\mathcal G,\{0\}}=\{0\}$
        \item $\{1,4\}$ lze definovat pomocí $E(x,y)$
        \item $\{2,3\}$ formulí $\neg E(x,y)\land \neg x=y$
    \end{itemize}
    $\mathrm{Df}^1(\mathcal G,\{0\})$ je podalgebra $\underline{\mathcal P(V(\mathcal G))}$, tedy uzavřená na doplněk, sjednocení, průnik, obsahuje $\emptyset$ a $V(\mathcal G)$. Podalgebra generovaná $\{\{0\},\{1,4\},\{2,3\}\}$ už ale obsahuje všechny podmnožiny zachovávající automorfismus $h$. Dostáváme:
    \begin{align*}        
        \mathrm{Df}^1(\mathcal G,\{0\})=\{&\emptyset, \{0\}, \{1,4\}, \{2,3\}, \{0,1,4\}, \{0,2,3\}, \\ &\{1,4,2,3\}, \{0,1,2,3,4\}\}        
    \end{align*}
    
\end{frame}


\section{9.3 $\omega$-kategorické teorie}


\begin{frame}{$\omega$-kategorické teorie}

    \vspace{-8pt}
    \alert{Izomorfní spektrum} $T$ je počet modelů $T$ kardinality $\kappa$ až na $\simeq$.\\
    $T$ je \alert{$\kappa$-kategorická} pokud $I(\kappa,T)=1$, \alert{$\omega$-kategorická} má-li jediný spočetně nekonečný model až na izomorfismus.

    \vspace{-3pt}
    \myblock{
        \textbf{Tvrzení:}
        Teorie DeLO je $\omega$-kategorická.
    }
    \textbf{Důkaz:}
    Buďte $\A,\B$ spočetně nekonečné modely, $A=\{a_i\mid i\in\mathbb N\}$, $B=\{b_i\mid i\in\mathbb N\}$. Z hustoty najdeme indukcí $h_0\subseteq h_1\subseteq h_2\subseteq\dots$ prosté parciální fce z $A$ do $B$ zach. usp., $\{a_0,\dots,a_{n-1}\}\subseteq\dom h_n$, $\{b_0,\dots,b_{n-1}\}\subseteq\rng h_n$. Potom $\A\simeq\B$ via $h=\bigcup_{n\in\mathbb N}h_n$.
    \hfill\qedsymbol

    \myblock{
        \textbf{Důsledek:}
        Izomorfní spektrum teorie DeLO*:
        \vspace{-3pt}
        \begin{itemize}
            \item $I(\kappa,DeLO^*)=0$ pro $\kappa\in\mathbb{N}$
            \item $I(\omega,DeLO^*)=4$
        \end{itemize}      
        \vspace{-3pt}
        Spočetné modely až na izomorfismus jsou například:
        \vspace{-9pt}
        $$ 
        \mathbb Q=\langle \mathbb Q,\leq\rangle\simeq\mathbb Q\upharpoonright(0,1), \ \mathbb Q\upharpoonright(0,1], \ \mathbb Q \upharpoonright [0,1), \ \mathbb Q \upharpoonright [0,1]
        $$
        \vspace{-20pt}
    }
    \textbf{Důkaz:}
    Husté uspořádání nemůže být konečné. Izomorfismus zobrazí minimum na minimum a maximum na maximum.
    \hfill\qedsymbol

\end{frame}


\begin{frame}{$\omega$-kategorické kritérium kompletnosti}

    %\alert{$\omega$-kategoricita} je zeslabení pojmu \alert{kompletnosti}

    \myblock{
        \textbf{Věta:}
        Buď $T$ $\omega$-kategorická ve spočetném jazyce $L$. Je-li
        \begin{enumerate}[(i)]
            \item $L$ bez rovnosti, nebo
            \item $L$ s rovností a $T$ nemá konečné modely,
        \end{enumerate}
        potom je $T$ kompletní.
    }

    \textbf{Důkaz:}
    \alert{(i)} Důsledek L.-S. věty bez rovnosti říká, že každý model je elementárně ekvivalentní nějakému spočetně nekonečnému, ten je ale až na izomorfismus jediný.
    
    \alert{(ii)} Důsledek L.-S. věty s rovností podobně říká, že všechny nekonečné modely jsou elementárně ekvivalentní. Mohla by mít elementárně neekvivalentní konečné modely, to jsme ale zakázali. \hfill\qedsymbol

    \medskip

    \myblock{
        \textbf{Důsledek:}
        $\DeLO$, $\DeLO^+$, $\DeLO^-$, a $\DeLO^\pm$ jsou kompletní, jsou to všechny (navzájem neekvivalentní) kompletní jedn. extenze $DeLO^*$.
    }

    Analogické kritérium platí i pro kardinality $\kappa$ větší než $\omega$.

\end{frame}


\section{9.4 Axiomatizovatelnost}


\begin{frame}{Axiomatizovatelnost}
        
    Třída struktur $K\subseteq\M_L$ je:
    
    \begin{itemize}
        \item \alert{axiomatizovatelná}, existuje-li teorie $T$ taková, že $\M_L(T)=K$
        \item \alert{konečně}/\alert{otevřeně} axiomatiz., je-li ax. konečnou/otevřenou $T$
        \item teorie $T'$ je \alert{konečně}/\alert{otevřeně} axiomatizovatelná, platí-li to o třídě jejích modelů $K=\M_L(T')$
    \end{itemize}

    \textbf{Pozorování:} Je-li $K$ axiomatizovatelná, musí být uzavřená na $\equiv$.

    \medskip
    
    \myexample{
    Například, jak ukážeme:
    \begin{itemize}
        \item grafy a částečná uspořádání jsou konečně i otevřeně ax.
        \item tělesa jsou konečně, ale ne otevřeně axiomatizovatelná
        \item nekonečné grupy jsou axiomatizovatelné, ale ne konečně
        \item konečné grafy nejsou axiomatizovatelné
    \end{itemize}
    }

\end{frame}


\begin{frame}{Neaxiomatizovatelnost konečných modelů}

    \smallskip

    \myblock{
        \textbf{Věta:}
        Má-li $T$ libovolně velké konečné modely, má i nekonečný model. Potom není třída jejích konečných modelů axiomatizovatelná.
    }

    \textbf{Důkaz:}
    \alert{Je-li jazyk bez rovnosti,} vezmeme  kanonický model pro bezespornou větev v tablu z $T$ pro $\F\bot$ ($T$ je bezesporná).
    
    \alert{Je-li jazyk s rovností,} přidáme spočetně mnoho nových konst. symbolů $c_i$ a vezmeme extenzi: \myalertinline{
        $T'=T \cup \{\neg c_i = c_j \mid i\neq j\in\mathbb N\}$
    }

    \vspace{-2pt}

    Každá \alert{konečná část} $T'$ má model: buď $k$ největší, že $c_k$ je v této konečné části: lib. $\geq(k+1)$-prvkový model,21 interpretuj $c_0,\dots,c_k$ jako různé prvky.

    \vspace{-2pt}

    \alert{Věta o kompaktnosti} dává model $T'$, ten je nekonečný, redukt na původní jazyk (zapomenutí $c_i^\A$) je nekonečný model $T$.\hfill\qedsymbol

    \begin{itemize}
        \item např. \myexampleinline{konečné grafy} nejsou axiomatizovatelné
        \item \alert{nekonečné modely} teorie jsou vždy axiomatizovatelné, máme-li rovnost: stačí přidat \myalertinline{`existuje alespoň $n$ prvků'} pro vš. $n\in\mathbb N$
    \end{itemize}

\end{frame}


\begin{frame}{Konečná axiomatizovatelnost}
    
    \smallskip

    \myblock{
        \textbf{Věta (O konečné axiomatizovatelnosti):}
        $K\subseteq \M_L$ je konečně axiomatizovatelná, právě když $K$ i $\overline{K}=\M_L\setminus K$ jsou axiomatizovatelné.
    }

    \textbf{Důkaz:} \alert{\Large$\Rightarrow$}
    Je-li $K$ axiomatizovatelná \alert{sentencemi} $\varphi_1,\dots,\varphi_n$ (vezmi gen. uzávěry), potom $\neg(\varphi_1\land\varphi_2\land\dots\land\varphi_n)$ axiomatizuje $\overline{K}$.

    \alert{\Large$\Leftarrow$} Buď $K=\M(T)$ a $\overline{K}=\M(S)$. Potom  \alert{$T\cup S$ je sporná}, neboť:
    $$
    \M(T\cup S)=\M(T)\cap \M(S)=K\cap\overline{K}=\emptyset
    $$
    \alert{Věta o kompaktnosti} dává konečné $T'\subseteq T$ a $S'\subseteq S$ takové, že:
    $$
    \emptyset = \M(T'\cup S')=\M(T')\cap \M(S')
    $$
    Nyní si všimněme, že platí:
    $$
    \M(T)\subseteq \M(T')\subseteq \overline{\M(S')}\subseteq \overline{\M(S)}=\M(T)
    $$
    Tím jsme dokázali, že \alert{$M(T)=M(T')$}, neboli $T'$ je konečná axiomatizace $K$. \hfill\qedsymbol

\end{frame}


\begin{frame}{Tělesa charakteristiky 0 nejsou konečně axiomatizovatelná}

    Buď $T$ teorie těles. Těleso $\A=\langle A,+,-,0,\cdot,1 \rangle$ je
    \vspace{-6pt}
    \begin{itemize}
        \item \alert{charakteristiky $p$}, je-li $p$ nejmenší prvočíslo takové, že $\A\models p1=0$, kde $p1$ je term $1+1+\dots+1$ (s $p$ jedničkami),
        \item \alert{charakteristiky 0}, pokud není charakteristiky $p$ pro žádné $p$.
        \item Tělesa charakteristiky $p$ jsou konečně axiomatizovatelná: \myalertmath{\small
            $$
            T_p=T\cup \{p1=0\}
            $$
        }
        \item Tělesa char. 0 jsou axiomatizovatelná, ale ne konečně: \myalertmath{\small
            $$
            T_0=T\cup \{\neg\, p1=0\mid p\text{ prvočíslo}\}
            $$
        }
    \end{itemize}
    \myblock{\vspace{2pt}
        \textbf{Tvrzení:}
        Třída $K$ těles char. $0$ není konečně axiomatizovatelná.
        \vspace{2pt}
    }
    
    \textbf{Důkaz:}
    Stačí ukázat, že $\overline{K}$ (tělesa nenulové char. a netělesa) není axiomatizovatelná. \alert{Sporem: $\overline{K}=\M(S)$.} Potom  
    $S'=S\cup T_0$ má model, neboť každá konečná část má model: těleso charakteristiky větší než jakékoliv $p$ z axiomu $T_0$ tvaru $\neg\, p1=0$. Je-li $\A$ je model $S'$, potom $\A\in\M(S)=\overline{K}$. Zároveň ale $\A\in\M(T_0)=K$, spor.\hfill\qedsymbol

\end{frame}


\begin{frame}{Otevřená axiomatizovatelnost}

    \smallskip

    \myblock{\vspace{2pt}
        \textbf{Tvrzení:}
        Je-li $T$ otevřeně axiomatizovatelná, potom je každá podstruktura modelu $T$ také modelem $T$.
        \vspace{2pt}  
    }
    
    \textbf{Důkaz:}
    Buď $T'$ otevřená axiomatizace $T$, $\A$ model $T'$, $\B\subseteq\A$. Pro každou $\varphi\in T'$ platí $\B\models\varphi$ ($\varphi$ je otevřená), tedy i $\B\models T'$.  
    \hfill\qedsymbol

    \textbf{Poznámka:} Platí i obráceně, je-li každá podstruktura modelu také model, potom je otevřeně axiomatizovatelná. (Důkaz neuvedeme.)

    \begin{itemize}
        \item \myexampleinline{DeLO} není otevřeně axiomatizovatelná, např. žádná konečná podstruktura modelu DeLO není hustá
        \item \myexampleinline{teorie těles} není otevřeně axiomatizovatelná, podstruktura $\mathbb Z\subseteq\mathbb Q$ není těleso, nemá inverzní prvek k $2$ vůči násobení
        \item pro dané $n\in\mathbb N$ jsou \myexampleinline{nejvýše $n$-prvkové grupy} otevřeně axiomatizovatelné (i jejich podgrupy jsou nejvýše $n$-prvkové); k (otevřené) teorii grup stačí přidat: \myalertinline{
            $\bigvee_{1\leq i<j\leq n+1}x_i=x_j$
        }
    \end{itemize}

\end{frame}


\section{\sc Kapitola 10: Nerozhodnutelnost a neúplnost}


\begin{frame}{Nerozhodnutelnost a neúplnost}

    Jak lze s teoriemi pracovat algoritmicky?

    \medskip

    + zlatý hřeb přednášky: \alert{Gödelovy věty o neúplnosti} (1931)
    \begin{itemize}
        \item ukazují limity formálního přístupu
        \item zastavily program formalizace matematiky
        \item pojem \alert{algoritmu} budeme chápat jen intuitivně
        \item technické podrobnosti důkazů vynecháme
        
    \end{itemize}

    Typicky potřebujeme spočetný jazyk.
     
\end{frame}


\section{10.1 Rekurzivní axiomatizace a rozhodnutelnost}


\begin{frame}{Rekurzivní axiomatizace}

    \begin{itemize}
        \item v dokazování povolujeme nekonečné teorie, jak jsou zadané?
        \item pro ověření že daný důkaz (např. tablo, rezoluční zamítnutí) je korektní potřebujeme algoritmický přístup ke všem axiomům
        \item mohli bychom požadovat \alert{enumerátor} pro $T$, tj. algoritmus, který vypisuje axiomy z $T$, a každý axiom někdy vypíše
        \item ale kdyby byl v důkazu chybný axiom, nikdy bychom se to nedozvěděli: stále bychom čekali, zda ho enumerátor vypíše
        \item proto požadujeme silnější vlastnost:
    \end{itemize}

    \myblock{
        $T$ je \alert{rekurzivně axiomatizovaná}, pokud existuje algoritmus, který pro každou vstupní formuli $\varphi$ doběhne a odpoví, zda $\varphi\in T$.
    }

    (ekvivalentní enumerátoru vypisujícímu axiomy v lexikograf. pořadí)  

\end{frame}


\begin{frame}{Rozhodnutelnost}

    \vspace{-3pt}
    Můžeme v dané teorii \alert{`algoritmicky rozhodovat pravdu'}?

    \vspace{-3pt}
    \myblock{
        \begin{itemize}
            \item $T$ je \alert{rozhodnutelná}, pokud existuje algoritmus, který pro každou vstupní formuli $\varphi$ doběhne a odpoví, zda $T\models\varphi$,
            \item $T$ je \alert{částečně rozhodnutelná}, existuje-li algoritmus, který: %pro každou vstupní formuli:
            \begin{itemize}
                \item pokud $T\models\varphi$, doběhne a odpoví ``ano''
                \item pokud $T\not\models\varphi$, buď nedoběhne, nebo doběhne a odpoví ``ne''
            \end{itemize}
        \end{itemize}
    }

    
    \myblock{
        \textbf{Tvrzení:}
        Je-li $T$ je rekurzivně axiomatizovaná, potom:
                
        (i) $T$ je část. rozhod. \hfill (ii) je-li navíc kompletní, je rozhodnutelná
    }
    
    
    \textbf{Důkaz:} \alert{(i)} Algoritmus konstruuje systematické tablo z $T$ pro $\F\varphi$; stačí enumerátor pro $T$, nebo postupně generovat vš. sentence a testovat, jsou-li v $T$. Je-li $T\models\varphi$, konstrukce skončí, ověříme, že je tablo sporné. (Jinak skončit nemusí.)
    
    \vspace{-3pt}  
    \alert{(ii)}
    Víme, že buď $T\proves\varphi$ nebo $T\proves\neg\varphi$. Paralelně konstruujeme tablo pro $\F\varphi$ a pro $\T\varphi$ (důkaz a zamítnutí $\varphi$ z $T$). Jedna z konstrukcí po konečně mnoha krocích skončí.
    \hfill\qedsymbol


\end{frame}


\begin{frame}{Rekurzivně spočetná kompletace}

    \myblock{
        $T$ má \alert{rekurzivně spočetnou kompletaci}, je-li (nějaká) množina až na $\sim$ všech jednoduchých kompletních extenzí $T$ \alert{rekurzivně spočetná}, tj. existuje algoritmus, který pro vstup $(i,j)$ vypíše $i$-tý axiom $j$-té extenze (v nějakém uspořádání), nebo odpoví, že už neexistuje.
    }
     
    \myblock{
        \textbf{Tvrzení:}
        Je-li $T$ rekurzivně axiomatizovaná a má rekurzivně spočetnou kompletaci, potom je rozhodnutelná.
    }
        
    \textbf{Důkaz:}
    Buď $T\proves\varphi$, nebo existuje protipříklad $\A\not\models\varphi$, tj. kompl. jedn. extenze $T_i$, že $T_i\not\proves\varphi$. Kompletnost $T_i$ dává $T_i\proves\neg\varphi$. 
    
    Algoritmus paralelně konstruuje tablo důkaz $\varphi$ z $T$ a (postupně) tablo důkazy $\neg\varphi$ ze všech kompletních jedn. extenzí $T_1,T_2,\dots$. (Je-li jich nekonečně mnoho, uděláme \alert{dovetailing}: 1. krok 1. tabla, potom 2. krok 1., 1. krok 2., 3. krok 1., 2. krok 2., 1. krok 3., atd.)
    
    Alespoň jedno z tabel je sporné, můžeme předpokládat konečné, algoritmus ho po konečně mnoha krocích zkonstruuje.
    \hfill\qedsymbol

\end{frame}


\begin{frame}{Příklady}
    
    Následující teorie jsou rekurzivně axiomatizované a mají rekurzivně spočetnou kompletaci, tedy jsou rozhodnutelné:

    \myexample{
    \begin{enumerate}[(a)]
    \item Teorie čisté rovnosti
    \item Teorie unárního predikátu ($T=\emptyset$, $L=\langle U \rangle$ s rovností)
    \item Teorie hustých lineárních uspořádání DeLO*
    \item Teorie Booleových algeber (Alfred Tarski 1940),
    \item Teorie algebraicky uzavřených těles (Tarski 1949),
    \item Teorie komutativních grup (Wanda Szmielew 1955).
    \end{enumerate}
    }

\end{frame}


\begin{frame}{Rekurzivní axiomatizovatelnost}

    Kdy lze třídu struktur `efektivně (algoritmicky) popsat'?

    \myblock{
        $K\subseteq\M_L$ je \alert{rek. axiomatizovatelná}, pokud existuje rek. axiomatizovaná $T$, že $K=M_L(T)$. $T'$ je \alert{rek. axiomatizovatelná}, platí-li to pro třídu jejích modelů (tj. je-li ekvivalentní rek. axiomatizované teorii).
    }

    (podobně lze definovat \alert{rek. spočetnou axiomatizovatelnost})

    \myblock{
        \textbf{Tvrzení:}
        Je-li $\A$ konečná struktura v konečném jazyce s rovností, potom je teorie $\Th(\A)$ rekurzivně axiomatizovatelná.
    }
        
    (z toho plyne i rozhodnutelnost $\Th(\A)$, ale $\A\models\varphi$ lze ověřit přímo)

    \textbf{Důkaz:}
    Buď $A=\{a_1,\dots,a_n\}$. $\Th(\A)$ axiomatizujeme sentencí ``existuje právě $n$ prvků $a_1,\dots,a_n$ splňujících právě ty \alert{základní vztahy} o funkčních hodnotách a relacích, které platí v $\A$''.
    
    Např. je-li $f^\A(a_4, a_2)=a_{17}$, přidej atom. formuli $f(x_{a_4},x_{a_2})=x_{a_{17}}$, je-li $(a_3,a_3,a_1)\notin R^\A$ přidej $\neg R(x_{a_3},x_{a_3},x_{a_1})$.\hfill\qedsymbol

\end{frame}


\begin{frame}{Příklady}

    Pro následující struktury je $\Th(\A)$ rekurzivně axiomatizovatelná:

    \myexample{
        \begin{itemize}
            \item $\langle\mathbb Z,\leq\rangle$, jde o tzv. teorii \alert{diskrétních lineárních uspořádání}        
            \item $\langle\mathbb Q,\leq\rangle$, jde o teorii DeLO
            \item $\langle\mathbb N,S,0\rangle$, teorie \alert{následníka s nulou}
            \item $\langle\mathbb N,S,+,0\rangle$, \alert{Presburgerova aritmetika}
            \item $\langle\mathbb R,+,-,\cdot,0,1\rangle$, teorie \alert{reálně uzavřených těles}, znamená že lze algoritmicky rozhodovat Euklid. geometrii (Tarski, 1949)
            \item $\langle \mathbb C,+,-,\cdot,0,1 \rangle$, teorie \alert{algebraicky uzavřených těles char. 0}
        \end{itemize}
    }

    \medskip

    \myblock{
        \textbf{Důsledek:}
        Pro struktury výše platí, že $\Th(\A)$ je rozhodnutelná.
    }
    \textbf{Důkaz:} $\Th(\A)$ je vždy kompletní.

    \myexample{
        Teorie \alert{standardního modelu aritmetiky} $\underline{\mathbb N}=\langle\mathbb N,S,+,\cdot,0,\leq\rangle$ ale \alert{není} rekurzivně axiomatizovatelná (viz První Gödelova věta o neúplnosti).
    }

\end{frame}



\end{document}


