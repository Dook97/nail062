\documentclass{beamer}

%% slide-specific

\usetheme[progressbar=frametitle]{metropolis}
%\usecolortheme{spruce}
%\metroset{block=fill}

% define Metropolis colors    
\definecolor{mAlert}{HTML}{EB811B}
\definecolor{mExample}{HTML}{14B03D}
\definecolor{mBlock}{HTML}{23373b}

% my blocks
\setlength\fboxsep{0pt}%

\newcommand{\myblock}[1]{\colorbox{mBlock!8}{\begin{minipage}{\linewidth}#1\end{minipage}}}
\newcommand{\myblockmath}[1]{\colorbox{mBlock!8}{\begin{minipage}{\linewidth}\vspace{-6pt}#1\end{minipage}}}
\newcommand{\myblockinline}[1]{\colorbox{mBlock!8}{#1}}
\newcommand{\myexample}[1]{\colorbox{mExample!8}{\begin{minipage}{\linewidth}#1\end{minipage}}}
\newcommand{\myexamplemath}[1]{\colorbox{mExample!8}{\begin{minipage}{\linewidth}\vspace{-6pt}#1\end{minipage}}}
\newcommand{\myexampleinline}[1]{\colorbox{mExample!8}{#1}}
\newcommand{\myalert}[1]{\colorbox{mAlert!8}{\begin{minipage}{\linewidth}#1\end{minipage}}}
\newcommand{\myalertmath}[1]{\colorbox{mAlert!8}{\begin{minipage}{\linewidth}\vspace{-6pt}#1\end{minipage}}}
\newcommand{\myalertinline}[1]{\colorbox{mAlert!8}{#1}}

%% other
\newcommand{\mystructure}[1]{\mathcal{#1}}




%% packages
\usepackage{amsmath,amssymb,amsthm}
\usepackage{booktabs}
\usepackage[czech]{babel}
\usepackage{enumerate}
\usepackage{forest}
\usepackage{multicol}
% \usepackage{tcolorbox}
\usepackage{tikz}
    \usetikzlibrary{arrows.meta}
%\usepackage[unicode]{hyperref}
\usepackage[utf8x]{inputenc}
\usepackage{xfrac}

% %% theorems
% \theoremstyle{plain}
%     \newtheorem{theorem}{Věta}[section]
%     \newtheorem*{theorem-unnumbered}{Věta}
%     \newtheorem{proposition}[theorem]{Tvrzení}
%     \newtheorem{corollary}[theorem]{Důsledek}
%     \newtheorem{lemma}[theorem]{Lemma}
%     \newtheorem{observation}[theorem]{Pozorování}
% \theoremstyle{definition}
%     \newtheorem{definition}[theorem]{Definice}
%     \newtheorem*{algorithm}{Algoritmus}
% \theoremstyle{remark}
%     \newtheorem{remark}[theorem]{Poznámka}
%     \newtheorem{example}[theorem]{Příklad}
%     \newtheorem{exercise}{Cvičení}[chapter]
%     \newtheorem*{solution}{Řešení}

%% macros and definitions
\DeclareMathOperator{\Aut}{Aut}
\DeclareMathOperator{\Conseq}{Csq}
\DeclareMathOperator{\DeLO}{DeLO}
\DeclareMathOperator{\dom}{dom}
\DeclareMathOperator{\Fm}{Fm}
\DeclareMathOperator{\M}{M}
%\DeclareMathOperator{\Proof}{Proof}
\DeclareMathOperator{\rng}{rng}
\DeclareMathOperator{\Term}{Term}
\DeclareMathOperator{\Th}{Th}
\DeclareMathOperator{\Thm}{Thm}
\DeclareMathOperator{\Tree}{Tree}
\DeclareMathOperator{\Var}{Var}
\DeclareMathOperator{\VF}{VF}

\newcommand{\A}{\structure{A}}
\newcommand{\B}{\structure{B}}
\newcommand{\Con}{\mathit{Con}}
\newcommand{\disjointunion}{\mathbin{\dot{\sqcup}}}
\newcommand{\F}{\ensuremath{\mathrm{F}}}
\newcommand{\landsymb}{{\land}}
\newcommand{\lbin}{\mathbin{\square}}
\newcommand{\lbinsymb}{{\lbin}}
\newcommand{\liff}{\mathbin{\leftrightarrow}}
\newcommand{\liffsymb}{{\liff}}
\newcommand{\limplies}{\mathbin{\rightarrow}}
\newcommand{\limpliessymb}{{\limplies}}
\newcommand{\lorsymb}{{\lor}}
\newcommand{\Prf}{\mathit{Prf}}
\newcommand{\proves}{\vdash}
%\newcommand{\structure}[1]{\mathcal{#1}}
\newcommand{\todo}{[TODO]}
\newcommand{\T}{\ensuremath{\mathrm{T}}}
\newcommand{\union}{\mathbin{\cup}}


\title{Třináctá přednáška}
\subtitle{NAIL062 Výroková a predikátová logika}
\author{Jakub Bulín (KTIML MFF UK)}
% \institute{KTIML MFF UK}
\date{Zimní semestr 2023}


\begin{document}


\frame{\titlepage}


\begin{frame}{Třináctá přednáška}

    \textbf{Program}
        \begin{itemize}
            \item aritmetické teorie
            \item nerozhodnutelnost predikátové logiky
            \item Gödelovy věty o neúplnosti
        \end{itemize}      
    
    \textbf{Materiály}

        \href{https://github.com/jbulin-mff-uk/nail062/raw/main/lecture/lecture-notes/lecture-notes.pdf}{\alert{\textbf{Zápisky z přednášky}}}, Sekce 10.2-10.4 z Kapitoly 10

\end{frame}


\section{10.2 Aritmetika}


\begin{frame}{Aritmetika}

    \begin{itemize}
        \item přirozená čísla hrají důležitou roli v matematice i v aplikacích
        \item \alert{jazyk aritmetiky} je $L=\langle S,+,\cdot,0,\leq\rangle$ s rovností
        \item \alert{standardní model aritmetiky}  $\underline{\mathbb N}=\langle\mathbb N,S,+,\cdot,0,\leq\rangle$ nemá rekurzivně axiomatizovatelnou teorii (První věta o neúplnosti)
        \item proto používáme rekurzivně axiomatizované teorie, které vlastnosti $\underline{\mathbb N}$ popisují částečně; říkáme jim \alert{aritmetiky}       
        \item představíme dvě: \alert{Robinsonovu} $Q$ a \alert{Peanovu} $PA$
    \end{itemize}

\end{frame}


\begin{frame}{Robinsonova aritmetika $Q$}    

    \myblockamsmath{ 
        \vspace{-12pt}       
        \begin{align*}
            &\neg S(x) = 0& &x\cdot 0=0\\
            &S(x)=S(y)\rightarrow x=y& &x\cdot S(y)=x\cdot y+x\\
            &x+0=x& &\neg x=0 \rightarrow (\exists y)(x=S(y))\\
            &x+S(y)=S(x+y)& &x\le y \leftrightarrow (\exists z)(z+x=y)\qquad
        \end{align*}
    }
    
    \begin{itemize}
        \item velmi slabá, nelze v ní dokázat např. komutativitu ani asociativitu $+$ či $\cdot$, nebo tranzitivitu $\leq$
        \item ale lze dokázat všechna \alert{existenční tvrzení o numerálech} pravdivá v $\underline{\mathbb N}$, tj. formule v PNF, jen $\exists$, za volné proměnné substituujeme \alert{numerály} $\underline{n}=S(\dots S(0)\dots)$
        \item např. pro \myexampleinline{
            $\varphi(x,y)=(\exists z)(x+z=y)$
         } je $Q\proves\varphi(\underline{1},\underline{2})$
    \end{itemize}

    \medskip
    
    \myblock{
        \textbf{Tvrzení:}
        Je-li $\varphi(x_1,\dots,x_n)$ existenční formule, $a_1,\dots,a_n\in\mathbb N$, pak
        $Q\proves\varphi(x_1/\underline{a_1},\dots,x_n/\underline{a_n})$ právě když $\underline{\mathbb{N}}\models \varphi[e(x_1/a_1,\dots,x_n/a_n)]
        $
    }

    (Důkaz vynecháme.)

\end{frame}


\begin{frame}{Peanova aritmetika $PA$}    

    \myblock{
        Extenze $Q$ o \alert{schéma indukce}, tj. pro každou $L$-formuli $\varphi(x,\overline{y})$:
        \vspace{-6pt}
        $$
        (\varphi(0,\overline{y}) \land (\forall x)(\varphi(x,\overline{y})\limplies \varphi(S(x),\overline{y}))) \limplies (\forall x)\varphi(x,\overline{y})
        $$
        \vspace{-21pt}
    }

    \begin{itemize}
        \item mnohem lepší aproximace $\Th(\underline{\mathbb N})$
        \item dokáže `základní' vlastnosti (např. komut. a asociativitu $+$) \item stále ale existují sentence platné v $\underline{\mathbb N}$ ale nezávislé v $PA$\\(opět dokážeme v První větě o neúplnosti)
    \end{itemize}

    \bigskip

    \textbf{Poznámka:} strukturu $\underline{\mathbb N}$ lze axiomatizovat (až na $\simeq$) v predikátové logice \alert{2. řádu}, extenzí $PA$ o tzv. \alert{axiom indukce}:
    $$
    (\forall X)((X(0) \land (\forall x)(X(x) \limplies X(S(x)))) \limplies (\forall x)X(x))
    $$

    \begin{itemize}
        \item $X$ reprezentuje (libovolnou) podmnožinu modelu
        \item použijeme na množinu všech následníků 0
        \item každý prvek je následník 0 $\Rightarrow$ izomorfismus s $\underline{\mathbb N}$
    \end{itemize}

\end{frame}


\section{10.3 Nerozhodnutelnost predikátové logiky}


\begin{frame}{Nerozhodnutelnost predikátové logiky}
    
    \myblock{
        \textbf{Věta (O nerozhodnutelnosti predikátové logiky):}
        Neexistuje algoritmus, který pro vstupní formuli $\varphi$ rozhodne, zda je logicky platná.
    }
    \begin{itemize}
        \item tj. zda je formule $\varphi$ [v lib. jazyce 1. řádu] tautologie ($\models\varphi$)
        \item neboli $T=\emptyset$ není rozhodnutelná
    \end{itemize}

    %Protože zatím neznáme potřebný formalismus týkající se algoritmů, např. pojem Turingova stroje, zvolíme jako výchozí bod jiný \alert{nerozhodnutelný problém}. Nejznámějším je tzv. \alert{Halting problem}, tj. otázka, zda se daný program zastaví na daném vstupu.\footnote{Jeho nerozhodnutelnost si dokážete v předmětech NTIN071 Automaty a gramatiky a poté znovu v NTIN090 Základy složitosti a vyčíslitelnosti.} My si ale usnadníme práci tím, že zvolíme jiný nerozhodnutelný problém, tzv. \alert{Hilbertův desátý problém}.\footnote{Hilbert jej vyslovil v roce 1900, a publikoval v roce 1902 spolu s 22 dalšími problémy, které významně ovlivnily matematiku 20., i 21. století. Některé zůstávají nevyřešeny, např. Riemannova hypotéza, \href{https://en.wikipedia.org/wiki/Logic_in_computer_science}{viz Wikipedia}.}

\end{frame}


\begin{frame}{Hilbertův desátý problém}

\end{frame}


\begin{frame}{Důkaz nerozhodnutelnosti}

\end{frame}


\section{10.4 Gödelovy věty}


\begin{frame}{První věta o neúplnosti}

    

\end{frame}


\begin{frame}{Aritmetizace dokazatelnosti}
    

\end{frame}


\begin{frame}{Self-reference}
    

\end{frame}


\begin{frame}{Věta o pevném bodě}
    

\end{frame}


\begin{frame}{Nedefinovatelnost pravdy}

    
\end{frame}


\begin{frame}{Důkaz První věty o neúplnosti}

    
\end{frame}


\begin{frame}{Důsledky První věty o neúplnosti}
    

\end{frame}


\begin{frame}{Druhá věta o neúplnosti}
    

\end{frame}


\begin{frame}{Důsledky Druhé věty o neúplnosti}

    
\end{frame}


\end{document}


