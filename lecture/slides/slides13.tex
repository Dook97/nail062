\documentclass{beamer}

%% slide-specific

\usetheme[progressbar=frametitle]{metropolis}
%\usecolortheme{spruce}
%\metroset{block=fill}

% define Metropolis colors    
\definecolor{mAlert}{HTML}{EB811B}
\definecolor{mExample}{HTML}{14B03D}
\definecolor{mBlock}{HTML}{23373b}

% my blocks
\setlength\fboxsep{0pt}%

\newcommand{\myblock}[1]{\colorbox{mBlock!8}{\begin{minipage}{\linewidth}#1\end{minipage}}}
\newcommand{\myblockmath}[1]{\colorbox{mBlock!8}{\begin{minipage}{\linewidth}\vspace{-6pt}#1\end{minipage}}}
\newcommand{\myblockinline}[1]{\colorbox{mBlock!8}{#1}}
\newcommand{\myexample}[1]{\colorbox{mExample!8}{\begin{minipage}{\linewidth}#1\end{minipage}}}
\newcommand{\myexamplemath}[1]{\colorbox{mExample!8}{\begin{minipage}{\linewidth}\vspace{-6pt}#1\end{minipage}}}
\newcommand{\myexampleinline}[1]{\colorbox{mExample!8}{#1}}
\newcommand{\myalert}[1]{\colorbox{mAlert!8}{\begin{minipage}{\linewidth}#1\end{minipage}}}
\newcommand{\myalertmath}[1]{\colorbox{mAlert!8}{\begin{minipage}{\linewidth}\vspace{-6pt}#1\end{minipage}}}
\newcommand{\myalertinline}[1]{\colorbox{mAlert!8}{#1}}

%% other
\newcommand{\mystructure}[1]{\mathcal{#1}}




%% packages
\usepackage{amsmath,amssymb,amsthm}
\usepackage{booktabs}
\usepackage[czech]{babel}
\usepackage{enumerate}
\usepackage{forest}
\usepackage{multicol}
% \usepackage{tcolorbox}
\usepackage{tikz}
    \usetikzlibrary{arrows.meta}
%\usepackage[unicode]{hyperref}
\usepackage[utf8x]{inputenc}
\usepackage{xfrac}

% %% theorems
% \theoremstyle{plain}
%     \newtheorem{theorem}{Věta}[section]
%     \newtheorem*{theorem-unnumbered}{Věta}
%     \newtheorem{proposition}[theorem]{Tvrzení}
%     \newtheorem{corollary}[theorem]{Důsledek}
%     \newtheorem{lemma}[theorem]{Lemma}
%     \newtheorem{observation}[theorem]{Pozorování}
% \theoremstyle{definition}
%     \newtheorem{definition}[theorem]{Definice}
%     \newtheorem*{algorithm}{Algoritmus}
% \theoremstyle{remark}
%     \newtheorem{remark}[theorem]{Poznámka}
%     \newtheorem{example}[theorem]{Příklad}
%     \newtheorem{exercise}{Cvičení}[chapter]
%     \newtheorem*{solution}{Řešení}

%% macros and definitions
\DeclareMathOperator{\Aut}{Aut}
\DeclareMathOperator{\Conseq}{Csq}
\DeclareMathOperator{\DeLO}{DeLO}
\DeclareMathOperator{\dom}{dom}
\DeclareMathOperator{\Fm}{Fm}
\DeclareMathOperator{\M}{M}
%\DeclareMathOperator{\Proof}{Proof}
\DeclareMathOperator{\rng}{rng}
\DeclareMathOperator{\Term}{Term}
\DeclareMathOperator{\Th}{Th}
\DeclareMathOperator{\Thm}{Thm}
\DeclareMathOperator{\Tree}{Tree}
\DeclareMathOperator{\Var}{Var}
\DeclareMathOperator{\VF}{VF}

\newcommand{\A}{\structure{A}}
\newcommand{\B}{\structure{B}}
\newcommand{\Con}{\mathit{Con}}
\newcommand{\disjointunion}{\mathbin{\dot{\sqcup}}}
\newcommand{\F}{\ensuremath{\mathrm{F}}}
\newcommand{\landsymb}{{\land}}
\newcommand{\lbin}{\mathbin{\square}}
\newcommand{\lbinsymb}{{\lbin}}
\newcommand{\liff}{\mathbin{\leftrightarrow}}
\newcommand{\liffsymb}{{\liff}}
\newcommand{\limplies}{\mathbin{\rightarrow}}
\newcommand{\limpliessymb}{{\limplies}}
\newcommand{\lorsymb}{{\lor}}
\newcommand{\Prf}{\mathit{Prf}}
\newcommand{\proves}{\vdash}
%\newcommand{\structure}[1]{\mathcal{#1}}
\newcommand{\todo}{[TODO]}
\newcommand{\T}{\ensuremath{\mathrm{T}}}
\newcommand{\union}{\mathbin{\cup}}


\title{Třináctá přednáška}
\subtitle{NAIL062 Výroková a predikátová logika}
\author{Jakub Bulín (KTIML MFF UK)}
% \institute{KTIML MFF UK}
\date{Zimní semestr 2023}


\begin{document}


\frame{\titlepage}


\begin{frame}{Třináctá přednáška}

    \textbf{Program}
        \begin{itemize}
            \item aritmetické teorie
            \item nerozhodnutelnost predikátové logiky
            \item Gödelovy věty o neúplnosti
        \end{itemize}      
    
    \textbf{Materiály}

        \href{https://github.com/jbulin-mff-uk/nail062/raw/main/lecture/lecture-notes/lecture-notes.pdf}{\alert{\textbf{Zápisky z přednášky}}}, Sekce 10.2-10.4 z Kapitoly 10

\end{frame}


\section{10.2 Aritmetika}


\begin{frame}{Aritmetika}

    \begin{itemize}
        \item přirozená čísla hrají důležitou roli v matematice i v aplikacích
        \item \alert{jazyk aritmetiky} je $L=\langle S,+,\cdot,0,\leq\rangle$ s rovností
        \item \alert{standardní model aritmetiky}  $\underline{\mathbb N}=\langle\mathbb N,S,+,\cdot,0,\leq\rangle$ nemá rekurzivně axiomatizovatelnou teorii (První věta o neúplnosti)
        \item proto používáme rekurzivně axiomatizované teorie, které vlastnosti $\underline{\mathbb N}$ popisují částečně; říkáme jim \alert{aritmetiky}       
        \item představíme dvě: \alert{Robinsonovu} $Q$ a \alert{Peanovu} $PA$
    \end{itemize}

\end{frame}


\begin{frame}{Robinsonova aritmetika $Q$}    

    \myblockamsmath{ 
        \vspace{-12pt}       
        \begin{align*}
            &\neg S(x) = 0& &x\cdot 0=0\\
            &S(x)=S(y)\rightarrow x=y& &x\cdot S(y)=x\cdot y+x\\
            &x+0=x& &\neg x=0 \rightarrow (\exists y)(x=S(y))\\
            &x+S(y)=S(x+y)& &x\le y \leftrightarrow (\exists z)(z+x=y)\qquad
        \end{align*}
    }
    
    \begin{itemize}
        \item velmi slabá, nelze v ní dokázat např. komutativitu ani asociativitu $+$ či $\cdot$, nebo tranzitivitu $\leq$
        \item ale lze dokázat všechna \alert{existenční tvrzení o numerálech} pravdivá v $\underline{\mathbb N}$, tj. formule v PNF, jen $\exists$, za volné proměnné substituujeme \alert{numerály} $\underline{n}=S(\dots S(0)\dots)$
        \item např. pro \myexampleinline{
            $\varphi(x,y)=(\exists z)(x+z=y)$
         } je $Q\proves\varphi(\underline{1},\underline{2})$
    \end{itemize}

    \medskip
    
    \myblock{
        \textbf{Tvrzení:}
        Je-li $\varphi(x_1,\dots,x_n)$ existenční formule, $a_1,\dots,a_n\in\mathbb N$, pak
        $Q\proves\varphi(x_1/\underline{a_1},\dots,x_n/\underline{a_n})$ právě když $\underline{\mathbb{N}}\models \varphi[e(x_1/a_1,\dots,x_n/a_n)]
        $
    }

    (Důkaz vynecháme.)

\end{frame}


\begin{frame}{Peanova aritmetika $PA$}    

    \myblock{
        Extenze $Q$ o \alert{schéma indukce}, tj. pro každou $L$-formuli $\varphi(x,\overline{y})$:
        \vspace{-6pt}
        $$
        (\varphi(0,\overline{y}) \land (\forall x)(\varphi(x,\overline{y})\limplies \varphi(S(x),\overline{y}))) \limplies (\forall x)\varphi(x,\overline{y})
        $$
        \vspace{-21pt}
    }

    \begin{itemize}
        \item mnohem lepší aproximace $\Th(\underline{\mathbb N})$
        \item dokáže `základní' vlastnosti (např. komut. a asociativitu $+$) \item stále ale existují sentence platné v $\underline{\mathbb N}$ ale nezávislé v $PA$\\(opět dokážeme v První větě o neúplnosti)
    \end{itemize}

    \bigskip

    \textbf{Poznámka:} strukturu $\underline{\mathbb N}$ lze axiomatizovat (až na $\simeq$) v predikátové logice \alert{2. řádu}, extenzí $PA$ o tzv. \alert{axiom indukce}:
    $$
    (\forall X)((X(0) \land (\forall x)(X(x) \limplies X(S(x)))) \limplies (\forall x)X(x))
    $$

    \begin{itemize}
        \item $X$ reprezentuje (libovolnou) podmnožinu modelu
        \item použijeme na množinu všech následníků 0
        \item každý prvek je následník 0 $\Rightarrow$ izomorfismus s $\underline{\mathbb N}$
    \end{itemize}

\end{frame}


\section{10.3 Nerozhodnutelnost predikátové logiky}


\begin{frame}{Nerozhodnutelnost predikátové logiky}
    
    \myblock{
        \textbf{Věta (O nerozhodnutelnosti predikátové logiky):}
        Neexistuje algoritmus, který pro vstupní formuli $\varphi$ rozhodne, zda je logicky platná.
    }

    \begin{itemize}
        \item tj. zda je formule $\varphi$ [v lib. jazyce 1. řádu] tautologie ($\models\varphi$)
        \item neboli $T=\emptyset$ není rozhodnutelná 
    \end{itemize}

    Nemáme formalismus pro algoritmy (Turingovy stroje), dokážeme redukcí na jiný nerozhodnutelný problém: \alert{\href{https://en.wikipedia.org/wiki/Hilbert\%27s_problems}{Hilbertův 10. problém}}

    \bigskip

    \myalert{
    \begin{quote}
        ``Najděte algoritmus, který po konečně mnoha krocích určí, zda daná diofantická rovnice s libovolným počtem proměnných a
        celočíselnými koeficienty má celočíselné řešení.''
    \end{quote}
    }

    \medskip

    \alert{diofantická rovnice}: $p(x_1,\dots,x_n)=0$, kde $p$ je celočíselný polynom


    ukážeme, že existuje \alert{redukce} `těžkého' Hilbertova 10. problému na náš problém, tedy i náš problém je `těžký'
    
\end{frame}


\begin{frame}{Nerozhodnutelnost Hilbertova desátého problému}

    \myblock{
        \textbf{Věta (Matiyasevich 1970):}    
        Problém existence celočíselného řešení dané diofantické rovnice s celočís. koeficienty je nerozhodnutelný.
    }

    (Důkaz neuvedeme.)

    \medskip

    \myblock{
        \textbf{Důsledek:}
        Neexistuje algoritmus rozhodující, mají-li dané polynomy $p(x_1,\dots,x_n),q(x_1,\dots,x_n)$ s \alert{přiroz.} koeficienty \alert{přirozené řešení}, tj.
        \vspace{-6pt}
        $$
        \underline{\mathbb N}\models(\exists x_1)\dots(\exists x_n)\ p(x_1,\dots,x_n)=q(x_1,\dots,x_n)
        $$
        \vspace{-16pt}
    }
    
    \textbf{Důkaz:} \alert{\href{https://en.wikipedia.org/wiki/Lagrange\%27s_four-square_theorem}{Lagrangeova věta o čtyřech čtvercích}} říká, že každé celé číslo lze vyjádřit jako rozdíl dvou přirozených, a naopak, každé přirozené číslo lze vyjádřit jako součet čtyř čtverců. Diofantickou rovnici lze tedy transformovat na rovnici z důsledku, a naopak.\hfill\qedsymbol

\end{frame}


\begin{frame}{Důkaz nerozhodnutelnosti predikátové logiky}

    Uvažme $\varphi$ tvaru $(\exists x_1)\dots(\exists x_n)\ p(x_1,\dots,x_n)=q(x_1,\dots,x_n)$ 
    kde $p$ a $q$ jsou přirozené polynomy. Dle Tvrzení o Robinsonově aritmetice:
    $$
    \underline{\mathbb N}\models \varphi\ \Leftrightarrow\  Q\proves \varphi
    $$

    Buď $\psi_Q$ konjunkce (gen. uzávěrů) axiomů $Q$ (je konečná). Zřejmě: 
    $$
    \alert{Q\proves\varphi}\ \Leftrightarrow\ \psi_Q\proves\varphi\ \Leftrightarrow\ \alert{\proves\psi_Q\limplies\varphi}
    $$
    Dle Věty o úplnosti je to ale ekvivalentní \alert{$\models\psi_Q\limplies\varphi$}. Dostáváme:
    $$
    \underline{\mathbb N}\models \varphi\ \Leftrightarrow\ \models \psi_Q\limplies\varphi
    $$
    \alert{Sporem:} Pokud bychom měli algoritmus rozhodující logickou platnost, mohli bychom rozhodovat i existenci přirozeného řešení rovnice $p(x_1,\dots,x_n)=q(x_1,\dots,x_n)$, tj. Hilbertův 10. problém.\hfill\qedsymbol


\end{frame}


\section{10.4 Gödelovy věty}


\begin{frame}{První věta o neúplnosti + důsledek o nekompletnosti}

    \myblock{
        \textbf{Věta (Gödel 1931):}
        Je-li $T$ bezesporná rekurzivně axiomatizovaná extenze Robinsonovy aritmetiky, potom existuje sentence, která je pravdivá v~$\underline{\mathbb N}$, ale není dokazatelná v $T$.
    }

    \begin{itemize}
        \item vlastnosti aritmetiky přir. čísel nelze `rozumně', efektivně popsat (v logice 1. řádu), takový popis je nutně `neúplný'
        \item \alert{pravdivost} je ve standardním modelu $\underline{\mathbb N}$ zatímco \alert{dokazatelnost} v $T$ (samozřejmě pravdivá v $T$ je v $T$ i dokazatelná)
        \item \alert{bezespornost} nutná (sporná teorie dokáže vše)
        \item bez \alert{rekurzivní axiomatizovatelnosti} by teorie nebyla `užitečná'
        \item extenze $Q$ znamená `základní aritmetická síla' (různé varianty předpokladu; nelze-li zakódovat přir. čísla s $+,\cdot$ je moc `slabá'
    \end{itemize}    

    \myblock{
        \textbf{Důsledek:}
        Splňuje-li teorie $T$ předpoklady První věty o neúplnosti a je-li navíc $\underline{\mathbb N}$ modelem $T$, potom $T$ není kompletní.
    }
    
    \textbf{Důkaz:}
        Vezměme Gödelovu sentenci $\varphi$ ($\underline{\mathbb N}\models\varphi$, $T\not\proves\varphi$). Je-li $T$ kompletní, víme $T\proves\neg\varphi$, z korektnosti $T\models\neg\varphi$, tedy $\underline{\mathbb N}\models\neg\varphi$.
    \hfill\qedsymbol    

\end{frame}


\begin{frame}{O důkazu}

    \begin{itemize}
        \item Gödelova sentence formalizuje \alert{``Nejsem dokazatelná v $T$''}
        \item převratná důkazová technika, dva hlavní principy:
        \item \alert{aritmetizace syntaxe}, zakódování sentencí a jejich dokazatelnosti do přirozených čísel
        \item \alert{self-reference}, sentence `mluví sama o sobě' (o svém kódu)
        \item všechny technické detaily vynecháme, viz např. V. Švejdar: \emph{Logika -- neúplnost, složitost a nutnost}, Academia 2002
    \end{itemize}
    
\end{frame}


\begin{frame}{Aritmetizace dokazatelnosti}
    
    \begin{itemize}
        \item \alert{Gödelovo číslování} `rozumně' kóduje konečné syntaktické objekty (termy, formule, tablo důkazy) do $\mathbb N$: lze algoritmicky [de-]kódovat, simulovat `manipulaci' s objekty na jejich kódech
        \item pro $\varphi$ bude \alert{$\lceil\varphi\rceil$} příslušný kód, \alert{$\underline{\varphi}$} odpovídající $\lceil\varphi\rceil$-tý numerál
        \item pro danou $T$ máme binární relaci $\MyProof_T\subseteq\mathbb N^2$ definovanou
        \hspace{-1cm}\myalertinline{
            $(n,m)\in\MyProof_T$ $\Leftrightarrow$
        $n=\lceil\varphi\rceil$, $m=\lceil\tau\rceil$, $\tau$ je tablo důkaz $\varphi$ z $T$
        }
        \item je-li $T$ rek. axiomatizovaná, je relace $\MyProof_T\subseteq\mathbb N^2$ \alert{rekurzivní} (lze algoritmicky ověřit korektnost tabla, tj. $(n,m)\in\MyProof_T$)
        \item klíčová část důkazu První věty (důkaz vynecháme):
    \end{itemize}

    \myblock{
        \textbf{Tvrzení:}
        Je-li $T$ rekurzivně axiomatizovaná extenze Robinsonovy aritmetiky, potom existuje formule $\Prf_T(x,y)$ v jazyce aritmetiky, která \alert{reprezentuje} relaci $\MyProof_T$, tj. pro každá $n,m\in\mathbb N$:
    \begin{itemize}
        \item je-li $(n,m)\in\MyProof_T$, potom $Q\proves\Prf_T(\underline{n},\underline{m})$
        \item jinak $Q\proves\neg\Prf_T(\underline{n},\underline{m})$
    \end{itemize} 
    }

\end{frame}


\begin{frame}
    \frametitle{Predikát dokazatelnosti}

    

\end{frame}


\begin{frame}{Self-reference}
    

\end{frame}


\begin{frame}{Věta o pevném bodě}
    

\end{frame}


\begin{frame}{Nedefinovatelnost pravdy}

    
\end{frame}


\begin{frame}{Důkaz První věty o neúplnosti}

    
\end{frame}


\begin{frame}{Důsledky První věty o neúplnosti}
    

\end{frame}


\begin{frame}{Gödelova Druhá věta o neúplnosti}
    

\end{frame}


\begin{frame}{Důsledky Druhé věty o neúplnosti}

    
\end{frame}


\end{document}


