\documentclass{beamer}

%% slide-specific

\usetheme[progressbar=frametitle]{metropolis}
%\usecolortheme{spruce}
%\metroset{block=fill}

% define Metropolis colors    
\definecolor{mAlert}{HTML}{EB811B}
\definecolor{mExample}{HTML}{14B03D}
\definecolor{mBlock}{HTML}{23373b}

% my blocks
\setlength\fboxsep{0pt}%

\newcommand{\myblock}[1]{\colorbox{mBlock!8}{\begin{minipage}{\linewidth}#1\end{minipage}}}
\newcommand{\myblockmath}[1]{\colorbox{mBlock!8}{\begin{minipage}{\linewidth}\vspace{-6pt}#1\end{minipage}}}
\newcommand{\myblockinline}[1]{\colorbox{mBlock!8}{#1}}
\newcommand{\myexample}[1]{\colorbox{mExample!8}{\begin{minipage}{\linewidth}#1\end{minipage}}}
\newcommand{\myexamplemath}[1]{\colorbox{mExample!8}{\begin{minipage}{\linewidth}\vspace{-6pt}#1\end{minipage}}}
\newcommand{\myexampleinline}[1]{\colorbox{mExample!8}{#1}}
\newcommand{\myalert}[1]{\colorbox{mAlert!8}{\begin{minipage}{\linewidth}#1\end{minipage}}}
\newcommand{\myalertmath}[1]{\colorbox{mAlert!8}{\begin{minipage}{\linewidth}\vspace{-6pt}#1\end{minipage}}}
\newcommand{\myalertinline}[1]{\colorbox{mAlert!8}{#1}}

%% other
\newcommand{\mystructure}[1]{\mathcal{#1}}




%% packages
\usepackage{amsmath,amssymb,amsthm}
\usepackage{booktabs}
\usepackage[czech]{babel}
\usepackage{enumerate}
\usepackage{forest}
\usepackage{multicol}
% \usepackage{tcolorbox}
\usepackage{tikz}
    \usetikzlibrary{arrows.meta}
%\usepackage[unicode]{hyperref}
\usepackage[utf8x]{inputenc}
\usepackage{xfrac}

% %% theorems
% \theoremstyle{plain}
%     \newtheorem{theorem}{Věta}[section]
%     \newtheorem*{theorem-unnumbered}{Věta}
%     \newtheorem{proposition}[theorem]{Tvrzení}
%     \newtheorem{corollary}[theorem]{Důsledek}
%     \newtheorem{lemma}[theorem]{Lemma}
%     \newtheorem{observation}[theorem]{Pozorování}
% \theoremstyle{definition}
%     \newtheorem{definition}[theorem]{Definice}
%     \newtheorem*{algorithm}{Algoritmus}
% \theoremstyle{remark}
%     \newtheorem{remark}[theorem]{Poznámka}
%     \newtheorem{example}[theorem]{Příklad}
%     \newtheorem{exercise}{Cvičení}[chapter]
%     \newtheorem*{solution}{Řešení}

%% macros and definitions
\DeclareMathOperator{\Aut}{Aut}
\DeclareMathOperator{\Conseq}{Csq}
\DeclareMathOperator{\DeLO}{DeLO}
\DeclareMathOperator{\dom}{dom}
\DeclareMathOperator{\Fm}{Fm}
\DeclareMathOperator{\M}{M}
%\DeclareMathOperator{\Proof}{Proof}
\DeclareMathOperator{\rng}{rng}
\DeclareMathOperator{\Term}{Term}
\DeclareMathOperator{\Th}{Th}
\DeclareMathOperator{\Thm}{Thm}
\DeclareMathOperator{\Tree}{Tree}
\DeclareMathOperator{\Var}{Var}
\DeclareMathOperator{\VF}{VF}

\newcommand{\A}{\structure{A}}
\newcommand{\B}{\structure{B}}
\newcommand{\Con}{\mathit{Con}}
\newcommand{\disjointunion}{\mathbin{\dot{\sqcup}}}
\newcommand{\F}{\ensuremath{\mathrm{F}}}
\newcommand{\landsymb}{{\land}}
\newcommand{\lbin}{\mathbin{\square}}
\newcommand{\lbinsymb}{{\lbin}}
\newcommand{\liff}{\mathbin{\leftrightarrow}}
\newcommand{\liffsymb}{{\liff}}
\newcommand{\limplies}{\mathbin{\rightarrow}}
\newcommand{\limpliessymb}{{\limplies}}
\newcommand{\lorsymb}{{\lor}}
\newcommand{\Prf}{\mathit{Prf}}
\newcommand{\proves}{\vdash}
%\newcommand{\structure}[1]{\mathcal{#1}}
\newcommand{\todo}{[TODO]}
\newcommand{\T}{\ensuremath{\mathrm{T}}}
\newcommand{\union}{\mathbin{\cup}}


\title{Pátá přednáška}
\subtitle{NAIL062 Výroková a predikátová logika}
\author{Jakub Bulín (KTIML MFF UK)}
% \institute{KTIML MFF UK}
\date{Zimní semestr 2024}


\begin{document}


\maketitle


\begin{frame}{Pátá přednáška}

    \textbf{Program}
        \begin{itemize}
            \item rezoluční metoda
            \item korektnost a úplnost rezoluce
            %\item LI-rezoluce a Horn-SAT
            \item úvod do predikátové logiky
            \item syntaxe predikátové logiky
        \end{itemize}

    \textbf{Materiály}

        \href{https://github.com/jbulin-mff-uk/nail062/raw/main/lecture/lecture-notes/lecture-notes.pdf}{\alert{\textbf{Zápisky z přednášky}}}, Sekce 5.1-5.3 z Kapitoly 5 (Sekci 5.4 zatím přeskočíme), Sekce 6.1-6.3 z Kapitoly 6

\end{frame}


\section{\sc Kapitola 5: Rezoluční metoda}


\begin{frame}{Rezoluční metoda}
    
    \begin{itemize}
        \item jiný důkazový systém než tablo metoda
        \item mnohem efektivnější implementace
        \item logické programování, automatické dokazování, SAT solvery (důkaz jako \alert{certifikát} nesplnitelnosti)
        \item pracuje s CNF (každý výrok/teorii lze převést do CNF)
        \item jediné inferenční pravidlo: \alert{rezoluční pravidlo}
        $$
        \frac{\{p\}\cup C_1,\{\neg p\}\cup C_2}{C_1\cup C_2}
        $$
        \item platí obecnější \alert{pravidlo řezu}:
        $$
        \frac{\varphi\lor\psi,\neg\varphi\lor\chi}{\psi\lor\chi}
        $$
    \end{itemize}

\end{frame}


\section{5.1 Množinová reprezentace}


\begin{frame}{Množinová reprezentace}

    \begin{itemize}
        \item \alert{literál} $\ell$ je $p$ nebo $\neg p$ (pro $p\in\mathbb P$), $\bar \ell$ je \alert{opačný literál} k $\ell$
        \item \alert{klauzule} $C$ je konečná množina literálů
        \item \alert{prázdná klauzule} $\square$ je nesplnitelná
        \item \alert{CNF formule} $S$ je množina klauzulí (může být i \alert{nekonečná}!)
        \item \alert{prázdná formule} $\emptyset$ je vždy splněna
    \end{itemize}
    Modely reprezentujeme jako množiny literálů:
    \begin{itemize}
        \item \alert{(částečné) ohodnocení} je libovolná \alert{konzistentní} množina literálů (tj. nesmí obsahovat dvojici opačných literálů) 
        \item\alert{úplné ohodnocení} obsahuje $p$ nebo $\neg p$ pro každý prvovýrok
        \item ohodnocení $\mathcal V$ \alert{splňuje} formuli $S$, píšeme \alert{$\mathcal V\models S$}, pokud $\mathcal V$ obsahuje nějaký literál z každé klauzule v $S$:
        $$
        \mathcal V\cap C\neq\emptyset\text{ pro každou }C\in S
        $$
    \end{itemize}

\end{frame}


\begin{frame}{Množinová reprezentace: příklad}
    
    $\varphi=(\neg p_1\lor p_2)\land(\neg p_1\lor\neg p_2\lor p_3)\land(\neg p_3\lor\neg p_4)\land(\neg p_4\lor \neg p_5)\land p_4$ 
        
    \begin{itemize}
        \item v množinové reprezentaci:
        \myexampleamsmath{
        $$
        S=\{\{\neg p_1,p_2\},\{\neg p_1,\neg p_2,p_3\},\{\neg p_3,\neg p_4\},\{\neg p_4,\neg p_5\},\{p_4\}\}
        $$
        }
        \item ohodnocení \myexampleinline{
            $\mathcal V=\{\neg p_1,\neg p_3,p_4,\neg p_5\}$
        } 
        splňuje $S$, \alert{$\mathcal V\models S$}
        \item není úplné, můžeme rozšířit libovolným literálem pro $p_2$, platí 
        \begin{itemize}
            \item \alert{$\mathcal V\union\{p_2\}\models S$}
            \item \alert{$\mathcal V\union\{\neg p_2\}\models S$}
        \end{itemize}        
        \item tato dvě úplná ohodnocení odpovídají modelům
         \begin{itemize}
            \item $(0,1,0,1,0)$
            \item $(0,0,0,1,0)$
         \end{itemize}
    \end{itemize}

\end{frame}


\section{5.2 Rezoluční důkaz}


\begin{frame}{Rezoluční pravidlo}    
    
    \myblock{
    Mějme klauzule $C_1$ a $C_2$ a literál $\ell$ takový, že $\ell\in C_1$ a $\bar\ell\in C_2$. Potom \alert{rezolventa} klauzulí $C_1$ a $C_2$ \alert{přes literál} $\ell$ je klauzule:
    $$
    C=(C_1\setminus\{\ell\})\union (C_2\setminus\{\bar\ell\})
    $$
    \vspace{-12pt}
    }
    
    tedy z první klauzule odstraníme $\ell$ a z druhé $\bar\ell$ (musely tam být!) a zbylé literály sjednotíme, mohli bychom také psát: 
    $$
    C_1'\union C_2'\text{ je rezolventou klauzulí }C_1'\disjointunion\{\ell\}\text{ a }C_2'\disjointunion\{\bar \ell\}
    $$
    \vspace{-12pt}
    \begin{itemize}
        \item \myexampleinline{
            z klauzulí $C_1=\{\neg q,r\}$ a $C_2=\{\neg p,\neg q,\neg r\}$
        }
        odvodíme klauzuli $\{\neg p,\neg q\}$ přes literál $r$
        \item \myexampleinline{
            z $\{p,q\}$ a $\{\neg p,\neg q\}$
        }
        odvodíme $\{p,\neg p\}$ přes literál $q$, nebo $\{q,\neg q\}$ přes literál $p$ (obojí jsou ale tautologie)
        \item nelze z nich ale odvodit $\square$ \emph{``rezolucí přes $p$ a $q$ najednou''}! ($S=\{\{p,q\},\{\neg p,\neg q\}\}$ je splnitelná, např. $(1,0)$ je model)
    \end{itemize}

\end{frame}


\begin{frame}{Rezoluční důkaz}
    
    Rezoluční pravidlo je \alert{korektní}, tj. pro libovolné ohodnocení $\mathcal V$ platí:
    \myalertmath{
    $$
    \text{Pokud }\mathcal V\models C_1\text{ a }\mathcal V\models C_2\text{, potom }\mathcal V\models C.
    $$
    }    
    
    V rezolučním důkazu můžeme vždy napsat buď axiom, nebo rezolventu již napsaných klauzulí; tím zaručíme korektnost důkazů:

    \medskip

    \myblock{
    \alert{Rezoluční důkaz (odvození)} klauzule $C$ z formule $S$ je konečná posloupnost klauzulí $C_0,C_1,\dots,C_n=C$ taková, že pro každé $i$:
    \begin{itemize}
        \item $C_i\in S$, nebo
        \item $C_i$ je rezolventou nějakých $C_j,C_k$ kde $j,k<i$
    \end{itemize}
    }
    \begin{itemize}
        \item existuje-li rez. důkaz, je $C$ \alert{rezolucí dokazatelná} z $S$, \alert{$S\proves_R C$}
        \item \alert{rezoluční zamítnutí} formule $S$ je rezoluční důkaz $\square$ z $S$
        \item v tom případě je $S$ \alert{rezolucí zamítnutelná}
   \end{itemize}   
   
\end{frame}


\begin{frame}{Příklad}

    Formule \myexampleinline{
    $S=\{\{p,\neg q,r\},\{p,\neg r\},\{\neg p,r\},\{\neg p,\neg r\},\{q,r\}\}$
    } je rezolucí zamítnutelná, jedno z možných zamítnutí je:
    $$
    \{p,\neg q,r\},\{q,r\},\{p,r\},\{\neg p,r\},\{r\},\{p,\neg r\},\{\neg p,\neg r\},\{\neg r\},\square
    $$

    Rezoluční důkaz má přirozeně stromovou strukturu, tzv. \alert{rezoluční strom}: na listech jsou axiomy, vnitřní vrcholy jsou rezoluční kroky.

    \begin{center}
        \begin{forest}
        for tree={grow=north}
        [$ \square $
            [$ \{\neg r\} $
                [{$ \{\neg p, \neg r\} $}]
                [{$ \{p, \neg r\} $}]
            ]
            [$ \{r\} $
                [{$ \{\neg p, r\} $}]
                [{$ \{p,r\} $}
                    [{$ \{q, r\} $}]
                    [{$ \{p,\neg q, r\} $}]
                ]
            ]
        ]
        \end{forest}
    \end{center}

\end{frame}


\begin{frame}{Rezoluční strom}

    \myblock{
        \alert{Rezoluční strom} klauzule $C$ z formule $S$ je konečný binární strom s vrcholy označenými klauzulemi, kde
        \begin{itemize}
            \item v kořeni je $C$,
            \item v listech jsou klauzule z $S$,
            \item v každém vnitřním vrcholu je rezolventa klauzulí ze synů tohoto vrcholu.
        \end{itemize}    
    }

    \medskip

    \textbf{Pozorování:} $C$ má rezoluční strom z $S$, právě když $S\proves_R C$. (Důkaz snadno indukcí dle hloubky stromu a délky důkazu.)
    \begin{itemize}
        \item rezolučnímu důkazu odpovídá \alert{jednoznačný} rezoluční strom
        \item z rezolučního stromu můžeme získat více důkazů (jsou dané libovolnou procházkou po vrcholech, která navštíví vnitřní vrchol až poté, co navštívila oba jeho syny)
    \end{itemize}

\end{frame}


\begin{frame}{Rezoluční uzávěr}
    
    jaké všechny klauzule se můžeme rezolucí \alert{`naučit'} z dané formule? (není praktické je všechny najít, jde o užitečný teoretický pohled)
    
    \medskip

    \myblock{
    \alert{Rezoluční uzávěr} $\mathcal R(S)$ formule $S$ je definován induktivně jako nejmenší množina klauzulí splňující:
    \begin{itemize}
        \item $C\in\mathcal R(S)$ pro všechna $C\in S$,
        \item jsou-li $C_1,C_2\in\mathcal R(S)$ a $C$ jejich rezolventa, potom i $C\in\mathcal R(S)$
    \end{itemize}
    }
    
    \bigskip

    Pro \myexampleinline{
        $S=\{\{p,\neg q,r\},\{p,\neg r\},\{\neg p,r\},\{\neg p,\neg r\},\{q,r\}\}$
    } máme:
    \begin{align*}
        \mathcal R(S)=\{&\textcolor{blue}{\{p,\neg q,r\},\{p,\neg r\},\{\neg p,r\},\{p,s\},\{q,r\}},\\&\{p,\neg q\},\{\neg q,r\},\{r,\neg r\},\{p,\neg p\},\{r,s\},\\&
        \{p,r\},\{p,q\},\{r\},\{p\}\}
    \end{align*}

\end{frame}


\section{5.3 Korektnost a úplnost rezoluční metody}


\begin{frame}{Korektnost rezoluce}
    
    Korektnost dokážeme snadno indukcí podle délky důkazu (nebo alternativně indukcí dle hloubky rezolučního stromu).

    \myblock{
    \textbf{Věta (O korektnosti rezoluce):} 
    Je-li CNF formule $S$ rezolucí zamítnutelná, potom je $S$ nesplnitelná.
    }

    \textbf{Důkaz:} Nechť $S\proves_R\square$, a vezměme nějaký rezoluční důkaz $C_0,C_1,\dots,C_n=\square$. \alert{Sporem:} nechť existuje ohodnocení $\mathcal V\models S$. Indukcí podle $i$ dokážeme, že \alert{$\mathcal V\models C_i$}. Potom i $\mathcal V\models \square$, což je spor. 
    
    Pro $i=0$ to platí, neboť $C_0\in S$. Pro $i>0$ máme dva případy:
    \begin{itemize}
        \item \alert{$C_i\in S$:} v tom případě $\mathcal V\models C_i$ plyne z předpokladu, že $\mathcal V\models S$,
        \item \alert{$C_i$ je rezolventou $C_j,C_k$, kde $j,k<i$:} z indukčního předpokladu víme $\mathcal V\models C_j$ a $\mathcal V\models C_k$, $\mathcal V\models C_i$ plyne z korektnosti rezolučního pravidla\hfill\qedsymbol
    \end{itemize}

\end{frame}


\begin{frame}{Dosazení}

    %K důkazu úplnosti budeme potřebovat:
    
    Je-li $S$ CNF formule a $\ell$ literál, potom \alert{dosazení} $\ell$ do $S$ je formule
    \myalertmath{
        $$
        S^\ell=\{C\setminus\{\bar\ell\}\mid \ell\notin C\in S\}
        $$
    }
        
    \begin{itemize}
        \item $S^\ell$ je výsledkem \alert{jednotkové propagace} aplikované na $S\union\{\{\ell\}\}$.
        \item $S^\ell$ neobsahuje v žádné klauzuli literál $\ell$ ani $\bar\ell$ (vůbec tedy neobsahuje prvovýrok z $\ell$)
        \item Pokud $S$ neobsahovala literál $\ell$ ani $\bar\ell$, potom $S^\ell=S$.
        \item Pokud $S$ obsahovala jednotkovou klauzuli $\{\bar\ell\}$, potom $\square\in S^\ell$, tedy $S^\ell$ je sporná.
    \end{itemize} 

\end{frame}


\begin{frame}{Větvení}
    
    \myblock{
    \textbf{Lemma:} 
    $S$ je splnitelná, právě když je splnitelná $S^\ell$ nebo $S^{\bar\ell}$.
    }

    \textbf{Důkaz:} \alert{$\Rightarrow$} Ohodnocení $\mathcal V\models S$ nemůže obsahovat $\ell$ i $\bar\ell$; BÚNO \alert{$\bar\ell\notin\mathcal V$}. Ukážeme, že potom $\mathcal V\models S^\ell$. 
    
    Vezměme libovolnou klauzuli v $S^\ell$. Ta je tvaru $C\setminus\{\bar\ell\}$ pro klauzuli $C\in S$ (neobsahující literál $\ell$). Víme, že $\mathcal V\models C$, protože ale $\mathcal V$ neobsahuje $\bar\ell$, muselo ohodnocení $\mathcal V$ splnit nějaký jiný literál v $C$, takže platí i $\mathcal V\models C\setminus\{\bar\ell\}$.

    \alert{$\Leftarrow$} BÚNO mějme ohodnocení \alert{$\mathcal V\models S^\ell$}. Protože se $\bar\ell$ (ani $\ell$) nevyskytuje v $S^\ell$, platí také \alert{$\mathcal V\setminus\{\bar\ell\}\models S^\ell$}. Ohodnocení \alert{$\mathcal V'=(\mathcal V\setminus\{\bar\ell\})\union\{\ell\}$} potom splňuje všechny $C\in S$, tedy $\mathcal V'\models S$: 
    \begin{itemize}
        \item pokud $\ell\in C$, potom $\ell\in C\cap\mathcal V'$ a $C\cap\mathcal V'\neq\emptyset$
        \item jinak $C\cap\mathcal V'=C\cap\mathcal (\mathcal V\setminus\{\bar\ell\})=(C\setminus\{\bar\ell\})\cap(\mathcal V\setminus\{\bar\ell\})\neq\emptyset$ neboť $\mathcal V\setminus\{\bar\ell\}\models C\setminus\{\bar\ell\}\in S^\ell$\hfill\qedsymbol
    \end{itemize}

\end{frame}


\begin{frame}{Strom dosazení}
    
    Zda je \emph{konečná} formule $S$ splnitelná můžeme zjišťovat rekurzivně, dosazením obou literálů pro některý prvovýrok $p$, a rozvětvením na $S^p,S^\bar p$ (jako v DPLL). Výslednému stromu říkáme \alert{strom dosazení}.
 
    Např. pro \myexampleinline{
        $S=\{\{p\},\{\neg q\},\{\neg p,\neg q\}\}$
    }:
    \vspace{-6pt}
    \begin{center}
        \begin{forest}    
        [{$S$}
            [{ $S^p=\{\{\neg q\}\}$ }
                [{$S^{pq}=\{\square\}$}, tikz={\node[fit to=tree,label=below:\textcolor{red}
                {$\otimes$}] {};}]
                [{$S^{p\bar q}=\emptyset$}, tikz={\node[fit to=tree,label=below:\textcolor{blue}{\small $\mathcal V=\{p,\bar q\}$}] {};}]
            ]
            [{ $S^{\bar p}=\{\square,\{\neg q\}\}$ }, tikz={\node[fit to=tree,label=below:\textcolor{red}{$\otimes$}] {};}]
        ]
        \end{forest}
    \end{center}
    \vspace{-12pt}
    % ! beamer+forest conflict: compile with care

    \begin{itemize}
        \item jakmile větev obsahuje $\square$, je nesplnitelná a nepokračujeme v ní
        \item listy jsou buď nesplnitelné, nebo prázdné teorie: v tom případě z posloupnosti dosazení získáme splňující ohodnocení.    
    \end{itemize}

\end{frame}


\begin{frame}{Strom dosazení a nesplnitelnost}
    
    \myblock{
    \textbf{Důsledek:} 
    CNF formule $S$ (ve spočetném jazyce, \alert{může být i nekonečná}) je nesplnitelná, právě když každá větev stromu dosazení obsahuje $\square$.
    }

    \textbf{Důkaz:} 
        Pro \alert{konečnou} $S$ snadno dokážeme indukcí dle $|\Var(S)|$: 
        \begin{itemize}
            \item Je-li $|\Var(S)|=0$, máme $S=\emptyset$ nebo $S=\{\square\}$, v obou případech je strom dosazení jednoprvkový a tvrzení platí. 
            \item V indukčním kroku vybereme libovolný literál $\ell\in\Var(S)$ a aplikujeme Lemma.
        \end{itemize} 
    Je-li $S$ \alert{nekonečná a splnitelná}, má splňující ohodnocení, to se `shoduje' s odpovídající (nekonečnou) větví ve stromu dosazení. 
    
    Je-li \alert{nekonečná a nesplnitelná}, dle Věty o kompaktnosti existuje konečná $S'\subseteq S$, která je také nesplnitelná. Po dosazení pro všechny proměnné z $\Var(S')$ bude v každé větvi $\square$, to nastane po konečně mnoha krocích.
    \hfill\qedsymbol

\end{frame}


\begin{frame}{Úplnost rezoluce}

    \myblock{
    \textbf{Věta (O úplnosti rezoluce):} 
    Je-li CNF formule $S$ nesplnitelná, je rezolucí zamítnutelná (tj. $S\proves_R\square$).
    }

    \textbf{Důkaz:} \alert{Je-li $S$ nekonečná}, má z kompaktnosti konečnou nesplnitelnou část, její rezoluční zamítnutí je také zamítnutí $S$. 
    
    \alert{Je-li $S$ konečná}, ukážeme indukcí dle počtu proměnných: Je-li $|\Var(S)|=0$, jediná možná nesplnitelná formule bez proměnných je $S=\{\square\}$, a máme jednokrokový důkaz $S\proves_R\square$. 
    
    Jinak vyberme $p\in\Var(S)$. Podle Lemmatu jsou $S^p$ i $S^{\bar p}$ nesplnitelné. Mají o proměnnou méně, tedy dle ind. předpokladu existují rezoluční stromy $T$ pro $S^p\proves_R\square$ a $T'$ pro $S^{\bar p}\proves_R\square$.

    Ukážeme, jak z $T$ vyrobit rezoluční strom $\widehat T$ pro $S\proves_R \neg p$. Analogicky $\widehat{T'}$ pro $S\proves_R p$ a potom už snadno vyrobíme rezoluční strom pro $S\proves_R\square$: ke kořeni $\square$ připojíme kořeny stromů $\widehat T$ a $\widehat{T'}$ jako levého a pravého syna (tj. získáme $\square$ rezolucí z $\{\neg p\}$ a $\{p\}$).

\end{frame}


\begin{frame}{Dokončení důkazu}

    Rezoluční strom \alert{$T$ pro $S^p\proves_R\square$} \scalebox{1.5}{$\rightsquigarrow$} \alert{$\widehat T$ pro $S\proves_R \neg p$}: 
    
    Vrcholy i uspořádání jsou stejné, jen do některých klauzulí ve vrcholech \alert{přidáme literál $\neg p$}. 
    
    Na každém listu stromu $T$ je nějaká klauzule $C\in S^p$, a
    \begin{itemize}
        \item buď $C\in S$,
        \item nebo $C\notin S$, ale $C\cup\{\neg p\}\in S$
    \end{itemize}
    
    V prvním případě necháme label stejný. Ve druhém případě přidáme do $C$ \alert{a do všech klauzulí nad tímto listem} literál $\neg p$. 
    
    Listy jsou nyní klauzule z $S$, a každý vnitřní vrchol je nadále rezolventou svých synů. V kořeni jsme $\square$ změnili na $\neg p$ (ledaže každý list $T$ už byl klauzule z $S$, to ale už $T$ dává $S\proves_R\square$). \hfill\qedsymbol

\end{frame}


\section{Zatím přeskočíme: 5.4 LI-rezoluce a Horn-SAT}


\begin{frame}{Lineární důkaz: neformálně}

    Rezoluční důkaz můžeme kromě rezolučního stromu \alert{zorganizovat i jinak}, jako tzv. \alert{lineární důkaz}:
    
    \begin{center}
        \begin{forest}
            for tree={math content,grow=west,text height=2ex, text depth=1ex}
            [C_{n+1}
                [,phantom]
                [C_n
                    [,phantom]
                    [\cdots\cdots\cdots
                        [C_2
                            [,phantom]
                            [C_1
                                [,phantom]
                                [C_0]
                                [B_0]
                            ]
                            [B_1]
                        ]
                    ]
                    [B_{n-1}]                    
                ]
                [B_n]
            ]
        \end{forest}  
    \end{center}
    
    \vspace{-6pt}

    \begin{itemize}
        \item v každém kroku máme jednu \alert{centrální} klauzuli
        \item tu rezolvujeme s \alert{boční} (`side') klauzulí
        \item boční klauzule je buď axiom z $S$, nebo některá z předchozích centrálních (jako bychom odvozené klauzule přidávali k axiomům)
        \item výsledná \alert{rezolventa je novou centrální klauzulí}
    \end{itemize}

    (Tento pohled lépe odpovídá procedurálnímu výpočtu, jde jen o to, jak vybírat vhodné boční klauzule.)

\end{frame}


\begin{frame}{Lineární důkaz: formálně}

    \begin{center}
        \begin{forest}
            for tree={math content,grow=west,text height=2ex, text depth=1ex}
            [C_{n+1}
                [,phantom]
                [C_n
                    [,phantom]
                    [\cdots\cdots\cdots
                        [C_2
                            [,phantom]
                            [C_1
                                [,phantom]
                                [C_0]
                                [B_0]
                            ]
                            [B_1]
                        ]
                    ]
                    [B_{n-1}]                    
                ]
                [B_n]
            ]
        \end{forest}  
    \end{center}

    \myblock{
    \alert{Lineární důkaz} klauzule $C$ z formule $S$ je konečná posloupnost
    $$
    \begin{bmatrix}
        C_0 \\
        B_0
    \end{bmatrix},
    \begin{bmatrix}
        C_1 \\
        B_1
    \end{bmatrix},\dots,
    \begin{bmatrix}
        C_n \\
        B_n
    \end{bmatrix},
    C_{n+1}
    $$
    kde $C_i$ říkáme \alert{centrální} klauzule, $C_0$ je \alert{počáteční}, $C_{n+1}=C$ je \alert{koncová}, $B_i$ jsou \alert{boční} klauzule, a platí:
    \begin{itemize}
        \item $C_0\in S$, pro $i\leq n$ je $C_{i+1}$ rezolventou $C_i$ a $B_i$,
        \item $B_0\in S$, pro $i\leq n$ je $B_i\in S$ nebo $B_i=C_j$ pro nějaké $j<i$. 
    \end{itemize}
    \alert{Lineární zamítnutí} $S$ je lineární důkaz $\square$ z $S$. 
    }    

\end{frame}


\begin{frame}{Příklad a ekvivalence s rezolučním důkazem}

    Lineární zamítnutí \myexampleinline{
        $S = \{\{p, q\},\{p, \neg q\}, \{\neg p, q\}, \{\neg p, \neg q\}\}$
    }:
    \begin{center}
        \begin{forest}
            for tree={math content,grow=west,text height=2ex, text depth=1ex, l sep=3em}
            [{\square}
                [,phantom]
                [{\{\neg p\}}
                    [,phantom]
                    [{\{q\}}
                        [,phantom]
                        [{\textcolor{blue}{\{p\}}}
                            [,phantom]
                            [{\{p,q\}}]
                            [{\{p,\neg q\}}]
                        ]
                        [{\{\neg p,q\}}]
                    ]
                    [{\{\neg p,\neg q\}}]                    
                ]
                [{\textcolor{red}{\{p\}}}]
            ]
        \end{forest}  
    \end{center}
    Poslední boční klauzule $\textcolor{red}{\{p\}}$ není z $S$, ale je rovna předchozí centrální klauzuli.

    \textbf{Poznámka:} $C$ má lineární důkaz z $S$, právě když $S\proves_R C$.

    \alert{$\Rightarrow$} Z lineárního důkazu snadno vyrobíme rezoluční strom. Indukcí dle délky důkazu: máme-li boční klauzuli $B_i\notin S$, potom $B_i=C_j$ pro nějaké $j<i$: místo $B_i$ připojíme rezoluční strom pro $C_j$ z $S$. 
    
    \alert{$\Leftarrow$} Plyne z úplnosti lineární rezoluce, důkaz najdete v učebnici.
        
\end{frame}


\begin{frame}{LI-rezoluce}
    
    \begin{itemize}
        \item \alert{lineární důkaz:} boční klauzule je \alert{axiom nebo dřívější centrální}
        \item co když požadujeme, aby boční klauzule byly \alert{pouze axiomy?}
        
        $\Rightarrow$ \alert{LI-rezoluce (linear-input)}        
            
    \end{itemize}
    
    \myblock{
    \alert{LI-důkaz} (rezolucí) klauzule $C$ z formule $S$ je lineární důkaz 
    $$
    \begin{bmatrix}
        C_0 \\
        B_0
    \end{bmatrix},
    \begin{bmatrix}
        C_1 \\
        B_1
    \end{bmatrix},\dots,
    \begin{bmatrix}
        C_n \\
        B_n
    \end{bmatrix},
    C
    $$
    ve kterém je každá boční klauzule $B_i$ axiom z $S$. Pokud LI-důkaz existuje, říkáme, že je $C$ \alert{LI-dokazatelná} z $S$, a píšeme \alert{$S\proves_{LI}C$}. Pokud $S\proves_{LI}\square$, je $S$ \alert{LI-zamítnutelná}.
    }

    \begin{itemize}
        \item LI-důkaz odpovídá rezolučnímu stromu tvaru \emph{``chlupaté cesty''}
        \item z toho plyne korektnost
        \item ztrácíme úplnost, ale hledání důkazů je snazší
        \item ukážeme \alert{úplnost pro Hornovy formule}, je základem Prologu
    \end{itemize}

\end{frame}


\begin{frame}{Hornovy formule}
        
    \begin{itemize}
        \item \alert{Hornova klauzule} má nejvýše jeden pozitivní literál
        \item \alert{Hornova formule} je množina Hornových klauzulí (i nekonečná)
        \item \alert{Fakt} je pozitivní jednotková klauzule, např. $\{p\}$
        \item \alert{Pravidlo} je klauzule s právě jedním pozitivním a alespoň jedním negativním literálem
        \item Pravidlům a faktům říkáme \alert{programové klauzule}
        \item \alert{Cíl} je neprázdná klauzule bez pozitivního literálu       
        \item dokazujeme sporem: \alert{cíl} je negací toho, co chceme dokázat (konjunkce faktů)
    \end{itemize}

    \myblock{
        \textbf{Pozorování:}
        Je-li Hornova formule $S$ nesplnitelná a $\square\notin S$, potom obsahuje fakt i cíl.
    }

    \textbf{Důkaz:} Neobsahuje-li fakt, ohodnotíme všechny proměnné~0; neobsahuje-li cíl, ohodnotíme 1.\hfill\qedsymbol

\end{frame}


\begin{frame}{Příklad konstrukce LI-zamítnutí}
    
    Ukážeme: \myexampleinline{
        $T=\{\{p,\neg r,\neg s\},\{\neg q,r\},\{q,\neg s\},\{s\}\}\ \models\ p\land q$
    } Sestrojíme LI-zamítnutí $T\cup\{G\}\proves_{LI}\square$ pro cíl $G=\{\neg p,\neg q\}$. 
    
    
    V $T$ \alert{najdeme fakt}, a provedeme \alert{jednotkovou propagaci} v $T\cup\{G\}$. Opakujeme, dokud není formule prázdná:
    \begin{itemize}
        \item $T=\{\{p,\neg r,\neg s\},\{\neg q,r\},\{q,\neg s\},\{s\}\}$, $G=\{\neg p,\neg q\}$
        \item $T^s=\{\{p,\neg r\},\{\neg q,r\},\{q\}\}$, $G^s=\{\neg p,\neg q\}$
        \item $T^{sq}=\{\{p,\neg r\},\{r\}\}$, $G^{sq}=\{\neg p\}$
        \item $T^{sqr}=\{\{p\}\}$, $G^{sqr}=\{\neg p\}$
        \item $T^{sqrp}=\emptyset$, $G^{sqrp}=\square$
    \end{itemize}

    To, že vždy najdeme fakt, plyne z Pozorování pro $T\cup\{G\}$.

    Nyní zpětným postupem sestrojíme LI-zamítnutí, podobně jako v důkazu úplnosti rezoluce.
    
\end{frame}


\begin{frame}{Konstrukce zamítnutí zpětným postupem}

    \begin{itemize}
        \item $T^{sqrp},G^{sqrp}\proves_{LI}\square$:
        \begin{center}
            \begin{forest}
                for tree={math content,grow=west,text height=2ex, text depth=1ex, l sep=3em}
                            [{\square}]                       ]
            \end{forest} 
        \end{center}
        \item $T^{sqr},G^{sqr}\proves_{LI}\square$:
        \begin{center}
            \begin{forest}
                for tree={math content,grow=west,text height=2ex, text depth=1ex, l sep=3em}
                            [{\square}
                                [,phantom]
                                [\{\alert{\neg p}\}]
                                [{\{p\}}]                        
                            ]
            \end{forest} 
        \end{center}
        
        \item $T^{sq},G^{sq}\proves_{LI}\square$:
        \begin{center}
            \begin{forest}
                for tree={math content,grow=west,text height=2ex, text depth=1ex, l sep=3em}
                        [{\square}
                            [,phantom]
                            [{\{\alert{\neg r}\}}
                                [,phantom]
                                [{\{\neg p\}}]
                                [{\{p,\alert{\neg r}\}}]                        
                            ]
                            [{\{r\}}]
                        ]
            \end{forest} 
        \end{center}
        
    \end{itemize}

\end{frame}


\begin{frame}{Konstrukce zamítnutí zpětným postupem -- pokračování}
    
    \begin{itemize}
        \item $T^{s},G^{s}\proves_{LI}\square$:
        
        \bigskip

        \begin{center}
            \begin{forest}
                for tree={math content,grow=west,text height=2ex, text depth=1ex, l sep=2em}
                    [{\square}
                        [,phantom]
                        [{\{\alert{\neg q}\}}
                            [,phantom]
                            [{\{\alert{\neg q},\neg r\}}
                                [,phantom]
                                [{\{\neg p,\alert{\neg q}\}}]
                                [{\{p,\neg r\}}]                        
                            ]
                            [{\{\alert{\neg q},r\}}]
                        ]
                        [{\{q\}}]                    
                    ]
            \end{forest} 
        \end{center}
        
        \item $T,G\proves_{LI}\square$
        
        \bigskip
            \begin{forest}
                for tree={math content,grow=west,text height=2ex, text depth=1ex, l sep=2em}
                [{\square}
                    [,phantom]
                    [{\{\alert{\neg s}\}}
                        [,phantom]
                        [{\{\neg q,\alert{\neg s}\}}
                            [,phantom]
                            [{\{\neg q,\neg r,\alert{\neg s}\}}
                                [,phantom]
                                [{\{\neg p,\neg q\}}]
                                [{\{p,\neg r,\alert{\neg s}\}}]                        
                            ]
                            [{\{\neg q,r\}}]
                        ]
                        [{\{q,\alert{\neg s}\}}]                    
                    ]
                    [{\{s\}}]
                ]
            \end{forest}
        
    \end{itemize}

\end{frame}


\begin{frame}{Úplnost pro Hornovy formule}
 
    \myblock{
    \textbf{Věta (O úplnosti LI-rezoluce pro Hornovy formule):} 
    Je-li Hornova formule $T$ splnitelná, a $T\cup\{G\}$ je nesplnitelná pro cíl $G$, potom $T\cup\{G\}\proves_{LI}\square$, a to LI-zamítnutím, které začíná cílem $G$.
    }

    \textbf{Důkaz:} Opět lze díky Větě o kompaktnosti předpokládat, že $T$ je konečná. LI-zamítnutí sestrojíme indukcí podle počtu proměnných.

    Z Pozorování víme, že $T$ obsahuje fakt $\{p\}$. Protože $T\cup\{G\}$ je nesplnitelná, je dle Lemmatu o dosazení \alert{nesplnitelná i} $(T\cup\{G\})^p$ $=$ \alert{$T^p\cup\{G^p\}$}, kde $G^p=G\setminus\{\neg p\}$.
    
    Všimněte si, že \alert{$T^p$ je splnitelná}. (Stejným ohodnocením jako $T$, neboť to musí obsahovat $p$ kvůli faktu $\{p\}$, tedy neobsahuje $\neg p$.) 
    
    Zároveň má $T^p$ méně proměnných než $T$. \alert{Je-li $G^p$ cíl}, využijeme indukčního předpokladu (následující slide). Co když ale $G^p$ není cíl?       
    
\end{frame}


\begin{frame}{Dokončení důkazu}

    \alert{Není-li $G^p$ cíl}, nutně $G^p=\square$ a $G=\{\neg p\}$. Potom je $\square$ rezolventou $G$ a faktu $\{p\}\in T$, a máme jednokrokové LI-zamítnutí $T\cup\{G\}$. (To dává i \alert{bázi indukce}.)


    \alert{Je-li $G^p$ cíl}, dle \alert{indukčního předpokladu} existuje LI-odvození $\square$ z $T^p\cup\{G^p\}$ začínající $G^p=G\setminus\{\neg p\}$.
    
    Hledané LI-zamítnutí $T\cup\{G\}$ začínající $G$ zkonstruujeme (podobně jako v důkazu Věty o úplnosti rezoluce):

    \begin{itemize}
        \item Přidáme literál $\neg p$ do všech listů, které už nejsou v $T\cup\{G\}$ (vznikly odebráním $\neg p$), a do všech vrcholů nad nimi.
        \item Tím získáme $T\cup\{G\}\proves_{LI}\neg p$.
        \item Na závěr přidáme boční klauzuli $\{p\}$ a odvodíme $\square$.\hfill\qedsymbol
    \end{itemize}

\end{frame}


%\section{5.5 Rezoluce v Prologu}


\begin{frame}[fragile]{Program v Prologu}
    
    síla Prologu vychází z \alert{unifikace} a rezoluce v predikátové logice, nyní si ale ukážeme příklad \alert{výrokového} programu:

    \begin{itemize}
        \item \alert{program} v Prologu je Hornova formule obsahující pouze \alert{programové klauzule}, tj. \alert{fakta} nebo \alert{pravidla}
        \item \alert{dotaz} je konjunkce faktů, negace dotazu je \alert{cíl}
    \end{itemize}
      
    Např.\myexampleinline{
        program $\{\{p,\neg r,\neg s\},\{\neg q,r\},\{q,\neg s\},\{s\}\}$, dotaz $p\land q$
    } 
    \begin{itemize}
        \item klauzule $\{p,\neg r,\neg s\}$ je ekvivalentní $r\land s\limplies p$, píšeme \texttt{p:-r,s.}
        \item výsledný program a dotaz:
        \begin{verbatim}
            p:-r,s.
            r:-q.
            q:-s.
            s.
            ?-p,q.    
        \end{verbatim}
    \end{itemize}
    
    Například klauzuli $\{p,\neg r,\neg s\}$, která je ekvivalentní $r\land s\limplies p$, zapíšeme v Prologu jako: \texttt{p:-r,s.}
    \begin{verbatim}
        p:-r,s.
        r:-q.
        q:-s.
        s.    
    \end{verbatim}
    A programu položíme dotaz:
    \begin{verbatim}
        ?-p,q.    
    \end{verbatim}

\end{frame}


\begin{frame}{Rezoluce v Prologu}
    
    \myblock{
        \textbf{Důsledek:} Mějme program $P$ a dotaz $Q=p_1\land\dots\land p_n$, a označme $G=\{\neg p_1,\dots,\neg p_n\}$ (tj. $G\sim \neg Q$). Následující podmínky jsou ekvivalentní:
        
        \begin{enumerate}[(i)]
            \item $P\models Q$,
            \item $P\cup\{G\}$ je nesplnitelná,
            \item $P\cup\{G\}\proves_{LI}\square$, a existuje LI-zamítnutí začínající cílem $G$.
        \end{enumerate}
    }
    
    \medskip

    \textbf{Důkaz:}
    \begin{itemize}
        \item \alert{(i)$\Leftrightarrow$(ii)} Věta o důkazu sporem
        \item \alert{(ii)$\Leftrightarrow$(iii)} Věta o úplnosti LI-rezoluce pro Hornovy formule (Program je vždy splnitelný)\hfill\qedsymbol
    \end{itemize}

\end{frame}


\section{ČÁST II -- PREDIKÁTOVÁ LOGIKA}


\section{\sc Kapitola 6: Syntaxe a sémantika predikátové logiky}


\section{6.1 Úvod}


\begin{frame}{Predikátová logika neformálně}

    \textbf{Výroková logika:} popis světa pomocí \alert{výroků} složených z \alert{prvovýroků} (\alert{výrokových proměnných}) -- bitů informace
    
    \textbf{Predikátová logika [prvního řádu]:}
    \begin{itemize}
        \item základní stavební kámen jsou \alert{proměnné} reprezentující \alert{individua} -- nedělitelné objekty z nějaké množiny (např. přirozená čísla, vrcholy grafu, stavy mikroprocesoru)
        \item tato individua mají určité vlastnosti a vzájemné vztahy (\alert{relace}), kterým říkáme \alert{predikáty}
        \begin{itemize}
            \item \myexampleinline{
                $\mathrm{Leaf}(x)$
                }nebo \myexampleinline{
                    $\mathrm{Edge}(x,y)$
                    } mluvíme-li o grafu
            \item \myexampleinline{
                $x\leq y$
                } v přirozených číslech
        \end{itemize}
        \item a mohou vstupovat do \alert{funkcí}
        \begin{itemize}
            \item \myexampleinline{
                $\mathrm{lowest\_common\_ancestor}(x,y)$
                } v zakořeněném stromu
            \item \myexampleinline{
                $\mathrm{succ}(x)$
                } nebo \myexampleinline{
                    $x+y$
                    } v přirozených číslech
        \end{itemize}
        \item a mohou být \alert{konstantami} se speciálním významem, např. \myexampleinline{
            $\mathrm{root}$
            } v zakořeněném stromu, 
            \myexampleinline{
                $0$
            } v tělese. 
    \end{itemize}    

\end{frame}


\begin{frame}{Syntaxe neformálně}

    \begin{itemize}
        \item \alert{atomické formule}: predikát (včetně \alert{rovnosti} $=$) o proměnných nebo o \alert{termech} (`výrazy' složené z funkcí popř. konstant)
        \item \alert{formule} jsou složené z atomických formulí pomocí logických spojek, a dvou \alert{kvantifikátorů}:
    \end{itemize}  

    \medskip
    \myalert{
        $\forall x$ ``pro všechna individua (reprezentovaná proměnnou $x$)''
            
        $\exists x$ ``existuje individuum (reprezentované proměnnou $x$)''
    }

    \bigskip

    Např. \myexampleinline{\textit{``Každý, kdo má dítě, je rodič.''}} lze formalizovat takto:
    \myexamplemath{
    $$
    (\forall x)((\exists y)\mathrm{child\_of}(y,x)\limplies\mathrm{is\_parent}(x))
    $$
    }

    \begin{itemize}
        \item \alert{$\mathrm{child\_of}(y,x)$} je binární predikát vyjadřující, že individuum reprezentované proměnnou $y$ je dítětem individua reprezentovaného proměnnou $x$
        \item \alert{$\mathrm{is\_parent}(x)$} je unární predikát vyjadřující, že individuum reprezentované $x$ je rodič
    \end{itemize}
    
\end{frame}


\begin{frame}{Sémantika neformálně}

    \myexamplemath{
    $$
    (\forall x)((\exists y)\mathrm{child\_of}(y,x)\limplies\mathrm{is\_parent}(x))
    $$
    }

    Platnost? Záleží na \alert{modelu} světa/systému, který nás zajímá:
    
    \alert{Model} je\dots
    \begin{itemize}
        \item (neprázdná) množina individuí, spolu
        \item s binární relací \alert{interpretující} binární relační symbol \alert{$\mathrm{child\_of}$}, a
        \item s unární relací (tj. podmnožinou) interpretující unární relační symbol \alert{$\mathrm{is\_parent}$}        
    \end{itemize}
    Obecně mohou být relace jakékoliv, snadno sestrojíme model, ve kterém formule neplatí, např. 
    $$
    \mystructure{A}=\langle\{0,1\},\{(0,0),(0,1),(1,0),(1,1)\},\emptyset\rangle
    $$

\end{frame}


\begin{frame}{Příklad s funkcemi a konstantami}

    ``Je-li $x_1\leq y_1$ a $x_2\leq y_2$, potom platí $(y_1 \cdot y_2)-(x_1\cdot x_2)\geq 0$.''

    \myexampleamsmath{
        $$
        \varphi=(x_1\leq y_1)\land (x_2\leq y_2)\limplies ((y_1 \cdot y_2)+(-(x_1\cdot x_2))\geq 0)
        $$
    }

    \begin{itemize}
        \item dva binární relační symboly ($\leq,\geq$), binární funkční symbol $+$, unární funkční symbol $-$, a konstantní symbol $0$
        \item \alert{model, ve kterém $\varphi$ platí:} $\mathbb N$ s binárními relacemi $\leq^\mathbb N,\geq^\mathbb N$, bin. funkcemi $+^\mathbb N,\cdot^\mathbb N$, unární funkcí $-^\mathbb N$, a konstantou $0^\mathbb N=0$ 
        \item vezmeme-li ale podobně množinu $\mathbb Z$, $\varphi$ už platit nebude
    \end{itemize}
    Poznámky:
    \begin{itemize}
        \item mohli bychom chápat `$-$' jako binární, obvykle ale bývá unární
        \item pro \alert{konstantní symbol} $0$ používáme (jak je zvykem) stejný symbol, jako pro přirozené číslo 0. Ale pozor, v našem modelu může být \alert{symbol} $0$ interpretován jako \alert{jiné číslo}, nebo náš model vůbec nemusí sestávat z čísel!
    \end{itemize}

\end{frame}


\begin{frame}{Ještě o syntaxi}

    \myexampleamsmath{
        $$
        \varphi=(x_1\leq y_1)\land (x_2\leq y_2)\limplies ((y_1 \cdot y_2)+(-(x_1\cdot x_2))\geq 0)
        $$
    }

    \begin{itemize}
        \item $\varphi$ nemá žádné kvantifikátory, tj. je \alert{otevřená}
        \item $x_1,x_2,y_1,y_2$ jsou \alert{volné proměnné} této formule (nejsou \alert{vázané} žádným kvantifikátorem), píšeme $\varphi(x_1,x_2,y_1,y_2)$
        \item sémantiku $\varphi$ chápeme stejně jako $(\forall x_1)(\forall x_2)(\forall y_1)(\forall y_2)\varphi$
        \item používáme \alert{konvence} (infixový zápis, vynechání závorek), jinak:
        $$
        \varphi=((\leq (x_1,y_1) \land \leq(x_2,y_2))\limplies \leq(+(\cdot (y_1,y_2),-(\cdot(x_1,x_2))),0))
        $$
        \item cvičení: definujte \alert{strom formule}, nakreslete ho pro $\varphi$
    \end{itemize}
   
\end{frame}


\begin{frame}{Termy vs. atomické formule}
    
    \myexampleamsmath{
        $$
        \varphi=(x_1\leq y_1)\land (x_2\leq y_2)\limplies ((y_1 \cdot y_2)+(-(x_1\cdot x_2))\geq 0)
        $$
    }
    
    \begin{itemize}
        \item výraz \alert{$(y_1 \cdot y_2)+(-(x_1\cdot x_2))$} je \alert{term}
        \item výrazy \alert{$(x_1\leq y_1)$}, \alert{$(x_2\leq y_2)$} a \alert{$((y_1 \cdot y_2)+(-(x_1\cdot x_2))\geq 0)$} jsou (všechny) \alert{atomické (pod)formule} $\varphi$ 
    \end{itemize}
   
    V čem je rozdíl? Máme-li konkrétní model, a konkrétní \alert{ohodnocení proměnných} individui (prvky) tohoto modelu:
    
    \begin{itemize}
        \item výsledkem termu (při daném ohodnocení proměnných) je konkrétní \alert{individuum z modelu}, zatímco
        \item atomickým formulí lze přiřadit \alert{pravdivostní hodnotu} (a tedy kombinovat je logickými spojkami)
    \end{itemize}   

\end{frame}


\section{6.2 Struktury}


\begin{frame}{Signatura}

    \begin{itemize}
        \item specifikuje jakého \alert{typu} bude daná struktura, tj. jaké  má relace, funkce (jakých arit) a konstanty, a symboly pro ně 
        \item \alert{konstanty} lze chápat jako funkce arity 0, tj. funkce bez vstupů
    \end{itemize}

    \myblock{
        \alert{Signatura} je dvojice $\langle\mathcal R,\mathcal F\rangle$, kde $\mathcal R,\mathcal F$ jsou disjunktní množiny symbolů (\alert{relační} a \alert{funkční}, ty zahrnují \alert{konstantní}) spolu s danými aritami (tj. danými funkcí $\mathrm{ar}\colon \mathcal R\cup\mathcal F\to\mathbb N$) a neobsahující symbol `$=$' (ten je rezervovaný pro \alert{rovnost}). 
    }

    \bigskip

    \begin{itemize}
        \item často zapíšeme jen výčtem symbolů, jsou-li arity a zda jsou relační nebo funkční zřejmé
        \item kromě běžně používaných symbolů typicky používáme:
        \begin{itemize}
            \item pro relační symboly $P,Q,R,\dots$
            \item pro funkční (nekonstantní) symboly $f,g,h,\dots$
            \item pro konstantní symboly $c,d,a,b,\dots$
        \end{itemize}
    \end{itemize}
    

\end{frame}


\begin{frame}{Příklady signatur}

    \begin{itemize}
        \item \myexampleinline{
        $\langle E \rangle$
        } signatura \alert{grafů}: $E$ je binární relační symbol (struktury jsou uspořádané grafy)
        \item \myexampleinline{
        $\langle \leq \rangle$
        } signatura \alert{částečných uspořádání}: stejná jako signatura grafů, jen jiný symbol (ne každá struktura v této signatuře je částečné uspořádání! k tomu musí splňovat příslušné \alert{axiomy})
        \item \myexampleinline{
        $\langle +, -, 0\rangle$
        } signatura \alert{grup}: $+$ je binární funkční, $-$ unární funkční, $0$ konstantní symbol
        \item \myexampleinline{
        $\langle +, -, 0,\cdot,1\rangle$
        } signatura \alert{těles}: $\cdot$ je binární funkční, $1$ konstantní symbol
        \item \myexampleinline{
        $\langle +, -, 0,\cdot,1,\leq\rangle$
        } signatura \alert{uspořádaných těles}: $\leq$ je binární relační symbol
        \item \myexampleinline{
        $\langle -,\landsymb,\lorsymb,\bot,\top\rangle$
        } signatura \alert{Booleových algeber}: $\landsymb,\lorsymb$ jsou binární funkční, $\bot,\top$ jsou konstantní symboly
        \item \myexampleinline{
        $\langle S,+,\cdot,0,\leq\rangle$
        } signatura \alert{aritmetiky}: $S$ je unární funkční symbol
    \end{itemize}
    
\end{frame}


\begin{frame}{Struktury}

    \alert{Strukturu} dané signatury získáme tak, že:
    \begin{itemize}
        \item zvolíme neprázdnou \alert{doménu}, a na ní
        \item zvolíme \alert{realizace} (také říkáme \alert{interpretace}) všech relačních a funkčních symbolů (včetně konstantních)
        \item to znamená \alert{konkrétní} relace resp. funkce příslušných arit
        \item realizací konstantního symbolu je zvolený prvek z domény
        \item na tom, jaké konkrétní symboly jsou v signatuře nezáleží (např. $+$ neznamená, že realizace musí souviset se sčítáním)
    \end{itemize}

\end{frame}


\begin{frame}{Příklady struktur 1/3}
    
    \begin{itemize}
        \item Struktura \alert{v prázdné signatuře} $\langle\ \rangle$ je libovolná neprázdná množina. (Nemusí být konečná, ani spočetná! Formálně to bude trojice $\langle A,\emptyset,\emptyset\rangle$, ale rozdíl zanedbáme.)            

        \item Struktura \alert{v signatuře grafů} je $\mathcal G=\langle V,E\rangle$, kde $V\neq\emptyset$ a $E\subseteq V^2$, říkáme jí \alert{orientovaný graf}. 
        \begin{itemize}
            \item je-li $E$ ireflexivní a symetrická, je to \alert{jednoduchý} graf
            \item je-li $E$ reflexivní, tranzitivní, a antisymetrická, jde o \alert{částečné uspořádání}
            \item je-li $E$ reflexivní, tranzitivní, a symetrická, je to \alert{ekvivalence}
        \end{itemize}
        \item Struktury \alert{v signatuře částečných uspořádání} jsou tytéž, jako v signatuře grafů, signatury se liší jen symbolem. (Ne každá struktura v signatuře částečných uspořádání je č. uspořádání!)
    \end{itemize}

\end{frame}


\begin{frame}{Příklady struktur 2/3}
    
    Struktury \alert{v signatuře grup} jsou například následující \alert{grupy}:
        \begin{itemize}
            \item \myexampleinline{
                $\underline{\mathbb Z_n}=\langle\mathbb Z_n,+,-,0\rangle$
            }, \alert{aditivní grupa celých čísel modulo $n$} (operace jsou modulo $n$). 
            
            \smallskip
            
            \textbf{Poznámka:} $\underline{\mathbb Z_n}$ znamená strukturu, zatímco $\mathbb Z_n$ jen její doménu. Často se to ale nerozlišuje a $\mathbb Z_n$ se používá i pro strukturu. Podobně $+,-,0$ jsou jak symboly, tak interpretace.

            \smallskip

            \item \myexampleinline{
                $\mathcal S_n=\langle \mathrm{Sym}_n,\circ,{}^{-1},\mathrm{id}\rangle$
             } je \alert{symetrická grupa} (grupa všech permutací) na $n$ prvcích.
            \item \myexampleinline{
                $\underline{\mathbb Q}^*=\langle \mathbb Q\setminus\{0\},\cdot,{}^{-1},1\rangle$
                } je \alert{multiplikativní grupa (nenulových) racionálních čísel}. \myalertinline{(Interpretací \alert{symbolu} $0$ je \alert{číslo} $1$!)}
        \end{itemize}
        Všechny tyto struktury \alert{splňují axiomy teorie grup}, snadno ale najdeme jiné, které axiomy nesplňují, nejsou tedy grupami.
    

\end{frame}


\begin{frame}{Příklady struktur 3/3}
    
    \begin{itemize}
        \item Struktury \myexampleinline{
            $\underline{\mathbb Q}=\langle \mathbb Q, +, -, 0,\cdot,1,\leq\rangle$
         }a\myexampleinline{
            $\underline{\mathbb Z}=\langle \mathbb Z, +, -, 0,\cdot,1,\leq\rangle$
          } (se standardními operacemi a uspořádáním) jsou \alert{v signatuře uspořádaných těles} (ale jen první z nich je uspořádané těleso).
        \item \myexampleinline{
            $\underline{\mathcal P(X)}=\langle \mathcal P(X),\bar{},\cap,\cup,\emptyset,X\rangle$
        }, tzv. \alert{potenční algebra} nad množinou $X$, je struktura \alert{v signatuře Booleových algeber}. (\alert{Booleova algebra} je to pokud $X\neq\emptyset$.)
        \item \myexampleinline{
            $\underline{\mathbb N}=\langle \mathbb N,S,+,\cdot,0,\leq\rangle$
        }, kde $S(x)=x+1$, a ostatní symboly jsou realizovány standardně, je \alert{standardní model aritmetiky}.
    \end{itemize}

\end{frame}


\begin{frame}{Definice struktury}
    
    \myblock{
    \alert{Struktura v signatuře} $\langle\mathcal R,\mathcal F\rangle$ je trojice $\A=\langle A, \mathcal R^\A,\mathcal F^\A \rangle$, kde
    \begin{itemize}
    \item  $A$ je neprázdná množina, říkáme jí \alert{doména} (také \alert{univerzum}),
    \item $\mathcal R^\A=\{R^\A\mid R\in\mathcal R\}$ kde $R^\A\subseteq A^{\mathrm{ar}(R)}$ je \alert{interpretace} relačního symbolu $R$,
    \item $\mathcal F^\A=\{f^\A\mid f\in\mathcal F\}$ kde $f^\A\colon A^{\mathrm{ar}(f)}\to A$ je \alert{interpretace} funkčního symbolu $f$ (speciálně pro konstantní symbol $c\in\mathcal F$ máme $c^\A\in A$).
    \end{itemize}
    }

    \medskip

    \myexampleinline{Příklad:} rozmyslete si, jak vypadají struktury v \alert{signatuře $n$ konstant} $\langle c_1,c_2,\dots,c_n\rangle$? Popište všechny 5-prvkové v signatuře 3 konstant. 

\end{frame}


\section{6.3 Syntaxe}


\begin{frame}{Jazyk}

    \myblock{
    Jazyk je daný \alert{signaturou} a informací, zda je \alert{s rovností} nebo ne. 
    }    

    Tj. specifikujeme `typ' modelů a zda můžeme používat symbol `\alert{$=$}' interpretovaný jako \alert{identita} prvků z domény; většinou to dovolíme. (Je-li jazyk bez  rovnosti, musí mít signatura relační symbol. Proč?)
    
    Do jazyka patří:
    \begin{itemize}
        \item spočetně mnoho \alert{proměnných} $x_0,x_1,x_2,\dots$ (píšeme také $x,y,z,\dots$; množinu všech proměnných označíme \alert{$\Var$})
        \item \alert{relační}, \alert{funkční} a \alert{konstantní symboly} ze signatury, symbol \alert{$=$} jde-li o jazyk s rovností (to jsou `\alert{mimologické}' symboly)
        \item \alert{univerzální} a \alert{existenční} \alert{kvantifikátory} $(\forall x),(\exists x)$ pro každou proměnnou $x\in\Var$ (kvantifikátor `$(\forall x)$' chápeme jako jediný symbol, tj. \alert{neobsahuje} proměnnou $x$)
        \item symboly pro log. spojky \alert{\( \neg,\landsymb,\lorsymb, \limpliessymb, \liffsymb \)}, závorky \alert{\( (,) \)}, a čárka `\alert{$,$}'
    \end{itemize}

\end{frame}


\begin{frame}{Jazyk: příklady}

    
    \begin{itemize}
        \item Jazyk \alert{$L=\langle\rangle$ s rovností} je jazyk \alert{čisté rovnosti}
        \item jazyk $L=\langle c_0,c_1,c_2,\dots\rangle$ s rovností je jazyk \alert{spočetně mnoha konstant}
        \item jazyk \alert{uspořádání} je $\langle \leq \rangle$ s rovností
        \item jazyk \alert{teorie grafů} je $\langle E \rangle$ s rovností
        \item jazyky \alert{teorie grup, teorie těles, teorie uspořádaných těles, Booleových algeber, aritmetiky} jsou jazyky \alert{s rovností} odpovídající daným signaturám
    \end{itemize}
    
\end{frame}


\begin{frame}{Termy}

    \vspace{-6pt}
    \alert{čistě syntaktické} `výrazy' z proměnných, konstantních symbolů, funkčních symbolů, závorek a čárek
    
    \myblock{
        \alert{Termy} jazyka $L$ jsou konečné nápisy definované induktivně:
        \begin{itemize}
            \item každá proměnná a každý konstantní symbol z $L$ je term,
            \item je-li $f$ funkční symbol z $L$ arity $n$ a jsou-li $t_1,\dots,t_n$ termy, potom nápis $f(t_1,t_2,\dots,t_n)$ je také term.
        \end{itemize}
        Množinu všech \alert{termů} jazyka $L$ označíme $\Term_L$. 
    }

    \begin{itemize}
        \item \alert{podterm} je podřetězec, který je sám termem
        \item term bez proměnných je \alert{konstantní (ground)}, např. \myexampleinline{$((S(0)+S(0))\cdot S(S(0)))$} v jazyce aritmetiky
        \item \myalertinline{termy nesmí obsahovat prvky struktury, jen symboly z jazyka}
        \item \myexampleinline{$(1+1)\cdot 2$} \alert{není} term, ledaže rozšíříme jazyk o \alert{symboly} $1$ a $2$
        \item jako lidé můžeme použít \alert{infixový} zápis, např. $(t_1+t_2)$ místo $+(t_1,t_2)$, vynechat závorky je-li struktura termu zřejmá
    \end{itemize}

\end{frame}


\begin{frame}{Strom termu}
    
    \vspace{-3pt}
    \alert{Strom termu} $t$, \alert{$\Tree(t)$}: v listech proměnné nebo konst. symboly, ve vnitřních vrcholech funkční symboly (arita je rovna počtu synů)

    \begin{center}

    \scalebox{0.8}{
        \tikzset{every label/.style = {text=red}}
        \begin{minipage}{.49\textwidth}
            \centering
            \begin{forest}
                for tree={math content,circle,draw=blue!20,fill=blue!10,minimum size=22pt}
                [\cdot 
                    [+ 
                        [S
                            [0]                    
                        ] 
                        [x]
                    ]
                    [y]
                ]
            \end{forest}
    
            (a) $(S(0) + x) \cdot y$ v jazyce aritmetiky
        \end{minipage}
        \begin{minipage}{.49\textwidth}
            \centering
            \begin{forest}
                for tree={math content,circle,draw=blue!20,fill=blue!10,minimum size=22pt}
                [\lor 
                    [\neg 
                        [\land
                            [x]
                            [y]                    
                        ]
                    ]
                    [\bot]
                ]
            \end{forest}
            
            (b) $\neg (x\land y)\lor \bot$ v jazyce Booleových algeber 
        \end{minipage}
    }
    \end{center}

    \begin{itemize}
        \item symboly $\neg,\land,\lor$ nejsou logické, ale mimologické ze signatury
        \item \alert{sémantika}: proměnné ohodnotíme prvky, konst. a funkční symboly nahradíme interpretacemi, výsledek je prvek z domény
    \end{itemize}
    
\end{frame}


\begin{frame}{Atomické formule}
    Termům nelze přiřadit \alert{pravdivostní hodnotu}, potřebujeme \alert{predikát} (relační symbol nebo $=$), který mluví \alert{o `vztahu' termů}: v dané struktuře při ohodnocení proměnných prvky je buď splněn, nebo ne.

    \medskip

    \alert{Formule} (`tvrzení o strukturách') skládáme z \alert{atomických formulí} pomocí  logických spojek a kvantifikátorů:  

    \medskip

    \myblock{
        \alert{Atomická formule} jazyka $L$ je nápis $R(t_1,\dots,t_n)$, kde $R$ je $n$-ární relační symbol z $L$ (včetně $=$ jde-li o jazyk s rovností) a $t_i\in\Term_L$. %Množinu všech \alert{atomických formulí} jazyka $L$ označíme $\AFm_L$. 
    }

    \medskip

    \begin{itemize}
        \item \myexampleinline{
            $R(f(f(x)),c, f(d))$
            } kde $R$ je ternární relační, $f$ unární funkční, $c,d$ konstantní symboly
            \item \myalertinline{ infixový zápis } $\leq(x,y)$, $=(t_1,t_2)$ píšeme jako $x\leq y$, $t_1=t_2$
        \item \myexampleinline{
            $(x\cdot x)+(y\cdot y)\leq (x+y)\cdot(x+y)$
            } v jazyce uspořád. těles
        \item \myexampleinline{
            $x\cdot y\leq (S(0)+x)\cdot y$
            } v jazyce aritmetiky
        \item \myexampleinline{
            $\neg(x\land y)\lor\bot=\bot$
            } v jazyce Booleových algeber        
    \end{itemize} 
    
\end{frame}


\begin{frame}{Formule}
    
    \myblock{
        \alert{Formule} jazyka $L$ jsou konečné nápisy definované induktivně: 
        \begin{itemize}
            \item každá \alert{atomická formule} jazyka $L$ je formule,
            \item je-li $\varphi$ formule, potom \alert{$(\neg\varphi)$} je také formule
            \item jsou-li $\varphi,\psi$ formule, potom \alert{$(\varphi\lbin\psi)$ pro $\lbinsymb\in\{\landsymb,\lorsymb,\limpliessymb,\liffsymb\}$} jsou také formule
            \item je-li $\varphi$ formule a $x$ proměnná, potom \alert{$((Q x)\varphi)$ pro $Q\in\{\forall,\exists\}$} jsou také formule
            \end{itemize}    
        %Množinu všech \alert{formulí} jazyka $L$ označíme $\Fm_L$.
    }
    \begin{itemize}
        \item \alert{podformule} je podřetězec, který je sám formulí
        \item při zápisu formulí jako lidé používáme obvyklé konvence
        \item kvantifikátory mají stejnou prioritu jako $\neg$, vyšší než ostatní logické spojky! místo $((\forall x)\varphi$) píšeme $(\forall x)\varphi$
        \item \myalertinline{pozor, 
        $(\forall x)\varphi\land\psi$ neznamená totéž, co $(\forall x)(\varphi\land\psi)$!}
        \item někde uvidíte $\forall x\varphi$ nebo $\forall_ x\varphi$, my ale budeme psát jen $(\forall x)\varphi$
    
    \end{itemize}
    

\end{frame}


\begin{frame}{Strom formule}

    \begin{columns}
        
        \column{0.65\textwidth}

        Příklad: \myexampleinline{
        $(\forall x)(x\cdot y\leq (S(0)+x)\cdot y)$
        }

        \bigskip

        \alert{Strom formule}, \alert{$\Tree(\varphi)$}:
        
        \begin{itemize}
            \item strom atomické formule $R(t_1,\dots,t_n)$: v kořeni $R$, připojíme stromy $\Tree(t_i)$
            \item pro složené formule podobně jako ve výrokové logice
            \item kvantifikátory mají jediného syna
        \end{itemize}

        \vspace{1.5cm}

        \column{0.3\textwidth}

        \begin{center}
            \scalebox{0.9}{        
                \begin{forest}
                    for tree={math content,circle,draw=blue!20,fill=blue!10,minimum size=22pt}
                    [\forall x
                        [\leq 
                            [\cdot [x] [y]] 
                            [\cdot [+ [S [0]] [x]] [y]]
                        ]
                    ]
                \end{forest}        
            }
        \end{center}

    \end{columns}

\end{frame}


\begin{frame}{Volné a vázané proměnné}

    Význam formule (\alert{pravdivostní hodnota}) může/nemusí záviset na proměnných~v~ní:\myexampleinline{
            $x\leq 0$
        }vs.\myexampleinline{
            $(\exists x)(x\leq 0)$
         }vs.\myexampleinline{
            $x\leq 0 \lor (\exists x)(x\leq 0)$
         }

    \begin{itemize}
        \item \alert{výskyt $x$ ve $\varphi$:} list $\Tree(\varphi)$ označený $x$ [v $(Qx)$ nemá výskyt!]
        \item \alert{vázaný}: součástí podformule začínající $(Qx)$, jinak \alert{volný}
        \item $x$ je \alert{volná} ve $\varphi$ má-li volný výskyt, \alert{vázaná} má-li vázaný výskyt
        \item zápis \alert{$\varphi(x_1,\dots,x_n)$} znamená, že mezi $x_1,\dots,x_n$ jsou všechny volné proměnné ve formuli $\varphi$
    \end{itemize}

    Proměnná může být \alert{volná i vázaná}, např.:
    \myexamplemath{
    $$
    \varphi=(\forall x)(\exists y)(x\leq y)\lor x\leq z
    $$
    }

    \begin{itemize}
        \item první výskyt $x$ je vázaný a druhý volný (nakreslete si strom!) 
        \item $y$ je vázaná a $z$ je volná, můžeme tedy psát $\varphi(x,z)$
    \end{itemize}

\end{frame}


\begin{frame}{Otevřené a uzavřené formule}

    \alert{otevřená formule:} nemá žádný kvantifikátor\\
    \alert{uzavřená formule (sentence):} nemá žádnou volnou proměnnou  

     \begin{itemize}
        \item \myexampleinline{
            $x+y\leq 0$
        } je otevřená formule
        \item \myexampleinline{
            $(\forall x)(\forall y)(x+y\leq 0)$
        } je uzavřená formule  neboli sentence
        \item \myexampleinline{
            $(\forall x)(x+y\leq 0)$
         } není ani otevřená, ani uzavřená
        \item \myexampleinline{
            $(0+1=1)\land (1+1=0)$
         } je otevřená i uzavřená
        \item atomické formule je otevřená, otevřené formule jsou kombinace atomických pomocí logických spojek
        \item je-li formule otevřená i uzavřená potom nemá žádné proměnné (všechny termy v ní jsou konstantní)
        \item formule bez vázané proměnné není nutně otevřená!\myexampleinline{
            $(\forall x)0=1$
        }
    \end{itemize}

    Uvidíme, že \alert{pravdivostní hodnota} závisí jen na ohodnocení volných proměnných; \alert{sentence} mají ve struktuře pravdiv. hodnotu 0 nebo 1
    
\end{frame}


\begin{frame}{Instance a varianty: neformálně}

    \begin{itemize}
        \item proměnná může hrát různé \myalertinline{`role'} (\myalertinline{`lokální'} vs. \myalertinline{`globální'})
        \item \alert{instance}: \myalertinline{`dosazení'} do \myalertinline{`globální'} proměnné (lépe \myalertinline{`nahrazení'} proměnné nějakým termem, který ji počítá, čistě syntaktické!)
        \item \alert{varianta}: \myalertinline{`přejmenování'} \myalertinline{`lokální'} proměnné
    \end{itemize}
    
    \myexamplemath{
    $$
    P(x)\land (\forall x)(Q(x) \land (\exists x)R(x))
    $$
    }

    \begin{itemize}
        \item první výskyt $x$ je volný, 2. je vázaný $(\forall x)$, 3. je vázaný  $(\exists x)$
        \item pokud \alert{substituujeme} za proměnnou $x$ term $t=1+1$, dostáváme \alert{instanci} formule $\varphi$, kterou označíme \alert{$\varphi(x/t)$}:
        \myexamplemath{
        $$
        P(1+1)\land (\forall x)(Q(x) \land (\exists x)R(x))
        $$
        }
        \item přejmenujeme-li kvantifikátory, získáme \alert{variantu} formule $\varphi$:
        \myexamplemath{
        $$
        P(x)\land (\forall y)(Q(y) \land (\exists z)R(z))
        $$
        }
    \end{itemize}

    \myalert{Kdy a jak to lze, aby instance byla \alert{důsledek} a varianta \alert{ekvivalentní}?}    

\end{frame}


\begin{frame}{Instance}
    
    Substituujeme-li do $\varphi$ za $x$ term $t$, chceme aby výsledná formule \myalertinline{`říkala o $t$ totéž, co $\varphi$ o $x$'}. Např. \myexampleinline{
        $\varphi(x)=(\exists y)(x+y=1)$
        }
    \begin{itemize}
        \item říká o $x$, že \myalertinline{`existuje $1-x$'}
        \item term $t=1$ lze: \myexampleinline{\small
            $\varphi(x/t)=(\exists y)(1+y=1)$} říká \myalertinline{`existuje {\small $1-1$}'}
        \item term $t=y$ nelze: \myexampleinline{
            $(\exists y)(y+y=1)$} říká \myalertinline{`1 je dělitelné 2'}\\
        \textbf{problém:} obsahuje $y$, po nahrazení bude nově vázané $(\exists y)$
    \end{itemize}
    \myblock{
    Term $t$ je \alert{substituovatelný} za proměnnou $x$ ve formuli $\varphi$, pokud po simultánním nahrazení všech volných výskytů $x$ za $t$ nevznikne žádný vázaný výskyt proměnné z $t$. Potom je vzniklá formule \alert{instance} $\varphi$ vzniklá substitucí $t$ za $x$, \alert{$\varphi(x/t)$}.
    }
    \begin{itemize}
        \item $t$ \alert{není} substituovatelný za $x$ do $\varphi$, právě když $x$ má volný výskyt v nějaké podformuli $\varphi$ tvaru $(Qy)\psi$ a $y$ se vyskytuje v $t$
        \item speciálně: konstantní termy jsou vždy substituovatelné
    \end{itemize}

\end{frame}


\begin{frame}{Varianta}
    
    Substituovat $t$ můžeme vždy do \alert{varianty} $\varphi$, ve které přejmenujeme všechny kvantifikované proměnné na nové (které nejsou v $t$ ani $\varphi$)

    \smallskip
    
    \myblock{
    Má-li formule $\varphi$ podformuli tvaru $(Qx)\psi$ a je-li $y$ proměnná, že
    \begin{enumerate}[(i)]
        \item $y$ je substituovatelná za $x$ do $\psi$, a
        \item $y$ nemá volný výskyt v $\psi$.
    \end{enumerate}    
    \alert{Varianta} $\varphi$ vznikne nahrazením $(Qx)\psi$ formulí $(Qy)\psi(x/y)$,  říkáme tak i výsledku postupné variace ve více podformulích.
    }

    \medskip

    Mějme \myexampleinline{
        $\varphi=(\exists x)(\forall y)(x\leq y)$}:
    \begin{itemize}
        \item \myexampleinline{
            $(\exists u)(\forall v)(u\leq v)$} je varianta $\varphi$
        \item \myexampleinline{
            $(\exists y)(\forall y)(y\leq y)$} není varianta kvůli (i): $y$ není substituovatelná za $x$ do $\psi=(\forall y)(y\leq y)$
        \item \myexampleinline{
            $(\exists x)(\forall x)(x\leq x)$} není varianta kvůli (ii): $x$ má volný výskyt v~$\psi=(x\leq y)$        
    \end{itemize}   

\end{frame}


\end{document}
