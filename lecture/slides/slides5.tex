\documentclass{beamer}

%% slide-specific

\usetheme[progressbar=frametitle]{metropolis}
%\usecolortheme{spruce}
%\metroset{block=fill}

% define Metropolis colors    
\definecolor{mAlert}{HTML}{EB811B}
\definecolor{mExample}{HTML}{14B03D}
\definecolor{mBlock}{HTML}{23373b}

% my blocks
\setlength\fboxsep{0pt}%

\newcommand{\myblock}[1]{\colorbox{mBlock!8}{\begin{minipage}{\linewidth}#1\end{minipage}}}
\newcommand{\myblockmath}[1]{\colorbox{mBlock!8}{\begin{minipage}{\linewidth}\vspace{-6pt}#1\end{minipage}}}
\newcommand{\myblockinline}[1]{\colorbox{mBlock!8}{#1}}
\newcommand{\myexample}[1]{\colorbox{mExample!8}{\begin{minipage}{\linewidth}#1\end{minipage}}}
\newcommand{\myexamplemath}[1]{\colorbox{mExample!8}{\begin{minipage}{\linewidth}\vspace{-6pt}#1\end{minipage}}}
\newcommand{\myexampleinline}[1]{\colorbox{mExample!8}{#1}}
\newcommand{\myalert}[1]{\colorbox{mAlert!8}{\begin{minipage}{\linewidth}#1\end{minipage}}}
\newcommand{\myalertmath}[1]{\colorbox{mAlert!8}{\begin{minipage}{\linewidth}\vspace{-6pt}#1\end{minipage}}}
\newcommand{\myalertinline}[1]{\colorbox{mAlert!8}{#1}}

%% other
\newcommand{\mystructure}[1]{\mathcal{#1}}




%% packages
\usepackage{amsmath,amssymb,amsthm}
\usepackage{booktabs}
\usepackage[czech]{babel}
\usepackage{enumerate}
\usepackage{forest}
\usepackage{multicol}
% \usepackage{tcolorbox}
\usepackage{tikz}
    \usetikzlibrary{arrows.meta}
%\usepackage[unicode]{hyperref}
\usepackage[utf8x]{inputenc}
\usepackage{xfrac}

% %% theorems
% \theoremstyle{plain}
%     \newtheorem{theorem}{Věta}[section]
%     \newtheorem*{theorem-unnumbered}{Věta}
%     \newtheorem{proposition}[theorem]{Tvrzení}
%     \newtheorem{corollary}[theorem]{Důsledek}
%     \newtheorem{lemma}[theorem]{Lemma}
%     \newtheorem{observation}[theorem]{Pozorování}
% \theoremstyle{definition}
%     \newtheorem{definition}[theorem]{Definice}
%     \newtheorem*{algorithm}{Algoritmus}
% \theoremstyle{remark}
%     \newtheorem{remark}[theorem]{Poznámka}
%     \newtheorem{example}[theorem]{Příklad}
%     \newtheorem{exercise}{Cvičení}[chapter]
%     \newtheorem*{solution}{Řešení}

%% macros and definitions
\DeclareMathOperator{\Aut}{Aut}
\DeclareMathOperator{\Conseq}{Csq}
\DeclareMathOperator{\DeLO}{DeLO}
\DeclareMathOperator{\dom}{dom}
\DeclareMathOperator{\Fm}{Fm}
\DeclareMathOperator{\M}{M}
%\DeclareMathOperator{\Proof}{Proof}
\DeclareMathOperator{\rng}{rng}
\DeclareMathOperator{\Term}{Term}
\DeclareMathOperator{\Th}{Th}
\DeclareMathOperator{\Thm}{Thm}
\DeclareMathOperator{\Tree}{Tree}
\DeclareMathOperator{\Var}{Var}
\DeclareMathOperator{\VF}{VF}

\newcommand{\A}{\structure{A}}
\newcommand{\B}{\structure{B}}
\newcommand{\Con}{\mathit{Con}}
\newcommand{\disjointunion}{\mathbin{\dot{\sqcup}}}
\newcommand{\F}{\ensuremath{\mathrm{F}}}
\newcommand{\landsymb}{{\land}}
\newcommand{\lbin}{\mathbin{\square}}
\newcommand{\lbinsymb}{{\lbin}}
\newcommand{\liff}{\mathbin{\leftrightarrow}}
\newcommand{\liffsymb}{{\liff}}
\newcommand{\limplies}{\mathbin{\rightarrow}}
\newcommand{\limpliessymb}{{\limplies}}
\newcommand{\lorsymb}{{\lor}}
\newcommand{\Prf}{\mathit{Prf}}
\newcommand{\proves}{\vdash}
%\newcommand{\structure}[1]{\mathcal{#1}}
\newcommand{\todo}{[TODO]}
\newcommand{\T}{\ensuremath{\mathrm{T}}}
\newcommand{\union}{\mathbin{\cup}}


\title{Pátá přednáška}
\subtitle{NAIL062 Výroková a predikátová logika}
\author{Jakub Bulín (KTIML MFF UK)}
% \institute{KTIML MFF UK}
\date{Zimní semestr 2023}


\begin{document}


\maketitle


\begin{frame}{Pátá přednáška}

    \textbf{Program}
        \begin{itemize}
            \item věta o kompaktnosti
            \item hilbertovský kalkulus
            \item rezoluční metoda
            \item korektnost a úplnost rezoluce
            \item LI-rezoluce a Horn-SAT
        \end{itemize}

    \textbf{Materiály}

        \href{https://github.com/jbulin-mff-uk/nail062/raw/main/lecture/lecture-notes/lecture-notes.pdf}{\alert{\textbf{Zápisky z přednášky}}}, Sekce 4.7-4.8 z Kapitoly 4, Kapitola 5

\end{frame}


\section{4.7 Věta o kompaktnosti}


\begin{frame}{Kompaktnost}

    \myblock{
        \textbf{Věta (O kompaktnosti):} Teorie má model, právě když každá její konečná část má model.
    }
    \smallskip

    \textbf{Důkaz:} \alert{$\Rightarrow$ Snadné:} Model $T$ je zjevně modelem každé její části.
    
    \alert{$\Leftarrow$ Nepřímo:} buď $T$ sporná, najdeme spornou konečnou $T'\subseteq T$.

    Z \alert{úplnosti} víme, že $T\proves\bot$, tedy existuje i \alert{konečný} tablo důkaz $\tau$ výroku $\bot$ z $T$. Konstrukce $\tau$ má konečně mnoho kroků, použili jsme tedy jen konečně 
    mnoho axiomů z $T$. Definujme:
    
    \myalertmath{
    $$
    T'=\{\alpha\in T\mid \mathrm{T}\alpha\text{ je položka v tablu $\tau$}\}
    $$
    }   

    Tedy $\tau$ je tablo jen z teorie $T'$, máme tablo důkaz $T'\proves\bot$, dle \alert{korektnosti} je $T'$ sporná.\hfill\qedsymbol

\end{frame}


\begin{frame}{Aplikace kompaktnosti}
     
    \begin{center}
        \alert{vlastnost nekonečného objektu~$\mathcal O$}
    
        $\Updownarrow$

        \alert{vlastnost všech konečných podobjektů~$\mathcal O'$}
    \end{center}
    
    \begin{itemize}
        \item vlastnost popíšeme pomocí (nekonečné) teorie $T$
        \item ke každé konečné $T'\subseteq T$ sestrojíme konečný podobjekt $\mathcal O'$
        \item $\mathcal O'$ splňuje danou vlastnost
        \item to nám dává model $T'$
        \item dle Věty o kompaktnosti má i $T$ model
        \item což ukazuje, že i nekonečný objekt $\mathcal O$ splňuje vlastnost
    \end{itemize}
     
    Věta o kompaktnosti má mnoho aplikací (několik z nich uvidíme později), následující příklad chápejte jako `šablonu'.

\end{frame}


\begin{frame}{Aplikace kompaktnosti: příklad}

    \myexample{
        \textbf{Důsledek:} Spočetně nekonečný graf je bipartitní, právě když je každý jeho konečný podgraf bipartitní.
    }

    \textbf{Důkaz:} \alert{$\Rightarrow$} Každý podgraf bipartitního grafu je bipartitní. 
    
    \alert{$\Leftarrow$} $G$ je bipartitní, právě když je obarvitelný 2 barvami. Mějme jazyk $\mathbb P=\{p_v\mid v\in V(G)\}$ (kde $p_v$ je barva $v$) a uvažme teorii
    
    \myalertmath{
    $$  
        T=\{p_u\liff\neg p_v\mid \{u,v\}\in E(G)\}
    $$
    }

    Zřejmě $G$ je bipartitní, právě když $T$ má model. Dle Věty o kompaktnosti stačí ukázat, že každá konečná $T'\subseteq T$ má model.
    
    Buď $G'$ podgraf $G$ indukovaný na vrcholech, o kterých $T'$ mluví:
    $$
    V(G')=\{v\in V(G)\mid p_v\in\Var(T')\}
    $$
    Protože je $T'$ konečná, je $G'$ také konečný, tedy je dle předpokladu 2-obarvitelný. Libovolné 2-obarvení $V(G')$ ale určuje model $T'$.\hfill\qedsymbol

\end{frame}


\section{4.8 Hilbertovský kalkulus}


\begin{frame}{Hilbertovský deduktivní systém}

    \begin{itemize}
        \item jiný, původní dokazovací systém 
        \item používá jen logické spojky $\neg$, $\limplies$
        \item \alert{schémata logických axiomů} ($\varphi,\psi,\chi$ jsou libovolné výroky)
        \begin{enumerate}[(i)]
            \item $\varphi \limplies (\psi \limplies \varphi)$
            \item $(\varphi\limplies (\psi \limplies \chi))\limplies ((\varphi \limplies \psi)\limplies(\varphi \limplies \chi))$
            \item $(\neg \varphi \limplies \neg \psi)\limplies(\psi \limplies \varphi)$
        \end{enumerate}
        \item \alert{odvozovací pravidlo}: tzv. \alert{modus ponens}
                $$\frac{\varphi, \varphi \limplies \psi}{\psi}$$       
        \item \alert{hilbertovský důkaz} výroku $\varphi$ z teorie $T$ je \alert{konečná} posloupnost výroků $\varphi_0, \dots, \varphi_n=\varphi$, ve které pro každé $i\le n$:
        \begin{itemize}
        \item $\varphi_i$ je \alert{logický axiom}, nebo
        \item $\varphi_i$ je \alert{axiom teorie} ($\varphi_i \in T$), nebo
        \item $\varphi_i$ lze odvodit z předchozích pomocí \alert{odvozovacího pravidla}
        \end{itemize}
        \item existuje-li hilbertovský důkaz, píšeme: \alert{$T\proves_H\varphi$}
    \end{itemize}

\end{frame}


\begin{frame}{Příklad hilbertovského důkazu}

    Ukažme, že pro teorii $T=\{\neg\varphi\}$ a pro libovolný výrok $\psi$ platí:  
    \myexamplemath{  
    $$
    T\proves_H\varphi\limplies\psi
    $$
    }

    Hilbertovským důkazem je následující posloupnost výroků:
    
    \begin{enumerate}\it
        \item $\neg\varphi$\hfill axiom teorie
        \item $\neg \varphi \limplies (\neg \psi \limplies \neg \varphi)$\hfill logický axiom (i)
        \item $\neg \psi \limplies \neg \varphi$\hfill modus ponens na 1. a 2.
        \item $(\neg \psi \limplies \neg \varphi)\limplies(\varphi \limplies \psi)$ \hfill logický axiom (iii)
        \item $\varphi \limplies \psi$ \hfill modus ponens na 3. a 4.
    \end{enumerate}    

\end{frame}


\begin{frame}{Korektnost a úplnost}

    \myblock{
    \textbf{Věta (o korektnosti hilbertovského kalkulu):}
    $T\proves_H\varphi \Rightarrow T\models\varphi$
    }

    \medskip

    \textbf{Důkaz:} Indukcí dle délky důkazu ukážeme, že každý výrok $\varphi_i$ z důkazu (tedy i $\varphi_n=\varphi$) platí v $T$.
    \begin{itemize}
        \item Je-li $\varphi_i$ logický axiom, $T \models \varphi_i$ platí protože logické axiomy jsou tautologie.
        \item Je-li $\varphi_i\in T$, jistě platí $T \models \varphi_i$.
        \item Získáme-li $\varphi_i$ pomocí modus ponens z $\varphi_j$ a $\varphi_k=\varphi_j\limplies\varphi_i$ (pro nějaká $j,k<i$), víme z indukčního předpokladu, že platí $T \models \varphi_j$ a $T \models \varphi_j\limplies\varphi_i$. Potom ale platí i $T \models \varphi_i$. (Modus ponens je \alert{korektní} odvozovací pravidlo)\hfill\qedsymbol
    \end{itemize}

    \myblock{
    \textbf{Věta (o úplnosti hilbertovského kalkulu):}
    $T\models\varphi\ \Rightarrow\ T\proves_H\varphi$
    }

    Důkaz vynecháme.
    
\end{frame}


\section{\sc Kapitola 5: Rezoluční metoda}


\begin{frame}{Rezoluční metoda}
    
    \begin{itemize}
        \item jiný důkazový systém než tablo metoda
        \item mnohem efektivnější implementace
        \item logické programování, automatické dokazování, SAT solvery (důkaz jako \alert{certifikát} nesplnitelnosti)
        \item pracuje s CNF (každý výrok/teorii lze převést do CNF)
        \item jediné inferenční pravidlo: \alert{rezoluční pravidlo}
        $$
        \frac{\{p\}\cup C_1,\{\neg p\}\cup C_2}{C_1\cup C_2}
        $$
        \item platí obecnější \alert{pravidlo řezu}:
        $$
        \frac{\varphi\lor\psi,\neg\varphi\lor\chi}{\psi\lor\chi}
        $$
    \end{itemize}

\end{frame}


\section{5.1 Množinová reprezentace}


\begin{frame}{Množinová reprezentace}

    \begin{itemize}
        \item \alert{literál} $\ell$ je $p$ nebo $\neg p$ (pro $p\in\mathbb P$), $\bar \ell$ je \alert{opačný literál} k $\ell$
        \item \alert{klauzule} $C$ je konečná množina literálů
        \item \alert{prázdná klauzule} $\square$ je nesplnitelná
        \item \alert{CNF formule} $S$ je množina klauzulí (může být i \alert{nekonečná}!)
        \item \alert{prázdná formule} $\emptyset$ je vždy splněna
    \end{itemize}
    Modely reprezentujeme jako množiny literálů:
    \begin{itemize}
        \item \alert{(částečné) ohodnocení} je libovolná \alert{konzistentní} množina literálů (tj. nesmí obsahovat dvojici opačných literálů) 
        \item\alert{úplné ohodnocení} obsahuje $p$ nebo $\neg p$ pro každý prvovýrok
        \item ohodnocení $\mathcal V$ \alert{splňuje} formuli $S$, píšeme \alert{$\mathcal V\models S$}, pokud $\mathcal V$ obsahuje nějaký literál z každé klauzule v $S$:
        $$
        \mathcal V\cap C\neq\emptyset\text{ pro každou }C\in S
        $$
    \end{itemize}

\end{frame}


\begin{frame}{Množinová reprezentace: příklad}
    
    $\varphi=(\neg p_1\lor p_2)\land(\neg p_1\lor\neg p_2\lor p_3)\land(\neg p_3\lor\neg p_4)\land(\neg p_4\lor \neg p_5)\land p_4$ 
        
    \begin{itemize}
        \item v množinové reprezentaci:
        \myexampleamsmath{
        $$
        S=\{\{\neg p_1,p_2\},\{\neg p_1,\neg p_2,p_3\},\{\neg p_3,\neg p_4\},\{\neg p_4,\neg p_5\},\{p_4\}\}
        $$
        }
        \item ohodnocení \myexampleinline{
            $\mathcal V=\{\neg p_1,\neg p_3,p_4,\neg p_5\}$
        } 
        splňuje $S$, \alert{$\mathcal V\models S$}
        \item není úplné, můžeme rozšířit libovolným literálem pro $p_2$, platí 
        \begin{itemize}
            \item \alert{$\mathcal V\union\{p_2\}\models S$}
            \item \alert{$\mathcal V\union\{\neg p_2\}\models S$}
        \end{itemize}        
        \item tato dvě úplná ohodnocení odpovídají modelům
         \begin{itemize}
            \item $(0,1,0,1,0)$
            \item $(0,0,0,1,0)$
         \end{itemize}
    \end{itemize}

\end{frame}


\section{5.2 Rezoluční důkaz}


\begin{frame}{Rezoluční pravidlo}    
    
    \myblock{
    Mějme klauzule $C_1$ a $C_2$ a literál $\ell$ takový, že $\ell\in C_1$ a $\bar\ell\in C_2$. Potom \alert{rezolventa} klauzulí $C_1$ a $C_2$ \alert{přes literál} $\ell$ je klauzule:
    $$
    C=(C_1\setminus\{\ell\})\union (C_2\setminus\{\bar\ell\})
    $$
    \vspace{-12pt}
    }
    
    tedy z první klauzule odstraníme $\ell$ a z druhé $\bar\ell$ (musely tam být!) a zbylé literály sjednotíme, mohli bychom také psát: 
    $$
    C_1'\union C_2'\text{ je rezolventou klauzulí }C_1'\disjointunion\{\ell\}\text{ a }C_2'\disjointunion\{\bar \ell\}
    $$
    \vspace{-12pt}
    \begin{itemize}
        \item \myexampleinline{
            z klauzulí $C_1=\{\neg q,r\}$ a $C_2=\{\neg p,\neg q,\neg r\}$
        }
        odvodíme klauzuli $\{\neg p,\neg q\}$ přes literál $r$
        \item \myexampleinline{
            z $\{p,q\}$ a $\{\neg p,\neg q\}$
        }
        odvodíme $\{p,\neg p\}$ přes literál $q$, nebo $\{q,\neg q\}$ přes literál $p$ (obojí jsou ale tautologie)
        \item nelze z nich ale odvodit $\square$ \emph{``rezolucí přes $p$ a $q$ najednou''}! ($S=\{\{p,q\},\{\neg p,\neg q\}\}$ je splnitelná, např. $(1,0)$ je model)
    \end{itemize}

\end{frame}


\begin{frame}{Rezoluční důkaz}
    
    Rezoluční pravidlo je \alert{korektní}, tj. pro libovolné ohodnocení $\mathcal V$ platí:
    \myalertmath{
    $$
    \text{Pokud }\mathcal V\models C_1\text{ a }\mathcal V\models C_2\text{, potom }\mathcal V\models C.
    $$
    }    
    
    V rezolučním důkazu můžeme vždy napsat buď axiom, nebo rezolventu již napsaných klauzulí; tím zaručíme korektnost důkazů:

    \medskip

    \myblock{
    \alert{Rezoluční důkaz (odvození)} klauzule $C$ z formule $S$ je konečná posloupnost klauzulí $C_0,C_1,\dots,C_n=C$ taková, že pro každé $i$:
    \begin{itemize}
        \item $C_i\in S$, nebo
        \item $C_i$ je rezolventou nějakých $C_j,C_k$ kde $j,k<i$
    \end{itemize}
    }
    \begin{itemize}
        \item existuje-li rez. důkaz, je $C$ \alert{rezolucí dokazatelná} z $S$, \alert{$S\proves_R C$}
        \item \alert{rezoluční zamítnutí} formule $S$ je rezoluční důkaz $\square$ z $S$
        \item v tom případě je $S$ \alert{rezolucí zamítnutelná}
   \end{itemize}   
   
\end{frame}


\begin{frame}{Příklad}

    Formule \myexampleinline{
    $S=\{\{p,\neg q,r\},\{p,\neg r\},\{\neg p,r\},\{\neg p,\neg r\},\{q,r\}\}$
    } je rezolucí zamítnutelná, jedno z možných zamítnutí je:
    $$
    \{p,\neg q,r\},\{q,r\},\{p,r\},\{\neg p,r\},\{r\},\{p,\neg r\},\{\neg p,\neg r\},\{\neg r\},\square
    $$

    Rezoluční důkaz má přirozeně stromovou strukturu, tzv. \alert{rezoluční strom}: na listech jsou axiomy, vnitřní vrcholy jsou rezoluční kroky.

    \begin{center}
        \begin{forest}
        for tree={grow=north}
        [$ \square $
            [$ \{\neg r\} $
                [{$ \{\neg p, \neg r\} $}]
                [{$ \{p, \neg r\} $}]
            ]
            [$ \{r\} $
                [{$ \{\neg p, r\} $}]
                [{$ \{p,r\} $}
                    [{$ \{q, r\} $}]
                    [{$ \{p,\neg q, r\} $}]
                ]
            ]
        ]
        \end{forest}
    \end{center}

\end{frame}


\begin{frame}{Rezoluční strom}

    \myblock{
        \alert{Rezoluční strom} klauzule $C$ z formule $S$ je konečný binární strom s vrcholy označenými klauzulemi, kde
        \begin{itemize}
            \item v kořeni je $C$,
            \item v listech jsou klauzule z $S$,
            \item v každém vnitřním vrcholu je rezolventa klauzulí ze synů tohoto vrcholu.
        \end{itemize}    
    }

    \medskip

    \textbf{Pozorování:} $C$ má rezoluční strom z $S$, právě když $S\proves_R C$. (Důkaz snadno indukcí dle hloubky stromu a délky důkazu.)
    \begin{itemize}
        \item rezolučnímu důkazu odpovídá \alert{jednoznačný} rezoluční strom
        \item z rezolučního stromu můžeme získat více důkazů (jsou dané libovolnou procházkou po vrcholech, která navštíví vnitřní vrchol až poté, co navštívila oba jeho syny)
    \end{itemize}

\end{frame}


\begin{frame}{Rezoluční uzávěr}
    
    jaké všechny klauzule se můžeme rezolucí \alert{`naučit'} z dané formule? (není praktické je všechny najít, jde o užitečný teoretický pohled)
    
    \medskip

    \myblock{
    \alert{Rezoluční uzávěr} $\mathcal R(S)$ formule $S$ je definován induktivně jako nejmenší množina klauzulí splňující:
    \begin{itemize}
        \item $C\in\mathcal R(S)$ pro všechna $C\in S$,
        \item jsou-li $C_1,C_2\in\mathcal R(S)$ a $C$ jejich rezolventa, potom i $C\in\mathcal R(S)$
    \end{itemize}
    }
    
    \bigskip

    Pro \myexampleinline{
        $S=\{\{p,\neg q,r\},\{p,\neg r\},\{\neg p,r\},\{\neg p,\neg r\},\{q,r\}\}$
    } máme:
    \begin{align*}
        \mathcal R(S)=\{&\textcolor{blue}{\{p,\neg q,r\},\{p,\neg r\},\{\neg p,r\},\{p,s\},\{q,r\}},\\&\{p,\neg q\},\{\neg q,r\},\{r,\neg r\},\{p,\neg p\},\{r,s\},\\&
        \{p,r\},\{p,q\},\{r\},\{p\}\}
    \end{align*}

\end{frame}


\section{5.3 Korektnost a úplnost rezoluční metody}


\begin{frame}{Korektnost rezoluce}
    
    Korektnost dokážeme snadno indukcí podle délky důkazu (nebo alternativně indukcí dle hloubky rezolučního stromu).

    \myblock{
    \textbf{Věta (O korektnosti rezoluce):} 
    Je-li CNF formule $S$ rezolucí zamítnutelná, potom je $S$ nesplnitelná.
    }

    \textbf{Důkaz:} Nechť $S\proves_R\square$, a vezměme nějaký rezoluční důkaz $C_0,C_1,\dots,C_n=\square$. \alert{Sporem:} nechť existuje ohodnocení $\mathcal V\models S$. Indukcí podle $i$ dokážeme, že \alert{$\mathcal V\models C_i$}. Potom i $\mathcal V\models \square$, což je spor. 
    
    Pro $i=0$ to platí, neboť $C_0\in S$. Pro $i>0$ máme dva případy:
    \begin{itemize}
        \item \alert{$C_i\in S$:} v tom případě $\mathcal V\models C_i$ plyne z předpokladu, že $\mathcal V\models S$,
        \item \alert{$C_i$ je rezolventou $C_j,C_k$, kde $j,k<i$:} z indukčního předpokladu víme $\mathcal V\models C_j$ a $\mathcal V\models C_k$, $\mathcal V\models C_i$ plyne z korektnosti rezolučního pravidla\hfill\qedsymbol
    \end{itemize}

\end{frame}


\begin{frame}{Dosazení}

    %K důkazu úplnosti budeme potřebovat:
    
    Je-li $S$ CNF formule a $\ell$ literál, potom \alert{dosazení} $\ell$ do $S$ je formule
    \myalertmath{
        $$
        S^\ell=\{C\setminus\{\bar\ell\}\mid \ell\notin C\in S\}
        $$
    }
        
    \begin{itemize}
        \item $S^\ell$ je výsledkem \alert{jednotkové propagace} aplikované na $S\union\{\{\ell\}\}$.
        \item $S^\ell$ neobsahuje v žádné klauzuli literál $\ell$ ani $\bar\ell$ (vůbec tedy neobsahuje prvovýrok z $\ell$)
        \item Pokud $S$ neobsahovala literál $\ell$ ani $\bar\ell$, potom $S^\ell=S$.
        \item Pokud $S$ obsahovala jednotkovou klauzuli $\{\bar\ell\}$, potom $\square\in S^\ell$, tedy $S^\ell$ je sporná.
    \end{itemize} 

\end{frame}


\begin{frame}{Větvení}
    
    \myblock{
    \textbf{Lemma:} 
    $S$ je splnitelná, právě když je splnitelná $S^\ell$ nebo $S^{\bar\ell}$.
    }

    \textbf{Důkaz:} \alert{$\Rightarrow$} Ohodnocení $\mathcal V\models S$ nemůže obsahovat $\ell$ i $\bar\ell$; BÚNO \alert{$\bar\ell\notin\mathcal V$}. Ukážeme, že potom $\mathcal V\models S^\ell$. 
    
    Vezměme libovolnou klauzuli v $S^\ell$. Ta je tvaru $C\setminus\{\bar\ell\}$ pro klauzuli $C\in S$ (neobsahující literál $\ell$). Víme, že $\mathcal V\models C$, protože ale $\mathcal V$ neobsahuje $\bar\ell$, muselo ohodnocení $\mathcal V$ splnit nějaký jiný literál v $C$, takže platí i $\mathcal V\models C\setminus\{\bar\ell\}$.

    \alert{$\Leftarrow$} BÚNO mějme ohodnocení \alert{$\mathcal V\models S^\ell$}. Protože se $\bar\ell$ (ani $\ell$) nevyskytuje v $S^\ell$, platí také \alert{$\mathcal V\setminus\{\bar\ell\}\models S^\ell$}. Ohodnocení \alert{$\mathcal V'=(\mathcal V\setminus\{\bar\ell\})\union\{\ell\}$} potom splňuje všechny $C\in S$, tedy $\mathcal V'\models S$: 
    \begin{itemize}
        \item pokud $\ell\in C$, potom $\ell\in C\cap\mathcal V'$ a $C\cap\mathcal V'\neq\emptyset$
        \item jinak $C\cap\mathcal V'=C\cap\mathcal (\mathcal V\setminus\{\bar\ell\})=(C\setminus\{\bar\ell\})\cap(\mathcal V\setminus\{\bar\ell\})\neq\emptyset$ neboť $\mathcal V\setminus\{\bar\ell\}\models C\setminus\{\bar\ell\}\in S^\ell$\hfill\qedsymbol
    \end{itemize}

\end{frame}


\begin{frame}{Strom dosazení}
    
    Zda je \emph{konečná} formule $S$ splnitelná můžeme zjišťovat rekurzivně, dosazením obou literálů pro některý prvovýrok $p$, a rozvětvením na $S^p,S^\bar p$ (jako v DPLL). Výslednému stromu říkáme \alert{strom dosazení}.
 
    Např. pro \myexampleinline{
        $S=\{\{p\},\{\neg q\},\{\neg p,\neg q\}\}$
    }:
    \vspace{-6pt}
    \begin{center}
        \begin{forest}    
        [{$S$}
            [{$S^p=\{\{\neg q\}\}$}
                [{$S^{pq}=\{\square\}$}, tikz={\node[fit to=tree,label=below:\textcolor{red}
                {$\otimes$}] {};]}]
                [{$S^{p\bar q}=\emptyset$}, tikz={\node[fit to=tree,label=below:\textcolor{blue}{\small $\mathcal V=\{p,\bar q\}$}] {};]}]
            ]
            [{$S^{\bar p}=\{\square,\{\neg q\}\}$}, tikz={\node[fit to=tree,label=below:\textcolor{red}{$\otimes$}] {};]}]
        ]
        \end{forest}
    \end{center}
    \vspace{-12pt}
    % ! beamer+forest conflict: compile with care

    \begin{itemize}
        \item jakmile větev obsahuje $\square$, je nesplnitelná a nepokračujeme v ní
        \item listy jsou buď nesplnitelné, nebo prázdné teorie: v tom případě z posloupnosti dosazení získáme splňující ohodnocení.    
    \end{itemize}

\end{frame}



\begin{frame}{Strom dosazení a nesplnitelnost}
    
    \myblock{
    \textbf{Důsledek:} 
    CNF formule $S$ (ve spočetném jazyce, \alert{může být i nekonečná}) je nesplnitelná, právě když každá větev stromu dosazení obsahuje $\square$.
    }

    \textbf{Důkaz:} 
        Pro \alert{konečnou} $S$ snadno dokážeme indukcí dle $|\Var(S)|$: 
        \begin{itemize}
            \item Je-li $|\Var(S)|=0$, máme $S=\emptyset$ nebo $S=\{\square\}$, v obou případech je strom dosazení jednoprvkový a tvrzení platí. 
            \item V indukčním kroku vybereme libovolný literál $\ell\in\Var(S)$ a aplikujeme Lemma.
        \end{itemize} 
    Je-li $S$ \alert{nekonečná a splnitelná}, má splňující ohodnocení, to se `shoduje' s odpovídající (nekonečnou) větví ve stromu dosazení. 
    
    Je-li \alert{nekonečná a nesplnitelná}, dle Věty o kompaktnosti existuje konečná $S'\subseteq S$, která je také nesplnitelná. Po dosazení pro všechny proměnné z $\Var(S')$ bude v každé větvi $\square$, to nastane po konečně mnoha krocích.
    \hfill\qedsymbol

\end{frame}


\begin{frame}{Úplnost rezoluce}

    \myblock{
    \textbf{Věta (O úplnosti rezoluce):} 
    Je-li CNF formule $S$ nesplnitelná, je rezolucí zamítnutelná (tj. $S\proves_R\square$).
    }

    \textbf{Důkaz:} \alert{Je-li $S$ nekonečná}, má z kompaktnosti konečnou nesplnitelnou část, její rezoluční zamítnutí je také zamítnutí $S$. 
    
    \alert{Je-li $S$ konečná}, ukážeme indukcí dle počtu proměnných: Je-li $|\Var(S)|=0$, jediná možná nesplnitelná formule bez proměnných je $S=\{\square\}$, a máme jednokrokový důkaz $S\proves_R\square$. 
    
    Jinak vyberme $p\in\Var(S)$. Podle Lemmatu jsou $S^p$ i $S^{\bar p}$ nesplnitelné. Mají o proměnnou méně, tedy dle ind. předpokladu existují rezoluční stromy $T$ pro $S^p\proves_R\square$ a $T'$ pro $S^{\bar p}\proves_R\square$.

    Ukážeme, jak z $T$ vyrobit rezoluční strom $\widehat T$ pro $S\proves_R \neg p$. Analogicky $\widehat{T'}$ pro $S\proves_R p$ a potom už snadno vyrobíme rezoluční strom pro $S\proves_R\square$: ke kořeni $\square$ připojíme kořeny stromů $\widehat T$ a $\widehat{T'}$ jako levého a pravého syna (tj. získáme $\square$ rezolucí z $\{\neg p\}$ a $\{p\}$).

\end{frame}


\begin{frame}{Dokončení důkazu}

    Rezoluční strom \alert{$T$ pro $S^p\proves_R\square$} \scalebox{1.5}{$\rightsquigarrow$} \alert{$\widehat T$ pro $S\proves_R \neg p$}: 
    
    Vrcholy i uspořádání jsou stejné, jen do některých klauzulí ve vrcholech \alert{přidáme literál $\neg p$}. 
    
    Na každém listu stromu $T$ je nějaká klauzule $C\in S^p$, a
    \begin{itemize}
        \item buď $C\in S$,
        \item nebo $C\notin S$, ale $C\cup\{\neg p\}\in S$
    \end{itemize}
    
    V prvním případě necháme label stejný. Ve druhém případě přidáme do $C$ \alert{a do všech klauzulí nad tímto listem} literál $\neg p$. 
    
    Listy jsou nyní klauzule z $S$, a každý vnitřní vrchol je nadále rezolventou svých synů. V kořeni jsme $\square$ změnili na $\neg p$ (ledaže každý list $T$ už byl klauzule z $S$, to ale už $T$ dává $S\proves_R\square$). \hfill\qedsymbol

\end{frame}


\section{5.4 LI-rezoluce a Horn-SAT}


\begin{frame}{Lineární důkaz: neformálně}

    Rezoluční důkaz můžeme kromě rezolučního stromu \alert{zorganizovat i jinak}, jako tzv. \alert{lineární důkaz}:
    
    \begin{center}
        \begin{forest}
            for tree={math content,grow=west,text height=2ex, text depth=1ex}
            [C_{n+1}
                [,phantom]
                [C_n
                    [,phantom]
                    [\cdots\cdots\cdots
                        [C_2
                            [,phantom]
                            [C_1
                                [,phantom]
                                [C_0]
                                [B_0]
                            ]
                            [B_1]
                        ]
                    ]
                    [B_{n-1}]                    
                ]
                [B_n]
            ]
        \end{forest}  
    \end{center}
    
    \vspace{-6pt}

    \begin{itemize}
        \item v každém kroku máme jednu \alert{centrální} klauzuli
        \item tu rezolvujeme s \alert{boční} (`side') klauzulí
        \item boční klauzule je buď axiom z $S$, nebo některá z předchozích centrálních (jako bychom odvozené klauzule přidávali k axiomům)
        \item výsledná \alert{rezolventa je novou centrální klauzulí}
    \end{itemize}

    (Tento pohled lépe odpovídá procedurálnímu výpočtu, jde jen o to, jak vybírat vhodné boční klauzule.)

\end{frame}


\begin{frame}{Lineární důkaz: formálně}

    \begin{center}
        \begin{forest}
            for tree={math content,grow=west,text height=2ex, text depth=1ex}
            [C_{n+1}
                [,phantom]
                [C_n
                    [,phantom]
                    [\cdots\cdots\cdots
                        [C_2
                            [,phantom]
                            [C_1
                                [,phantom]
                                [C_0]
                                [B_0]
                            ]
                            [B_1]
                        ]
                    ]
                    [B_{n-1}]                    
                ]
                [B_n]
            ]
        \end{forest}  
    \end{center}

    \myblock{
    \alert{Lineární důkaz} klauzule $C$ z formule $S$ je konečná posloupnost
    $$
    \begin{bmatrix}
        C_0 \\
        B_0
    \end{bmatrix},
    \begin{bmatrix}
        C_1 \\
        B_1
    \end{bmatrix},\dots,
    \begin{bmatrix}
        C_n \\
        B_n
    \end{bmatrix},
    C_{n+1}
    $$
    kde $C_i$ říkáme \alert{centrální} klauzule, $C_0$ je \alert{počáteční}, $C_{n+1}=C$ je \alert{koncová}, $B_i$ jsou \alert{boční} klauzule, a platí:
    \begin{itemize}
        \item $C_0\in S$, pro $i\leq n$ je $C_{i+1}$ rezolventou $C_i$ a $B_i$,
        \item $B_0\in S$, pro $i\leq n$ je $B_i\in S$ nebo $B_i=C_j$ pro nějaké $j<i$. 
    \end{itemize}
    \alert{Lineární zamítnutí} $S$ je lineární důkaz $\square$ z $S$. 
    }    

\end{frame}


\begin{frame}{Příklad a ekvivalence s rezolučním důkazem}

    Lineární zamítnutí \myexampleinline{
        $S = \{\{p, q\},\{p, \neg q\}, \{\neg p, q\}, \{\neg p, \neg q\}\}$
    }:
    \begin{center}
        \begin{forest}
            for tree={math content,grow=west,text height=2ex, text depth=1ex, l sep=3em}
            [{\square}
                [,phantom]
                [{\{\neg p\}}
                    [,phantom]
                    [{\{q\}}
                        [,phantom]
                        [{\textcolor{blue}{\{p\}}}
                            [,phantom]
                            [{\{p,q\}}]
                            [{\{p,\neg q\}}]
                        ]
                        [{\{\neg p,q\}}]
                    ]
                    [{\{\neg p,\neg q\}}]                    
                ]
                [{\textcolor{red}{\{p\}}}]
            ]
        \end{forest}  
    \end{center}
    Poslední boční klauzule $\textcolor{red}{\{p\}}$ není z $S$, ale je rovna předchozí centrální klauzuli.

    \textbf{Poznámka:} $C$ má lineární důkaz z $S$, právě když $S\proves_R C$.

    \alert{$\Rightarrow$} Z lineárního důkazu snadno vyrobíme rezoluční strom. Indukcí dle délky důkazu: máme-li boční klauzuli $B_i\notin S$, potom $B_i=C_j$ pro nějaké $j<i$: místo $B_i$ připojíme rezoluční strom pro $C_j$ z $S$. 
    
    \alert{$\Leftarrow$} Plyne z úplnosti lineární rezoluce, důkaz najdete v učebnici.
        
\end{frame}


\begin{frame}{LI-rezoluce}
    
    \begin{itemize}
        \item \alert{lineární důkaz:} boční klauzule je \alert{axiom nebo dřívější centrální}
        \item co když požadujeme, aby boční klauzule byly \alert{pouze axiomy?}
        
        $\Rightarrow$ \alert{LI-rezoluce (linear-input)}        
            
    \end{itemize}
    
    \myblock{
    \alert{LI-důkaz} (rezolucí) klauzule $C$ z formule $S$ je lineární důkaz 
    $$
    \begin{bmatrix}
        C_0 \\
        B_0
    \end{bmatrix},
    \begin{bmatrix}
        C_1 \\
        B_1
    \end{bmatrix},\dots,
    \begin{bmatrix}
        C_n \\
        B_n
    \end{bmatrix},
    C
    $$
    ve kterém je každá boční klauzule $B_i$ axiom z $S$. Pokud LI-důkaz existuje, říkáme, že je $C$ \alert{LI-dokazatelná} z $S$, a píšeme \alert{$S\proves_{LI}C$}. Pokud $S\proves_{LI}\square$, je $S$ \alert{LI-zamítnutelná}.
    }

    \begin{itemize}
        \item LI-důkaz odpovídá rezolučnímu stromu tvaru \emph{``chlupaté cesty''}
        \item z toho plyne korektnost
        \item ztrácíme úplnost, ale hledání důkazů je snazší
        \item ukážeme \alert{úplnost pro Hornovy formule}, je základem Prologu
    \end{itemize}

\end{frame}


\begin{frame}{Hornovy formule}
        
    \begin{itemize}
        \item \alert{Hornova klauzule} má nejvýše jeden pozitivní literál
        \item \alert{Hornova formule} je množina Hornových klauzulí (i nekonečná)
        \item \alert{Fakt} je pozitivní jednotková klauzule, např. $\{p\}$
        \item \alert{Pravidlo} je klauzule s právě jedním pozitivním a alespoň jedním negativním literálem
        \item Pravidlům a faktům říkáme \alert{programové klauzule}
        \item \alert{Cíl} je neprázdná klauzule bez pozitivního literálu       
        \item dokazujeme sporem: \alert{cíl} je negací toho, co chceme dokázat (konjunkce faktů)
    \end{itemize}

    \myblock{
        \textbf{Pozorování:}
        Je-li Hornova formule $S$ nesplnitelná a $\square\notin S$, potom obsahuje fakt i cíl.
    }

    \textbf{Důkaz:} Neobsahuje-li fakt, ohodnotíme všechny proměnné~0; neobsahuje-li cíl, ohodnotíme 1.\hfill\qedsymbol

\end{frame}


\begin{frame}{Příklad konstrukce LI-zamítnutí}
    
    Ukážeme: \myexampleinline{
        $T=\{\{p,\neg r,\neg s\},\{\neg q,r\},\{q,\neg s\},\{s\}\}\ \models\ p\land q$
    } Sestrojíme LI-zamítnutí $T\cup\{G\}\proves_{LI}\square$ pro cíl $G=\{\neg p,\neg q\}$. 
    
    
    V $T$ \alert{najdeme fakt}, a provedeme \alert{jednotkovou propagaci} v $T\cup\{G\}$. Opakujeme, dokud není formule prázdná:
    \begin{itemize}
        \item $T=\{\{p,\neg r,\neg s\},\{\neg q,r\},\{q,\neg s\},\{s\}\}$, $G=\{\neg p,\neg q\}$
        \item $T^s=\{\{p,\neg r\},\{\neg q,r\},\{q\}\}$, $G^s=\{\neg p,\neg q\}$
        \item $T^{sq}=\{\{p,\neg r\},\{r\}\}$, $G^{sq}=\{\neg p\}$
        \item $T^{sqr}=\{\{p\}\}$, $G^{sqr}=\{\neg p\}$
        \item $T^{sqrp}=\emptyset$, $G^{sqrp}=\square$
    \end{itemize}

    To, že vždy najdeme fakt, plyne z Pozorování pro $T\cup\{G\}$.

    Nyní zpětným postupem sestrojíme LI-zamítnutí, podobně jako v důkazu úplnosti rezoluce.
    
\end{frame}


\begin{frame}{Konstrukce zamítnutí zpětným postupem}

    \begin{itemize}
        \item $T^{sqrp},G^{sqrp}\proves_{LI}\square$:
        \begin{center}
            \begin{forest}
                for tree={math content,grow=west,text height=2ex, text depth=1ex, l sep=3em}
                            [{\square}]                       ]
            \end{forest} 
        \end{center}
        \item $T^{sqr},G^{sqr}\proves_{LI}\square$:
        \begin{center}
            \begin{forest}
                for tree={math content,grow=west,text height=2ex, text depth=1ex, l sep=3em}
                            [{\square}
                                [,phantom]
                                [\{\alert{\neg p}\}]
                                [{\{p\}}]                        
                            ]
            \end{forest} 
        \end{center}
        
        \item $T^{sq},G^{sq}\proves_{LI}\square$:
        \begin{center}
            \begin{forest}
                for tree={math content,grow=west,text height=2ex, text depth=1ex, l sep=3em}
                        [{\square}
                            [,phantom]
                            [{\{\alert{\neg r}\}}
                                [,phantom]
                                [{\{\neg p\}}]
                                [{\{p,\alert{\neg r}\}}]                        
                            ]
                            [{\{r\}}]
                        ]
            \end{forest} 
        \end{center}
        
    \end{itemize}

\end{frame}


\begin{frame}{Konstrukce zamítnutí zpětným postupem -- pokračování}
    
    \begin{itemize}
        \item $T^{s},G^{s}\proves_{LI}\square$:
        
        \bigskip

        \begin{center}
            \begin{forest}
                for tree={math content,grow=west,text height=2ex, text depth=1ex, l sep=2em}
                    [{\square}
                        [,phantom]
                        [{\{\alert{\neg q}\}}
                            [,phantom]
                            [{\{\alert{\neg q},\neg r\}}
                                [,phantom]
                                [{\{\neg p,\alert{\neg q}\}}]
                                [{\{p,\neg r\}}]                        
                            ]
                            [{\{\alert{\neg q},r\}}]
                        ]
                        [{\{q\}}]                    
                    ]
            \end{forest} 
        \end{center}
        
        \item $T,G\proves_{LI}\square$
        
        \bigskip
            \begin{forest}
                for tree={math content,grow=west,text height=2ex, text depth=1ex, l sep=2em}
                [{\square}
                    [,phantom]
                    [{\{\alert{\neg s}\}}
                        [,phantom]
                        [{\{\neg q,\alert{\neg s}\}}
                            [,phantom]
                            [{\{\neg q,\neg r,\alert{\neg s}\}}
                                [,phantom]
                                [{\{\neg p,\neg q\}}]
                                [{\{p,\neg r,\alert{\neg s}\}}]                        
                            ]
                            [{\{\neg q,r\}}]
                        ]
                        [{\{q,\alert{\neg s}\}}]                    
                    ]
                    [{\{s\}}]
                ]
            \end{forest}
        
    \end{itemize}

\end{frame}


\begin{frame}{Úplnost pro Hornovy formule}
 
    \myblock{
    \textbf{Věta (O úplnosti LI-rezoluce pro Hornovy formule):} 
    Je-li Hornova formule $T$ splnitelná, a $T\cup\{G\}$ je nesplnitelná pro cíl $G$, potom $T\cup\{G\}\proves_{LI}\square$, a to LI-zamítnutím, které začíná cílem $G$.
    }

    \textbf{Důkaz:} Opět lze díky Větě o kompaktnosti předpokládat, že $T$ je konečná. LI-zamítnutí sestrojíme indukcí podle počtu proměnných.

    Z Pozorování víme, že $T$ obsahuje fakt $\{p\}$. Protože $T\cup\{G\}$ je nesplnitelná, je dle Lemmatu o dosazení \alert{nesplnitelná i} $(T\cup\{G\})^p$ $=$ \alert{$T^p\cup\{G^p\}$}, kde $G^p=G\setminus\{\neg p\}$.
    
    Všimněte si, že \alert{$T^p$ je splnitelná}. (Stejným ohodnocením jako $T$, neboť to musí obsahovat $p$ kvůli faktu $\{p\}$, tedy neobsahuje $\neg p$.) 
    
    Zároveň má $T^p$ méně proměnných než $T$. \alert{Je-li $G^p$ cíl}, využijeme indukčního předpokladu (následující slide). Co když ale $G^p$ není cíl?       
    
\end{frame}


\begin{frame}{Dokončení důkazu}

    \alert{Není-li $G^p$ cíl}, nutně $G^p=\square$ a $G=\{\neg p\}$. Potom je $\square$ rezolventou $G$ a faktu $\{p\}\in T$, a máme jednokrokové LI-zamítnutí $T\cup\{G\}$. (To dává i \alert{bázi indukce}.)


    \alert{Je-li $G^p$ cíl}, dle \alert{indukčního předpokladu} existuje LI-odvození $\square$ z $T^p\cup\{G^p\}$ začínající $G^p=G\setminus\{\neg p\}$.
    
    Hledané LI-zamítnutí $T\cup\{G\}$ začínající $G$ zkonstruujeme (podobně jako v důkazu Věty o úplnosti rezoluce):

    \begin{itemize}
        \item Přidáme literál $\neg p$ do všech listů, které už nejsou v $T\cup\{G\}$ (vznikly odebráním $\neg p$), a do všech vrcholů nad nimi.
        \item Tím získáme $T\cup\{G\}\proves_{LI}\neg p$.
        \item Na závěr přidáme boční klauzuli $\{p\}$ a odvodíme $\square$.\hfill\qedsymbol
    \end{itemize}

\end{frame}


%\section{5.5 Rezoluce v Prologu}


\begin{frame}[fragile]{Program v Prologu}
    
    síla Prologu vychází z \alert{unifikace} a rezoluce v predikátové logice, nyní si ale ukážeme příklad \alert{výrokového} programu:

    \begin{itemize}
        \item \alert{program} v Prologu je Hornova formule obsahující pouze \alert{programové klauzule}, tj. \alert{fakta} nebo \alert{pravidla}
        \item \alert{dotaz} je konjunkce faktů, negace dotazu je \alert{cíl}
    \end{itemize}
      
    Např.\myexampleinline{
        program $\{\{p,\neg r,\neg s\},\{\neg q,r\},\{q,\neg s\},\{s\}\}$, dotaz $p\land q$
    } 
    \begin{itemize}
        \item klauzule $\{p,\neg r,\neg s\}$ je ekvivalentní $r\land s\limplies p$, píšeme \texttt{p:-r,s.}
        \item výsledný program a dotaz:
        \begin{verbatim}
            p:-r,s.
            r:-q.
            q:-s.
            s.
            ?-p,q.    
        \end{verbatim}
    \end{itemize}
    
    Například klauzuli $\{p,\neg r,\neg s\}$, která je ekvivalentní $r\land s\limplies p$, zapíšeme v Prologu jako: \texttt{p:-r,s.}
    \begin{verbatim}
        p:-r,s.
        r:-q.
        q:-s.
        s.    
    \end{verbatim}
    A programu položíme dotaz:
    \begin{verbatim}
        ?-p,q.    
    \end{verbatim}

\end{frame}


\begin{frame}{Rezoluce v Prologu}
    
    \myblock{
        \textbf{Důsledek:} Mějme program $P$ a dotaz $Q=p_1\land\dots\land p_n$, a označme $G=\{\neg p_1,\dots,\neg p_n\}$ (tj. $G\sim \neg Q$). Následující podmínky jsou ekvivalentní:
        
        \begin{enumerate}[(i)]
            \item $P\models Q$,
            \item $P\cup\{G\}$ je nesplnitelná,
            \item $P\cup\{G\}\proves_{LI}\square$, a existuje LI-zamítnutí začínající cílem $G$.
        \end{enumerate}
    }
    
    \medskip

    \textbf{Důkaz:}
    \begin{itemize}
        \item \alert{(i)$\Leftrightarrow$(ii)} Věta o důkazu sporem
        \item \alert{(ii)$\Leftrightarrow$(iii)} Věta o úplnosti LI-rezoluce pro Hornovy formule (Program je vždy splnitelný)\hfill\qedsymbol
    \end{itemize}

\end{frame}


\end{document}
