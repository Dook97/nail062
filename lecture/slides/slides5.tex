\documentclass{beamer}

%% slide-specific

\usetheme[progressbar=frametitle]{metropolis}
%\usecolortheme{spruce}
%\metroset{block=fill}

% define Metropolis colors    
\definecolor{mAlert}{HTML}{EB811B}
\definecolor{mExample}{HTML}{14B03D}
\definecolor{mBlock}{HTML}{23373b}

% my blocks
\setlength\fboxsep{0pt}%

\newcommand{\myblock}[1]{\colorbox{mBlock!8}{\begin{minipage}{\linewidth}#1\end{minipage}}}
\newcommand{\myblockmath}[1]{\colorbox{mBlock!8}{\begin{minipage}{\linewidth}\vspace{-6pt}#1\end{minipage}}}
\newcommand{\myblockinline}[1]{\colorbox{mBlock!8}{#1}}
\newcommand{\myexample}[1]{\colorbox{mExample!8}{\begin{minipage}{\linewidth}#1\end{minipage}}}
\newcommand{\myexamplemath}[1]{\colorbox{mExample!8}{\begin{minipage}{\linewidth}\vspace{-6pt}#1\end{minipage}}}
\newcommand{\myexampleinline}[1]{\colorbox{mExample!8}{#1}}
\newcommand{\myalert}[1]{\colorbox{mAlert!8}{\begin{minipage}{\linewidth}#1\end{minipage}}}
\newcommand{\myalertmath}[1]{\colorbox{mAlert!8}{\begin{minipage}{\linewidth}\vspace{-6pt}#1\end{minipage}}}
\newcommand{\myalertinline}[1]{\colorbox{mAlert!8}{#1}}

%% other
\newcommand{\mystructure}[1]{\mathcal{#1}}




%% packages
\usepackage{amsmath,amssymb,amsthm}
\usepackage{booktabs}
\usepackage[czech]{babel}
\usepackage{enumerate}
\usepackage{forest}
\usepackage{multicol}
% \usepackage{tcolorbox}
\usepackage{tikz}
    \usetikzlibrary{arrows.meta}
%\usepackage[unicode]{hyperref}
\usepackage[utf8x]{inputenc}
\usepackage{xfrac}

% %% theorems
% \theoremstyle{plain}
%     \newtheorem{theorem}{Věta}[section]
%     \newtheorem*{theorem-unnumbered}{Věta}
%     \newtheorem{proposition}[theorem]{Tvrzení}
%     \newtheorem{corollary}[theorem]{Důsledek}
%     \newtheorem{lemma}[theorem]{Lemma}
%     \newtheorem{observation}[theorem]{Pozorování}
% \theoremstyle{definition}
%     \newtheorem{definition}[theorem]{Definice}
%     \newtheorem*{algorithm}{Algoritmus}
% \theoremstyle{remark}
%     \newtheorem{remark}[theorem]{Poznámka}
%     \newtheorem{example}[theorem]{Příklad}
%     \newtheorem{exercise}{Cvičení}[chapter]
%     \newtheorem*{solution}{Řešení}

%% macros and definitions
\DeclareMathOperator{\Aut}{Aut}
\DeclareMathOperator{\Conseq}{Csq}
\DeclareMathOperator{\DeLO}{DeLO}
\DeclareMathOperator{\dom}{dom}
\DeclareMathOperator{\Fm}{Fm}
\DeclareMathOperator{\M}{M}
%\DeclareMathOperator{\Proof}{Proof}
\DeclareMathOperator{\rng}{rng}
\DeclareMathOperator{\Term}{Term}
\DeclareMathOperator{\Th}{Th}
\DeclareMathOperator{\Thm}{Thm}
\DeclareMathOperator{\Tree}{Tree}
\DeclareMathOperator{\Var}{Var}
\DeclareMathOperator{\VF}{VF}

\newcommand{\A}{\structure{A}}
\newcommand{\B}{\structure{B}}
\newcommand{\Con}{\mathit{Con}}
\newcommand{\disjointunion}{\mathbin{\dot{\sqcup}}}
\newcommand{\F}{\ensuremath{\mathrm{F}}}
\newcommand{\landsymb}{{\land}}
\newcommand{\lbin}{\mathbin{\square}}
\newcommand{\lbinsymb}{{\lbin}}
\newcommand{\liff}{\mathbin{\leftrightarrow}}
\newcommand{\liffsymb}{{\liff}}
\newcommand{\limplies}{\mathbin{\rightarrow}}
\newcommand{\limpliessymb}{{\limplies}}
\newcommand{\lorsymb}{{\lor}}
\newcommand{\Prf}{\mathit{Prf}}
\newcommand{\proves}{\vdash}
%\newcommand{\structure}[1]{\mathcal{#1}}
\newcommand{\todo}{[TODO]}
\newcommand{\T}{\ensuremath{\mathrm{T}}}
\newcommand{\union}{\mathbin{\cup}}


\title{Pátá přednáška}
\subtitle{NAIL062 Výroková a predikátová logika}
\author{Jakub Bulín (KTIML MFF UK)}
% \institute{KTIML MFF UK}
\date{Zimní semestr 2023}


\begin{document}


\frame{\titlepage}


\begin{frame}{Pátá přednáška}

    \textbf{Program}
        \begin{itemize}
            \item věta o kompaktnosti
            \item hilbertovský kalkulus
            \item rezoluční metoda
            \item korektnost a úplnost rezoluce
            \item LI-rezoluce a Horn-SAT
        \end{itemize}

    \textbf{Materiály}

        \href{https://github.com/jbulin-mff-uk/nail062/raw/main/lecture/lecture-notes/lecture-notes.pdf}{\alert{\textbf{Zápisky z přednášky}}}, Sekce 4.7-4.8 z Kapitoly 4, Kapitola 5

\end{frame}


\section{4.7 Věta o kompaktnosti}


\begin{frame}{Kompaktnost}

    \myblock{
        \textbf{Věta (O kompaktnosti):} Teorie má model, právě když každá její konečná část má model.
    }
    \smallskip

    \textbf{Důkaz:} \alert{$\Rightarrow$ Snadné:} Model $T$ je zjevně modelem každé její části.
    
    \alert{$\Leftarrow$ Nepřímo:} buď $T$ sporná, najdeme spornou konečnou $T'\subseteq T$.

    Z \alert{úplnosti} víme, že $T\proves\bot$, tedy existuje i \alert{konečný} tablo důkaz $\tau$ výroku $\bot$ z $T$. Konstrukce $\tau$ má konečně mnoho kroků, použili jsme tedy jen konečně 
    mnoho axiomů z $T$. Definujme:
    
    \myalertmath{
    $$
    T'=\{\alpha\in T\mid \mathrm{T}\alpha\text{ je položka v tablu $\tau$}\}
    $$
    }   

    Tedy $\tau$ je tablo jen z teorie $T'$, máme tablo důkaz $T'\proves\bot$, dle \alert{korektnosti} je $T'$ sporná.\hfill\qedsymbol

\end{frame}


\begin{frame}{Aplikace kompaktnosti}
     
    \begin{center}
        vlastnost nekonečného objektu~$\mathcal O$
    
        $\Updownarrow$

        vlastnost všech konečných podobjektů~$\mathcal O'$
    \end{center}
    
    \begin{itemize}
        \item vlastnost popíšeme pomocí (nekonečné) teorie $T$
        \item ke každé konečné $T'\subseteq T$ sestrojíme konečný podobjekt $\mathcal O'$
        \item $\mathcal O'$ splňuje danou vlastnost
        \item to nám dává model $T'$
        \item dle Věty o kompaktnosti má i $T$ model
        \item což ukazuje, že i nekonečný objekt $\mathcal O$ splňuje vlastnost
    \end{itemize}
     
    Věta o kompaktnosti má mnoho aplikací (několik z nich uvidíme později), následující příklad chápejte jako `šablonu'.

\end{frame}


\begin{frame}{Aplikace kompaktnosti: příklad}

    \myexample{
        \textbf{Důsledek:} Spočetně nekonečný graf je bipartitní, právě když je každý jeho konečný podgraf bipartitní.
    }

    \textbf{Důkaz:} \alert{$\Rightarrow$} Každý podgraf bipartitního grafu je bipartitní. 
    
    \alert{$\Leftarrow$} $G$ je bipartitní, právě když je obarvitelný 2 barvami. Mějme jazyk $\mathbb P=\{p_v\mid v\in V(G)\}$ (kde $p_v$ je barva $v$) a uvažme teorii
    
    \myalertmath{
    $$  
        T=\{p_u\limplies\neg p_v\mid \{u,v\}\in E(G)\}
    $$
    }

    Zřejmě $G$ je bipartitní, právě když $T$ má model. Dle Věty o kompaktnosti stačí ukázat, že každá konečná $T'\subseteq T$ má model.
    
    Buď $G'$ podgraf $G$ indukovaný na vrcholech, o kterých $T'$ mluví:
    $$
    V(G')=\{v\in V(G)\mid p_v\in\Var(T')\}
    $$
    Protože je $T'$ konečná, je $G'$ také konečný, tedy je dle předpokladu 2-obarvitelný. Libovolné 2-obarvení $V(G')$ ale určuje model $T'$.\hfill\qedsymbol

\end{frame}


\section{4.8 Hilbertovský kalkulus}


\begin{frame}{Hilbertovský deduktivní systém}

    \begin{itemize}
        \item jiný, původní dokazovací systém 
        \item používá jen logické spojky $\neg$, $\limplies$
        \item \alert{schémata logických axiomů} ($\varphi,\psi,\chi$ jsou libovolné výroky)
        \begin{enumerate}[(i)]
            \item $\varphi \limplies (\psi \limplies \varphi)$
            \item $(\varphi\limplies (\psi \limplies \chi))\limplies ((\varphi \limplies \psi)\limplies(\varphi \limplies \chi))$
            \item $(\neg \varphi \limplies \neg \psi)\limplies(\psi \limplies \varphi)$
        \end{enumerate}
        \item \alert{odvozovací pravidlo}: tzv. \emph{modus ponens}
                $$\frac{\varphi, \varphi \limplies \psi}{\psi}$$       
        \item \alert{hilbertovský důkaz} výroku $\varphi$ z teorie $T$ je \emph{konečná} posloupnost výroků $\varphi_0, \dots, \varphi_n=\varphi$, ve které pro každé $i\le n$:
        \begin{itemize}
        \item $\varphi_i$ je \alert{logický axiom}, nebo
        \item $\varphi_i$ je \alert{axiom teorie} ($\varphi_i \in T$), nebo
        \item $\varphi_i$ lze odvodit z předchozích pomocí \alert{odvozovacího pravidla}
        \end{itemize}
        \item existuje-li hilbertovský důkaz, píšeme: \alert{$T\proves_H\varphi$}
    \end{itemize}

\end{frame}


\begin{frame}{Příklad hilbertovského důkazu}

    Ukažme, že pro teorii $T=\{\neg\varphi\}$ a pro libovolný výrok $\psi$ platí:  
    \myexamplemath{  
    $$
    T\proves_H\varphi\limplies\psi
    $$
    }

    Hilbertovským důkazem je následující posloupnost výroků:
    
    \begin{enumerate}\it
        \item $\neg\varphi$\hfill axiom teorie
        \item $\neg \varphi \limplies (\neg \psi \limplies \neg \varphi)$\hfill logický axiom (i)
        \item $\neg \psi \limplies \neg \varphi$\hfill modus ponens na 1. a 2.
        \item $(\neg \psi \limplies \neg \varphi)\limplies(\varphi \limplies \psi)$ \hfill logický axiom (iii)
        \item $\varphi \limplies \psi$ \hfill modus ponens na 3. a 4.
    \end{enumerate}    

\end{frame}


\begin{frame}{Korektnost a úplnost}

    \myblock{
    \textbf{Věta (o korektnosti hilbertovského kalkulu):}
    $T\proves_H\varphi \Rightarrow T\models\varphi$
    }

    \medskip

    \textbf{Důkaz:} Indukcí dle délky důkazu ukážeme, že každý výrok $\varphi_i$ z důkazu (tedy i $\varphi_n=\varphi$) platí v $T$.
    \begin{itemize}
        \item Je-li $\varphi_i$ logický axiom, $T \models \varphi_i$ platí protože logické axiomy jsou tautologie.
        \item Je-li $\varphi_i\in T$, jistě platí $T \models \varphi_i$.
        \item Získáme-li $\varphi_i$ pomocí modus ponens z $\varphi_j$ a $\varphi_k=\varphi_j\limplies\varphi_i$ (pro nějaká $j,k<i$), víme z indukčního předpokladu, že platí $T \models \varphi_j$ a $T \models \varphi_j\limplies\varphi_i$. Potom ale platí i $T \models \varphi_i$. (Modus ponens je \alert{korektní} odvozovací pravidlo)\hfill\qedsymbol
    \end{itemize}

    \myblock{
    \textbf{Věta (o úplnosti hilbertovského kalkulu):}
    $T\models\varphi\ \Rightarrow\ T\proves_H\varphi$
    }

    Důkaz vynecháme.
    
\end{frame}

\end{document}
