
Klíčovou, ale velmi technickou částí důkazu První věty je následující tvrzení, které ponecháme bez důkazu.

\begin{proposition}
Je-li $T$ navíc extenzí Robinsonovy aritmetiky $Q$, potom existuje formule $\Prf_T(x,y)$ v jazyce aritmetiky, která \alert{reprezentuje} relaci $\MyProof_T$, tj. pro každá $n,m\in\mathbb N$ platí:
\begin{itemize}
    \item Je-li $(n,m)\in\MyProof_T$, potom $Q\proves\Prf_T(\underline{n},\underline{m})$,
    \item jinak $Q\proves\neg\Prf_T(\underline{n},\underline{m})$.
\end{itemize} 
\end{proposition}

Formule $\Prf_T(x,y)$ tedy vyjadřuje \alert{``$y$ je důkaz $x$ v $T$''}.\footnote{Přesněji, tablo jehož kódem je $y$ je důkazem sentence jejíž kódem je $x$.} Potom můžeme vyjádřit, že \alert{``$x$ je dokazatelná v $T$''}, a to formulí $(\exists y)\Prf_T(x,y)$. Všimněte si, že platí následující pozorování, neboť svědek poskytuje kód nějakého tablo důkazu, a $\underline{\mathbb N}$ splňuje axiomy $Q$:
\begin{observation}\label{observation:proof-predicate}
$T\proves\varphi$ právě když $\underline{\mathbb N}\models (\exists y)\Prf_T(\underline{\varphi},y)$.  
\end{observation}
Budeme potřebovat i následující důsledek, který vyslovíme také bez důkazu:
\begin{corollary}[O predikátu dokazatelnosti]\label{corollary:predicate-of-provability}
    Je-li $T\proves\varphi$, potom $T\proves (\exists y)\Prf_T(\underline{\varphi},y)$.
\end{corollary}

Umíme tedy vyjádřit, že daná sentence je, nebo není, dokazatelná. Jak ale může sentence říci `sama o sobě', že není dokazatelná? K tomu použijeme \alert{princip self-reference}.

\subsubsection*{Self-reference}

Abychom ilustrovali princip self-reference, pro názornost si místo logické sentence představme větu v češtině, a místo vlastnosti ``být dokazatelný'' tvrzení o počtu písmen. Podívejme se na následující větu:
\begin{quote}
    \texttt{Tato věta má 22 znaků.}
\end{quote}
V přirozeném jazyce snadno vyjádříme self-referenci zájmenem ``Tato'', z kontextu víme, že myslíme větu samou. Ve formálních systémech ale typicky nemáme self-referenci přímo k dispozici. \alert{Přímou referenci} obvykle máme k dispozici, stačí umět `mluvit' o posloupnostech symbolů, jako v následujícím příkladě:
\begin{quote}
    \texttt{Následující věta má 29 znaků. "Následující věta má 29 znaků."}
\end{quote}
Zde se ale není žádná self-reference. Pomůžeme si trikem, kterému budeme říkat `zdvojení':
\begin{quote}
    \texttt{Následující věta zapsaná jednou a ještě jednou v uvozovkách má 149\\ znaků. "Následující věta zapsaná jednou a ještě jednou v uvozovkách\\ má 149 znaků."}
\end{quote}
Pomocí přímé reference a zdvojení tedy můžeme získat self-referenci.\begin{remark}
    Stejný princip lze použít k sestrojení programu v C, jehož výstupem je jeho vlastní kód (34 je ASCII kód uvozovek):    
{\small
\begin{verbatim}
main(){char *c="main(){char *c=%c%s%c; printf(c,34,c,34);}"; printf(c,34,c,34);}    
\end{verbatim}
}
\end{remark}


\subsection{Důkaz a důsledky}

V této podsekci dokážeme První Gödelovu větu o neúplnosti a řekneme si i něco o jejích důsledcích. Budeme potřebovat následující větu, která popisuje, jak technicky využijeme princip self-reference. Lze na ní nahlížet jako na formu `diagonalizačního argumentu',\footnote{Diagonalizací se myslí argument připomínající \alert{Cantorův diagonální argument}, známý z důkazu nespočetnosti $\mathbb R$. Podobný argument, používající self-referenci, potkáme třeba v \alert{Holičově paradoxu}, nebo v důkazu nerozhodnutelnosti \alert{Halting problému}.} proto se tomuto tvrzení také někdy říká \alert{diagonální lemma}.

\begin{theorem}[Věta o pevném bodě]
Je-li $T$ extenzí Robinsonovy aritmetiky, potom pro každou formuli $\varphi(x)$ (v jazyce teorie $T$) existuje sentence $\psi$ taková, že platí: 
$$
T\proves \psi \liff \varphi(\underline{\psi})
$$
\end{theorem}
Sentence $\psi$ je tedy \alert{self-referenční}, říká o sobě: ``splňuji vlastnost $\varphi$''.\footnote{Přesněji, říká to o numerálu odpovídajícímu jejímu kódu.} Vysvětlíme si jen myšlenku důkazu. Všimněte si, jak se v důkazu použije přímá reference a zdvojení.
\begin{proof} Uvažme \alert{zdvojující funkci}, funkci $d\colon\mathbb N\to\mathbb N$ takovou, že pro každou formuli $\chi(x)$ platí:
$$
d(\lceil \chi(x)\rceil)=\lceil\chi(\underline{\chi(x)})\rceil
$$
Funkce $d$ tedy dostane na vstupu přirozené číslo $n$, které dekóduje jako formuli v jedné proměnné, dosadí do této formule numerál $\underline{n}$,\footnote{Zde \alert{numerál} odpovídá `uvozovkám' z předchozího neformálního popisu self-reference, a $d(\lceil\chi\rceil)$ znamená ``$\chi$ napsaná jednou a ještě jednou v uvozovkách.''} a výslednou sentenci znovu zakóduje.

S využitím předpokladu, že $T$ je extenzí $Q$, lze dokázat, že tato funkce je v $T$ \alert{reprezentovatelná}. Pro jednoduchost předpokládejme, že je reprezentovatelná termem,\footnote{Byť ve skutečnosti je reprezentovaná (složitou) formulí.} a označme ho také $d$. To znamená, že pro každou formuli $\chi(x)$ platí:
$$
T\proves d(\underline{\chi(x)})=\underline{\chi(\underline{\chi(x)})}
$$
Tedy Robinsonova aritmetika dokazuje \alert{o numerálech}, že $d$ opravdu `zdvojuje'.

Hledaná self-referenční sentence $\psi$ je sentence:\footnote{Následující věta zapsaná jednou a ještě jednou v uvozovkách má vlastnost $\varphi$. ``Následující věta zapsaná jednou a ještě jednou v uvozovkách má vlastnost $\varphi$.''}
$$
\varphi(d(\underline{\varphi(d(x))}))
$$
Chceme dokázat, že platí $T\proves \psi \liff \varphi(\underline{\psi})$, neboli $T \proves \varphi(d(\underline{\varphi(d(x))}))\liff\varphi(\underline{\varphi(d(\underline{\varphi(d(x))}))})$. K~tomu stačí ověřit, že:
$$
T \proves d(\underline{\varphi(d(x))})=\underline{\varphi(d(\underline{\varphi(d(x))}))}
$$
To ale víme z reprezentovatelnosti $d$, kde za formuli $\chi(x)$ dosadíme $\varphi(d(x))$.
\end{proof}

Než přistoupíme k samotnému důkazu Gödelovy věty, ukážeme si jako rozcvičku jeden důsledek Věty o pevném bodě: \alert{Definicí pravdy} v aritmetické teorii $T$ myslíme formuli $\tau(x)$ takovou, že pro každou sentenci $\psi$ platí: 
$$
T\proves\psi\liff\tau(\underline{\psi})
$$
Pokud by definice pravdy existovala, znamenalo by to, že místo dokazování sentence stačí spočíst její kód, substituovat příslušný numerál do $\tau$, a vyhodnotit.

\begin{theorem}[Nedefinovatelnost pravdy]
    V žádném bezesporném rozšíření Robinsonovy aritmetiky neexistuje definice pravdy.
\end{theorem}
Důkaz využívá \alert{Paradox lháře}, vyjádříme větu ``Nejsem pravdivá v $T$''.
\begin{proof}
Předpokládejme pro spor, že existuje definice pravdy $\tau(x)$.
Použijeme Větu o pevném bodě, kde za formuli $\varphi(x)$ vezmeme $\neg\tau(x)$. Dostáváme existenci sentence $\psi$ takové, že:
$$
T\proves\psi\liff\neg\tau(\underline{\psi})
$$
Protože $\tau(x)$ je definice pravdy, platí ale i $T\proves\psi\liff\tau(\underline{\psi})$, tedy i $T\proves\tau(\underline{\psi})\liff\neg\tau(\underline{\psi})$. To by ale znamenalo, že $T$ je sporná.
\end{proof}

Důkaz Gödelovy věty používá tentýž trik, ale pro větu ``Nejsem dokazatelná v $T$''.

\begin{proof}[Důkaz První věty o neúplnosti]
Mějme bezespornou rekurzivně axiomatizovanou extenzi $T$ Robinsonovy aritmetiky. Chceme najít Gödelovu sentenci $\psi_T$, která je pravdivá v $\underline{\mathbb N}$, ale není dokazatelná v $T$.

Takovou sentenci získáme z Věty o pevném bodě jako sentenci vyjadřující ``Nejsem dokazatelná v $T$''. Nechť $\varphi(x)$ je formule $\neg(\exists y)\Prf_T(x,y)$ (``$x$ není dokazatelná v $T$''). Podle Věty o pevném bodě existuje sentence $\psi_T$ splňující:
$$
T\proves\psi_T\liff\neg(\exists y)\Prf_T(\underline{\psi_T},y)
$$
Sentence $\psi_T$ je tedy v $T$ ekvivalentní sentenci, která vyjadřuje, že $\psi_T$ není dokazatelná v $T$. Lze ukázat, že stejná ekvivalence platí i v $\underline{\mathbb N}$ (neboť tak jsme $\Prf_T$ a $\psi_T$ zkonstruovali):
$$
\underline{\mathbb N}\models\psi_T\ \text{ právě když }\ \underline{\mathbb N}\models\neg(\exists y)\Prf_T(\underline{\psi_T},y)
$$
Z Pozorování \ref{observation:proof-predicate} získáváme, že 
$$
\underline{\mathbb N}\models\psi_T\ \text{ právě když }\ T\not\proves\psi_T
$$
neboli $\psi_T$ je pravdivá v $\underline{\mathbb N}$, právě když není dokazatelná v $T$. Stačí tedy ukázat, že $\psi_T$ není dokazatelná v $T$. Předpokládejme pro spor, že $T\proves\psi_T$. Ze self-reference víme, že platí
$T\proves\neg(\exists y)\Prf_T(\underline{\psi_T},y)$.
Z Důsledku \ref{corollary:predicate-of-provability} o predikátu dokazatelnosti ale dostáváme $T\proves (\exists y)\Prf_T(\underline{\psi_T},y)$, což by znamenalo, že $T$ je sporná.
\end{proof}


Na závěr si ukážeme dva důsledky a jedno zesílení. Následující okamžitý důsledek už jsme zmínili dříve:

\begin{corollary}
Je-li $T$ rekurzivně axiomatizovaná extenze Robinsonovy aritmetiky a je-li navíc $\underline{\mathbb N}$ modelem teorie $T$, potom $T$ není kompletní.
\end{corollary}
\begin{proof}
Sporná teorie není kompletní. Pokud by $T$ byla bezesporná, splňovala by předpoklady První věty o neúplnosti, tedy by v ní nebyla dokazatelná Gödelova sentence $\psi_T$. Pokud by byla kompletní, musela by dokazovat $\neg\psi_T$. To by ale znamenalo, že platí i $\underline{\mathbb N}\models\neg\psi_T$, přičemž víme, že $\psi_T$ je v $\underline{\mathbb N}$ pravdivá.  
\end{proof}

Z toho plyne, že nelze rekurzivně axiomatizovat standardní model přirozených čísel:
\begin{corollary}
Teorie $\Th(\underline{\mathbb N})$ není rekurzivně axiomatizovatelná.    
\end{corollary}
\begin{proof}
Teorie $\Th(\underline{\mathbb N})$ je extenzí Robinsonovy aritmetiky a platí v modelu $\underline{\mathbb N}$. Pokud by byla rekurzivně axiomatizovaná, podle předchozího důsledku by nemohla být kompletní, což ale je.
\end{proof}

Jedním ze zesílení Gödelovy První věty je následující tvrzení, které uvedeme bez důkazu. Ukazuje, že předpoklad $\underline{\mathbb N}\models T$ v prvním důsledku výše je ve skutečnosti nadbytečný.

\begin{theorem}[Rosserův trik, 1936]
V každé bezesporné rekurzivně axiomatizované extenzi Robinsonovy aritmetiky existuje nezávislá sentence. Tedy taková teorie není kompletní.    
\end{theorem}


\subsection{Druhá věta o neúplnosti}

Druhá Gödelova věta o neúplnosti říká, neformálně řečeno, že efektivně daná, dostatečně bohatá teorie nemůže sama dokázat svou bezespornost. Bezespornost (``konzistenci'') vyjádříme následující sentencí, kterou označíme jako $\Con_T$:
$$
\neg(\exists y)\Prf_T(\underline{0=S(0)},y)
$$
Všimněte si, že platí $\underline{\mathbb N}\models\Con_T$, právě když $T\not\proves 0=S(0)$. Neboli sentence $\Con_T$ opravdu vyjadřuje, že \alert{``Teorie $T$ je bezesporná''.}

\begin{theorem}[Druhá věta o neúplnosti]
Pro každou bezespornou rekurzivně axiomatizovanou extenzi $T$ Peanovy aritmetiky platí, že $\Con_T$ není dokazatelná v $T$.    
\end{theorem}

Všimněte si, že sentence $\Con_T$ je přitom pravdivá v $\underline{\mathbb N}$ (neboť $T$ je opravdu bezesporná). Zmiňme také, že není třeba plná síla Peanovy aritmetiky, stačí slabší předpoklad. Nyní si ukážeme hlavní myšlenku důkazu Druhé věty:

\begin{proof}[Důkaz Druhé věty o neúplnosti]
Vezměme Gödelovu sentenci $\psi_T$ vyjadřující ``nejsem dokazatelná v $T$''. V důkazu První věty o neúplnosti (konkrétně v první části) jsme ukázali, že:
\begin{quote}
    ``Pokud je $T$ bezesporná, potom $\psi_T$ není dokazatelná v $T$.''
\end{quote}
Z toho jednak plyne, že $T\not\proves\psi_T$, neboť $T$ bezesporná je. Na druhou stranu to lze formulovat jako ``platí $\Con_T\to \psi_T$'' a je-li $T$ extenze Peanovy aritmetiky, lze důkaz tohoto tvrzení zformalizovat v rámci teorie $T$, tedy ukázat, že:
$$
T\proves\Con_T\to\psi_T
$$
Kdyby platilo $T\proves\Con_T$, dostali bychom i $T\proves\psi_T$, což by byl spor.
\end{proof}

Na závěr si ukážeme tři důsledky Druhé věty.

\begin{corollary}
    Existuje model {\it PA}, ve kterém platí sentence $(\exists y)\Prf_{\text{\it PA}}(\underline{0=S(0)},y)$.
\end{corollary}
\begin{proof}
    Sentence $\Con_{\text{\it PA}}$ není dokazatelná, tedy ani pravdivá v $\text{\it PA}$. Platí ale v $\underline{\mathbb N}$ (neboť $\text{\it PA}$ je bezesporná), což znamená, že je $\Con_{\text{\it PA}}$ nezávislá v $\text{\it PA}$. V nějakém modelu tedy musí platit její negace, která je ekvivalentní $(\exists y)\Prf_{\text{\it PA}}(\underline{0=S(0)},y)$.
        
\end{proof}
Uvědomme si, že musí jít o nestandardní model $\text{\it PA}$, svědkem musí být    
\alert{nestandardní} prvek (tj. takový, který není hodnotou žádného numerálu).

\begin{corollary}
    Existuje bezesporná rekurzivně axiomatizovaná extenze $T$    Peanovy aritmetiky, která `dokazuje svou spornost', tj. taková, že $T\proves \neg \Con_T$.
\end{corollary}
\begin{proof}
Uvažme teorii $T=\text{\it PA} \cup \{\neg \Con_{\text{\it PA}}\}$. Tato teorie je bezesporná, neboť $\text{\it PA}\not\proves\Con_{\text{\it PA}}$. Také triviálně platí $T\proves\neg\Con_{\text{\it PA}}$ (tj. $T$ `dokazuje spornost' teorie $\text{\it PA}$). Protože je $\text{\it PA}\subseteq T$, platí i $T\proves\neg\Con_T$.
\end{proof}
Zde si uvědomme, že $\underline{\mathbb{N}}$ nemůže být modelem teorie $T$.

Nakonec se podívejme na teorii ZFC, tj. Zermelovu–Fraenkelovu teorii množin s axiomem výběru, na které je založena formalizace matematiky. Tato teorie není formálně vzato extenzí $\text{\it PA}$, ale není problém v ní Peanovu aritmetiku (v jistém smyslu) `interpretovat'. To znamená, že ani tato teorie neumí dokázat svou vlastní bezespornost.

\begin{corollary}
    Je-li teorie množin ZFC bezesporná, nemůže být sentence $\Con_{ZFC}$ v teorii ZFC dokazatelná.
\end{corollary}

Pokud by tedy někdo v rámci teorie ZFC dokázal, že je ZFC bezesporná, znamenalo by to, že je ZFC sporná. Což bude taková pěkná tečka za naší přednáškou.
