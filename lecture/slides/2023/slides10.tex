\documentclass{beamer}

%% slide-specific

\usetheme[progressbar=frametitle]{metropolis}
%\usecolortheme{spruce}
%\metroset{block=fill}

% define Metropolis colors    
\definecolor{mAlert}{HTML}{EB811B}
\definecolor{mExample}{HTML}{14B03D}
\definecolor{mBlock}{HTML}{23373b}

% my blocks
\setlength\fboxsep{0pt}%

\newcommand{\myblock}[1]{\colorbox{mBlock!8}{\begin{minipage}{\linewidth}#1\end{minipage}}}
\newcommand{\myblockmath}[1]{\colorbox{mBlock!8}{\begin{minipage}{\linewidth}\vspace{-6pt}#1\end{minipage}}}
\newcommand{\myblockinline}[1]{\colorbox{mBlock!8}{#1}}
\newcommand{\myexample}[1]{\colorbox{mExample!8}{\begin{minipage}{\linewidth}#1\end{minipage}}}
\newcommand{\myexamplemath}[1]{\colorbox{mExample!8}{\begin{minipage}{\linewidth}\vspace{-6pt}#1\end{minipage}}}
\newcommand{\myexampleinline}[1]{\colorbox{mExample!8}{#1}}
\newcommand{\myalert}[1]{\colorbox{mAlert!8}{\begin{minipage}{\linewidth}#1\end{minipage}}}
\newcommand{\myalertmath}[1]{\colorbox{mAlert!8}{\begin{minipage}{\linewidth}\vspace{-6pt}#1\end{minipage}}}
\newcommand{\myalertinline}[1]{\colorbox{mAlert!8}{#1}}

%% other
\newcommand{\mystructure}[1]{\mathcal{#1}}




%% packages
\usepackage{amsmath,amssymb,amsthm}
\usepackage{booktabs}
\usepackage[czech]{babel}
\usepackage{enumerate}
\usepackage{forest}
\usepackage{multicol}
% \usepackage{tcolorbox}
\usepackage{tikz}
    \usetikzlibrary{arrows.meta}
%\usepackage[unicode]{hyperref}
\usepackage[utf8x]{inputenc}
\usepackage{xfrac}

% %% theorems
% \theoremstyle{plain}
%     \newtheorem{theorem}{Věta}[section]
%     \newtheorem*{theorem-unnumbered}{Věta}
%     \newtheorem{proposition}[theorem]{Tvrzení}
%     \newtheorem{corollary}[theorem]{Důsledek}
%     \newtheorem{lemma}[theorem]{Lemma}
%     \newtheorem{observation}[theorem]{Pozorování}
% \theoremstyle{definition}
%     \newtheorem{definition}[theorem]{Definice}
%     \newtheorem*{algorithm}{Algoritmus}
% \theoremstyle{remark}
%     \newtheorem{remark}[theorem]{Poznámka}
%     \newtheorem{example}[theorem]{Příklad}
%     \newtheorem{exercise}{Cvičení}[chapter]
%     \newtheorem*{solution}{Řešení}

%% macros and definitions
\DeclareMathOperator{\Aut}{Aut}
\DeclareMathOperator{\Conseq}{Csq}
\DeclareMathOperator{\DeLO}{DeLO}
\DeclareMathOperator{\dom}{dom}
\DeclareMathOperator{\Fm}{Fm}
\DeclareMathOperator{\M}{M}
%\DeclareMathOperator{\Proof}{Proof}
\DeclareMathOperator{\rng}{rng}
\DeclareMathOperator{\Term}{Term}
\DeclareMathOperator{\Th}{Th}
\DeclareMathOperator{\Thm}{Thm}
\DeclareMathOperator{\Tree}{Tree}
\DeclareMathOperator{\Var}{Var}
\DeclareMathOperator{\VF}{VF}

\newcommand{\A}{\structure{A}}
\newcommand{\B}{\structure{B}}
\newcommand{\Con}{\mathit{Con}}
\newcommand{\disjointunion}{\mathbin{\dot{\sqcup}}}
\newcommand{\F}{\ensuremath{\mathrm{F}}}
\newcommand{\landsymb}{{\land}}
\newcommand{\lbin}{\mathbin{\square}}
\newcommand{\lbinsymb}{{\lbin}}
\newcommand{\liff}{\mathbin{\leftrightarrow}}
\newcommand{\liffsymb}{{\liff}}
\newcommand{\limplies}{\mathbin{\rightarrow}}
\newcommand{\limpliessymb}{{\limplies}}
\newcommand{\lorsymb}{{\lor}}
\newcommand{\Prf}{\mathit{Prf}}
\newcommand{\proves}{\vdash}
%\newcommand{\structure}[1]{\mathcal{#1}}
\newcommand{\todo}{[TODO]}
\newcommand{\T}{\ensuremath{\mathrm{T}}}
\newcommand{\union}{\mathbin{\cup}}


\title{Desátá přednáška}
\subtitle{NAIL062 Výroková a predikátová logika}
\author{Jakub Bulín (KTIML MFF UK)}
% \institute{KTIML MFF UK}
\date{Zimní semestr 2023}


\begin{document}


\frame{\titlepage}


\begin{frame}{Desátá přednáška}

    \textbf{Program}
        \begin{itemize}
            \item grounding, Herbrandova věta
            \item unifikace, unifikační algoritmus
            \item rezoluční pravidlo, rezoluční důkaz
        \end{itemize}    

    \textbf{Materiály}

        \href{https://github.com/jbulin-mff-uk/nail062/raw/main/lecture/lecture-notes/lecture-notes.pdf}{\alert{\textbf{Zápisky z přednášky}}}, Sekce 8.3-8.5 z Kapitoly 8

\end{frame}


\section{8.3 Grounding}


\begin{frame}{Grounding}

    \begin{itemize}
        \item \alert{základní (ground) instance} otevřené $\varphi$ ve volných proměnných $x_1,\dots,x_n$ je $\varphi(x_1/t_1,\dots,x_n/t_n)$, kde vš. $t_i$ jsou konstantní
        
        \bigskip
        
        \myblock{\textbf{Herbrandova věta} říká, že je-li \alert{otevřená} teorie \alert{nesplnitelná}, lze to doložit ``na konkrétních prvcích'': existuje konečně mnoho \alert{základních instancí} axiomů, jejichž konjunkce je nesplnitelná}
        
        \bigskip
        \item např. pro \myexampleinline{
            $T=\{P(x,y)\lor R(x,y),\neg P(c,y),\neg R(x,f(x))\}$
        } substituujeme \alert{konstantní} termy $\{x/c,y/f(c)\}$:
        
        \myalertmath{\vspace{-12pt}
            $$
            (P(c,f(c))\lor R(c,f(c)))\ \land\ \neg P(c,f(c))\ \land\ \neg R(c,f(c))
            $$
        } 

        \item základní atomické sentence chápeme jako prvovýroky: 
        
        \myalertmath{
            $$
            (p_1 \lor p_2) \land \neg p_1 \land \neg p_2
            $$
        }

        \item to už snadno zamítneme výrokovou rezolucí
        \item $p_1$ znamená ``platí $P(c,f(c))$'', $p_2$ znamená ``platí $R(c,f(c))$''        
    \end{itemize}
    
\end{frame}


\begin{frame}{Přímá redukce do výrokové logiky}

    Herbrandova věta + korektnost a úplnost výrokové rezoluce dává následující, neefektivní postup ($S'$ je moc velká, i nekonečná):
    
    \medskip

    \myalert{
    \begin{enumerate}
        \item $S$ $\rightsquigarrow$ $S'$ = množina všech základních instancí klauzulí z $S$
        \item atomické sentence v $S'$ chápeme jako prvovýroky
        \item $S$ nesplnitelná $\Leftrightarrow$ $S'$ zamítnutelná `na úrovni výrokové logiky'
    \end{enumerate}
    }
    
    \medskip

    Např. pro \myexampleinline{
        $S=\{\{P(x,y),R(x,y)\},\{\neg P(c,y)\},\{\neg R(x,f(x))\}\}$
    }
    {\footnotesize
    \begin{align*}
        S'=\{&\{P(c,c),R(c,c)\},\{P(c,f(c)),R(c,f(c))\},\{P(f(c),c),R(f(c),c)\},\dots,\\ 
        &\{\neg P(c,c)\}, \{\neg P(c,f(c))\},\{\neg P(c,f(f(c)))\},\{\neg P(c,f(f(f(c))))\}, \dots,\\
        &\{\neg R(c,f(c))\}, \{\neg R(f(c),f(f(c)))\},\{\neg R(f(f(c)),f(f(f(c))))\},\dots\}    
    \end{align*}
    }
    $S'$ je nesplnitelná obsahuje konečnou nesplnitelnou podmnožinu:
    $$
    \{\{P(c,f(c)),R(c,f(c))\},\{\neg P(c,f(c))\},\{\neg R(c,f(c))\}\}\proves_R\square
    $$

    \myalertinline{\textbf{Efektivnější} je hledat vhodné základní instance \alert{unifikací} [za chvíli]}

\end{frame}


\begin{frame}{Herbrandův model}

    \myblock{
    Mějme jazyk $L=\langle\mathcal R,\mathcal F\rangle$ s alespoň jedním konstantním symbolem. $L$-struktura $\A=\langle A,\mathcal R^\A,\mathcal F^\A\rangle$ je \alert{Herbrandův model}, jestliže:
    \begin{itemize}
        \item $A$ je množina všech konst. $L$-termů (\alert{Herbrandovo univerzum})
        \item pro každý $n$-ární $f\in\mathcal F$ a (konstantní) $\text{``$t_1$''},\dots,\text{``$t_n$''}\in A$:
        \myalertmath{
        $$
        f^\A(\text{``$t_1$''},\dots,\text{``$t_n$''})=\text{``$f(t_1,\dots,t_n)$''}
        $$
        }
        \item speciálně, pro konstantní symbol $c\in\mathcal F$ je $c^\A=\text{``$c$''}$
        \item na relační symboly neklademe podmínky
    \end{itemize}
    }

    \medskip

    Např. \myexampleinline{
        $L=\langle P,f,c\rangle$
    } ($P$ unární rel., $f$ binární funkční, $c$ konstantní)
    \alert{Herbrandův model} je každá struktura $\A=\langle A,P^\A,f^\A,c^\A\rangle$, kde
    \begin{itemize}\small
        \item  $A=\{\text{``$c$''},\text{``$f(c,c)$''},\text{``$f(c,f(c,c))$''},\text{``$f(f(c,c),c)$''}\dots\}$
        \item $c^\A=\text{``$c$''}$
        \item $f^\A(\text{``$c$''},\text{``$c$''})=\text{``$f(c,c)$''}$, $f^\A(\text{``$c$''},\text{``$f(c,c)$''})=\text{``$f(c,f(c,c))$''}$, $f^\A(\text{``$f(c,c)$''},\text{``$c$''})=\text{``$f(f(c,c),c)$''}$, atd.
        \item $P^\A\subseteq A$ může být libovolná
    \end{itemize}

\end{frame}


\begin{frame}{Herbrandova věta}

    \myblock{
    \textbf{Věta (Herbrandova):}
        Je-li $T$ otevřená, v jazyce bez rovnosti a s alespoň jedním konstantním symbolem, potom:
        \begin{itemize}
            \item buď má $T$ \alert{Herbrandův model}, nebo
            \item existuje konečně mnoho základních instancí axiomů $T$, jejichž konjunkce je nesplnitelná.
        \end{itemize}
    }
    
    \textbf{Důkaz:}
        \alert{$T_\text{ground}$} = \myalertinline{množina všech základních instancí axiomů $T$}

        \vspace{-3pt}
        
        Zkonstruujeme \alert{``systematické tablo''} $\tau$ z $T_\text{ground}$ s $\F\bot$ v kořeni, ale z jazyka $L$, bez rozšíření o pomocné konstantní symboly na $L_C$. (Nepotřebujeme je, protože v $T_\text{ground}$ nejsou kvantifikátory.)
        
        Pokud má $\tau$ bezespornou větev, je \alert{``kanonický model''} (opět bez pomocných symbolů) Herbrandovým modelem $T$. 
        
        Jinak je $\tau$ \alert{důkaz sporu}, $T_\text{ground}$ (a tedy i $T$) je nesplnitelná. Tablo $\tau$ je konečné, používá jen konečně mnoho $\alpha_\text{ground}\in T_\text{ground}$, jejich konjunkce už je nesplnitelná.\hfill\qedsymbol
    

\end{frame}


\begin{frame}{Poznámky}

    \begin{itemize}
        \item konstatní symbol potřebujeme, aby existovaly vůbec nějaké konstantní termy (ale není-li v $L$ žádný, můžeme ho přidat)
        \item Herbrandův model je podobný kanonickému, ale nepřidáváme pomocné symboly, a neříkáme nic o relacích
        \item je-li jazyk s rovností, najdeme Herbrandův model pro $T^*$ (přidané axiomy rovnosti) a faktorizujeme podle $=^A$
    \end{itemize}

\end{frame}


\begin{frame}{Důsledky Herbrandovy věty}

    \myblock{
    \textbf{Důsledek:}
    Je-li $T$ otevřená v jazyce s konstantním symbolem, potom $T$ má model, právě když má model teorie $T_\text{ground}$.
    }

    \textbf{Důkaz:}
    \alert{\Large$\Rightarrow$} V modelu $T$ platí i všechny základní instance axiomů. Je tedy i modelem $T_\text{ground}$. 
    
    \alert{\Large$\Leftarrow$} Pokud $T$ nemá model, podle Herbrandovy věty je nějaká konečná podmnožina teorie $T_\text{ground}$ nesplnitelná.\hfill\qedsymbol

    \bigskip

    \myblock{
    \textbf{Důsledek:}
    Mějme otevřenou $\varphi(x_1,\dots,x_n)$ v $L$ s konst. symbolem. Potom existuje $m\in\mathbb N$ a konstantní $L$-termy $t_{ij}$ ($i\in[m],j\in[n]$), že sentence \myalertinline{
        $(\exists x_1)\dots(\exists x_n)\varphi(x_1,\dots,x_n)$
    } je pravdivá, právě když je následující formule (výroková) tautologie:

    \vspace{-12pt}
    $$
    \varphi(x_1/t_{11},\dots,x_n/t_{1n})\lor \dots \lor \varphi(x_1/t_{m1},\dots,x_n/t_{mn})
    $$
    \vspace{-16pt}
    }

    \textbf{Důkaz:}
    Je \alert{pravdivá}, právě když $(\forall x_1)\dots(\forall x_n)\neg\varphi$ neboli $\neg\varphi$ je \alert{nesplnitelná}. Stačí aplikovat Herbrandovu větu na $T=\{\neg\varphi\}$.\hfill\qedsymbol

\end{frame}


\section{8.4 Unifikace}


\begin{frame}{Příklady substitucí}

    Místo \alert{všech základních} použijeme \alert{`vhodné'} substituce (\alert{unifikace}):

    \myexampleinline{
        1. $\{P(x),Q(x,a)\}$ a $\{\neg P(y),\neg Q(b,y)\}$
        } \vspace{-6pt}
        \begin{itemize}
            \item substitucí \myalertinline{
            $\{x/b,y/a\}$
            } získáme $\{P(b),Q(b,a)\}$ a $\{\neg P(a),\neg Q(b,a)\}$, z nich rezolucí $\{P(b),\neg P(a)\}$
            \item nebo \myalertinline{
            $\{x/y\}$
            } a rezolucí přes $P(y)$ máme $\{Q(y,a),\neg Q(b,y)\}$
            \item šlo by např. \myalertinline{
                $\{x/a\}$
            }, získat $\{Q(a,a),\neg Q(b,a)\}$, ale to je \alert{horší}   
        \end{itemize}         
          
    \myexampleinline{
        2. $\{P(x),Q(x,z)\}$ a $\{\neg P(y),\neg Q(f(y),y)\}$
    } \vspace{-6pt}
        \begin{itemize}
            \item lze použít \myalertinline{
                $\{x/f(a),y/a,z/a\}$
                }, získat $\{P(f(a)),Q(f(a),a)\}$ a $\{\neg P(a),\neg Q(f(a),a)\}$, rezolucí \alert{$\{P(f(a)),\neg P(a)\}$}
            \item \alert{lepší} je \myalertinline{
                $\{x/f(z),y/z\}$
            }, dává $\{P(f(z)),Q(f(z),z)\}$ a $\{\neg P(z),\neg Q(f(z),z)\}$, rezolventu \alert{$\{P(f(z)),\neg P(z)\}$}
            \item proč lepší? \alert{obecnější}, rezolventa `říká více': {\small$\{P(f(a)),\neg P(a)\}$} je důsledkem {\small$\{P(f(z)),\neg P(z)\}$}, ale nejsou ekvivalentní
            \item $\{x/f(a),y/a,z/a\}$ získáme \alert{složením} $\{x/f(z),y/z\}$ a $\{z/a\}$
        \end{itemize}            

\end{frame}


\begin{frame}{Substituce formálně}

    \myblock{
    \begin{itemize}
        \item \alert{substituce} je konečná množina \alert{$\sigma=\{x_1/t_1,\dots,x_n/t_n\}$}, kde $x_i$ jsou navzájem různé proměnné, $t_i$ jsou termy, $t_i$ není $x_i$
        \begin{itemize}
            \item \alert{základní}: všechny termy $t_i$ jsou konstantní
            \item \alert{přejmenování proměnných}: vš. $t_i$ navzájem různé proměnné
        \end{itemize}
        \item \alert{výraz} je term nebo literál (atomická formule nebo její negace)
        \item \alert{instance} výrazu $E$ \alert{při substituci} $\sigma=\{x_1/t_1,\dots,x_n/t_n\}$, \alert{$E\sigma$}: simultánně nahradíme  všechny výskyty $x_i$ za termy $t_i$
        \item pro množinu výrazů $S$ je \alert{$S\sigma=\{E\sigma\mid E\in S\}$}
    \end{itemize}
    }

    \bigskip
    
    \begin{itemize}
        \item simultánně proto, aby výskyt $x_i$ v termu $t_j$ nevedl ke zřetězení
        \item např. \myexampleinline{
            $S=\{P(x),R(y,z)\}$, $\sigma=\{x/f(y,z),y/x,z/c\}$
        }
        $$
        S\sigma=\{P(f(y,z)),R(x,c)\}
        $$
    \end{itemize}

\end{frame}


\begin{frame}{Skládání substitucí}
    
    \begin{itemize}
        \item substituce lze \alert{skládat}, \alert{$\sigma\tau$} znamená nejprve $\sigma$ a potom~$\tau$
        \item chceme, aby platilo \alert{$E(\sigma\tau)=(E\sigma)\tau$}, pro libovolný výraz $E$
        \item např. pro výraz \myexampleinline{
            $E=P(x,w,u)$
        } a substituce
        $$
        \sigma=\{x/f(y),w/v\}\qquad\tau=\{x/a,y/g(x),v/w,u/c\}
        $$
        máme $E\sigma=P(f(y),v,u)$ a $(E\sigma)\tau=P(f(g(x)),w,c)$, takže:
        $$
        \sigma\tau=\{x/f(g(x)),y/g(x),v/w,u/c\}
        $$
        \item skládání není komutativní, $\sigma\tau$ je (typicky) jiná než $\tau\sigma$, zde
        $$
        \tau\sigma=\{x/a,y/g(f(y)),u/c,w/v\}
        $$
        \item ale je asociativní (takže nemusíme psát závorky v $\sigma_1\sigma_2\cdots\sigma_n$)
    \end{itemize}

    \myblock{
        Buď $\sigma=\{x_1/t_1,\dots,x_n/t_n\}$ a $\tau=\{y_1/s_1,\dots,y_m/s_m\}$, označme $X=\{x_1,\dots,x_n\}$ a $Y=\{y_1,\dots,y_m\}$. \alert{Složení $\sigma$ a $\tau$} je substituce
        \vspace{-6pt}
        $$
        \sigma\tau=\{x_i/t_i\tau\mid x_i\in X,x_i\neq t_i\tau\}\cup\{y_j/s_j\mid y_j\in Y\setminus X\}
        $$        
    }

\end{frame}


\begin{frame}{Vlastnosti skládání}

    \myblock{
        \textbf{Tvrzení:} Pro libovolné substituce $\sigma$, $\tau$, $\varrho$ a výraz $E$ platí:
        \vspace{-6pt}
        \begin{center}
            (i)  $(E\sigma)\tau=E(\sigma\tau)$ \qquad
            (ii) $(\sigma\tau)\varrho=\sigma(\tau\varrho)$
        \end{center}
        
    }

    \medskip

    \textbf{Důkaz:} \alert{(i)} Buď $\sigma=\{x_1/t_1,\dots,x_n/t_n\}$ a $\tau=\{y_1/s_1,\dots,y_m/s_m\}$. Stačí pro $E$ proměnnou (substituce nemění ostatní symboly):
    \begin{itemize}
        \item pro $E=x_i$ je $E\sigma=t_i$ a $(E\sigma)\tau=t_i\tau=E(\sigma\tau)$
        \item pro $E=y_j\notin X$ je $E\sigma=E$ a $(E\sigma)\tau=E\tau=s_j=E(\sigma\tau)$
        \item je-li $E$ jiná proměnná, potom $(E\sigma)\tau=E=E(\sigma\tau)$.
    \end{itemize}
    \alert{(i)} opakovaným užitím (i) máme pro lib. výraz, tedy i proměnnou:
    $$
    E((\sigma\tau)\varrho)=(E(\sigma\tau))\varrho=((E\sigma)\tau)\varrho=(E\sigma)(\tau\varrho)=E(\sigma(\tau\varrho))
    $$
    Z toho plyne, že $(\sigma\tau)\varrho$ a $\sigma(\tau\varrho)$ jsou touž substitucí. 
    
    (Podrobněji, zřejmě platí: $\pi=\{z_1/v_1,\dots,z_k/v_k\}$ právě když $z_i\pi=v_i$ a $E\pi=E$ je-li $E$ proměnná různá od všech $z_i$.) \hfill\qedsymbol
 

\end{frame}


\begin{frame}{Unifikace}
        
    \myblock{
    \begin{itemize}
        \item \alert{unifikace} pro $S=\{E_1,\dots,E_n\}$ je substituce $\sigma$ taková, že $E_1\sigma=E_2\sigma=\cdots =E_n\sigma$, tj.  $S\sigma$ obsahuje jediný výraz
        \item pokud má $S$ unifikaci, je \alert{unifikovatelná}
        \item unifikace pro $S$ je \alert{nejobecnější}, pokud pro každou unifikaci $\tau$ pro $S$ existuje substituce $\lambda$ taková, že $\tau=\sigma\lambda$        
    \end{itemize}
    }
    
    NB: různé nejobecnějších unifikace pro $S$ se liší jen přejmenováním proměnných
    
    Např. pro \myexampleinline{
        $S=\{P(f(x),y),P(f(a),w)\}$
    } 

    \begin{itemize}
        \item $\sigma=\{x/a,y/w\}$ je nejobecnější unifikace
        \item $\tau=\{x/a,y/b,w/b\}$ je unifikace, ale není nejobecnější, nelze z ní získat např. unifikaci $\varrho=\{x/a, y/c, w/c\}$
        \item z nejobecnější unifikace $\sigma$ získáme $\tau=\sigma\lambda$ pro $\lambda=\{w/b\}$
    \end{itemize}

\end{frame}


\begin{frame}{Unifikační algoritmus}
   
    \vspace{-6pt}
    \begin{itemize}
        \item postupně od začátku výrazů aplikuje substituce %tak, aby se výrazy stávaly více podobnými
        \item buď $p$ nejlevější pozice, na které se nějaké dva výrazy z $S$ liší
        \item \alert{$D(S)$} je množina všech podvýrazů začínajících na pozici $p$
        \item \myexampleinline{\small
            $S=\{P(x,y),P(f(x),z),P(z,f(x))\}, p=3, D(S)=\{x,f(x),z\}$
        }  
    \end{itemize}
    
    \medskip

    \myblock{
    \textbf{vstup}: konečná množina výrazů $S\neq\emptyset$\\
    \textbf{výstup:} nejobecnější unifikace $\sigma$ nebo info, že není unifikovatelná
    
    \begin{enumerate}[(1)]\setcounter{enumi}{-1}
        \item nastav $S_0:=S$, $\sigma_0:=\emptyset$, $k:=0$
        \item pokud $|S_k|=1$, vrať $\sigma=\sigma_0\sigma_1\cdots \sigma_k$
        \item zjisti, zda je v $D(S_k)$ proměnná $x$ a term $t$ \alert{neobsahující} $x$
        \item pokud ano, nastav $\sigma_{k+1}:=\{x/t\}$, $S_{k+1}:=S_k\sigma_{k+1}$, $k:=k+1$, a jdi na (1)
        \item pokud ne, odpověz, že $S$ není unifikovatelná
    \end{enumerate}
    }

    NB: hledání $x$ a $t$ v kroku (2) je relativně výpočetně náročné

\end{frame}


\begin{frame}{Ukázkový běh}
    \vspace{-6pt}
    \myexampleinline{\small
    $S=S_0=\{P(f(y,g(z)),h(b)),\ P(f(h(w),g(a)),t),\ P(f(h(b),g(z)),y)\}$
    }

    \vspace{-1pt}
    
    \alert{($k=0$)} $|S_0|>1$, $D(S_0)=\{y,h(w),h(b)\}$, proměnná $y$ není v $h(w)$, nastavíme $\sigma_1:=\{y/h(w)\}$ a $S_1=S_0\sigma_1$
    
    \myexampleinline{\small
    $S_1=\{P(f(h(w),g(z)),h(b)),\ P(f(h(w),g(a)),t),\ P(f(h(b),g(z)),h(w))\}$
    }

    \vspace{-1pt}

    \alert{($k=1$)} $D(S_1)=\{w,b\}$, $\sigma_2=\{w/b\}$, $S_2=S_1\sigma_2$    
    
    \myexampleinline{\small
    $S_2=\{P(f(h(b),g(z)),h(b)),\ P(f(h(b),g(a)),t)\}$
    }

    \vspace{-1pt}
        
    \alert{($k=2$)} $D(S_2)=\{z,a\}$, $\sigma_3=\{z/a\}$, $S_3=S_2\sigma_3$    
    
    \myexampleinline{\small
    $S_3=\{P(f(h(b),g(a)),h(b)),\ P(f(h(b),g(a)),t)\}$
    }

    \vspace{-1pt}

    \alert{($k=3$)} $D(S_3)=\{h(b),t\}$, $\sigma_4=\{t/h(b)\}$, $S_4=S_3\sigma_4$    
    
    \myexampleinline{\small
    $S_4=\{P(f(h(b),g(a)),h(b))\}$
    }

    \vspace{-1pt}

    \alert{($k=4$)} $|S_4|=1$, nejobecnější unifikace pro $S$ je
    $\sigma=\sigma_1\sigma_2\sigma_3\sigma_4=$\\    
    $\{y/h(w)\}\{w/b\}\{z/a\}\{t/h(b)\}=$\myalertinline{
        $\{y/h(b),w/b,z/a,t/h(b)\}$
    }    

\end{frame}


\begin{frame}{Důkaz korektnosti}
    
    \myblock{
    \textbf{Tvrzení:}
    Unifikační algoritmus je korektní. Pro sestrojenou $\sigma$ navíc platí, že je-li $\tau$ libovolná unifikace, potom $\tau=\sigma\tau$.
    }
    
    \textbf{Důkaz:} Algoritmus vždy skončí, neboť v každém kroku eliminuje proměnnou. Skončí-li neúspěchem, nelze unifikovat $S_k$, tedy ani $S$.
    
    Odpoví-li $\sigma=\sigma_0\sigma_1\cdots\sigma_k$, zjevně jde o unifikaci. Zbývá dokázat, že je nejobecnější, k tomu stačí dokázat vlastnost `navíc': Buď $\tau$ lib. unifikace pro $S$. Indukcí pro $0\leq i\leq k$ ukážeme \alert{$\tau=\sigma_0\sigma_1\cdots\sigma_i\tau$}
    
    \alert{(báze indukce)} Pro $i=0$ je $\sigma_0=\emptyset$, $\tau=\sigma_0\tau$ tedy platí triviálně.

    \alert{(indukční krok)} Buď $\sigma_{i+1}=\{x/t\}$. Ukažme, že pro lib. proměnnou platí: \alert{$u\sigma_{i+1}\tau=u\tau$} Z toho okamžitě plyne i $\tau=\sigma_0\sigma_1\cdots\sigma_i\sigma_{i+1}\tau$.
    
    Pro \alert{$u\neq x$} je $u\sigma_{i+1}=u$, tedy i $u\sigma_{i+1}\tau=u\tau$. Je-li \alert{$u=x$}, máme $u\sigma_{i+1}=x\sigma_{i+1}=t$. Protože $\tau$ unifikuje $S_i=S\sigma_0\sigma_1\cdots\sigma_i$ a $x,t\in D(S_i)$, $\tau$ unifikuje i $x$ a $t$, tzn. $t\tau=x\tau$, tj. $u\sigma_{i+1}\tau=u\tau$.\hfill\qedsymbol

\end{frame}


\section{8.5 Rezoluční metoda}


\begin{frame}{Příklad rezolučního kroku}

    Chceme-li ukázat $T\models\varphi$, skolemizací najdeme CNF formuli $S$ ekvisplnitelnou s $T\cup\{\neg\varphi\}$. Stačí najít rezoluční zamítnutí $S$.

    Jediným podstatným rozdílem bude \alert{rezoluční pravidlo}. 

    Rezolventou dvojice klauzulí bude klauzule, kterou lze odvodit aplikací \alert{(nejobecnější) unifikace}. Nejprve příklad:

    \myexampleinline{
        $C_1=\{P(x),Q(x,y),Q(x,f(z))\}, C_2=\{\neg P(u),\neg Q(f(u),u)\}$
    } 

    Vyberme z $C_1$ \alert{oba} pozitivní literály začínající $Q$, z $C_2$ negativní. 

    $S=\{Q(x,y),Q(x,f(z)),Q(f(u),u)\}$ lze unifikovat pomocí nejobecnější unifikace \myalertinline{
        $\sigma=\{x/f(f(z)),y/f(z),u/f(z)\}$
    }

    \begin{itemize}
        \item $C_1\sigma=\{P(f(f(z))),Q(f(f(z)),f(z))\}$
        \item $C_2\sigma=\{\neg P(f(z)),\neg Q(f(f(z)),f(z))\}$
    \end{itemize}

    z nich odvodíme rezolventu $C=\{P(f(f(z))),\neg P(f(z))\}$
    
\end{frame}


\begin{frame}{Rezoluční pravidlo}

    \myblock{
    Mějme klauzule $C_1$ a $C_2$ s disjunktními množinami proměnných tvaru
    $$
    C_1=C_1'\sqcup \{A_1,\dots,A_n\},\quad C_2=C_2'\sqcup \{\neg B_1,\dots,\neg B_m\}
    $$
    kde $n,m\ge 1$ a $S=\{A_1,\dots,A_n,B_1,\dots,B_m\}$ lze unifikovat. Buď $\sigma$ nejobecnější unifikace $S$. \alert{Rezolventa} $C_1$ a $C_2$ je potom klauzule
    $$
    C=C_1'\sigma \cup C_2'\sigma
    $$
    }

    \medskip

    \begin{itemize}
        \item Disjunktní množ. proměnných získáme přejmenováním. Proč? Z~\myexampleinline{
            $\{\{P(x)\},\{\neg P(f(x))\}\}$
        } odvodíme $\square$, nahradíme-li $\{P(x)\}$ klauzulí $\{P(y)\}$. Ale $S=\{P(x),P(f(x))\}$ není unifikovatelná.
    
        \item Proč potřebujeme z klauzule odstranit více literálů najednou? \myexampleinline{
            $S=\{\{P(x),P(y)\},\{\neg P(x),\neg P(y)\}\}$
        } je zamítnutelná, ale nemá zamítnutí, které by v každém kroku odstranilo jen jeden.  
    \end{itemize}
        
\end{frame}


\begin{frame}{Rezoluční důkaz}

    \myblock{
    \alert{Rezoluční důkaz (odvození)} klauzule $C$ z formule $S$ je konečná posloupnost klauzulí $C_0,C_1,\dots,C_n=C$ taková, že pro každé $i$ je buď
    \vspace{-12pt}
    \begin{itemize}
        \item $C_i=C_i'\sigma$ pro nějakou 
            \myalertinline{
                $C'_i\in S$ a přejmenování proměnných $\sigma$
            } 
        \item nebo $C_i$ je rezolventou nějakých $C_j,C_k$ kde $j<i$ a $k<i$.
    \end{itemize}
    Existuje-li, je $C$ \alert{rezolucí dokazatelná} z $S$, \alert{$S\proves_R C$}. \alert{(Rezoluční) zamítnutí}  $S$ je rez. důkaz $\square$ z $S$, potom je $S$ \alert{(rezolucí) zamítnutelná}.
    }


\end{frame}


\begin{frame}{Příklad}

    \myexampleamsmath{\small
    \smallskip
    \begin{align*}
        S=\{&\{\neg P(x,y),\neg P(y,z), P(x,z)\},\{\neg P(x,x)\},\\
        & \{\neg P(x,y),P(y,x)\},\{P(x,f(x))\}\}
    \end{align*} 
    }

    rezoluční zamítnutí:
    
    \myalertamsmath{\small
    \smallskip
    \begin{align*}
        &\{\neg P(x,y),\neg P(y,z), P(x,z)\},\ 
        \{P(x',f(x'))\},\ 
        \textcolor{blue}{\{\neg P(f(x),z),P(x,z)\}},\\
        &\{\neg P(x,y),P(y,x)\},\ 
        \textcolor{blue}{\{P(f(x'),x')\}},\ 
        \textcolor{blue}{\{P(x,x)\}},\ 
        \{P(x',x')\},\ 
        \textcolor{blue}{\square}  
    \end{align*}
    }

    \medskip

    rezoluční strom:

    \medskip

    \scalebox{0.85}{
    \hspace{-1cm}\begin{forest}
        for tree={l=1.5cm, grow=north}
        [{$ \square $}, label=left:{\footnotesize\textcolor{blue}{$x'/x$}}
            [{$ \{\neg P(x',x')\} $}]
            [{$ \{P(x,x)\} $}, label=left:{\footnotesize\textcolor{blue}{$z/x,x'/x$}}
                [{$ \{P(f(x'),x')\} $}, label=right:{\footnotesize\textcolor{blue}{$x/x',y/f(x')$}}
                    [{$ \{P(x',f(x'))\} $}]
                    [{$ \{\neg P(x,y),P(y,x)\} $}]            
                ]
                [{$ \{\neg P(f(x),z),P(x,z)\} $}, label=left:{\footnotesize\textcolor{blue}{$y/f(x'),x'/x$}}
                    [{$ \{P(x',f(x'))\} $}]
                    [{$ \{\neg P(x,y),\neg P(y,z), P(x,z) \} $}]                
                ]
            ]
        ]
    \end{forest}
    }    

\end{frame}


\end{document}


