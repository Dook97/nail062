\documentclass{beamer}

%% slide-specific

\usetheme[progressbar=frametitle]{metropolis}
%\usecolortheme{spruce}
%\metroset{block=fill}

% define Metropolis colors    
\definecolor{mAlert}{HTML}{EB811B}
\definecolor{mExample}{HTML}{14B03D}
\definecolor{mBlock}{HTML}{23373b}

% my blocks
\setlength\fboxsep{0pt}%

\newcommand{\myblock}[1]{\colorbox{mBlock!8}{\begin{minipage}{\linewidth}#1\end{minipage}}}
\newcommand{\myblockmath}[1]{\colorbox{mBlock!8}{\begin{minipage}{\linewidth}\vspace{-6pt}#1\end{minipage}}}
\newcommand{\myblockinline}[1]{\colorbox{mBlock!8}{#1}}
\newcommand{\myexample}[1]{\colorbox{mExample!8}{\begin{minipage}{\linewidth}#1\end{minipage}}}
\newcommand{\myexamplemath}[1]{\colorbox{mExample!8}{\begin{minipage}{\linewidth}\vspace{-6pt}#1\end{minipage}}}
\newcommand{\myexampleinline}[1]{\colorbox{mExample!8}{#1}}
\newcommand{\myalert}[1]{\colorbox{mAlert!8}{\begin{minipage}{\linewidth}#1\end{minipage}}}
\newcommand{\myalertmath}[1]{\colorbox{mAlert!8}{\begin{minipage}{\linewidth}\vspace{-6pt}#1\end{minipage}}}
\newcommand{\myalertinline}[1]{\colorbox{mAlert!8}{#1}}

%% other
\newcommand{\mystructure}[1]{\mathcal{#1}}




%% packages
\usepackage{amsmath,amssymb,amsthm}
\usepackage{booktabs}
\usepackage[czech]{babel}
\usepackage{enumerate}
\usepackage{forest}
\usepackage{multicol}
% \usepackage{tcolorbox}
\usepackage{tikz}
    \usetikzlibrary{arrows.meta}
%\usepackage[unicode]{hyperref}
\usepackage[utf8x]{inputenc}
\usepackage{xfrac}

% %% theorems
% \theoremstyle{plain}
%     \newtheorem{theorem}{Věta}[section]
%     \newtheorem*{theorem-unnumbered}{Věta}
%     \newtheorem{proposition}[theorem]{Tvrzení}
%     \newtheorem{corollary}[theorem]{Důsledek}
%     \newtheorem{lemma}[theorem]{Lemma}
%     \newtheorem{observation}[theorem]{Pozorování}
% \theoremstyle{definition}
%     \newtheorem{definition}[theorem]{Definice}
%     \newtheorem*{algorithm}{Algoritmus}
% \theoremstyle{remark}
%     \newtheorem{remark}[theorem]{Poznámka}
%     \newtheorem{example}[theorem]{Příklad}
%     \newtheorem{exercise}{Cvičení}[chapter]
%     \newtheorem*{solution}{Řešení}

%% macros and definitions
\DeclareMathOperator{\Aut}{Aut}
\DeclareMathOperator{\Conseq}{Csq}
\DeclareMathOperator{\DeLO}{DeLO}
\DeclareMathOperator{\dom}{dom}
\DeclareMathOperator{\Fm}{Fm}
\DeclareMathOperator{\M}{M}
%\DeclareMathOperator{\Proof}{Proof}
\DeclareMathOperator{\rng}{rng}
\DeclareMathOperator{\Term}{Term}
\DeclareMathOperator{\Th}{Th}
\DeclareMathOperator{\Thm}{Thm}
\DeclareMathOperator{\Tree}{Tree}
\DeclareMathOperator{\Var}{Var}
\DeclareMathOperator{\VF}{VF}

\newcommand{\A}{\structure{A}}
\newcommand{\B}{\structure{B}}
\newcommand{\Con}{\mathit{Con}}
\newcommand{\disjointunion}{\mathbin{\dot{\sqcup}}}
\newcommand{\F}{\ensuremath{\mathrm{F}}}
\newcommand{\landsymb}{{\land}}
\newcommand{\lbin}{\mathbin{\square}}
\newcommand{\lbinsymb}{{\lbin}}
\newcommand{\liff}{\mathbin{\leftrightarrow}}
\newcommand{\liffsymb}{{\liff}}
\newcommand{\limplies}{\mathbin{\rightarrow}}
\newcommand{\limpliessymb}{{\limplies}}
\newcommand{\lorsymb}{{\lor}}
\newcommand{\Prf}{\mathit{Prf}}
\newcommand{\proves}{\vdash}
%\newcommand{\structure}[1]{\mathcal{#1}}
\newcommand{\todo}{[TODO]}
\newcommand{\T}{\ensuremath{\mathrm{T}}}
\newcommand{\union}{\mathbin{\cup}}


\title{Devátá přednáška}
\subtitle{NAIL062 Výroková a predikátová logika}
\author{Jakub Bulín (KTIML MFF UK)}
% \institute{KTIML MFF UK}
\date{Zimní semestr 2023}


\begin{document}


\frame{\titlepage}


\begin{frame}{Devátá přednáška}

    \textbf{Program}
        \begin{itemize}
            \item Löwenheim-Skolemova věta
            \item věta o kompaktnosti
            \item hilbertovský kalkulus.
            \item úvod do rezoluce v predikátové logice
            \item skolemizace
        \end{itemize}

    \textbf{Materiály}

        \href{https://github.com/jbulin-mff-uk/nail062/raw/main/lecture/lecture-notes/lecture-notes.pdf}{\alert{\textbf{Zápisky z přednášky}}}, Sekce 7.5-7.6 z Kapitoly 7, Sekce 8.1-8.2 z Kapitoly 8

\end{frame}


\section{7.5 Důsledky korektnosti a úplnosti}


\begin{frame}{$\proves\ =\ \models$}

    Syntaktickou analogií \alert{důsledků} jsou \alert{teorémy}:
    $$
    \Thm_L(T)=\{\varphi\mid \varphi\text{ je $L$-sentence a } T\proves\varphi\}
    $$
    
    Z korektnosti a úplnosti okamžitě dostáváme:
        \begin{itemize}
            \item $T\proves\varphi$ právě když $T\models\varphi$
            \item $\Thm_L(T)=\Conseq_L(T)$
        \end{itemize}
    
    Všude můžeme nahradit `\alert{platnost}' pojmem `\alert{dokazatelnost}'.  Např:
    \begin{itemize}
        \item $T$ je \alert{sporná}, je-li v ní dokazatelný spor (tj. \alert{$T\proves\bot$})
        \item $T$ je \alert{kompletní}, je-li pro každou sentenci buď $T\proves\varphi$ nebo $T\proves\neg\varphi$, ale ne obojí (jinak by byla sporná)
    \end{itemize}

    \myblock{
        \textbf{Věta (O dedukci):} $T,\varphi\proves\psi$ právě když  $T\proves\varphi\to\psi$.
    }

    \textbf{Důkaz:} Stačí dokázat: $T,\varphi\models\psi\Leftrightarrow T\models\varphi\to\psi$. To je snadné.\hfill\qedsymbol

\end{frame}


\begin{frame}{Löwenheim-Skolemova věta \& Věta o kompaktnosti}
    
    \medskip
    
    \myblock{
    \textbf{Věta (Löwenheim-Skolemova):}
    Je-li $L$ spočetný bez rovnosti, potom každá bezesporná $L$-teorie má spočetně nekonečný model.
    }

    (Později ukážeme i verzi s rovností, kan. model může být konečný.)

    \textbf{Důkaz:} V $T$ není dokazatelný spor. Dokončené tablo z $T$ s $\F\bot$ v kořeni tedy musí obsahovat bezespornou větev. Hledaný model je $L$-redukt kanonického modelu pro tuto větev.\hfill\qedsymbol

    \bigskip

    Věta o kompaktnosti, vč. důkazu, je stejná jako ve výrokové logice:

    \smallskip
    \myblock{
    \textbf{Věta (O kompaktnosti):}
    Teorie má model, právě když každá její konečná část má model.\vspace{2pt}  
    }  
    
    \textbf{Důkaz:} Model teorie je modelem každé části. Naopak, pokud $T$ nemá model, je sporná, tedy $T\proves\bot$. Vezměme nějaký \alert{konečný} tablo důkaz $\bot$ z $T$. K jeho konstrukci stačí konečně mnoho axiomů $T$, ty tvoří konečnou podteorii $T'\subseteq T$, která nemá model.\hfill\qedsymbol

\end{frame}


\begin{frame}{Nestandardní model přirozených čísel}

    \begin{itemize}
        \item $\underline{\mathbb N}=\langle\mathbb N,S,+,\cdot,0,\leq\rangle$ je \alert{standardní model} přirozených čísel
        \item \alert{teorie struktury $\Th(\underline{\mathbb N})$:} všechny sentence \alert{pravdivé} v~$\underline{\mathbb N}$
        \item \alert{$n$-tý numerál:} term $\underline n=S(S(\cdots (S(0)\cdots))$, kde $S$ je $n$-krát
    \end{itemize}
    
    Přidáme nový konstantní symbol $c$ a vyjádříme, že je ostře větší než každý $n$-tý numerál:
    $$
    T=\Th(\underline{\mathbb N})\cup\{\underline n<c\mid n\in \mathbb N\}
    $$
        
    \begin{itemize}
        \item každá konečná část $T$ má model
        \item dle věty o kompaktnosti: i $T$ má model
        \item říkáme mu \alert{nestandardní model} (označme $\A$)
        \item platí v něm tytéž sentence, které platí ve standardním modelu
        \item ale zároveň obsahuje prvek $c^\A$, který je větší než každé $n\in \mathbb N$ (tzn. větší než hodnota termu $\underline n$ v nestandardním modelu $\A$)
    \end{itemize}    

\end{frame}


\section{7.6 Hilbertovský kalkulus v predikátové logice}



\begin{frame}{Hilbertovský kalkulus v predikátové logice}

    \vspace{-6pt}
    \begin{itemize}
        \item používá jen $\neg$ a $\limplies$, dokazuje lib. formule (nejen sentence)
        \item \alert{schémata log. axiomů} ($\varphi,\psi,\chi$ formule, $t$ term, $x$ proměnná)
        \begin{enumerate}[(i)]
            \item $\varphi \limplies (\psi \limplies \varphi)$
            \item $(\varphi\limplies (\psi \limplies \chi))\limplies ((\varphi \limplies \psi)\limplies(\varphi \limplies \chi))$
            \item $(\neg \varphi \limplies \neg \psi)\limplies(\psi \limplies \varphi)$            
        \end{enumerate}

        \medskip

        \myalert{
        \begin{enumerate}[(iv)]
            \item $(\forall x)\varphi \limplies \varphi(x/t)$ \hfill je-li $t$ substituovatelný za $x$ do $\varphi$
            \item $(\forall x)(\varphi \to \psi) \limplies (\varphi \limplies (\forall x)\psi)$ \hfill není-li $x$ volná ve $\varphi$
        \end{enumerate}
        
        a navíc \textbf{axiomy rovnosti}, je-li jazyk s rovností
        }

        \medskip

        \item \alert{odvozovací pravidla}:\vspace{-6pt}
        \begin{center}
            \begin{minipage}{0.45\textwidth}
                $$
                \frac{\varphi, \varphi \limplies \psi}{\psi}\ \text{(modus ponens)}
                $$
            \end{minipage}          
            \begin{minipage}{0.45\textwidth}
                \myalertmath{
                $$
                \frac{\varphi}{(\forall x)\varphi} \ \text{(generalizace)} 
                $$
                }
            \end{minipage}            
        \end{center}        
        
        \item \alert{hilbertovský důkaz} formule $\varphi$ z $T$ je \alert{konečná} posloupnost $\varphi_0, \dots, \varphi_n=\varphi$, kde $\varphi_i$ je \alert{logický axiom} (vč. axiomů rovnosti), \alert{axiom teorie}, nebo lze odvodit z předchozích pomocí pravidel
        \item existuje-li, píšeme \alert{$T\proves_H\varphi$}
    \end{itemize}

\end{frame}


\begin{frame}{Korektnost a úplnost}

    \myblock{
    \textbf{Věta (o korektnosti hilbertovského kalkulu):}
    $T\proves_H\varphi \Rightarrow T\models\varphi$
    }

    \medskip

    \textbf{Důkaz:} Indukcí dle délky důkazu: každá $\varphi_i$ (vč. $\varphi_n=\varphi$) platí v $T$
    \begin{itemize}
        \item logické axiomy (vč. axiomů rovnosti) jsou tautologie, platí v $T$
        \item axiomy z $T$ jistě v $T$ také platí
         \item modus ponens i generalizace jsou \alert{korektní} inferenční pravidla:
        \begin{itemize}
            \item je-li $T\models\varphi$ a $T\models\varphi\to\psi$, potom $T\models\psi$
            \item je-li $T\models\varphi$, potom $T\models(\forall x)\varphi$
            \hfill\qedsymbol
        \end{itemize}
    \end{itemize}

    \bigskip
    
    \myblock{
    \textbf{Věta (o úplnosti hilbertovského kalkulu):}
    $T\models\varphi\ \Rightarrow\ T\proves_H\varphi$
    }

    Důkaz vynecháme.
    
\end{frame}


\section{\sc Kapitola 8: Rezoluce v predikátové logice}


\section{8.1 Úvod}


\begin{frame}{Rezoluce v predikátové logice}

    $T\models\varphi$? {\Large$\rightsquigarrow$} $T\cup\{\neg \varphi\}$ {\Large$\rightsquigarrow$} CNF formule $S$ {\Large$\rightsquigarrow$} rezoluční zamítnutí

    \begin{itemize}
        \item[] (pozor: $\varphi$ musí být \alert{sentence}!)        
        \item \alert{literál} je \alert{atomická formule} $R(t_1,\dots,t_n)$ nebo její negace
        \item \alert{klauzule} je konečná množina literálů, \alert{formule} množina klauzulí
        \item otevřenou formuli snadno převedeme do CNF, i univerzální kvantifikátor na začátku:{\small\myexampleinline{
            $(\forall x)(P(x)\lor \neg Q(x))\sim \{P(x),\neg Q(x)\}$
            }}
        \item co s existenčními kvantifikátory? nové symboly pro `svědky'
        {\small\myexampleinline{
        $(\exists x)(P(x)\lor \neg Q(x))\rightsquigarrow \{P(c),\neg Q(c)\}$
        }} ``\alert{skolemizace}''
        \item není ekvivalentní, ale zachovává \alert{[ne]splnitelnost}, to nám stačí
        \item rezoluční krok? literály nemusí být stejné, stačí \alert{unifikovatelné}
        {\myexampleinline{
        z klauzulí $\{P(x),\neg Q(x)\}$ a $\{Q(f(c))\}$ odvodíme $\{P(f(c))\}$
        }}
        \item \alert{unifikace} je substituce $\{x/f(c)\}$
    \end{itemize}

\end{frame}


\begin{frame}{Příklady}
    
    1. \myexampleinline{
        $T=\{(\forall x)P(x),(\forall x)(P(x)\limplies Q(x))\}$, $\varphi=(\exists x)Q(x)$
    }
    $$\neg\varphi=\neg(\exists x)Q(x)\sim(\forall x)\neg Q(x)\sim\neg Q(x)$$ 
    $T\cup\{\neg \varphi\}$ je \alert{ekvivalentní} \myalertinline{\small
        $S = \{\{P(x)\},\{\neg P(x),Q(x)\},\{\neg Q(x)\}\}$
    }
    rezoluční zamítnutí: představte si $p$ místo $P(x)$, $q$ místo $Q(x)$

    2. \myexampleinline{
        $T=\{(\forall x)(\exists y)R(x,y),R(x,y)\limplies Q(x)\}$, $\varphi=(\exists x)Q(x)$
    }
    $$
    T\cup\{\neg \varphi\}\sim\{(\forall x)(\exists y)R(x,y),\neg R(x,y)\lor Q(x),\neg Q(x)\}
    $$
    formuli \alert{$(\forall x)(\exists y)R(x,y)$} nahradíme \alert{$R(x,f(x))$}, kde $f$ je nový unární funkční symbol (reprezentuje \alert{výběr svědka}):

    \myalertmath{
    $$
    S = \{\{R(x,f(x))\},\{\neg R(x,y),Q(x)\},\{\neg Q(x)\}\}
    $$
    }

    není ekvivalentní, ale \alert{ekvisplnitelná} (zde obě nesplnitelné), vidíme po \alert{substituci $y/f(x)$}, která \alert{unifikuje} $R(x,f(x))$ a $R(x,y)$
 
\end{frame}


\begin{frame}{Unifikace}

    \myexamplemath{
        $$
        S = \{\{R(x,f(x))\},\{\neg R(x,y),Q(x)\},\{\neg Q(x)\}\}
        $$
    }

    \begin{itemize}
        \item na \alert{úrovni výrokové logiky} (\alert{ground level}):
        \myalertmath{ 
        $$
        \{\{r\},\{\neg p,q\},\{\neg q,p\},\{\neg q\}\}
        $$
        }
        není nesplnitelné! musíme využít, že $R(x,f(x))$ a $R(x,y)$ mají `podobnou strukturu' (jsou \alert{unifikovatelné})
        \item klauzule $\{\neg R(x,y),Q(x)\}$ platí i po provedení libovolné substituce: \alert{$\{\neg R(x/t),Q(x/t)\}$} je důsledek $S$ pro lib. term $t$
        \item představme si `přidání' všech takto získaných klauzulí do $S$: potom už je na ground level nesplnitelné (ale nekonečné)
        \item \alert{unifikační algoritmus} nám dá správnou substituci \myalertinline{
            $y/f(x)$
        }
        \item zahrneme už do \alert{rezolučního pravidla}, tedy \alert{rezolventou} klauzulí $\{P(c)\}$ a $\{\neg P(x),Q(x)\}$ bude klauzule $\{Q(c)\}$.
    \end{itemize}
 
\end{frame}

\begin{frame}{Rezoluční pravidlo}
    
    \begin{itemize}

        \item zahrnuje aplikaci unifikace
        \item lze vybrat \alert{více literálů najednou}, ale musí být unifikovatelné:
        
        \bigskip

        např. z \myexampleinline{
        $\{R(x,f(x)),R(g(y),z)\},\{\neg R(g(c),u),P(u)\}$
        } odvodíme rezolventu \myalertinline{
            $\{P(f(g(c)))\}$
        } za použití \alert{unifikace} 

        \medskip

        \myalertmath{
            $$
            \{x/g(c),y/c,z/f(g(c)),u/f(g(c))\}
            $$
        }

        \bigskip

        \item budeme vyžadovat disjunktní množiny proměnných v klauzulích; lze přejmenovat, proměnné mají \alert{lokální význam}:
        $$
        \models(\forall x)(\psi \land \chi) \leftrightarrow (\forall x)\psi \land (\forall x)\chi
        $$
        
    \end{itemize}

\end{frame}


\section{8.2 Skolemizace}


\begin{frame}{Ekvisplnitelná otevřená teorie}

    \begin{itemize}
        \item teorie $T$ v jazyce $L$ a $T'$ v (ne nutně stejném) jazyce $L'$ jsou  \alert{ekvisplnitelné}, pokud platí: $T$ má model $\Leftrightarrow$ $T'$ má model
        \item zajímá nás jen [ne]splnitelnost (dokazujeme sporem)
        \item pro převod do CNF a rezoluci potřebujeme otevřené formule 
    \end{itemize}

    \myblock{
    \textbf{Cíl:} Ke každé teorii $T$ sestrojíme \alert{ekvisplnitelnou, otevřenou} $T'$.
    }

    \begin{enumerate}
        \item převod do \alert{prenexní normální formy} (vytkneme kvantifikátory)
        \item nahradíme generálními uzávěry \myalertinline{(potřebujeme sentence!)}
        \item nahradíme sentence \alert{Skolemovými variantami} (odstranění $\exists$)
        \item odstraníme zbývající $\forall$, máme otevřené formule
    \end{enumerate}

\end{frame}


\begin{frame}{Prenexní normální forma}

    
    \myblock{Formule $\varphi$ je v \alert{prenexní normální formě (PNF)}, je-li následujícího tvaru, kde $Q_i\in\{\forall,\exists\}$ a formule $\varphi'$ je otevřená:
    $$
    (Q_1x_1)\dots(Q_nx_n)\varphi'
    $$
    \vspace{-16pt}
    }
    \begin{itemize}
        \item $(Q_1x_1)\dots(Q_nx_n)$ je \alert{kvantifikátorový prefix}, $\varphi'$ \alert{otevřené jádro}
        \item \alert{univerzální} formule: v PNF a všechny kvantifikátory jsou $\forall$
    \end{itemize}

    \myblock{
    \textbf{Tvrzení:}
        Ke každé formuli $\varphi$ existuje \alert{ekvivalentní} formule v PNF.
    }    
    
    \textbf{Důkaz:} nahrazujeme podformule ekvivalentními s cílem posunout kvantifikátory blíž kořeni $\Tree(\varphi)$, dle pravidel z násl. Lemmatu.\hfill\qedsymbol


    \myblock{
    \textbf{Důsledek:} Existuje i ekvivalentní PNF \alert{sentence} (generální uzávěr).
    }

\end{frame}


\begin{frame}{Pravidla vytýkání kvantifikátorů}

    \myblock{
    \textbf{Lemma:} Označme $\overline{Q}$ opačný kvantifikátor ke $Q$. Jsou-li $\varphi$ a $\psi$ formule, kde \alert{$x$ není volná v $\psi$}, potom:
    \begin{align*}
        \neg (Qx)\varphi\ &\sim\ (\overline{Q}x)\neg\varphi\\
        (Qx)\varphi \land \psi\ &\sim\ (Qx)(\varphi \land \psi)\\
        (Qx)\varphi \lor \psi\ &\sim\ (Qx)(\varphi \lor \psi)\\
        (Qx)\varphi \limplies \psi\ &\sim\ \alert{(\overline{Q}x)}(\varphi \limplies \psi)\\
        \psi \limplies (Qx)\varphi\ &\sim\ (Qx)(\psi \limplies \varphi)
    \end{align*}
    }

    \textbf{Důkaz:} snadno ověříme sémanticky, nebo tablo metodou (potom ale nejsou-li sentence, musíme nahradit generálními uzávěry)\hfill\qedsymbol

    \smallskip

    \textbf{Pozorování:} Nahradíme-li ve $\varphi$ podformuli $\psi$ ekvivalentní $\psi'$, je i výsledná formule $\varphi'$ ekvivalentní $\varphi$. (Připomeňme: $\varphi\sim\varphi'$ právě když mají stejné modely, tj. $\models\varphi\liff\varphi'$)
    
\end{frame}


\begin{frame}{Převod do PNF: příklad}

    \myexamplemath{
        $$
        (\forall z)P(x,z)\land P(y,z)\ \limplies\ \neg(\exists x)P(x,y)
        $$
    }      
    \begin{align*}
        \sim &\ 
        \alert{(\forall u)} P(x,\alert{u})\land P(y,z)\ \limplies\ (\forall x)\neg P(x,y)\\ \sim &\ 
        (\forall u)(P(x,u)\land P(y,z))\ \limplies\ \alert{(\forall v)}\neg P(\alert{v},y)\\ \sim &\ 
        (\exists u)( P(x,u)\land P(y,z)\ \limplies\ (\forall v)\neg P(v,y))\\ \sim &\ 
        (\exists u)(\forall v)( P(x,u)\land P(y,z)\ \limplies\ \neg P(v,y))
    \end{align*}
    
    \begin{itemize}
        \item v prvním kroku přejmenujeme $z$ na $u$, \alert{nesmí být volná v $P(y,z)$}
        \item podobně ve druhém kroku $x$ na $v$
        \item která pravidla používáme? sledujte postup na stromu formule
        \item chceme-li sentenci:
        $$
        \alert{(\forall x)(\forall y)(\forall z)}(\exists u)(\forall v)( P(x,u)\land P(y,z)\limplies\neg P(v,y))
        $$
    \end{itemize}

\end{frame}


\begin{frame}{Poznámky}

    \begin{enumerate}
        \item proč se při vytýkání z \alert{antecedentu} mění kvantifikátor?
        \begin{align*}
            \alert{(Qx)\varphi \limplies \psi}&\sim\neg(Qx)\varphi \lor\psi\\&\sim\alert{(\overline{Q}x)}(\neg\varphi)\lor\psi\\
            &\sim(\overline{Q}x)(\neg\varphi\lor\psi)\sim \alert{(\overline{Q}x)(\varphi \limplies \psi)}    
        \end{align*}

        \medskip
    
        \item proč nesmí být $x$ volná v $\psi$? neplatilo by, např:
        \myexamplemath{
        $$
        (\exists x)P(x)\land Q(x)\ \not\sim\ (\exists x)(P(x)\land Q(x))
        $$
        }
        musíme přejmenovat vázanou proměnnou $x$ na novou: 
        
        \myexampleinline{
        $
        (\exists x)P(x)\land Q(x)\ \sim\ (\exists y)P(y)\land Q(x) \sim (\exists y)(P(y)\land Q(x))
        $
        }

        \medskip

        \item PNF není jednoznačná, lze vytýkat v různém pořadí; lepší je nejprve vytknout ty, \alert{ze kterých se nakonec stanou existenční:}
        
        \medskip

        \myalertmath{
            $$
            (\exists y)(\forall x)\varphi(x,y)\text{ je lepší než }(\forall x)(\exists y)\varphi(x,y)
            $$
        }
        
        \medskip

        (protože ``$y$ nezávisí na $x$'')
    \end{enumerate}

\end{frame}


\begin{frame}{Skolemova varianta}

    Je-li PNF sentence \alert{univerzální}, tvaru $(\forall x_1)\dots(\forall x_n)\psi(x_1,\dots,x_n)$, nahradíme otevřeným jádrem $\psi$. Jinak musíme provést \alert{skolemizaci}:

    \medskip

    \myblock{
    Buď $\varphi$ $L$-\alert{sentence} v PNF, všechny vázané proměnné různé. Nechť
    \begin{itemize}
        \item existenční kvantifikátory jsou $(\exists y_1),\dots,(\exists y_n)$ (v tom pořadí)
        \item pro každé $i$ jsou $(\forall x_1),\dots,(\forall x_{n_i})$ právě všechny univerzální kvantifikátory předcházející $(\exists y_i)$ v prefixu $\varphi$
    \end{itemize}
    Buď $L'$ rozšíření $L$ o \alert{nové} funkční symboly $f_1,\dots,f_n$, kde $f_i$ je $n_i$-ární. \alert{Skolemova varianta} $\varphi$ je $L'$-sentence $\varphi_S$ vzniklá \myalertinline{
        odstraněním $(\exists y_i)$
    } a substitucí termu \myalertinline{
        $f_i(x_1,\dots,x_{n_i})$ za $y_i$
    }, postupně pro $i=1,\dots,n$. 
    }

    \medskip

    \myexampleamsmath{
    \begin{align*}
        \varphi&=\textcolor{red}{(\exists y_1)}(\forall x_1)(\forall x_2)\textcolor{blue}{(\exists y_2)}(\forall x_3)\ R(\textcolor{red}{y_1},x_1,x_2,\textcolor{blue}{y_2},x_3)\\
        \varphi_S&=(\forall x_1)(\forall x_2)(\forall x_3)\ R(\textcolor{red}{f_1},x_1,x_2,\textcolor{blue}{f_2(x_1,x_2)},x_3) 
    \end{align*}    
    }

    \begin{itemize}
        \item \myalertinline{musí být sentence!} pro \myexampleinline{
            $(\exists y)E(x,y)$} ne \xcancel{$E(x,c)$} ale $E(x,f(x))$
        \item \myalertinline{nové symboly!} (jedinou rolí je reprezentovat `svědky' ve $\varphi$)
    \end{itemize}

\end{frame}


\begin{frame}{Je to konzervativní extenze}

    \myblock{
    \textbf{Lemma:} Buď $\varphi$ $L$-sentence $(\forall x_1)\dots(\forall x_n)(\exists y)\psi$, $f$ nový funkční symbol, a $\varphi'$ sentence 
    $(\forall x_1)\dots(\forall x_n)\psi(y/f(x_1,\dots,x_n))$. Potom:
    \begin{enumerate}[(i)]
        \item $L$-redukt každého modelu $\varphi'$ je modelem $\varphi$, a 
        \item každý model $\varphi$ lze expandovat na model $\varphi'$.
    \end{enumerate}
    }

    \textbf{Důkaz:} \alert{(i)} Buď $\A'$ model $\varphi'$, $\A$ jeho $L$-redukt, $e:\Var\to\A$. $\A\models\varphi[e]$ platí neboť $\A\models\psi[e(y/a)]$ pro $a=(f(x_1,\dots,x_n))^{\A'}[e]$.
    
    \alert{(ii)} Protože $\A\models\varphi$, existuje funkce $f^A:A^n\to A$, že pro každé ohodnocení $e$ platí $\A\models \psi[e(y/a)]$ pro $a=f^A(e(x_1),\dots,e(x_n))$. To znamená, že expanze o funkci $f^A$ splňuje $\varphi'$.\hfill\qedsymbol 
    \begin{itemize}
        \item říká, že $\{\varphi'\}$ je konzervativní extenze $\{\varphi\}$, opakovaná aplikace dává \alert{Skolemovu větu} (výsledek skolemizace je otevřená konzervativní extenze, speciálně je ekvisplnitelná)  
        \item expanze v (ii) není jednoznačná (na rozdíl od extenze o definici nového funkčního symbolu)      
    \end{itemize}
   
\end{frame}


\begin{frame}{Skolemova věta (shrnutí postupu)}

    \myblock{
    \textbf{Věta:} Každá teorie má otevřenou konzervativní extenzi.
    }

    \textbf{Důkaz} Mějme $L$-teorii $T$. Axiomy nahradíme generálními uzávěry a převedeme do PNF, máme ekvivalentní $L$-teorii $T'$. V ní každý axiom nahradíme jeho Skolemovou variantou. 
    
    Tím získáme teorii $T''$ v rozšířeném jazyce~$L'$. Lemma říká:
    \begin{itemize}
        \item $L$-redukt každého modelu $T''$ je model $T'$
        \item každý model $T'$ lze expandovat do $L'$ na model $T''$
       \end{itemize}
    Neboli $T''$ je konzervativní extenzí $T'$, tedy i $T$. Je axiomatizovaná  univerzálními sentencemi, odstraníme kvantifikátorové prefixy (vezmeme jádra) a máme ekvivalentní otevřenou teorii $T'''$.    
    \hfill\qedsymbol

    \medskip

    \myblock{
    \textbf{Důsledek:} Ke každé teorii můžeme pomocí skolemizace najít ekvisplnitelnou otevřenou teorii. (A tu už snadno převedeme do CNF.)
    }

\end{frame}


\end{document}
