\documentclass{beamer}

%% slide-specific

\usetheme[progressbar=frametitle]{metropolis}
%\usecolortheme{spruce}
%\metroset{block=fill}

% define Metropolis colors    
\definecolor{mAlert}{HTML}{EB811B}
\definecolor{mExample}{HTML}{14B03D}
\definecolor{mBlock}{HTML}{23373b}

% my blocks
\setlength\fboxsep{0pt}%

\newcommand{\myblock}[1]{\colorbox{mBlock!8}{\begin{minipage}{\linewidth}#1\end{minipage}}}
\newcommand{\myblockmath}[1]{\colorbox{mBlock!8}{\begin{minipage}{\linewidth}\vspace{-6pt}#1\end{minipage}}}
\newcommand{\myblockinline}[1]{\colorbox{mBlock!8}{#1}}
\newcommand{\myexample}[1]{\colorbox{mExample!8}{\begin{minipage}{\linewidth}#1\end{minipage}}}
\newcommand{\myexamplemath}[1]{\colorbox{mExample!8}{\begin{minipage}{\linewidth}\vspace{-6pt}#1\end{minipage}}}
\newcommand{\myexampleinline}[1]{\colorbox{mExample!8}{#1}}
\newcommand{\myalert}[1]{\colorbox{mAlert!8}{\begin{minipage}{\linewidth}#1\end{minipage}}}
\newcommand{\myalertmath}[1]{\colorbox{mAlert!8}{\begin{minipage}{\linewidth}\vspace{-6pt}#1\end{minipage}}}
\newcommand{\myalertinline}[1]{\colorbox{mAlert!8}{#1}}

%% other
\newcommand{\mystructure}[1]{\mathcal{#1}}




%% packages
\usepackage{amsmath,amssymb,amsthm}
\usepackage{booktabs}
\usepackage[czech]{babel}
\usepackage{enumerate}
\usepackage{forest}
\usepackage{multicol}
% \usepackage{tcolorbox}
\usepackage{tikz}
    \usetikzlibrary{arrows.meta}
%\usepackage[unicode]{hyperref}
\usepackage[utf8x]{inputenc}
\usepackage{xfrac}

% %% theorems
% \theoremstyle{plain}
%     \newtheorem{theorem}{Věta}[section]
%     \newtheorem*{theorem-unnumbered}{Věta}
%     \newtheorem{proposition}[theorem]{Tvrzení}
%     \newtheorem{corollary}[theorem]{Důsledek}
%     \newtheorem{lemma}[theorem]{Lemma}
%     \newtheorem{observation}[theorem]{Pozorování}
% \theoremstyle{definition}
%     \newtheorem{definition}[theorem]{Definice}
%     \newtheorem*{algorithm}{Algoritmus}
% \theoremstyle{remark}
%     \newtheorem{remark}[theorem]{Poznámka}
%     \newtheorem{example}[theorem]{Příklad}
%     \newtheorem{exercise}{Cvičení}[chapter]
%     \newtheorem*{solution}{Řešení}

%% macros and definitions
\DeclareMathOperator{\Aut}{Aut}
\DeclareMathOperator{\Conseq}{Csq}
\DeclareMathOperator{\DeLO}{DeLO}
\DeclareMathOperator{\dom}{dom}
\DeclareMathOperator{\Fm}{Fm}
\DeclareMathOperator{\M}{M}
%\DeclareMathOperator{\Proof}{Proof}
\DeclareMathOperator{\rng}{rng}
\DeclareMathOperator{\Term}{Term}
\DeclareMathOperator{\Th}{Th}
\DeclareMathOperator{\Thm}{Thm}
\DeclareMathOperator{\Tree}{Tree}
\DeclareMathOperator{\Var}{Var}
\DeclareMathOperator{\VF}{VF}

\newcommand{\A}{\structure{A}}
\newcommand{\B}{\structure{B}}
\newcommand{\Con}{\mathit{Con}}
\newcommand{\disjointunion}{\mathbin{\dot{\sqcup}}}
\newcommand{\F}{\ensuremath{\mathrm{F}}}
\newcommand{\landsymb}{{\land}}
\newcommand{\lbin}{\mathbin{\square}}
\newcommand{\lbinsymb}{{\lbin}}
\newcommand{\liff}{\mathbin{\leftrightarrow}}
\newcommand{\liffsymb}{{\liff}}
\newcommand{\limplies}{\mathbin{\rightarrow}}
\newcommand{\limpliessymb}{{\limplies}}
\newcommand{\lorsymb}{{\lor}}
\newcommand{\Prf}{\mathit{Prf}}
\newcommand{\proves}{\vdash}
%\newcommand{\structure}[1]{\mathcal{#1}}
\newcommand{\todo}{[TODO]}
\newcommand{\T}{\ensuremath{\mathrm{T}}}
\newcommand{\union}{\mathbin{\cup}}


\title{Devátá přednáška}
\subtitle{NAIL062 Výroková a predikátová logika}
\author{Jakub Bulín (KTIML MFF UK)}
% \institute{KTIML MFF UK}
\date{Zimní semestr 2023}


\begin{document}


\frame{\titlepage}


\begin{frame}{Devátá přednáška}

    \textbf{Program}
        \begin{itemize}
            \item Löwenheim-Skolemova věta
            \item věta o kompaktnosti
            \item hilbertovský kalkulus.
            \item úvod do rezoluce v predikátové logice
            \item skolemizace
        \end{itemize}

    \textbf{Materiály}

        \href{https://github.com/jbulin-mff-uk/nail062/raw/main/lecture/lecture-notes/lecture-notes.pdf}{\alert{\textbf{Zápisky z přednášky}}}, Sekce 7.5-7.6 z Kapitoly 7, Sekce 8.1-8.2 z Kapitoly 8

\end{frame}


\section{7.5 Důsledky korektnosti a úplnosti}


\begin{frame}{$\proves\ =\ \models$}

    Syntaktickou analogií \alert{důsledků} jsou \alert{teorémy}:
    $$
    \Thm_L(T)=\{\varphi\mid \varphi\text{ je $L$-sentence a } T\proves\varphi\}
    $$
    
    Z korektnosti a úplnosti okamžitě dostáváme:
        \begin{itemize}
            \item $T\proves\varphi$ právě když $T\models\varphi$
            \item $\Thm_L(T)=\Conseq_L(T)$
        \end{itemize}
    
    Všude můžeme nahradit `\alert{platnost}' pojmem `\alert{dokazatelnost}'.  Např:
    \begin{itemize}
        \item $T$ je \alert{sporná}, je-li v ní dokazatelný spor (tj. \alert{$T\proves\bot$})
        \item $T$ je \alert{kompletní}, je-li pro každou sentenci buď $T\proves\varphi$ nebo $T\proves\neg\varphi$, ale ne obojí (jinak by byla sporná)
    \end{itemize}

    \myblock{
        \textbf{Věta (O dedukci):} $T,\varphi\proves\psi$ právě když  $T\proves\varphi\to\psi$.
    }

    \textbf{Důkaz:} Stačí dokázat: $T,\varphi\models\psi\Leftrightarrow T\models\varphi\to\psi$. To je snadné.\hfill\qedsymbol

\end{frame}


\begin{frame}{Löwenheim-Skolemova věta \& Věta o kompaktnosti}
    
    \medskip
    
    \myblock{
    \textbf{Věta (Löwenheim-Skolemova):}
    Je-li $L$ spočetný bez rovnosti, potom každá bezesporná $L$-teorie má spočetně nekonečný model.
    }

    (Později ukážeme i verzi s rovností, kan. model může být konečný.)

    \textbf{Důkaz:} V $T$ není dokazatelný spor. Dokončené tablo z $T$ s $\F\bot$ v kořeni tedy musí obsahovat bezespornou větev. Hledaný model je $L$-redukt kanonického modelu pro tuto větev.\hfill\qedsymbol

    \bigskip

    Věta o kompaktnosti, vč. důkazu, je stejná jako ve výrokové logice:

    \smallskip
    \myblock{
    \textbf{Věta (O kompaktnosti):}
    Teorie má model, právě když každá její konečná část má model.\vspace{2pt}  
    }  
    
    \textbf{Důkaz:} Model teorie je modelem každé části. Naopak, pokud $T$ nemá model, je sporná, tedy $T\proves\bot$. Vezměme nějaký \alert{konečný} tablo důkaz $\bot$ z $T$. K jeho konstrukci stačí konečně mnoho axiomů $T$, ty tvoří konečnou podteorii $T'\subseteq T$, která nemá model.\hfill\qedsymbol

\end{frame}


\begin{frame}{Nestandardní model přirozených čísel}

    \begin{itemize}
        \item $\underline{\mathbb N}=\langle\mathbb N,S,+,\cdot,0,\leq\rangle$ je \alert{standardní model} přirozených čísel
        \item \alert{teorie struktury $\Th(\underline{\mathbb N})$:} všechny sentence \alert{pravdivé} v~$\underline{\mathbb N}$
        \item \alert{$n$-tý numerál:} term $\underline n=S(S(\cdots (S(0)\cdots))$, kde $S$ je $n$-krát
    \end{itemize}
    
    Přidáme nový konstantní symbol $c$ a vyjádříme, že je ostře větší než každý $n$-tý numerál:
    $$
    T=\Th(\underline{\mathbb N})\cup\{\underline n<c\mid n\in \mathbb N\}
    $$
        
    \begin{itemize}
        \item každá konečná část $T$ má model
        \item dle věty o kompaktnosti: i $T$ má model
        \item říkáme mu \alert{nestandardní model} (označme $\A$)
        \item platí v něm tytéž sentence, které platí ve standardním modelu
        \item ale zároveň obsahuje prvek $c^\A$, který je větší než každé $n\in \mathbb N$ (tzn. větší než hodnota termu $\underline n$ v nestandardním modelu $\A$)
    \end{itemize}    

\end{frame}


\section{7.6 Hilbertovský kalkulus v predikátové logice}



\begin{frame}{Hilbertovský kalkulus v predikátové logice}

    \begin{itemize}
        \item používá jen logické spojky $\neg$, $\limplies$
        \item dokazují se libovolné formule (nejen sentence)
        \item \alert{schémata logických axiomů} ($\varphi,\psi,\chi$ jsou libovolné formule, $t$ term, $x$ proměnná)
        \begin{enumerate}[(i)]
            \item $\varphi \limplies (\psi \limplies \varphi)$
            \item $(\varphi\limplies (\psi \limplies \chi))\limplies ((\varphi \limplies \psi)\limplies(\varphi \limplies \chi))$
            \item $(\neg \varphi \limplies \neg \psi)\limplies(\psi \limplies \varphi)$            
        \end{enumerate}
        \myalert{
        \begin{enumerate}[(iv)]
            \item $(\forall x)\varphi \limplies \varphi(x/t)$ \hspace{1cm} je-li $t$ substituovatelný za $x$ do $\varphi$
            \item $(\forall x)(\varphi \to \psi) \limplies (\varphi \limplies (\forall x)\psi)$ \hspace{0.5cm} není-li $x$ volná proměnná ve $\varphi$}
        \end{enumerate}
        }
        \item \alert{odvozovací pravidlo}: tzv. \alert{modus ponens}
                $$\frac{\varphi, \varphi \limplies \psi}{\psi}$$       
        \item \alert{hilbertovský důkaz} výroku $\varphi$ z teorie $T$ je \alert{konečná} posloupnost výroků $\varphi_0, \dots, \varphi_n=\varphi$, ve které pro každé $i\le n$:
        \begin{itemize}
        \item $\varphi_i$ je \alert{logický axiom}, nebo
        \item $\varphi_i$ je \alert{axiom teorie} ($\varphi_i \in T$), nebo
        \item $\varphi_i$ lze odvodit z předchozích pomocí \alert{odvozovacího pravidla}
        \end{itemize}
        \item existuje-li hilbertovský důkaz, píšeme: \alert{$T\proves_H\varphi$}
    \end{itemize}

\end{frame}


\begin{frame}{Příklad hilbertovského důkazu}

    Ukažme, že pro teorii $T=\{\neg\varphi\}$ a pro libovolný výrok $\psi$ platí:  
    \myexamplemath{  
    $$
    T\proves_H\varphi\limplies\psi
    $$
    }

    Hilbertovským důkazem je následující posloupnost výroků:
    
    \begin{enumerate}\it
        \item $\neg\varphi$\hfill axiom teorie
        \item $\neg \varphi \limplies (\neg \psi \limplies \neg \varphi)$\hfill logický axiom (i)
        \item $\neg \psi \limplies \neg \varphi$\hfill modus ponens na 1. a 2.
        \item $(\neg \psi \limplies \neg \varphi)\limplies(\varphi \limplies \psi)$ \hfill logický axiom (iii)
        \item $\varphi \limplies \psi$ \hfill modus ponens na 3. a 4.
    \end{enumerate}    

\end{frame}


\begin{frame}{Korektnost a úplnost}

    \myblock{
    \textbf{Věta (o korektnosti hilbertovského kalkulu):}
    $T\proves_H\varphi \Rightarrow T\models\varphi$
    }

    \medskip

    \textbf{Důkaz:} Indukcí dle délky důkazu ukážeme, že každý výrok $\varphi_i$ z důkazu (tedy i $\varphi_n=\varphi$) platí v $T$.
    \begin{itemize}
        \item Je-li $\varphi_i$ logický axiom, $T \models \varphi_i$ platí protože logické axiomy jsou tautologie.
        \item Je-li $\varphi_i\in T$, jistě platí $T \models \varphi_i$.
        \item Získáme-li $\varphi_i$ pomocí modus ponens z $\varphi_j$ a $\varphi_k=\varphi_j\limplies\varphi_i$ (pro nějaká $j,k<i$), víme z indukčního předpokladu, že platí $T \models \varphi_j$ a $T \models \varphi_j\limplies\varphi_i$. Potom ale platí i $T \models \varphi_i$. (Modus ponens je \alert{korektní} odvozovací pravidlo)\hfill\qedsymbol
    \end{itemize}

    \myblock{
    \textbf{Věta (o úplnosti hilbertovského kalkulu):}
    $T\models\varphi\ \Rightarrow\ T\proves_H\varphi$
    }

    Důkaz vynecháme.
    
\end{frame}


\section{\sc Kapitola 8: Rezoluce v predikátové logice}


\section{8.1 Úvod}


\section{8.2 Skolemizace}


\end{document}


