\documentclass{beamer}

%% slide-specific

\usetheme[progressbar=frametitle]{metropolis}
%\usecolortheme{spruce}
%\metroset{block=fill}

% define Metropolis colors    
\definecolor{mAlert}{HTML}{EB811B}
\definecolor{mExample}{HTML}{14B03D}
\definecolor{mBlock}{HTML}{23373b}

% my blocks
\setlength\fboxsep{0pt}%

\newcommand{\myblock}[1]{\colorbox{mBlock!8}{\begin{minipage}{\linewidth}#1\end{minipage}}}
\newcommand{\myblockmath}[1]{\colorbox{mBlock!8}{\begin{minipage}{\linewidth}\vspace{-6pt}#1\end{minipage}}}
\newcommand{\myblockinline}[1]{\colorbox{mBlock!8}{#1}}
\newcommand{\myexample}[1]{\colorbox{mExample!8}{\begin{minipage}{\linewidth}#1\end{minipage}}}
\newcommand{\myexamplemath}[1]{\colorbox{mExample!8}{\begin{minipage}{\linewidth}\vspace{-6pt}#1\end{minipage}}}
\newcommand{\myexampleinline}[1]{\colorbox{mExample!8}{#1}}
\newcommand{\myalert}[1]{\colorbox{mAlert!8}{\begin{minipage}{\linewidth}#1\end{minipage}}}
\newcommand{\myalertmath}[1]{\colorbox{mAlert!8}{\begin{minipage}{\linewidth}\vspace{-6pt}#1\end{minipage}}}
\newcommand{\myalertinline}[1]{\colorbox{mAlert!8}{#1}}

%% other
\newcommand{\mystructure}[1]{\mathcal{#1}}




%% packages
\usepackage{amsmath,amssymb,amsthm}
\usepackage{booktabs}
\usepackage[czech]{babel}
\usepackage{enumerate}
\usepackage{forest}
\usepackage{multicol}
% \usepackage{tcolorbox}
\usepackage{tikz}
    \usetikzlibrary{arrows.meta}
%\usepackage[unicode]{hyperref}
\usepackage[utf8x]{inputenc}
\usepackage{xfrac}

% %% theorems
% \theoremstyle{plain}
%     \newtheorem{theorem}{Věta}[section]
%     \newtheorem*{theorem-unnumbered}{Věta}
%     \newtheorem{proposition}[theorem]{Tvrzení}
%     \newtheorem{corollary}[theorem]{Důsledek}
%     \newtheorem{lemma}[theorem]{Lemma}
%     \newtheorem{observation}[theorem]{Pozorování}
% \theoremstyle{definition}
%     \newtheorem{definition}[theorem]{Definice}
%     \newtheorem*{algorithm}{Algoritmus}
% \theoremstyle{remark}
%     \newtheorem{remark}[theorem]{Poznámka}
%     \newtheorem{example}[theorem]{Příklad}
%     \newtheorem{exercise}{Cvičení}[chapter]
%     \newtheorem*{solution}{Řešení}

%% macros and definitions
\DeclareMathOperator{\Aut}{Aut}
\DeclareMathOperator{\Conseq}{Csq}
\DeclareMathOperator{\DeLO}{DeLO}
\DeclareMathOperator{\dom}{dom}
\DeclareMathOperator{\Fm}{Fm}
\DeclareMathOperator{\M}{M}
%\DeclareMathOperator{\Proof}{Proof}
\DeclareMathOperator{\rng}{rng}
\DeclareMathOperator{\Term}{Term}
\DeclareMathOperator{\Th}{Th}
\DeclareMathOperator{\Thm}{Thm}
\DeclareMathOperator{\Tree}{Tree}
\DeclareMathOperator{\Var}{Var}
\DeclareMathOperator{\VF}{VF}

\newcommand{\A}{\structure{A}}
\newcommand{\B}{\structure{B}}
\newcommand{\Con}{\mathit{Con}}
\newcommand{\disjointunion}{\mathbin{\dot{\sqcup}}}
\newcommand{\F}{\ensuremath{\mathrm{F}}}
\newcommand{\landsymb}{{\land}}
\newcommand{\lbin}{\mathbin{\square}}
\newcommand{\lbinsymb}{{\lbin}}
\newcommand{\liff}{\mathbin{\leftrightarrow}}
\newcommand{\liffsymb}{{\liff}}
\newcommand{\limplies}{\mathbin{\rightarrow}}
\newcommand{\limpliessymb}{{\limplies}}
\newcommand{\lorsymb}{{\lor}}
\newcommand{\Prf}{\mathit{Prf}}
\newcommand{\proves}{\vdash}
%\newcommand{\structure}[1]{\mathcal{#1}}
\newcommand{\todo}{[TODO]}
\newcommand{\T}{\ensuremath{\mathrm{T}}}
\newcommand{\union}{\mathbin{\cup}}


\title{Devátá přednáška}
\subtitle{NAIL062 Výroková a predikátová logika}
\author{Jakub Bulín (KTIML MFF UK)}
% \institute{KTIML MFF UK}
\date{Zimní semestr 2023}


\begin{document}


\frame{\titlepage}


\begin{frame}{Devátá přednáška}

    \textbf{Program}
        \begin{itemize}
            \item Löwenheim-Skolemova věta
            \item věta o kompaktnosti
            \item hilbertovský kalkulus.
            \item úvod do rezoluce v predikátové logice
            \item skolemizace
        \end{itemize}

    \textbf{Materiály}

        \href{https://github.com/jbulin-mff-uk/nail062/raw/main/lecture/lecture-notes/lecture-notes.pdf}{\alert{\textbf{Zápisky z přednášky}}}, Sekce 7.5-7.6 z Kapitoly 7, Sekce 8.1-8.2 z Kapitoly 8

\end{frame}


\section{7.5 Důsledky korektnosti a úplnosti}


\begin{frame}{$\proves\ =\ \models$}

    Syntaktickou analogií \alert{důsledků} jsou \alert{teorémy}:
    $$
    \Thm_L(T)=\{\varphi\mid \varphi\text{ je $L$-sentence a } T\proves\varphi\}
    $$
    
    Z korektnosti a úplnosti okamžitě dostáváme:
        \begin{itemize}
            \item $T\proves\varphi$ právě když $T\models\varphi$
            \item $\Thm_L(T)=\Conseq_L(T)$
        \end{itemize}
    
    Všude můžeme nahradit `\alert{platnost}' pojmem `\alert{dokazatelnost}'.  Např:
    \begin{itemize}
        \item $T$ je \alert{sporná}, je-li v ní dokazatelný spor (tj. \alert{$T\proves\bot$})
        \item $T$ je \alert{kompletní}, je-li pro každou sentenci buď $T\proves\varphi$ nebo $T\proves\neg\varphi$, ale ne obojí (jinak by byla sporná)
    \end{itemize}

    \myblock{
        \textbf{Věta (O dedukci):} $T,\varphi\proves\psi$ právě když  $T\proves\varphi\to\psi$.
    }

    \textbf{Důkaz:} Stačí dokázat: $T,\varphi\models\psi\Leftrightarrow T\models\varphi\to\psi$. To je snadné.\hfill\qedsymbol

\end{frame}


\begin{frame}{Löwenheim-Skolemova věta \& Věta o kompaktnosti}
    
    \medskip
    
    \myblock{
    \textbf{Věta (Löwenheim-Skolemova):}
    Je-li $L$ spočetný bez rovnosti, potom každá bezesporná $L$-teorie má spočetně nekonečný model.
    }

    (Později ukážeme i verzi s rovností, kan. model může být konečný.)

    \textbf{Důkaz:} V $T$ není dokazatelný spor. Dokončené tablo z $T$ s $\F\bot$ v kořeni tedy musí obsahovat bezespornou větev. Hledaný model je $L$-redukt kanonického modelu pro tuto větev.\hfill\qedsymbol

    \bigskip

    Věta o kompaktnosti, vč. důkazu, je stejná jako ve výrokové logice:

    \smallskip
    \myblock{
    \textbf{Věta (O kompaktnosti):}
    Teorie má model, právě když každá její konečná část má model.\vspace{2pt}  
    }  
    
    \textbf{Důkaz:} Model teorie je modelem každé části. Naopak, pokud $T$ nemá model, je sporná, tedy $T\proves\bot$. Vezměme nějaký \alert{konečný} tablo důkaz $\bot$ z $T$. K jeho konstrukci stačí konečně mnoho axiomů $T$, ty tvoří konečnou podteorii $T'\subseteq T$, která nemá model.\hfill\qedsymbol

\end{frame}


\begin{frame}{Nestandardní model přirozených čísel}

    \begin{itemize}
        \item $\underline{\mathbb N}=\langle\mathbb N,S,+,\cdot,0,\leq\rangle$ je \alert{standardní model} přirozených čísel
        \item \alert{teorie struktury $\Th(\underline{\mathbb N})$:} všechny sentence \alert{pravdivé} v~$\underline{\mathbb N}$
        \item \alert{$n$-tý numerál:} term $\underline n=S(S(\cdots (S(0)\cdots))$, kde $S$ je $n$-krát
    \end{itemize}
    
    Přidáme nový konstantní symbol $c$ a vyjádříme, že je ostře větší než každý $n$-tý numerál:
    $$
    T=\Th(\underline{\mathbb N})\cup\{\underline n<c\mid n\in \mathbb N\}
    $$
        
    \begin{itemize}
        \item každá konečná část $T$ má model
        \item dle věty o kompaktnosti: i $T$ má model
        \item říkáme mu \alert{nestandardní model} (označme $\A$)
        \item platí v něm tytéž sentence, které platí ve standardním modelu
        \item ale zároveň obsahuje prvek $c^\A$, který je větší než každé $n\in \mathbb N$ (tzn. větší než hodnota termu $\underline n$ v nestandardním modelu $\A$)
    \end{itemize}    

\end{frame}


\section{7.6 Hilbertovský kalkulus v predikátové logice}



\begin{frame}{Hilbertovský kalkulus v predikátové logice}

    \vspace{-6pt}
    \begin{itemize}
        \item používá jen $\neg$ a $\limplies$, dokazuje lib. formule (nejen sentence)
        \item \alert{schémata log. axiomů} ($\varphi,\psi,\chi$ formule, $t$ term, $x$ proměnná)
        \begin{enumerate}[(i)]
            \item $\varphi \limplies (\psi \limplies \varphi)$
            \item $(\varphi\limplies (\psi \limplies \chi))\limplies ((\varphi \limplies \psi)\limplies(\varphi \limplies \chi))$
            \item $(\neg \varphi \limplies \neg \psi)\limplies(\psi \limplies \varphi)$            
        \end{enumerate}

        \medskip

        \myalert{
        \begin{enumerate}[(iv)]
            \item $(\forall x)\varphi \limplies \varphi(x/t)$ \hfill je-li $t$ substituovatelný za $x$ do $\varphi$
            \item $(\forall x)(\varphi \to \psi) \limplies (\varphi \limplies (\forall x)\psi)$ \hfill není-li $x$ volná ve $\varphi$
        \end{enumerate}
        
        a navíc \textbf{axiomy rovnosti}, je-li jazyk s rovností
        }

        \medskip

        \item \alert{odvozovací pravidla}:\vspace{-6pt}
        \begin{center}
            \begin{minipage}{0.45\textwidth}
                $$
                \frac{\varphi, \varphi \limplies \psi}{\psi}\ \text{(modus ponens)}
                $$
            \end{minipage}          
            \begin{minipage}{0.45\textwidth}
                \myalertmath{
                $$
                \frac{\varphi}{(\forall x)\varphi} \ \text{(generalizace)} 
                $$
                }
            \end{minipage}            
        \end{center}        
        
        \item \alert{hilbertovský důkaz} formule $\varphi$ z $T$ je \alert{konečná} posloupnost $\varphi_0, \dots, \varphi_n=\varphi$, kde $\varphi_i$ je \alert{logický axiom} (vč. axiomů rovnosti), \alert{axiom teorie}, nebo lze odvodit z předchozích pomocí pravidel
        \item existuje-li, píšeme \alert{$T\proves_H\varphi$}
    \end{itemize}

\end{frame}


\begin{frame}{Korektnost a úplnost}

    \myblock{
    \textbf{Věta (o korektnosti hilbertovského kalkulu):}
    $T\proves_H\varphi \Rightarrow T\models\varphi$
    }

    \medskip

    \textbf{Důkaz:} Indukcí dle délky důkazu: každá $\varphi_i$ (vč. $\varphi_n=\varphi$) platí v $T$
    \begin{itemize}
        \item logické axiomy (vč. axiomů rovnosti) jsou tautologie, platí v $T$
        \item axiomy z $T$ jistě v $T$ také platí
         \item modus ponens i generalizace jsou \alert{korektní} inferenční pravidla:
        \begin{itemize}
            \item je-li $T\models\varphi$ a $T\models\varphi\to\psi$, potom $T\models\psi$
            \item je-li $T\models\varphi$, potom $T\models(\forall x)\varphi$
            \hfill\qedsymbol
        \end{itemize}
    \end{itemize}

    \bigskip
    
    \myblock{
    \textbf{Věta (o úplnosti hilbertovského kalkulu):}
    $T\models\varphi\ \Rightarrow\ T\proves_H\varphi$
    }

    Důkaz vynecháme.
    
\end{frame}


\section{\sc Kapitola 8: Rezoluce v predikátové logice}


\section{8.1 Úvod}


\begin{frame}{Rezoluce v predikátové logice}

    $T\models\varphi$? {\Large$\rightsquigarrow$} $T\cup\{\neg \varphi\}$ {\Large$\rightsquigarrow$} CNF formule $S$ {\Large$\rightsquigarrow$} rezoluční zamítnutí

    \begin{itemize}
        \item \alert{literál} je \alert{atomická formule} $R(t_1,\dots,t_n)$ nebo její negace
        \item \alert{klauzule} je konečná množina literálů, \alert{formule} množina klauzulí
        \item otevřenou formuli snadno převedeme do CNF, i univerzální kvantifikátor na začátku:{\small\myexampleinline{
            $(\forall x)(P(x)\lor \neg Q(x))\sim \{P(x),\neg Q(x)\}$
            }}
        \item co s existenčními kvantifikátory? nové symboly pro `svědky'
        {\small\myexampleinline{
        $(\exists x)(P(x)\lor \neg Q(x))\rightsquigarrow \{P(c),\neg Q(c)\}$
        }} ``\alert{skolemizace}''
        \item není ekvivalentní, ale zachovává \alert{[ne]splnitelnost}, to nám stačí
        \item rezoluční krok? literály nemusí být stejné, stačí \alert{unifikovatelné}
        {\myexampleinline{
        z klauzulí $\{P(x),\neg Q(x)\}$ a $\{Q(f(c))\}$ odvodíme $\{P(f(c))\}$
        }}
        \item \alert{unifikace} je substituce $\{x/f(c)\}$
    \end{itemize}

\end{frame}


\begin{frame}{Příklady}
    
    1. Nechť \myexampleinline{
        $T=\{(\forall x)P(x),(\forall x)(P(x)\limplies Q(x))\}$
    } a \myexampleinline{
            $\varphi=(\exists x)Q(x)$
        }. 
    $$\neg\varphi=\neg(\exists x)Q(x)\sim(\forall x)\neg Q(x)\sim\neg Q(x)$$ 
    Teorii $T\cup\{\neg \varphi\}$ tedy můžeme převést na \alert{ekvivalentní} CNF formuli
    $$
    S = \{\{P(x)\},\{\neg P(x),Q(x)\},\{\neg Q(x)\}\}
    $$
    kterou snadno zamítneme rezolucí ve dvou krocích. (Představte si místo $P(x)$ prvovýrok $p$ a místo $Q(x)$ prvovýrok $q$.)

    % 2. Máme-li $T=\{(\exists x)P(x),P(x)\liff Q(x)\}$ a $\varphi=(\exists x)Q(x)$, potom můžeme převést do CNF jako obvykle, dostáváme:
    % $$
    % T\cup\{\neg \varphi\}\sim\{(\exists x)P(x),\neg P(x)\lor Q(x),\neg Q(x)\lor P(x),\neg Q(x)\}
    % $$
    % Formuli $(\exists x)P(x)$ nyní nahradíme $P(c)$, kde $c$ je nový konstantní symbol. Tím dostáváme CNF formuli:
    % $$
    % S = \{\{P(c)\},\{\neg P(x),Q(x)\},\{\neg Q(x),P(x)\},\{\neg Q(x)\}\}
    % $$
    % Ta není ekvivalentní teorii $T\cup\{\neg \varphi\}$, ale je s ní \alert{ekvisplnitelná} (v tomto případě jsou obě nesplnitelné).


\end{frame}



\section{8.2 Skolemizace}


\end{document}


