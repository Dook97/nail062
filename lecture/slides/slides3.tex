\documentclass{beamer}

\usepackage{a4wide}
\usepackage{amsmath}
\usepackage{amssymb}
\usepackage{amsthm}
\usepackage[czech]{babel}
\usepackage{bookmark}
\usepackage{enumerate}
\usepackage[T1]{fontenc}
\usepackage{forest}
\usepackage{hyperref}
\usepackage[utf8]{inputenc}
\usepackage{lmodern}
\usepackage{multicol}
\usepackage{tikz}

\theoremstyle{definition}
    \newtheorem{problem}{Příklad}

% \theoremstyle{remark}
%     \newtheorem*{steps}{Postup řešení}

\theoremstyle{plain}
    \newtheorem*{solution}{Řešení}
    

\DeclareRobustCommand\proves{\mathrel{|}\joinrel\mkern-.5mu\mathrel{-}}
\DeclareMathOperator{\Conseq}{Csq}
\DeclareMathOperator{\M}{M}

% hide solutions
\newif\ifhidesolutions
    \hidesolutionstrue
    % \hidesolutionsfalse

\ifhidesolutions
    \usepackage{environ}
    \NewEnviron{hide}{}
    \let\solution\hide
    \let\endsolution\endhide
\fi








\title{Třetí přednáška}
\subtitle{NAIL062 Výroková a predikátová logika}
\author{Jakub Bulín (KTIML MFF UK)}
% \institute{KTIML MFF UK}
\date{Zimní semestr 2023}


\begin{document}


\frame{\titlepage}


\begin{frame}{Třetí přednáška}

    \textbf{Program}
        \begin{itemize}
            \item algebra výroků
            \item problém splnitelnosti, SAT solvery
            \item 2-SAT a implikační graf
            \item Horn-SAT a jednotková propagace
            \item algoritmus DPLL
        \end{itemize}

    \textbf{Materiály}

        \href{https://github.com/jbulin-mff-uk/nail062/raw/main/lecture/lecture-notes/lecture-notes.pdf}{\alert{\textbf{Zápisky z přednášky}}}, Sekce 2.5 z Kapitoly 2, Kapitola 3

\end{frame}


\section{2.5 Algebra výroků}


\begin{frame}{Výroky až na ekvivalenci}
    Kolik existuje výroků nad $\mathbb P=\{p,q,r\}$? Nekonečně mnoho. \alert{Až na ekvivalenci?} Tolik, kolik je možných množin modelů: $2^{2^3}=256$.

    \myalertinline{Výroky až na ekvivalenci studujeme pomocí jejich množin modelů.}

    Ekvivalenční třídy: \alert{$\sfrac{\VF_\mathbb P}{\sim}$}, např. \myexampleinline{\small
        $[p\limplies q]_\sim=\{p\limplies q,\neg p \lor q,\dots\}$
    }

    Přiřazení modelů: \myalertinline{
        \small $h:\sfrac{\VF_\mathbb P}{\sim}\to\mathcal P(\M_\mathbb P)$ definované $h([\varphi]_\sim)=\M(\varphi)$
    }
    (je dobře definované, prosté, pro konečný jazyk bijekce)

    Na $\sfrac{\VF_\mathbb P}{\sim}$ zavedeme operace $\neg,\landsymb,\lorsymb$ \alert{pomocí reprezentantů}:
    {\small
    \begin{align*}
        \neg [\varphi]_\sim &=[\neg\varphi]_\sim\\
        [\varphi]_\sim \land [\psi]_\sim &= [\varphi\land\psi]_\sim\\
        [\varphi]_\sim \lor [\psi]_\sim &= [\varphi\lor\psi]_\sim\\
    \end{align*}
    }

    \vspace{-18pt}
    přidáme konstanty {\small $\bot=[\bot]_\sim,\top=[\top]_\sim$}, máme \emph{Booleovu algebru}: \alert{algebru výroků} jazyka $\mathbb P$; totéž relativně k teorii $T$ (\alert{použijeme $\sim_T$})
\end{frame}


\begin{frame}{Algebra výroků}

\myblock{
\alert{Algebra výroků} jazyka $\mathbb P$ resp. teorie $T$: \vspace{-6pt}
\begin{align*}
    \mathbf{AV}_\mathbb P&=\langle \sfrac{\VF_\mathbb P}{\sim}; \neg, \landsymb, \lorsymb, \bot, \top\rangle\\
    \mathbf{AV}_\mathbb P(T)&=\langle \sfrac{\VF_\mathbb P}{\sim_T}; \neg_T, \landsymb_T, \lorsymb_T, \bot_T, \top_T\rangle
\end{align*}
}

přiřazení modelů $h$ je prosté zobrazení algebry výroků jazyka do \alert{potenční algebry} \myalertinline{
    $\mathbf{\mathcal P(\M_\mathbb P)}=\langle \mathcal P(\M_\mathbb P); \overline{\phantom{x}}, \cap, \cup, \emptyset, \M_\mathbb P\rangle$
} \alert{zachovávající} operace a konstanty: $h(\bot)=\emptyset$, $h(\top)=\M_\mathbb P$, a
{\small
\begin{align*}
    h(\neg[\varphi]_\sim)&=\overline{h([\varphi]_\sim)}=\overline{\M(\varphi)}=\M_\mathbb P\setminus\M(\varphi)\\
    h([\varphi]_\sim\land[\psi]_\sim)&=h([\varphi]_\sim)\cap h([\psi]_\sim)=\M(\varphi)\cap\M(\psi)\\
    h([\varphi]_\sim\lor[\psi]_\sim)&=h([\varphi]_\sim)\cup h([\psi]_\sim)=\M(\varphi)\cup\M(\psi)
\end{align*}
}
tj. je to  \alert{homomorfismus} Booleových algeber, a nad konečným jazykem bijekce, tzv. \alert{izomorfismus}; stejně pro algebru výroků teorie

\myblock{
\textbf{Důsledek:}
Pro bezespornou teorii $T$ nad \emph{konečným} jazykem $\mathbb P$ je algebra výroků   $\mathbf{AV}_\mathbb P(T)$ izomorfní potenční algebře $\mathbf{\mathcal P(\M_\mathbb P(T))}$ prostřednictvím zobrazení $h([\varphi]_{\sim_T})=M(T,\varphi)$.
}

\end{frame}


\begin{frame}{Počítání až na ekvivalenci}

    \myblock{
    \textbf{Tvrzení:}
        Mějme $n$-prvkový jazyk $\mathbb P$ a bezespornou teorii $T$ mající právě $k$ modelů. Potom v jazyce $\mathbb P$ existuje \alert{až na ekvivalenci}:
        \begin{itemize}
            \item $2^{2^n}$ výroků (resp. teorií),
            \item $2^{2^n-k}$ výroků pravdivých (resp. lživých) v $T$,
            \item $2^{2^n}-2\cdot 2^{2^n-k}$ výroků nezávislých v $T$,
            \item $2^k$ jednoduchých extenzí teorie $T$ (z toho 1 sporná),
            \item $k$ kompletních jednoduchých extenzí $T$.
        \end{itemize}
        Dále \alert{až na $T$-ekvivalenci} existuje:
        \begin{itemize}
            \item $2^k$ výroků,
            \item $1$ výrok pravdivý v $T$, $1$ lživý v $T$,
            \item $2^k-2$ výroků nezávislých v $T$.
        \end{itemize}
    }

    \textbf{Důkaz:} stačí spočítat možné množiny modelů\hfill\qedsymbol
        
\end{frame}


\section{\sc Kapitola 3: Problém splnitelnosti}


\begin{frame}{Problém splnitelnosti Booleovských formulí}
   
\textbf{Problém SAT:}
\begin{itemize}
    \item vstup: výrok $\varphi$ v CNF
    \item otázka: je $\varphi$ splnitelný? 
\end{itemize}

\alert{univerzální problém}: každou teorii nad konečným jazykem lze převést do CNF

\alert{Cook-Levinova věta}: SAT je NP-úplný (důkaz: formalizuj výpočet nedeterministického Turingova stroje ve výrokové logice)

ale některé \emph{fragmenty} jsou v P, efektivně řešitelné, např. 2-SAT a Horn-SAT (viz Sekce 3.2 a 3.3)

\alert{praktický problém}: moderní \emph{SAT solvery} (viz Sekce 3.1) se používají v řadě odvětví aplikované informatiky, poradí si s obrovskými instancemi
    
\end{frame}


\section{3.1 SAT solvery}


\begin{frame}{SAT solvery}

    \begin{itemize}
        \item existují od 60. let 20. století, v 21. století dramatický rozvoj dnes až $~10^8$ proměnných, viz \href{http://www.satcompetition.org}{\alert{www.satcompetition.org}}.
        \item nejčastěji založeny na jednoduchém \alert{algoritmu DPLL} (viz Sekce 3.4), umí i najít řešení (model)
        \item řada technologií pro efektivnější řešení instancí pocházejících z různých aplikačních domén, heuristiky pro řízení prohledávání (za použití ML, NN) --- desítky tisíc řádků kódu
    \end{itemize}
    
    \textbf{Praktická ukázka:}

    \myexample{\emph{Boardomino:}
    Lze pokrýt šachovnici s chybějícími dvěma protilehlými rohy perfektně pokrýt kostkami domina?
    }

    \vspace{-6pt}
    těžká instance SATu (proč?), jak zakódovat? 
    
    řešič \href{http://www.labri.fr/perso/lsimon/glucose/}{\alert{Glucose}}, formát vstupu: \href{http://people.sc.fsu.edu/~jburkardt/data/cnf/cnf.html}{\alert{DIMACS CNF}}
    
\end{frame}


\section{3.2 2-SAT a implikační graf}


\begin{frame}{2-SAT vs. 3-SAT}

    \begin{itemize}
        \item \alert{$k$-CNF}: CNF a každá klauzule nejvýše $k$ literálů
        \item \alert{$k$-SAT}: je daný $k$-CNF výrok splnitelný?
        \item $k$-SAT je NP-úplný pro $k\geq 3$ (ke každému výroku lze sestrojit \alert{ekvisplnitelný} 3-CNF výrok)
        \item ale \myalertinline{2-SAT je v P, dokonce řešitelný v lineárním čase}
        \item algoritmus využívá tzv. \alert{implikační graf}:         
        \begin{itemize}
            \item 2-klauzule $p\lor q$ je ekvivalentní $\neg p\limplies q$ a také $\neg q\limplies p$
            \item $p\sim p\lor p$ je ekvivalentní $\neg p\limplies p$
            \item vrcholy jsou literály
            \item hrany dané implikacemi
            \item \textbf{myšlenka}: ohodnotíme-li vrchol 1, všude kam se dostaneme po hranách (\alert{komponenta} silné souvislosti) musí být také 1
        \end{itemize}        
    \end{itemize}    
    
\end{frame}


\begin{frame}{Implikační graf}

\begin{itemize}
    \item $V(\mathcal G_\varphi) = \{p, \neg p \mid p\in\Var(\varphi)\}$,
    \item $E(\mathcal G_\varphi) = \{(\overline{\ell_1},\ell_2), (\overline{\ell_2},\ell_1)\mid \ell_1\lor \ell_2\text{ je klauzule }\varphi\}\union\{(\overline{\ell},\ell)\mid \ell \text{ je jednotková klauzule }\varphi\}$
\end{itemize} 

\medskip
\myexamplemath{\vspace{-6pt}\small
$$
    (\neg p_1 \lor p_2) \land (\neg p_2 \lor \neg p_3) 
    \land  (p_1\lor p_3) \land (p_3\lor \neg p_4)\land  (\neg p_1\lor p_5)\land (p_2\lor p_5)\land p_1\land \neg p_4    
$$    
}
\vspace{-18pt}
\begin{columns}
    \begin{column}{0.45\textwidth}
    
    \begin{itemize}
        \item najdeme komponenty silné souvislosti
        \item literály v komponentě musí být ohodnoceny stejně (jinak ``$1\limplies 0$'')
        \item pokud má nějaká komponenta opačné literály, je $\varphi$ nesplnitelný
        \item jinak sestrojíme model:
    \end{itemize}
    
    \end{column}
    \begin{column}{0.55\textwidth}
        \scalebox{0.85}{
        \begin{tikzpicture}[every node/.style={circle,fill=gray!5,draw,minimum width=1cm,node distance=2cm}
        ]
        \node[fill=red!10] (-5) {$\neg p_5$};
        \node[fill=blue!10,right of=-5] (-1) {$\neg p_1$};
        \node[fill=blue!10,above of=-1] (-2) {$\neg p_2$};
        \node[fill=blue!10,below of=-1] (3) {$p_3$};    
        \node[fill=green!10,left of=3] (4) {$p_4$};
        \node[fill=blue!70,right of=-1] (1) {$p_1$}; 
        \node[fill=blue!70,above of=1] (2) {$p_2$};
        \node[fill=blue!70,below of=1] (-3) {$\neg p_3$};       
        \node[fill=red!70,right of=1] (5) {$p_5$};            
        \node[fill=green!70,right of=-3] (-4) {$\neg p_4$};
        \draw[-{Latex[length=2.5mm]}] (-5) -- (-1);
        \draw[-{Latex[length=2.5mm]}] (-1) -- (3);
        \draw[-{Latex[length=2.5mm]}] (3) edge[bend right] (-2);
        \draw[-{Latex[length=2.5mm]}] (-2) -- (-1);
        \draw[-{Latex[length=2.5mm]}] (4) -- (3);
        \draw[-{Latex[length=2.5mm]}] (-1) -- (1);
        \draw[-{Latex[length=2.5mm]}] (-3) -- (1);
        \draw[-{Latex[length=2.5mm]}] (2) edge[bend left] (-3);
        \draw[-{Latex[length=2.5mm]}] (1) -- (2);
        \draw[-{Latex[length=2.5mm]}] (-5) edge[bend left=70] (2);
        \draw[-{Latex[length=2.5mm]}] (4) edge[bend right] (-4);
        \draw[-{Latex[length=2.5mm]}] (1) -- (5);
        \draw[-{Latex[length=2.5mm]}] (-2) edge[bend left=70] (5);
        \draw[-{Latex[length=2.5mm]}] (-3) -- (-4);
        \end{tikzpicture}
        }
    \end{column}    
\end{columns}

\end{frame}


\begin{frame}{Konstrukce modelu}

\myalert{
stačí, aby z žádné komponenty ohodnocené 1 nevedla hrana do komponenty ohodnocené 0
}

\vspace{-9pt}
provedeme \alert{kontrakci komponent}, výsledný graf $\mathcal G_\varphi^\ast$ je \alert{acyklický}

\vspace{-9pt}
\begin{center}
    \scalebox{0.7}{
        \begin{tikzpicture}[every node/.style={circle,fill=gray!5,draw,minimum width=1cm,node distance=2cm}
        ]
        \node[fill=red!10] (C1) {$C_1$};
        \node[fill=blue!10,below right of=C1] (C3) {$C_3$};
        \node[fill=green!10,below left of=C3] (C2) {$C_2$};
        \node[fill=blue!70,right of=C3] (barC3) {$\overline{C_3}$}; 
        \node[fill=red!70,above right of=barC3] (barC1) {$\overline{C_1}$};                    
        \node[fill=green!70,below right of=barC3] (barC2) {$\overline{C_2}$};
        \draw[-{Latex[length=2.5mm]}] (C1) -- (C3);
        \draw[-{Latex[length=2.5mm]}] (C2) -- (C3);
        \draw[-{Latex[length=2.5mm]}] (C3) -- (barC3);
        \draw[-{Latex[length=2.5mm]}] (barC3) -- (barC1);
        \draw[-{Latex[length=2.5mm]}] (barC3) -- (barC2);
        \draw[-{Latex[length=2.5mm]}] (C1) edge[bend left] (barC3);
        \draw[-{Latex[length=2.5mm]}] (C3) edge[bend left] (barC1);
        \draw[-{Latex[length=2.5mm]}] (C2) edge[bend right] (barC2);
        \end{tikzpicture}
    }
\end{center}
\vspace{-9pt}


najdeme nějaké \alert{topologické uspořádání}; v něm najdeme nejlevější dosud neohodnocenou komponentu, ohodnotíme ji 0, opačnou komponentu ohodnotíme 1, a opakujeme

\vspace{-9pt}
\begin{center}
\scalebox{0.7}{
    \begin{tikzpicture}[every node/.style={circle,fill=gray!5,draw,minimum width=1cm,node distance=2cm}
        ]
        \node[fill=red!10,label={below:0}] (C1) {$C_1$};
        \node[fill=green!10,right of=C1,label={below:0}] (C2) {$C_2$};
        \node[fill=blue!10,right of=C2,label={below:0}] (C3) {$C_3$};
        \node[fill=blue!70,right of=C3,label={below:1}] (barC3) {$\overline{C_3}$}; 
        \node[fill=green!70,right of=barC3,label={below:1}] (barC2) {$\overline{C_2}$};
        \node[fill=red!70,right of=barC2,label={below:1}] (barC1) {$\overline{C_1}$};                    
        
        \draw[-{Latex[length=2.5mm]}] (C1) edge[bend left] (C3);
        \draw[-{Latex[length=2.5mm]}] (C2) -- (C3);
        \draw[-{Latex[length=2.5mm]}] (C3) -- (barC3);
        \draw[-{Latex[length=2.5mm]}] (barC3) edge[bend left] (barC1);
        \draw[-{Latex[length=2.5mm]}] (barC3) -- (barC2);
        \draw[-{Latex[length=2.5mm]}] (C1) edge[bend left] (barC3);
        \draw[-{Latex[length=2.5mm]}] (C3) edge[bend left] (barC1);
        \draw[-{Latex[length=2.5mm]}] (C2) edge[bend left] (barC2);
        \end{tikzpicture}
    }
\end{center}

\end{frame}


\begin{frame}{Shrnutí}
     
\myblock{
\textbf{Tvrzení:} $\varphi$ je splnitelný, právě když žádná silně souvislá komponenta v $\mathcal G_\varphi$ neobsahuje dvojici opačných literálů.
}

\vspace{-12pt}
\textbf{Důkaz:} $\Rightarrow$ literály v komponentě musí být ohodnoceny stejně

$\Leftarrow$ ohodnocení zkonstruované výše je model $\varphi$:
\begin{itemize}
    \item \alert{jednotková} klauzule $\ell$ platí\myalertinline{
        kvůli hraně ${\ell}\limplies\ell$}, komponenta s $\overline{\ell}$ byla ohodnocena dříve, a to 0, takže $v(\ell)=1$
    \item podobně pro \alert{2-klauzuli} $\ell_1\lor\ell_2$,\myalertinline{
        máme hrany $\overline{\ell_1}\limplies\ell_2$, $\overline{\ell_2}\limplies\ell_1$}   
    
    pokud jsme $\ell_1$ ohodnotili dříve než $\ell_2$, museli jsme jako první narazit na komponentu s $\overline{\ell_1}$ a ohodnotit ji 0, tedy $\ell_1$ platí; v~opačném případě symetricky platí $\ell_2$ \hfill\qedsymbol
\end{itemize}


\myblock{
\textbf{Důsledek:} 2-SAT je řešitelný v lineárním čase, včetně konstrukce modelu (pokud existuje). 
}

\vspace{-12pt}
\textbf{Důkaz:}
Komponenty silné souvislosti i topologické uspořádání najdeme v čase $\mathcal O(|V|+|E|)$, stačí je projít jednou \hfill\qedsymbol

\end{frame}


\section{3.3 Horn-SAT a jednotková propagace}


\begin{frame}{Horn-SAT}

    \begin{itemize}
        \item \alert{hornovská klauzule}: \emph{nejvýše jeden *pozitivní* literál}
        $$
        \alert{\neg p_1\lor \neg p_2\lor \dots \lor \neg p_n\lor q}\ \sim\ (p_1\land p_2\land\dots \land p_n)\limplies q
        $$
        základ logického programování (Prolog \myblockinline{\texttt{q:-p1,p2,...,pn.}})
        \item \myalertinline{Horn-SAT}, tj. splnitelnost \alert{hornovského} výroku (konjunkce hornovských klauzulí) \myalertinline{je opět v P, v lineárním čase}
        \item algoritmus využívá tzv. \alert{jednotkovou propagaci}:  
        \begin{itemize}
            \item jednotková klauzule vynucuje hodnotu výrokové proměnné
            \item tím můžeme výrok zjednodušit, např. pro $\neg p$ ($p=0$):\\
            odstraníme klauzule s literálem $\neg p$, už jsou splněné\\
            odstraníme literál $p$ (nemůže být splněný)
            \item žádná jednotková klauzule $\Rightarrow$ každá klauzule má \alert{aspoň jeden negativní literál} $\Rightarrow$ vše nastavíme na 0
        \end{itemize}    
    \end{itemize}    
    
\end{frame}


\begin{frame}{Jednotková propagace}

\myexamplemath{\vspace{-9pt}
    $$
    \varphi=(\neg p_1\lor p_2)\land(\neg p_1\lor\neg p_2\lor p_3)\land(\neg p_2\lor\neg p_3)\land(\neg p_5\lor \neg p_4)\land \alert{p_4}
    $$
}

\vspace{-12pt}
\begin{itemize}
    \item nastav $v(p_4)=1$, odstraň klauzule obsahující literál $p_4$, z ostatních klauzulí odstraň $\neg p_4$
    $$
    \varphi^{p_4}=(\neg p_1\lor p_2)\land(\neg p_1\lor\neg p_2\lor p_3)\land (\neg p_2\lor\neg p_3)\land\alert{\neg p_5}
    $$
    \item nastav $v(p_5)=0$, proveď jednotkovou propagaci $\neg p_5$
    $$
    (\varphi^{p_4})^{\neg p_5}=(\neg p_1\lor p_2)\land(\neg p_1\lor\neg p_2\lor p_3)\land(\neg p_2\lor\neg p_3)
    $$
    \item už žádná jednotková klauzule, v každé klauzuli alespoň dva literály ale \alert{nejvýše jeden pozitivní, tj. alespoň jeden negativní}: $v(p_1)=v(p_2)=v(p_3)=0$, model $v=(0,0,0,1,0)$
\end{itemize}

\myalertmath{
    $$
    \varphi^\ell=\{C\setminus\{\overline{\ell}\}\mid C\in\varphi,\ell\notin C\}\qquad\text{(množinový zápis)}
    $$
}
\myblock{
\textbf{Pozorování:} $\varphi^\ell$ neobsahuje $\ell$ ani $\overline{\ell}$, modely = modely $\varphi$ splňující $\ell$
}

\vspace{-12pt}
\myexampleinline{
    $\psi=p\land (\neg p\lor q)\land (\neg q\lor r)\land\neg r$} je nesplnitelný, co se stane?

\end{frame}


\begin{frame}{Algoritmus pro Horn-SAT}

    \textbf{vstup}: výrok $\varphi$ v Hornově tvaru,\\ \textbf{výstup}: model $\varphi$ nebo informace, že $\varphi$ není splnitelný
    \begin{enumerate}
        \item Pokud $\varphi$ obsahuje dvojici opačných jednotkových klauzulí $\ell,\overline{\ell}$, není splnitelný.
        \item Pokud $\varphi$ neobsahuje žádnou jednotkovou klauzuli, je splnitelný, ohodnoť všechny zbývající proměnné 0.
        \item Pokud $\varphi$ obsahuje jednotkovou klauzuli $\ell$, ohodnoť literál $\ell$ hodnotou 1, proveď jednotkovou propagaci, nahraď $\varphi$ výrokem $\varphi^\ell$, a vrať se na začátek.
    \end{enumerate}

\myblock{
\textbf{Tvrzení:} Algoritmus je korektní.\\
\textbf{Důsledek:} Horn-SAT lze řešit v lineárním čase.\vspace{3pt}
}

\vspace{-12pt}
\textbf{Důkaz:} {\small Korektnost plyne z pozorování a z diskuze. V každém kroku stačí projít, výrok zkrátíme (kvadratický horní odhad, ale při vhodné implementaci lineární)}

\end{frame}


\section{3.4 DPLL algoritmus pro řešení problému SAT}


\begin{frame}{Algoritmus DPLL (Davis-Putnam-Logemann-Loveland, 1961)}
\alert{čistý výskyt} $p$ = buď jen v pozitivních nebo jen v negativních literálech (lze mu nastavit příslušnou hodnotu!)

\myblock{
DPLL = jednotková propagace + čistý výskyt + větvení (rekurze)
}
        \textbf{vstup}: výrok $\varphi$ v CNF,\\ 
        \textbf{výstup}: model $\varphi$ nebo informace, že  $\varphi$ není splnitelný
        \begin{enumerate}                
            \item Dokud $\varphi$ obsahuje jednotkovou klauzuli $\ell$, ohodnoť literál $\ell$ hodnotou 1, proveď \alert{jednotkovou propagaci}, a nahraď $\varphi$ výrokem $\varphi^\ell$.
             \item Dokud existuje literál $\ell$, který má ve $\varphi$ \alert{čistý výskyt}, ohodnoť $\ell$ hodnotou 1, a odstraň klauzule obsahující $\ell$.
            \item Pokud $\varphi$ neobsahuje žádnou klauzuli, je splnitelný.
            \item Pokud $\varphi$ obsahuje prázdnou klauzuli, není splnitelný.
            \item Jinak zvol dosud neohodnocenou výrokovou proměnnou $p$, a \alert{zavolej algoritmus rekurzivně} na $\varphi\land p$ a na $\varphi\land \neg p$.
        \end{enumerate}

\end{frame}


\begin{frame}{Ukázkový běh}

\medskip

\myexamplemath{\small\vspace{-6pt}
    \begin{align*}
        &(\neg p\lor q\lor \neg r)\land(\neg p\lor\neg q\lor\neg s)\land(p\lor \neg r\lor \neg s)\land \\ &(q\lor \neg r\lor s)\land (p\lor s)\land(p\lor\neg s)\land(q\lor s)
    \end{align*}
}

žádná jednotková klauzule, $\neg r$ má \alert{čistý výskyt}: nastav  $v(r)=0$ a odstraň klauzule obsahující $\neg r$:
    $$
    (\neg p\lor\neg q\lor\neg s)\land (p\lor s)\land(p\lor\neg s)\land(q\lor s)
    $$
    už žádný čistý výskyt, rekurzivně zavolej na:
    \begin{enumerate}
        \item $(\neg p\lor\neg q\lor\neg s)\land (p\lor s)\land(p\lor\neg s)\land(q\lor s)\alert{\land p}$
        \item $(\neg p\lor\neg q\lor\neg s)\land (p\lor s)\land(p\lor\neg s)\land(q\lor s)\alert{\land \neg p}$
    \end{enumerate}
    a pokračuj dále v obou větvích výpočtu
    \begin{gather*}
        \vdots \\
        \M_\varphi=\{(1,a,0,b,c)\mid a,b,c\in\{0,1\}\}
    \end{gather*}
 
    
\end{frame}


\end{document}

