

\begin{proposition}\label{proposition:semantic-conditions-for-extensions}
    Mějme jazyky $L\subseteq L'$, teorii $T$ v jazyce $L$, a teorii $T'$ v jazyce $L'$.
    \begin{enumerate}[(i)]
        \item $T'$ je extenzí teorie $T$, právě když redukt každého modelu $T'$ na jazyk $L$ je modelem $T$.
        \item Pokud je $T'$ extenzí teorie $T$, a každý model $T$ lze expandovat do jazyka $L'$ na nějaký model teorie $T'$, potom je $T'$ je konzervativní extenzí teorie $T$.
    \end{enumerate}
\end{proposition}
\begin{remark}
    V části (ii) platí i opačná implikace, důkaz ale není tak jednoduchý, jako ve výrokové logice, a proto ho neuvedeme. (Problémem je jak získat z modelu $T$ který nelze expandovat na model $T'$ $L$-sentenci, která platí v $T$ ale ne v $T'$.)
\end{remark} 
%\todo ukázat i opačnou implikaci?
%$T'$ je konzervativní extenzí teorie $T$, právě když je extenzí, a každý model $T$ lze expandovat do jazyka $L'$ na nějaký model teorie $T'$.
\begin{proof}
    Nejprve dokažme (i): Mějme model $\A'$ teorie $T'$ a označme jako $\A$ jeho redukt na jazyk $L$. Protože $T'$ je extenzí teorie $T$, platí v $T'$, a tedy i v $\A'$, každý axiom $\varphi\in T$. Potom ale i $\A\models\varphi$ ($\varphi$ obsahuje jen symboly z jazyka $L$), tedy $\A$ je modelem $T$.

    Na druhou stranu, mějme $L$-sentenci $\varphi$ takovou, že $T\models\varphi$. Chceme ukázat, že $T'\models\varphi$. Pro libovolný model $\A'\in\M_{L'}(T')$ víme, že jeho $L$-redukt $\A$ je modelem $T$, tedy $\A\models\varphi$. Z toho plyne i $\A'\models\varphi$ (opět proto, že $\varphi$ je v jazyce $L$).

    Nyní (ii): Vezměme libovolnou $L$-sentenci $\varphi$, která platí v teorii $T'$, a ukažme, že platí i v $T$. Každý model $\A$ teorie $T$ lze expandovat na nějaký model $\A'$ teorie $T'$. Víme, že $\A'\models\varphi$, takže i $\A\models\varphi$. Tím jsme dokázali, že $T\models\varphi$, tj. jde o konzervativní extenzi.
    
\end{proof}


\subsection{Extenze o definice}\label{subsection:extension-by-definition}

Nyní si ukážeme speciální druh konzervativní extenze, tzv. extenzi \emph{o definice} nových (relačních, funkčních, konstantních) symbolů.


\subsubsection*{Definice relačního symbolu}

Nejjednodušším případem je definování nového relačního symbolu $R(x_1,\dots,x_n)$. Jako definice může sloužit libovolná formule s $n$ volnými proměnnými $\psi(x_1,\dots,x_n)$.

\begin{example}
Uveďme nejprve několik příkladů:
\begin{itemize}
    \item Jakoukoliv teorii v jazyce s rovností můžeme rozšířit o binární relační symbol $\neq$, který \emph{definujeme} formulí $\neg x_1=x_2$. To znamená, že požadujeme, aby platilo: $x_1\neq x_2\liff\neg x_1=x_2$.
    \item Teorii uspořádání můžeme rozšířit o symbol $<$ pro ostré uspořádání, který \emph{definujeme} formulí $x_1\leq x_2\land \neg x_1=x_2$. To znamená, že požadujeme, aby platilo $x_1<x_2 \liff x_1\leq x_2\land \neg x_1=x_2$.
    \item V aritmetice můžeme zavést symbol $\leq$, pomocí $x_1\leq x_2\liff(\exists y)(x_1+y=x_2)$.
\end{itemize}
\end{example}
Nyní uvedeme definici:
\begin{definition}[Definice relačního symbolu]
    Mějme teorii $T$ a formuli $\psi(x_1,\dots,x_n)$ v jazyce $L$. Označme jako $L'$ rozšíření jazyka $L$ o nový $n$-ární relační symbol $R$. \emph{Extenze teorie $T$ o definici $R$ formulí $\psi$} je $L'$-teorie:
    $$
    T'=T\cup\{R(x_1,\dots,x_n)\ \liff\ \psi(x_1,\dots,x_n)\}
    $$
\end{definition}
Všimněte si, že každý model $T$ lze \emph{jednoznačně} expandovat na model $T'$. Z Tvrzení \ref{proposition:semantic-conditions-for-extensions} potom ihned plyne následující:
\begin{corollary}
    $T'$ je konzervativní extenze $T$.
\end{corollary}

Ukážeme si ještě, že nový symbol lze ve formulích nahradit jeho definicí, a získat tak ($T'$-ekvivalentní) formuli v původním jazyce:

\begin{proposition}
    Pro každou $L'$-formuli $\varphi'$ existuje $L$-formule $\varphi$ taková, že $T'\models\varphi'\liff\varphi$.
\end{proposition}
\begin{proof}
    Je třeba nahradit atomické podformule s novým symbolem $R$, tj. tvaru $R(t_1,\dots,t_n)$. Takovou podformuli nahradíme formulí $\psi'(x_1/t_1,\dots,x_n/t_n)$, kde $\psi'$ je varianta $\psi$ zaručující substituovatelnost všech termů, tj. například přejmenujeme všechny vázané proměnné $\psi$ na zcela nové (nevyskytující se ve formuli $\varphi'$).
\end{proof}

\subsubsection*{Definice funkčního symbolu}

Nový funkční symbol definujeme obdobným způsobem, musíme si ale být jisti, že definice dává jednoznačnou možnost, jak nový symbol interpretovat.

\begin{example}
    Opět začneme příklady:
    \begin{itemize}
        \item V teorii grup můžeme zavést \emph{binární} funkční symbol $-_b$ pomocí $+$ a unárního $-$ takto:
        $$
        x_1 -_b x_2 = y\ \liff\ x_1 + (-x_2) = y
        $$
        Je zřejmé, že pro každá $x,y$ \emph{existuje} \emph{jednoznačné} $z$ splňující definici.        
        \item Uvažme teorii \emph{lineárních uspořádání}, tj. teorii uspořádání spolu s axiomem linearity $x\leq y\lor y\leq x$. Definujme binární funkční symbol $\min$ takto: 
        $$
        \min(x_1,x_2)=y\ \liff\ y\leq x_1\land y\leq x_2\land (\forall z)(z\leq x_1\land z\leq x_2\limplies z\leq y)
        $$
        Existence a jednoznačnost platí díky linearitě. Pokud bychom ale měli pouze teorii uspořádání, taková formule by nebyla dobrou definicí: v některých modelech by $\min(x_1,x_2)$ pro některé prvky neexistovalo, selhala by tedy požadovaná \emph{jednoznačnost}.
    \end{itemize}
\end{example}

\begin{definition}[Definice funkčního symbolu]
    Mějme teorii $T$ a formuli $\psi(x_1,\dots,x_n,y)$ v jazyce $L$. Označme jako $L'$ rozšíření jazyka $L$ o nový $n$-ární funkční symbol $f$. Nechť v teorii $T$ platí:
    \begin{itemize}
        \item \emph{axiom existence} $(\exists y)\psi(x_1,\dots,x_n,y)$, 
        \item \emph{axiom jednoznačnosti} $\psi(x_1,\dots,x_n,y)\land \psi(x_1,\dots,x_n,z)\limplies y=z$.
    \end{itemize}
    Potom \emph{extenze teorie $T$ o definici $f$ formulí $\psi$} je $L'$-teorie: 
    $$T'=T\cup\{f(x_1,\dots,x_n)=y\ \liff\ \psi(x_1,\dots,x_n,y)\}$$
\end{definition}

Formule $\psi$ tedy definuje v každém modelu $(n+1)$-ární relaci, a po této relaci požadujeme, aby byla funkcí, tj. aby pro každou $n$-tici prvků existovala jednoznačná možnost, jak ji rozšířit do $(n+1)$-tice, která je prvkem této relace. Všimněte si, že je-li definující formule $\psi$ tvaru $t(x_1,\dots,x_n)=y$, kde $x_1,\dots,x_n$ jsou proměnné $L$-termu $t$, potom axiomy existence a jednoznačnosti vždy platí.

Opět platí, že každý model $T$ lze \emph{jednoznačně} expandovat na model $T'$, tedy:
\begin{corollary}
    $T'$ je konzervativní extenze $T$.
\end{corollary}

A platí také analogické tvrzení o rozvádění definic:

\begin{proposition}
    Pro každou $L'$-formuli $\varphi'$ existuje $L$-formule $\varphi$ taková, že $T'\models\varphi'\liff\varphi$.
\end{proposition}
\begin{proof}
    Stačí dokázat pro formuli $\varphi'$ s jediným výskytem symbolu $f$; je-li výskytů více, aplikujeme postup induktivně, v případě vnořených výskytů v jednom termu $f(\dots f(\dots)\dots)$ postupujeme od vnitřních k vnějším.

    Označme $\varphi^*$ formuli vzniklou z $\varphi'$ nahrazením termu $f(t_1,\dots,t_n)$ \emph{novou} proměnnou $z$. Formuli $\varphi$ zkonstruujeme takto:
    $$
    (\exists z)(\varphi^*\land \psi'(x_1/t_1,\dots,x_n/t_n,y/z))
    $$
    kde $\psi'$ je varianta $\psi$ zaručující substituovatelnost všech termů.

    Mějme model $\A$ teorie $T'$ a ohodnocení $e$. Označme $a=f^\A(t_1,\dots,t_n)[e]$. Díky existenci a jednoznačnosti platí:
    $$
    \A\models\psi'(x_1/t_1,\dots,x_n/t_n,y/z)[e]\ \text{ právě když }\ e(z)=a 
    $$
    Máme tedy $\A\models\varphi[e]$, právě když $\A\models\varphi^*[e(z/a)]$, právě když $\A\models\varphi'[e]$. To platí pro libovolné ohodnocení $e$, tedy $\A\models\varphi'\liff\varphi$ pro každý model $T'$, tedy $T'\models\varphi'\liff\varphi$.
\end{proof}

\subsubsection*{Definice konstantního symbolu}

Konstantní symbol je speciálním případem funkčního symbolu arity $0$. Platí tedy stejná tvrzení. Axiomy existence a jednoznačnosti jsou: $(\exists y)\psi(y)$ a $\psi(y)\land\psi(z)\limplies y=z$. A extenze o definici konstantního symbolu $c$ formulí $\psi(y)$ je teorie $T'=T\cup\{c=y\liff \psi(y)\}$.

\begin{example}
    Ukážeme si dva příklady:
    \begin{itemize}
        \item Teorii aritmetiky můžeme rozšířit o definici konstantního symbolu $1$ formulí $\psi(y)$ tvaru $y=S(0)$, přidáme tedy axiom $1=y\ \liff\ y=S(0)$.
        \item Uvažme teorii těles a nový symbol $\frac{1}{2}$, definovaný formulí $y\cdot (1+1)=1$, tj. přidáním axiomu: 
        $$
        \frac{1}{2}=y\ \liff\ y\cdot (1+1)=1
        $$
        Zde nejde o korektní extenzi o definici, neboť neplatí axiom existence. Ve dvouprvkovém tělese $\mathbb Z_2$ (a v každém tělese \emph{charakteristiky 2}) nemá rovnice $y\cdot (1+1)=1$ řešení, neboť $1+1=0$.
        
        
        Pokud ale vezmeme teorii těles charakteristiky různé od 2, tj. přidáme-li k teorii těles axiom $\neg (1+1=0)$, potom už půjde o korektní extenzi o definici. Například v tělese $\mathbb Z_3$ máme $\frac{1}{2}^{\mathbb Z_3}=2$.
    \end{itemize}
\end{example}


\subsubsection*{Extenze o definice}

Máme-li $L$-teorii $T$ a $L'$-teorii $T'$, potom řekneme, že $T'$ je  \emph{extenzí} $T$ \emph{o definice}, pokud vznikla z $T$ postupnou extenzí o definice relačních a funkčních (příp. konstantních) symbolů. Vlastnosti, které jsme dokázali o extenzích o jeden symbol (ať už relační nebo funkční), se snadno rozšíří indukcí na více symbolů:

\begin{corollary}
   Je-li $T'$ extenze teorie $T$ o definice, potom platí:
   \begin{itemize}
    \item Každý model teorie $T$ lze jednoznačně expandovat na model $T'$.
    \item $T'$ je konzervativní extenze $T$.
    \item Pro každou $L'$-formuli $\varphi'$ existuje $L$-formule $\varphi$ taková, že $T'\models\varphi'\liff\varphi$.
   \end{itemize} 
\end{corollary}

Na závěr ještě jeden příklad, na kterém si ukážeme i rozvádění definic:
\begin{example}
    V teorii $T=\{(\exists y)(x+y=0),(x+y=0)\land(x+z=0)\limplies y=z\}$ jazyka $L=\langle +,0,\leq\rangle$ s rovností lze zavést $<$ a unární funkční symbol $-$ přidáním axiomů:
    \begin{align*}
        -x=y\ &\liff\ x+y=0\\
        x<y\ &\liff\ x\leq y\land\neg(x=y)
    \end{align*}
    Formule $-x<y$ (v jazyce $L'=\langle +,-,0,\leq,<\rangle)$ s rovností) je v této extenzi o definice ekvivalentní následující formuli:
    $$
    (\exists z)((z\leq y\land\neg(z=y))\land x+z=0)
    $$
\end{example}


\section{Definovatelnost ve struktuře}\label{section:definability}

Formuli s jednou volnou proměnnou $x$ můžeme chápat jako \emph{vlastnost} prvků. V dané struktuře taková formule \emph{definuje} množinu prvků, které tuto vlastnost splňují, tj. takových, že formule platí při ohodnocení $e$, ve kterém $e(x)=a$. Máme-li formuli se dvěma volnými proměnnými, definuje binární relaci, atp. Nyní tento koncept formalizujeme. Připomeňme, že zápis $\varphi(x_1,\dots,x_n)$ znamená, že $x_1,\dots,x_n$ jsou právě všechny volné proměnné formule $\varphi$.

\begin{definition}[Definovatelné množiny]
    Mějme formuli $\varphi(x_1,\dots,x_n)$ a strukturu $\A$ v témž jazyce. \emph{Množina definovaná formulí $\varphi(x_1,\dots,x_n)$ ve struktuře $\A$}, značíme $\varphi^\A(x_1,\dots,x_n)$, je:
    $$
    \varphi^\A(x_1,\dots,x_n)=\{(a_1,\dots,a_n)\in A^n\mid\A\models\varphi[e(x_1/a_1,\dots,x_n/a_n)]\}
    $$
\end{definition}
Zkráceně totéž zapíšeme také jako $\varphi^\A(\bar x)=\{\bar a\in A^n\mid\A\models\varphi[e(\bar x/\bar a)]\}$.

\begin{example} Uveďme několik příkladů:
    \begin{itemize}
        \item Formule $\neg(\exists y)E(x,y)$ definuje množinu všech \emph{izolovaných} vrcholů v daném grafu.
        \item Uvažme těleso reálných čísel $\underline{\mathbb R}$. Formule $(\exists y)(y\cdot y=x)\land\neg (x=0)$ definuje množinu všech kladných reálných čísel.
        \item Formule $x\leq y\land \neg (x=y)$ definuje v dané uspořádané množině $\langle S,\leq^S\rangle$ relaci \emph{ostrého uspořádání} $<^S$. 
    \end{itemize}
\end{example}

Často se také hodí mluvit o vlastnostech prvků relativně k jiným prvkům dané struktury. To nelze vyjádřit čistě syntakticky, ale můžeme za některé z volných proměnných dosadit prvky struktury jako \emph{parametry}. Zápisem $\varphi(\bar x,\bar y)$ myslíme, že formule $\varphi$ má volné proměnné $x_1,\dots,x_n,y_1,\dots,y_k$ (pro nějaká $n,k$).

\begin{definition}
    Mějme formuli $\varphi(\bar x,\bar y)$, kde $|\bar x|=n$ a $|\bar y|=k$, strukturu $\A$ v témž jazyce, a $k$-tici prvků $\bar b\in A^k$. \emph{Množina definovaná formulí $\varphi(\bar x,\bar y)$ s parametry $\bar b$ ve struktuře $\A$}, značíme $\varphi^{\A,\bar b}(\bar x,\bar y)$, je:
    $$
    \varphi^{\A,\bar b}(\bar x,\bar y)=\{\bar a\in A^n\mid\A\models\varphi[e(\bar x/\bar a,\bar y/\bar b)]\}
    $$
    Pro strukturu $\A$ a podmnožinu $B\subseteq A$ označíme $\mathrm{Df}^n(\A,B)$ množinu všech množin definovatelných ve struktuře $\A$ s parametry pocházejícími z $B$.
\end{definition}


\begin{example}
    Pro  $\varphi(x,y)=E(x,y)$ je $\varphi^{\mathcal G,v}(x,y)$ množina všech sousedů vrcholu $v$.
\end{example}

\begin{observation}
Množina $\mathrm{Df}^n(\A,B)$ je uzavřená na doplněk, průnik, sjednocení, a obsahuje $\emptyset$ a $A^n$. Jde tedy o podalgebru potenční algebry $\mathcal P(A^n)$.
\end{observation}

\subsection{Databázové dotazy}

Definovatelnost nachází přirozenou aplikaci v relačních databázích, např. ve známém dotazovacím jazyce SQL. \emph{Relační databáze} sestává z jedné nebo více \emph{tabulek}, někdy se jim říká \emph{relace}, řádky jedné tabulky jsou \emph{záznamy (records)}, nebo také \emph{tice (tuples)}. Jde tedy v principu o strukturu v čistě relačním jazyce. Představme si databázi obsahující dvě tabulky, Program a Movies, znázorněné na Obrázku \ref{figure:datbase}.

\begin{figure}[htbp]
\begin{multicols}{2}
    \ttfamily\small
    \begin{tabular}{lll}
        cinema         & title          & time   \\ \hline
        Atlas          & Forrest Gump   & 20:00  \\
        Lucerna        & Forrest Gump   & 21:00  \\
        Lucerna        & Philadelphia   & 18:30  \\
        \vdots         & \vdots         & \vdots
    \end{tabular}

    \begin{tabular}{lll}
        title          & director    & actor \\ \hline
        Forrest Gump   & R. Zemeckis & T. Hanks      \\
        Philadelphia   & J. Demme    & T. Hanks      \\
        Batman Returns & T. Burton   & M. Keaton     \\
        \vdots         & \vdots      & \vdots
    \end{tabular}    
\end{multicols}
\caption{Tabulky Program a Movies}
\label{figure:datbase}
\end{figure}

SQL dotaz ve své nejjednodušší formě (pomineme-li např. \emph{agregační funkce}) je v podstatě formule, a výsledkem dotazu je množina definovaná touto formulí (s parametry). Například, kdy a kde můžeme vidět film s Tomem Hanksem?

\begin{quote}
    \textbf{select} Program.cinema, Program.time \textbf{from} Program, Movies \textbf{where}\\ Program.title = Movies.title  \textbf{and} Movies.actor = `T. Hanks'   
\end{quote}

Výsledkem bude množina $\varphi^{\text{Database},\text{`T. Hanks'}} (x_\mathrm{cinema},x_\mathrm{time},y_\mathrm{actor})$ definovaná ve struktuře $\text{Database}=\langle D, \mathrm{Program}, \mathrm{Movies}\rangle$, kde $D=\{\text{`Atlas'},\text{`Lucerna'},\dots,\text{`M. Keaton'}\}$, s parametrem `T. Hanks' následující formulí $\varphi(x_\mathrm{cinema},x_\mathrm{time},y_\mathrm{actor})$ :
$$
(\exists y_\mathrm{title})(\exists y_\mathrm{director})(\mathrm{Program}(x_\mathrm{cinema},y_\mathrm{title},x_\mathrm{time}) \land \mathrm{Movies}(y_\mathrm{title},y_\mathrm{director},\text{`T. Hanks'}))
$$


\section{Vztah výrokové a predikátové logiky}
\label{section:relationship-propositional-predicate-logic}

Nyní si ukážeme, jak lze výrokovou logiku `simulovat' v logice predikátové, a to v teorii Booleových algeber. Nejprve představíme axiomy této teorie:

\begin{definition}[Booleovy algebry]
    \emph{Teorie Booleových algeber} je teorie jazyka $L=\langle -,\landsymb,\lorsymb,\bot,\top\rangle$ s rovností sestávající z následujících axiomů:\footnote{Všimněte si \emph{duality}: záměnou $\land$ s $\lor$ a $\bot$ s $\top$ získáme tytéž axiomy.}
    \begin{multicols}{2}
        \begin{itemize}
            \item \emph{asociativita} $\land$ a $\lor$:
            \begin{align*}
                x\land(y\land z) &=(x\land y)\land z\\
                x\lor(y\lor z) &=(x\lor y)\lor z
            \end{align*}
            \item \emph{komutativita} $\land$ a $\lor$:
            \begin{align*}
                x\land y &= y\land x\\
                x\lor y &= y\lor x
            \end{align*}
            \item \emph{distributivita} $\land$ vůči $\lor$ a $\lor$ vůči $\land$:
            \begin{align*}
                x\land(y\lor z) &= (x\land y)\lor (x\land z)\\
                x\lor(y\land z) &= (x\lor y)\land (x\lor z)
            \end{align*}
            \item \emph{absorpce}:
            \begin{align*}
                x\land(x\lor y) &= x\\
                x\lor(x\land y) &= x
            \end{align*}
            \item \emph{komplementace}:
            \begin{align*}
                x\land(-x) &= \bot \\
                x\lor(-x) &= \top
            \end{align*}
            \item \emph{netrivialita}:
            \begin{align*}
                \neg (\bot &= \top)
            \end{align*}

        \end{itemize}
    \end{multicols}
    Nejmenším modelem je \emph{2-prvková Booleova algebra} $\langle \{0,1\},f_\neg,f_\land,f_\lor,0,1\rangle$. Konečné Booleovy algebry jsou (až na \emph{izomorfismus}) právě $\langle \{0,1\}^n,f_\neg^n,f_\land^n,f_\lor^n,(0,\dots,0),(1,\dots,1)\rangle$, kde $f^n$ znamená, že funkci $f$ aplikujeme po složkách.\footnote{Tyto Booleovy algebry jsou izomorfní \emph{potenčním algebrám} $\mathcal P(\{1,\dots,n\})$, izomorfismus je daný bijekcí mezi podmnožinami a jejich charakteristickými vektory.}

    Výroky tedy můžeme chápat jako \emph{Booleovské termy} (a konstanty $\bot,\top$ představují pravdu a lež), pravdivostní hodnota výroku při daném ohodnocení prvovýroků je potom dána hodnotou odpovídajícího termu v 2-prvkové Booleově algebře.   Kromě toho, \emph{algebra výroků} daného výrokového jazyka nebo teorie je Booleovou algebrou (to platí i pro nekonečné jazyky).

    Na druhou stranu, máme-li \emph{otevřenou} formuli $\varphi$ (bez rovnosti), můžeme reprezentovat atomické výroky pomocí prvovýroků, a získat tak výrok, který platí, právě když platí $\varphi$. Více o tomto směru se dozvíme v Kapitole \ref{chapter:predicate-resolution} (o rezoluci v predikátové logice), kde se nejprve zbavíme kvantifikátorů pomocí tzv. \emph{Skolemizace}.

    Výrokovou logiku bychom také mohli zavést jako fragment logiky predikátové, pokud bychom povolili \emph{nulární relace} (a nulární relační symboly v jazyce): $A^0=\{\emptyset\}$, tedy na libovolné množině jsou právě dvě nulární relace $R^A\subseteq A^0$: $R^A=\emptyset=0$ a $R^A=\{\emptyset\}=\{0\}=1$. To ale dělat nebudeme. 
    
\end{definition}
