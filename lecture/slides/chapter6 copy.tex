
\begin{exercise}
    Dokažte podrobně všechny uvedené vlastnosti platnosti při ohodnocení.
\end{exercise}

A jak je tomu s pojmem pravdivosti (platnosti) ve struktuře?
\begin{itemize}
    \item Pokud $\A\models\varphi$, potom $\A\not\models\neg\varphi$. Je-li $\varphi$ sentence, potom platí i opačná implikace (tj. platí `právě když').
    \item $\A\models\varphi\land\psi$ právě když $\A\models\varphi$ a $\A\models\psi$,
    \item Pokud $\A\models\varphi$ nebo $\A\models\psi$, potom $\A\models\varphi\lor\psi$. Je-li $\varphi$ sentence, potom platí i opačná implikace (tj. platí `právě když').
    \item $\A\models\varphi$ právě když $\A\models
    (\forall x)\varphi$.
\end{itemize}
\emph{Generální uzávěr} formule $\varphi(x_1,\dots,x_n)$ (tj. $x_1,\dots,x_n$ jsou všechny volné proměnné formule $\varphi$) je sentence $(\forall x_1)\cdots(\forall x_n)\varphi$. Z posledního bodu plyne, že formule platí ve struktuře, právě když v ní platí její generální uzávěr.

\begin{exercise}
    Dokažte podrobně všechny uvedené vlastnosti platnosti ve struktuře.
\end{exercise}

\begin{exercise}
    Najděte příklad struktury $\A$ a formule $\varphi$ takových, že $\A\not\models\varphi$ a zároveň $\A\not\models\neg\varphi$.
\end{exercise}

\begin{exercise}
    Najděte příklad struktury $\A$ a formulí $\varphi,\psi$ takových, že $\A\models\varphi\lor\psi$ ale $\A\not\models\varphi$ ani $\A\not\models\psi$.
\end{exercise}

\section{Vlastnosti teorií}

Na základě pojmu \emph{platnosti} vybudujeme syntaktickou terminologii obdobně jako ve výrokové logice. \emph{Teorie} jazyka $L$ je libovolná množina $T$ $L$-formulí, prvkům teorie říkáme \emph{axiomy}. \emph{Model} teorie $T$ je $L$-struktura, ve které platí všechny axiomy teorie $T$, tj. $\A\models\varphi$ pro všechna $\varphi\in T$, což značíme $\A\models T$. \emph{Třída modelů}\footnote{Připomeňme, že nemůžeme říkat `množina'.} teorie $T$ je:
$$
\M_L(T)=\{\A\in\M_L\mid\A\models T\}
$$
Stejně jako ve výrokové logice budeme často vynechávat jazyk $L$, bude-li zřejmý z kontextu, a budeme psát $M(\varphi_1,\dots,\varphi_n)$ místo $M(\{\varphi_1,\dots,\varphi_n\})$ a $M(T,\varphi)$ místo $M(T\cup\{\varphi\})$.

\subsection{Platnost v teorii}

Je-li $T$ teorie v jazyce $L$ a $\varphi$ $L$-formule, potom říkáme, že $\varphi$ je: 
\begin{itemize}
    \item \emph{pravdivá (platí) v $T$}, značíme $T\models\varphi$, pokud $\A\models\varphi$ pro všechna $\A\in\M(T)$ (neboli: $\M(T)\subseteq\M(\varphi)$),
    \item \emph{lživá v $T$}, pokud $T\models\neg\varphi$, tj. pokud je lživá v každém modelu $T$ (neboli: $\M(T)\cap\M(\varphi)=\emptyset$),
    \item \emph{nezávislá v $T$}, pokud není pravdivá v $T$ ani lživá v $T$.
\end{itemize}
Máme-li prázdnou teorii $T=\emptyset$ (tj. $\M(T)=\M_L$), potom teorii $T$ vynecháváme, píšeme $\models\varphi$, a říkáme, že $\varphi$ \emph{je pravdivá (v logice), (logicky) platí, je tautologie}; podobně pro ostatní pojmy.

Teorie je \emph{sporná}, jestliže v ní platí \emph{spor} $\bot$, který v predikátové logice můžeme definovat jako $R(x_1,\dots,x_n)\land \neg R(x_1,\dots,x_n)$, kde $R$ je libovolný (třeba první) relační symbol z jazyka nebo rovnost (nemá-li jazyk relační symbol, musí být s rovností). Teorie je \emph{sporná}, právě když v ní platí každá formule, nebo, ekvivalentně, právě když nemá žádný model. Jinak říkáme, že je teorie \emph{bezesporná} (neplatí-li v ní spor, ekvivalentně má-li alespoň jeden model).

\emph{Sentencím} pravdivým v $T$ říkáme \emph{důsledky} $T$; \emph{množina všech důsledků} $T$ v jazyce $L$ je:
$$
\Conseq_L(T)=\{\varphi\mid\text{$\varphi$ je sentence a }T\models \varphi\}
$$

\subsubsection{Kompletnost v predikátové logice }
Jak je tomu s pojmem \emph{kompletnosti} teorie?\footnote{Připomeňme, že \emph{výroková} teorie je kompletní, je-li bezesporná a každý výrok v ní buď platí, nebo platí jeho negace. Ekvivalentně, má právě 
jeden model.}

\begin{definition}
    Teorie je \emph{kompletní}, je-li bezesporná a každá \emph{sentence} je v ní buď pravdivá, nebo lživá.
\end{definition}

Nemůžeme ale říci, že je teorie kompletní, právě když má jediný model. Máme-li totiž jeden model, dostáváme z něj nekonečně mnoho jiných, ale \emph{izomorfních} modelů, tj. lišících se jen pojmenováním prvků univerza.\footnote{Formálně pojem \emph{izomorfismu} definujeme později v části o \emph{teorii modelů}, v Sekci \ref{section:isomorphism-of-structures}, jde ale o zobecnění izomorfismu který znáte z teorie grafů.} Uvažovat jediný model `až na izomorfismus' by ale nebylo dostatečné. Správným pojmem je tzv. \emph{elementární ekvivalence}:

\begin{definition}
    Struktury $\A,\B$ (v témž jazyce) jsou \emph{elementárně ekvivalentní}, pokud v nich platí tytéž sentence. Značíme $\A\equiv\B$.
\end{definition}

\begin{example}\label{example:elementary-equivalence-of-orders-R-Q}
    Příkladem struktur, které jsou elementárně ekvivalentní, ale ne izomorfní, jsou uspořádané množiny $\A=\langle\mathbb Q,\leq\rangle$ a $\B=\langle\mathbb R,\leq\rangle$. Izomorfní nejsou proto, že $\mathbb Q$ je spočetná zatímco $\mathbb R$ nespočetná množina, neexistuje tedy dokonce žádná \emph{bijekce} mezi jejich univerzy. Není těžké ukázat, že pro každou sentenci $\varphi$ platí $\A\models\varphi\Leftrightarrow\B\models\varphi$: indukcí podle struktury formule $\varphi$, jediný netriviální případ je existenční kvantifikátor, a klíčovou vlastností je \emph{hustota} obou uspořádání, tj. následující vlastnost:
    $$
    (x\leq y\land \neg x=y)\limplies(\exists z)(x\leq z\land z\leq y\land \neg x=z\land\neg y=z)
    $$

\end{example}
\begin{observation}
    Teorie je kompletní, právě když má právě jeden model až na elementární ekvivalenci.    
\end{observation}



\subsubsection{Platnost pomocí nesplnitelnosti}

Otázku pravdivosti (platnosti) v dané teorii lze převést na problém existence modelu:
\begin{proposition}[O nesplnitelnosti a pravdivosti]
    Je-li $T$ teorie a $\varphi$ \emph{sentence} (ve stejném jazyce), potom platí: $T\cup\{\neg\varphi\}$ nemá model, právě když $T\models\varphi$.
\end{proposition}
\begin{proof}
    Platí následující ekvivalence: $T\cup\{\neg\varphi\}$ nemá model, právě když $\neg\varphi$ neplatí v žádném modelu $T$, právě když (neboť je to sentence) $\varphi$ platí v každém modelu $T$. 
\end{proof}

Předpoklad, že $\varphi$ je sentence, je nutný: uvažte teorii $T=\{P(c)\}$ a formuli $\varphi=P(x)$ (což není sentence). Potom $\{P(c),\neg P(x)\}$ nemá model, ale $P(c)\not\models P(x)$. (Zde $P$ je unární relační, a $c$ konstantní symbol.)


\subsection{Příklady teorií}

Uveďme několik příkladů důležitých teorií.

\subsubsection{Teorie grafů}
\emph{Teorie grafů} je teorie v jazyce $L=\langle E\rangle$ s rovností, splňující axiomy \emph{ireflexivity} a \emph{symetrie}:
$$
T_\text{graph}=\{\neg E(x, x),E(x,y)\limplies E(y,x)\}
$$
Modely $T_\text{graph}$ jsou struktury $\mathcal G=\langle G,E^\mathcal G\rangle$, kde $E^\mathcal G$ je symetrická ireflexivní relace, jde tedy o tzv. \emph{jednoduché} grafy, kde hranu $\{x,y\}$ reprezentuje dvojice uspořádaných hran $(x,y),(y,x)$.
\begin{itemize}
    \item Formule $\neg x=y\limplies E(x,y)$ platí v grafu, právě když je \emph{úplný}. Je tedy nezávislá v $T_\text{graph}$.
    \item Formule $(\exists y_1)(\exists y_2)(\neg y_1=y_2\land E(x,y_1)\land E(x,y_2)\land (\forall z)(E(x,z)\limplies z=y_1\lor z=y_2)$ vyjadřuje, že každý vrchol má stupeň právě 2. Platí tedy právě v grafech, které jsou disjunktní sjednocení kružnic, a je nezávislá v teorii $T_\text{graph}$.
\end{itemize}


\subsubsection{Teorie uspořádání}

\emph{Teorie uspořádání} je teorie v jazyce uspořádání $L=\langle\leq\rangle$ s rovností, jejíž axiomy jsou:
\begin{align*}
    T=\{& x\leq x,\\
        & x\leq y\land y\leq x\limplies x=y,\\
        & x\leq y\land y\leq z\limplies x\leq z\}\\
\end{align*}
Těmto axiomům říkáme \emph{reflexivita}, \emph{antisymetrie}, \emph{tranzitivita}. Modely $T$ jsou $L$-struktury $\langle S,\leq^S\rangle$, ve kterých platí axiomy $T$, tzv. \emph{(částečně) uspořádané množiny}. Např: $\A=\langle\mathbb N,\leq\rangle$, $\B=\langle\mathcal P(X),\subseteq\rangle$ pro $X=\{0,1,2\}$.
\begin{itemize}
    \item Formule $x\leq y\lor y\leq x$ (\emph{linearita}) platí v $\A$, ale neplatí v $\B$, neboť neplatí např. při ohodnocení kde $e(x)=\{0\}$, $e(y)=\{1\}$ (píšeme $\B\not\models\varphi[e]$). Je tedy nezávislá v $T$.
    \item Sentence $(\exists x)(\forall y)(y\leq x)$ (označme ji $\psi$) je pravdivá v $\B$ a lživá v $\A$, píšeme $\B\models\psi$, $\A\models\neg\psi$. Je tedy také nezávislá v $T$.
    \item Formule $(x\leq y\land y\leq z\land z\leq x)\limplies (x=y\land y=z)$ (označme ji $\chi$) je pravdivá v $T$, píšeme $T\models\chi$. Totéž platí pro její \emph{generální uzávěr} $(\forall x)(\forall y)(\forall z)\chi$.
\end{itemize}

\subsubsection{Algebraické teorie}

\begin{itemize}
    \item \emph{Teorie grup} je teorie v jazyce $L=\langle +,-,0\rangle$ s rovností, jejíž axiomy jsou:
    \begin{align*}
        T_1=\{& x + (y + z) = (x + y) + z,\\
            & 0 + x = x,\ x + 0 = 0,\\
            & x + (-x) = 0,\ (-x) + x = 0\}\\
    \end{align*}
    Těmto vlastnostem říkáme \emph{asociativita $+$}, \emph{neutralita $0$ vůči $+$}, a \emph{$-x$ je inverzní prvek k $x$ (vůči $+$ a $0$)}.
    \item \emph{Teorie komutativních grup} má navíc axiom $x+y=y+x$ (\emph{komutativita $+$}), je tedy: 
    $$
    T_2=T_1\cup\{x+y=y+x\}
    $$
    \item \emph{Teorie okruhů} je v jazyce $L=\langle +,-,0,\cdot,1\rangle$ s rovností, má navíc axiomy:
    \begin{align*}
        T_3=T_2\cup\{   & 1 \cdot x = x \cdot 1,\\
        & x \cdot (y \cdot z) = (x \cdot y) \cdot z,\\
        & x \cdot (y + z) = x \cdot y + x \cdot z,\\
        & (x + y) \cdot z = x \cdot z + y \cdot z\}
    \end{align*}
    Těmto vlastnostem říkáme \emph{neutralita $1$ vůči $\cdot$}, \emph{asociativita $\cdot$}, a \emph{(levá i pravá) distributivita $\cdot$ vůči $+$}.
    \item \emph{Teorie komutativních okruhů} má navíc axiom \emph{komutativity $\cdot$}, máme tedy:
    $$
    T_4 = T_3 \cup \{x \cdot y = y \cdot x\}
    $$
    \item \emph{Teorie těles} je ve stejném jazyce, ale má navíc axiomy \emph{existence inverzního prvku k $\cdot$} a \emph{netriviality}:
    $$
    T_5 = T_4 \cup \{\neg\,x=0 \limplies (\exists y)(x\cdot y = 1), \neg\,0=1\}
    $$
    \item \emph{Teorie uspořádaných těles} je v jazyce $\langle +, -, 0,\cdot,1,\leq\rangle$ s rovností, sestává z axiomů teorie těles, teorie uspořádání spolu s axiomem linearity, a z následujících axiomů \emph{kompatibility uspořádání}: $x\leq y\limplies (x+z\leq y+z)$ a $(0\leq x\land 0\leq y)\limplies 0\leq x\cdot y$. (Modely jsou tedy tělesa s \emph{lineárním (totálním)} uspořádáním, které je kompatibilní s tělesovými operacemi v tomto smyslu.)
\end{itemize}


\section{Podstruktura, expanze, redukt}

V této sekci se podíváme na způsoby, jak můžeme vytvářet nové struktury z existujících.

\subsubsection{Podstruktura}

Pojem \emph{podstruktury} zobecňuje podgrupy, podprostory vektorového tělesa, a indukované podgrafy grafu: vybereme nějakou podmnožinu $B$ univerza struktury $\A$, a vytvoříme na ní strukturu $\B$ stejné signatury, která `zdědí' relace, operace, a konstanty. Abychom to mohli provést, potřebujeme, aby byla množina $B$ \emph{uzavřená} na všechny operace a obsahovala všechny konstanty.\footnote{Stejně jako ne každá množina vektorů je podprostor, k tomu musí obsahovat nulový vektor, ke každému vektoru obsahovat všechny jeho skalární násobky, a pro každou dvojici vektorů obsahovat jejich součet. Jinými slovy, jen (neprázdné) množiny uzavřené na \emph{lineární kombinace} vektorů dávají vzniknout podprostorům.}

\begin{definition}[Podstruktura]
Mějme strukturu $\A=\langle A,\mathcal R^\mathcal A,\mathcal F^\mathcal A\rangle$ v signatuře $\langle\mathcal R,\mathcal F\rangle$. Struktura $\B=\langle B,\mathcal R^\mathcal B,\mathcal F^\mathcal B\rangle$ je \emph{(indukovaná) podstruktura $\A$}, značíme $\B\subseteq\A$, jestliže
\begin{itemize}
    \item $\emptyset\neq B\subseteq A$,
    \item $R^\B=R^\A\cap B^{\mathrm{ar(R)}}$ pro každý relační symbol $R\in \mathcal R$,
    \item $f^\B=f^\A\cap (B^{\mathrm{ar(f)}}\times B)$ pro každý funkční symbol $f\in \mathcal F$ (tj. funkce $f^\B$ je restrikce $f^\A$ na množinu $B$, a její výstupy jsou všechny také z $B$),
    \item speciálně, pro každý konstantní symbol $c\in\mathcal F$ máme $c^\B=c^\A\in B$.
\end{itemize}
\end{definition}
Množina $C\subseteq A$ je \emph{uzavřená} na funkci $f:A^n\to A$, pokud $f(x_1,\dots,x_n)\in C$ pro všechna $x_i\in C$. Platí:
\begin{observation}
    Množina $\emptyset\neq C\subseteq A$ je univerzem podstruktury struktury $\A$, právě když je $C$ uzavřená na všechny funkce struktury $\A$ (včetně konstant).
\end{observation}
V tom případě říkáme této podstruktuře \emph{restrikce} $\A$ na množinu $C$, a značíme ji $\A\restriction C$.

\begin{example}
    $\underline{\mathbb Z}=\langle Z,+,\cdot,0\rangle$ je podstrukturou $\underline{\mathbb Q}=\langle Q,+,\cdot,0\rangle$, můžeme psát $\underline{\mathbb Z}=\underline{\mathbb Q}\restriction\mathbb Z$. Struktura $\underline{\mathbb N}=\langle N,+,\cdot,0\rangle$ je podstrukturou obou těchto struktur, $\underline{\mathbb N}=\underline{\mathbb Q}\restriction\mathbb N=\underline{\mathbb Z}\restriction\mathbb N$.
\end{example}

\subsubsection{Platnost v podstruktuře}

Jak je tomu s platností formulí v podstruktuře? Uveďme několik jednoduchých pozorování o \emph{otevřených} formulích.

\begin{observation}
    Je-li $\B\subseteq\A$, potom pro každou \emph{otevřenou} formuli $\varphi$ a ohodnocení proměnných $e\colon\Var\to B$ platí: $\B\models\varphi[e]$ právě když $\A\models\varphi[e]$.
\end{observation}
\begin{proof}
    Pro atomické formule je zřejmé, dále snadno dokážeme indukcí podle struktury formule.
\end{proof}

\begin{corollary}
    \emph{Otevřená} formule platí ve struktuře $\A$, právě když platí v každé podstruktuře $\B\subseteq\A$.
\end{corollary}

Říkáme, že teorie $T$ je \emph{otevřená}, jsou-li všechny její axiomy otevřené formule.

\begin{corollary}
    Modely otevřené teorie jsou uzavřené na podstruktury, tj. každá podstruktura modelu otevřené teorie je také model této teorie.
\end{corollary}

\begin{example}
    Teorie grafů je otevřená. Každá podstruktura grafu (modelu teorie grafů) je také graf, říkáme mu (indukovaný) \emph{podgraf}.\footnote{Samotný pojem \emph{podgraf} v teorii grafů často znamená jen $E^\B\subseteq B\times B$, nikoliv $E^\B=B\times B$. My ale budeme používat slovo \emph{podgraf} ve striktnějším smyslu, jako indukovaný podgraf.} Podobně např. pro podgrupy nebo Booleovy podalgebry.
\end{example}

\begin{example}
    Teorie těles není otevřená. Jak si ukážeme později, není dokonce ani \emph{otevřeně axiomatizovatelná}, tj. neexistuje jí ekvivalentní otevřená teorie -- kvantifikátoru v axiomu o existenci inverzního prvku se nelze nijak zbavit. Podstruktura tělesa reálných čísel $\mathbb Q$ na množině všech celých čísel $\mathbb Q\restriction\mathbb Z$ není těleso. (Je to tzv. \emph{okruh}, ale nenulové prvky kromě $1,-1$ nemají multiplikativní inverz, např. rovnice $2\cdot x=1$ nemá v $\mathbb Z$ řešení).
\end{example}

\subsubsection{Generovaná podstruktura}

Co dělat, máme-li podmnožinu univerza, která \emph{není} uzavřená na operace struktury? V tom případě uvážíme \emph{uzávěr} této množiny na operace.\footnote{Viz pojem \emph{lineárního obalu} množiny vektorů.}

\begin{definition}
    Mějme strukturu $\A=\langle A,\mathcal R^\mathcal A,\mathcal F^\mathcal A\rangle$ a neprázdnou podmnožinu $X\subseteq A$. Označme jako $B$ nejmenší podmnožinu $A$, která obsahuje množinu $X$ a je uzavřená na všechny funkce struktury $\A$ (tj. také obsahuje všechny konstanty). Potom o podstruktuře $\A\restriction B$ říkáme, že je \emph{generovaná} množinou $X$, a značíme ji $\A\langle X\rangle$.
\end{definition}

\begin{example}
    Uvažme struktury $\underline{\mathbb Q}=\langle Q,+,\cdot,0\rangle$, $\underline{\mathbb Z}=\langle Z,+,\cdot,0\rangle$, a $\underline{\mathbb N}=\langle N,+,\cdot,0\rangle$. Potom $\underline{\mathbb Q}\langle\{1\}\rangle=\underline{\mathbb N}$, $\underline{\mathbb Q}\langle\{-1\}\rangle=\underline{\mathbb Z}$, a $\underline{\mathbb Q}\langle\{2\}\rangle$ je podstruktura $\underline{\mathbb N}$ na množině všech sudých čísel.
\end{example}

\begin{example}
    Pokud $\A$ nemá žádné operace (ani konstanty), např. je-li to graf či uspořádání, potom není čím generovat, a $\A\langle X\rangle=\A\restriction X$.
\end{example}

\subsubsection{Expanze a redukt}

Prozatím jsme konstruovali nové struktury změnou univerza. Můžeme ale také nechat univerzum stejné, a přidat resp. odebrat relace, operace, a konstanty. Výsledku takové operace říkáme \emph{expanze} resp. \emph{redukt}. Všimněte si, že jde o strukturu v jiné signatuře.

\begin{definition}[Expanze a redukt]
    Mějme jazyky $L\subseteq L'$, $L$-strukturu $\A$, a $L'$-strukturu $\A'$ na stejné doméně $A=A'$. Jestliže je interpretace každého symbolu [relačního, funkčního, konstantního] stejná [relace, funkce, konstanta] v $\A$ i v $\A'$ potom říkáme, že struktura $\A'$ je \emph{expanzí} struktury $\A$ do jazyka $L'$ (také říkáme, že je \emph{$L'$-expanzí}) a že struktura $\A$ je \emph{reduktem} struktury $\A'$ na jazyk $L$ (také říkáme, že je \emph{$L$-reduktem}).    
\end{definition}

\begin{example}
    Mějme grupu celých čísel $\langle\mathbb Z,+,-,0\rangle$. Potom struktura $\langle \mathbb Z,+\rangle$ je jejím reduktem, zatímco struktura $\langle\mathbb Z,+,-,0,\cdot,1\rangle$ (\emph{okruh} celých čísel) je její expanzí.
\end{example}

\begin{example}
    Mějme graf $\mathcal G=\langle G, E^\mathcal G\rangle$. Potom struktura $\langle G, E^G,c_v^\mathcal G\rangle_{v\in G}$ v jazyce $\langle E,c_v\rangle_{v\in G}$, kde $c_v^\mathcal G=v$ pro všechny vrcholy $v\in G$, je \emph{expanzí $\mathcal G$ o jména prvků (z množiny G)}.
\end{example}


\subsection{Věta o konstantách}

\emph{Věta o konstantách} říká (neformálně), že splnit formuli s volnou proměnnou je totéž, co splnit sentenci, ve které je tato volná proměnná nahrazena (substituována) \emph{novým} konstantním symbolem (který není nijak svázaný žádnými axiomy). Klíčem je fakt, že tento nový symbol může být v modelech interpretován jako libovolný (tj. každý) prvek. Tento trik později využijeme v tablo metodě.

\begin{theorem}[O konstantách]\label{theorem:on-constants}
    Mějme formuli $\varphi$ v jazyce $L$ s volnými proměnnými $x_1,\dots,x_n$. Označme $L'$ rozšíření jazyka o nové konstantní symboly $c_1,\dots,c_n$ a buď $T'$ stejná teorie jako $T$ ale v jazyce $L'$. Potom platí:
    $$
    T\models\varphi\text{ právě když }T'\models\varphi(x_1/c_1,\dots,x_n/c_n)
    $$
\end{theorem}
\begin{proof}
    Tvrzení stačí dokázat pro jednu volnou proměnnou $x$ a jednu konstantu $c$, indukcí se snadno rozšíří na $n$ konstant.
    
    Předpokládejme nejprve, že $\varphi$ platí v každém modelu teorie $T$. Chceme ukázat, že $\varphi(x/c)$ platí v každém modelu $\A'$ teorie $T'$. Vezměme tedy takový model $\A'$ a libovolné ohodnocení $e\colon\Var\to A$ a ukažme, že $\A'\models\varphi(x/c)[e]$.

    Označme jako $\A$ redukt $\A'$ na jazyk $L$ (`zapomeneme' konstantu $c^{\A'}$). Všimněte si, že $\A$ je model teorie $T$ (axiomy $T$ jsou tytéž jako $T'$, neobsahují symbol $c$) tedy v něm platí $\varphi$. Protože dle předpokladu platí $\A\models\varphi[e']$ pro \emph{libovolné} ohodnocení $e'$, platí i pro ohodnocení $e(x/c^{\A'})$ ve kterém ohodnotíme proměnnou $x$ interpretací konstantního symbolu $c$ ve struktuře $\A'$, máme tedy $\A\models\varphi[e(x/c^{\A'})]$. To ale znamená, že $\A'\models\varphi(x/c)[e]$, což jsme chtěli dokázat.
    
    Naopak, předpokládejme, že $\varphi(x/c)$ platí v každém modelu teorie $T'$ a ukažme, že $\varphi$ platí v každém modelu $\A$ teorie $T$. Zvolme tedy takový model $\A$ a nějaké ohodnocení $e\colon\Var\to A$ a ukažme, že $\A\models\varphi[e]$.

    Označme jako $\A'$ expanzi $\A$ do jazyka $L'$, kde konstantní symbol $c$ interpretujeme jako prvek $c^{\A'}=e(x)$. Protože dle předpokladu platí $\A'\models\varphi(x/c)[e']$ pro všechna ohodnocení $e'$, platí i $\A'\models\varphi(x/c)[e]$, což ale znamená, že $\A'\models\varphi[e]$. (Neboť $e=e(x/c)$ a $\A'\models\varphi(x/c)[e(x/c)]$ platí právě když $\A'\models\varphi[e(x/c)]$.) Formule $\varphi$ ale neobsahuje $c$ (zde používáme, že $c$ je \emph{nový}), máme tedy i $\A\models\varphi[e]$.
\end{proof}


\section{Extenze teorií}

Pojem \emph{extenze} teorie definujeme stejně jako ve výrokové logice:

\begin{definition}[Extenze teorie]
    Mějme teorii $T$ v jazyce $L$.
    \begin{itemize}
        \item \emph{Extenze} teorie $T$ je libovolná teorie $T'$ v jazyce $L'\supseteq L$ splňující $\Conseq_L(T)\subseteq\Conseq_{L'}(T')$,
        \item je to \emph{jednoduchá extenze}, pokud $L'=L$,
        \item je to \emph{konzervativní extenze}, pokud $\Conseq_L(T)=\Conseq_L(T')=\Conseq_{L'}(T')\cap \Fm_L$, kde $\Fm_L$ značí množinu všech formulí v jazyce $L$.
        \item Teorie $T'$ (v jazyce $L$) je \emph{ekvivalentní} teorii $T$, pokud je $T'$ extenzí $T$ a $T$ extenzí $T'$.
    \end{itemize}
\end{definition}

Podobně jako ve výrokové logice, pro teorie ve stejném jazyce platí následující sémantický popis těchto pojmů:

\begin{observation}
Mějme teorie $T,T'$ v jazyce $L$. Potom:
\begin{itemize}
    \item $T'$ je extenze $T$, právě když $\M_L(T')\subseteq\M_L(T)$.
    \item $T'$ je ekvivalentní s $T$, právě když $\M_L(T')=\M_L(T)$.
\end{itemize}
\end{observation}

Jak je tomu v případě, kdy teorie $T'$ je nad větším jazykem než $T$? Připomeňme situaci ve výrokové logice, popsanou v Pozorování \ref{observation:extensions-semantic-description-propositional}. Zformulujeme a dokážeme analogické tvrzení: Zatímco ve výrokové logice jsme přidávali hodnoty pro nové prvovýroky, resp. je zapomínali, v predikátové logice budeme expandovat resp. redukovat struktury, tj. přidávat nebo zapomínat interpretace relačních, funkčních, a konstantních symbolů. Princip tvrzení ale zůstává stejný.

\begin{proposition}\label{proposition:semantic-conditions-for-extensions}
    Mějme jazyky $L\subseteq L'$, teorii $T$ v jazyce $L$, a teorii $T'$ v jazyce $L'$.
    \begin{enumerate}[(i)]
        \item $T'$ je extenzí teorie $T$, právě když redukt každého modelu $T'$ na jazyk $L$ je modelem $T$.
        \item Pokud je $T'$ extenzí teorie $T$, a každý model $T$ lze expandovat do jazyka $L'$ na nějaký model teorie $T'$, potom je $T'$ je konzervativní extenzí teorie $T$.
    \end{enumerate}
\end{proposition}
\begin{remark}
    V části (ii) platí i opačná implikace, důkaz ale není tak jednoduchý, jako ve výrokové logice, a proto ho neuvedeme. (Problémem je jak získat z modelu $T$ který nelze expandovat na model $T'$ $L$-sentenci, která platí v $T$ ale ne v $T'$.)
\end{remark} 
%\todo ukázat i opačnou implikaci?
%$T'$ je konzervativní extenzí teorie $T$, právě když je extenzí, a každý model $T$ lze expandovat do jazyka $L'$ na nějaký model teorie $T'$.
\begin{proof}
    Nejprve dokažme (i): Mějme model $\A'$ teorie $T'$ a označme jako $\A$ jeho redukt na jazyk $L$. Protože $T'$ je extenzí teorie $T$, platí v $T'$, a tedy i v $\A'$, každý axiom $\varphi\in T$. Potom ale i $\A\models\varphi$ ($\varphi$ obsahuje jen symboly z jazyka $L$), tedy $\A$ je modelem $T$.

    Na druhou stranu, mějme $L$-sentenci $\varphi$ takovou, že $T\models\varphi$. Chceme ukázat, že $T'\models\varphi$. Pro libovolný model $\A'\in\M_{L'}(T')$ víme, že jeho $L$-redukt $\A$ je modelem $T$, tedy $\A\models\varphi$. Z toho plyne i $\A'\models\varphi$ (opět proto, že $\varphi$ je v jazyce $L$).

    Nyní (ii): Vezměme libovolnou $L$-sentenci $\varphi$, která platí v teorii $T'$, a ukažme, že platí i v $T$. Každý model $\A$ teorie $T$ lze expandovat na nějaký model $\A'$ teorie $T'$. Víme, že $\A'\models\varphi$, takže i $\A\models\varphi$. Tím jsme dokázali, že $T\models\varphi$, tj. jde o konzervativní extenzi.
    
\end{proof}


\subsection{Extenze o definice}\label{subsection:extension-by-definition}

Nyní si ukážeme speciální druh konzervativní extenze, tzv. extenzi \emph{o definice} nových (relačních, funkčních, konstantních) symbolů.


\subsubsection*{Definice relačního symbolu}

Nejjednodušším případem je definování nového relačního symbolu $R(x_1,\dots,x_n)$. Jako definice může sloužit libovolná formule s $n$ volnými proměnnými $\psi(x_1,\dots,x_n)$.

\begin{example}
Uveďme nejprve několik příkladů:
\begin{itemize}
    \item Jakoukoliv teorii v jazyce s rovností můžeme rozšířit o binární relační symbol $\neq$, který \emph{definujeme} formulí $\neg x_1=x_2$. To znamená, že požadujeme, aby platilo: $x_1\neq x_2\liff\neg x_1=x_2$.
    \item Teorii uspořádání můžeme rozšířit o symbol $<$ pro ostré uspořádání, který \emph{definujeme} formulí $x_1\leq x_2\land \neg x_1=x_2$. To znamená, že požadujeme, aby platilo $x_1<x_2 \liff x_1\leq x_2\land \neg x_1=x_2$.
    \item V aritmetice můžeme zavést symbol $\leq$, pomocí $x_1\leq x_2\liff(\exists y)(x_1+y=x_2)$.
\end{itemize}
\end{example}
Nyní uvedeme definici:
\begin{definition}[Definice relačního symbolu]
    Mějme teorii $T$ a formuli $\psi(x_1,\dots,x_n)$ v jazyce $L$. Označme jako $L'$ rozšíření jazyka $L$ o nový $n$-ární relační symbol $R$. \emph{Extenze teorie $T$ o definici $R$ formulí $\psi$} je $L'$-teorie:
    $$
    T'=T\cup\{R(x_1,\dots,x_n)\ \liff\ \psi(x_1,\dots,x_n)\}
    $$
\end{definition}
Všimněte si, že každý model $T$ lze \emph{jednoznačně} expandovat na model $T'$. Z Tvrzení \ref{proposition:semantic-conditions-for-extensions} potom ihned plyne následující:
\begin{corollary}
    $T'$ je konzervativní extenze $T$.
\end{corollary}

Ukážeme si ještě, že nový symbol lze ve formulích nahradit jeho definicí, a získat tak ($T'$-ekvivalentní) formuli v původním jazyce:

\begin{proposition}
    Pro každou $L'$-formuli $\varphi'$ existuje $L$-formule $\varphi$ taková, že $T'\models\varphi'\liff\varphi$.
\end{proposition}
\begin{proof}
    Je třeba nahradit atomické podformule s novým symbolem $R$, tj. tvaru $R(t_1,\dots,t_n)$. Takovou podformuli nahradíme formulí $\psi'(x_1/t_1,\dots,x_n/t_n)$, kde $\psi'$ je varianta $\psi$ zaručující substituovatelnost všech termů, tj. například přejmenujeme všechny vázané proměnné $\psi$ na zcela nové (nevyskytující se ve formuli $\varphi'$).
\end{proof}

\subsubsection*{Definice funkčního symbolu}

Nový funkční symbol definujeme obdobným způsobem, musíme si ale být jisti, že definice dává jednoznačnou možnost, jak nový symbol interpretovat.

\begin{example}
    Opět začneme příklady:
    \begin{itemize}
        \item V teorii grup můžeme zavést \emph{binární} funkční symbol $-_b$ pomocí $+$ a unárního $-$ takto:
        $$
        x_1 -_b x_2 = y\ \liff\ x_1 + (-x_2) = y
        $$
        Je zřejmé, že pro každá $x,y$ \emph{existuje} \emph{jednoznačné} $z$ splňující definici.        
        \item Uvažme teorii \emph{lineárních uspořádání}, tj. teorii uspořádání spolu s axiomem linearity $x\leq y\lor y\leq x$. Definujme binární funkční symbol $\min$ takto: 
        $$
        \min(x_1,x_2)=y\ \liff\ y\leq x_1\land y\leq x_2\land (\forall z)(z\leq x_1\land z\leq x_2\limplies z\leq y)
        $$
        Existence a jednoznačnost platí díky linearitě. Pokud bychom ale měli pouze teorii uspořádání, taková formule by nebyla dobrou definicí: v některých modelech by $\min(x_1,x_2)$ pro některé prvky neexistovalo, selhala by tedy požadovaná \emph{jednoznačnost}.
    \end{itemize}
\end{example}

\begin{definition}[Definice funkčního symbolu]
    Mějme teorii $T$ a formuli $\psi(x_1,\dots,x_n,y)$ v jazyce $L$. Označme jako $L'$ rozšíření jazyka $L$ o nový $n$-ární funkční symbol $f$. Nechť v teorii $T$ platí:
    \begin{itemize}
        \item \emph{axiom existence} $(\exists y)\psi(x_1,\dots,x_n,y)$, 
        \item \emph{axiom jednoznačnosti} $\psi(x_1,\dots,x_n,y)\land \psi(x_1,\dots,x_n,z)\limplies y=z$.
    \end{itemize}
    Potom \emph{extenze teorie $T$ o definici $f$ formulí $\psi$} je $L'$-teorie: 
    $$T'=T\cup\{f(x_1,\dots,x_n)=y\ \liff\ \psi(x_1,\dots,x_n,y)\}$$
\end{definition}

Formule $\psi$ tedy definuje v každém modelu $(n+1)$-ární relaci, a po této relaci požadujeme, aby byla funkcí, tj. aby pro každou $n$-tici prvků existovala jednoznačná možnost, jak ji rozšířit do $(n+1)$-tice, která je prvkem této relace. Všimněte si, že je-li definující formule $\psi$ tvaru $t(x_1,\dots,x_n)=y$, kde $x_1,\dots,x_n$ jsou proměnné $L$-termu $t$, potom axiomy existence a jednoznačnosti vždy platí.

Opět platí, že každý model $T$ lze \emph{jednoznačně} expandovat na model $T'$, tedy:
\begin{corollary}
    $T'$ je konzervativní extenze $T$.
\end{corollary}

A platí také analogické tvrzení o rozvádění definic:

\begin{proposition}
    Pro každou $L'$-formuli $\varphi'$ existuje $L$-formule $\varphi$ taková, že $T'\models\varphi'\liff\varphi$.
\end{proposition}
\begin{proof}
    Stačí dokázat pro formuli $\varphi'$ s jediným výskytem symbolu $f$; je-li výskytů více, aplikujeme postup induktivně, v případě vnořených výskytů v jednom termu $f(\dots f(\dots)\dots)$ postupujeme od vnitřních k vnějším.

    Označme $\varphi^*$ formuli vzniklou z $\varphi'$ nahrazením termu $f(t_1,\dots,t_n)$ \emph{novou} proměnnou $z$. Formuli $\varphi$ zkonstruujeme takto:
    $$
    (\exists z)(\varphi^*\land \psi'(x_1/t_1,\dots,x_n/t_n,y/z))
    $$
    kde $\psi'$ je varianta $\psi$ zaručující substituovatelnost všech termů.

    Mějme model $\A$ teorie $T'$ a ohodnocení $e$. Označme $a=f^\A(t_1,\dots,t_n)[e]$. Díky existenci a jednoznačnosti platí:
    $$
    \A\models\psi'(x_1/t_1,\dots,x_n/t_n,y/z)[e]\ \text{ právě když }\ e(z)=a 
    $$
    Máme tedy $\A\models\varphi[e]$, právě když $\A\models\varphi^*[e(z/a)]$, právě když $\A\models\varphi'[e]$. To platí pro libovolné ohodnocení $e$, tedy $\A\models\varphi'\liff\varphi$ pro každý model $T'$, tedy $T'\models\varphi'\liff\varphi$.
\end{proof}

\subsubsection*{Definice konstantního symbolu}

Konstantní symbol je speciálním případem funkčního symbolu arity $0$. Platí tedy stejná tvrzení. Axiomy existence a jednoznačnosti jsou: $(\exists y)\psi(y)$ a $\psi(y)\land\psi(z)\limplies y=z$. A extenze o definici konstantního symbolu $c$ formulí $\psi(y)$ je teorie $T'=T\cup\{c=y\liff \psi(y)\}$.

\begin{example}
    Ukážeme si dva příklady:
    \begin{itemize}
        \item Teorii aritmetiky můžeme rozšířit o definici konstantního symbolu $1$ formulí $\psi(y)$ tvaru $y=S(0)$, přidáme tedy axiom $1=y\ \liff\ y=S(0)$.
        \item Uvažme teorii těles a nový symbol $\frac{1}{2}$, definovaný formulí $y\cdot (1+1)=1$, tj. přidáním axiomu: 
        $$
        \frac{1}{2}=y\ \liff\ y\cdot (1+1)=1
        $$
        Zde nejde o korektní extenzi o definici, neboť neplatí axiom existence. Ve dvouprvkovém tělese $\mathbb Z_2$ (a v každém tělese \emph{charakteristiky 2}) nemá rovnice $y\cdot (1+1)=1$ řešení, neboť $1+1=0$.
        
        
        Pokud ale vezmeme teorii těles charakteristiky různé od 2, tj. přidáme-li k teorii těles axiom $\neg (1+1=0)$, potom už půjde o korektní extenzi o definici. Například v tělese $\mathbb Z_3$ máme $\frac{1}{2}^{\mathbb Z_3}=2$.
    \end{itemize}
\end{example}


\subsubsection*{Extenze o definice}

Máme-li $L$-teorii $T$ a $L'$-teorii $T'$, potom řekneme, že $T'$ je  \emph{extenzí} $T$ \emph{o definice}, pokud vznikla z $T$ postupnou extenzí o definice relačních a funkčních (příp. konstantních) symbolů. Vlastnosti, které jsme dokázali o extenzích o jeden symbol (ať už relační nebo funkční), se snadno rozšíří indukcí na více symbolů:

\begin{corollary}
   Je-li $T'$ extenze teorie $T$ o definice, potom platí:
   \begin{itemize}
    \item Každý model teorie $T$ lze jednoznačně expandovat na model $T'$.
    \item $T'$ je konzervativní extenze $T$.
    \item Pro každou $L'$-formuli $\varphi'$ existuje $L$-formule $\varphi$ taková, že $T'\models\varphi'\liff\varphi$.
   \end{itemize} 
\end{corollary}

Na závěr ještě jeden příklad, na kterém si ukážeme i rozvádění definic:
\begin{example}
    V teorii $T=\{(\exists y)(x+y=0),(x+y=0)\land(x+z=0)\limplies y=z\}$ jazyka $L=\langle +,0,\leq\rangle$ s rovností lze zavést $<$ a unární funkční symbol $-$ přidáním axiomů:
    \begin{align*}
        -x=y\ &\liff\ x+y=0\\
        x<y\ &\liff\ x\leq y\land\neg(x=y)
    \end{align*}
    Formule $-x<y$ (v jazyce $L'=\langle +,-,0,\leq,<\rangle)$ s rovností) je v této extenzi o definice ekvivalentní následující formuli:
    $$
    (\exists z)((z\leq y\land\neg(z=y))\land x+z=0)
    $$
\end{example}


\section{Definovatelnost ve struktuře}\label{section:definability}

Formuli s jednou volnou proměnnou $x$ můžeme chápat jako \emph{vlastnost} prvků. V dané struktuře taková formule \emph{definuje} množinu prvků, které tuto vlastnost splňují, tj. takových, že formule platí při ohodnocení $e$, ve kterém $e(x)=a$. Máme-li formuli se dvěma volnými proměnnými, definuje binární relaci, atp. Nyní tento koncept formalizujeme. Připomeňme, že zápis $\varphi(x_1,\dots,x_n)$ znamená, že $x_1,\dots,x_n$ jsou právě všechny volné proměnné formule $\varphi$.

\begin{definition}[Definovatelné množiny]
    Mějme formuli $\varphi(x_1,\dots,x_n)$ a strukturu $\A$ v témž jazyce. \emph{Množina definovaná formulí $\varphi(x_1,\dots,x_n)$ ve struktuře $\A$}, značíme $\varphi^\A(x_1,\dots,x_n)$, je:
    $$
    \varphi^\A(x_1,\dots,x_n)=\{(a_1,\dots,a_n)\in A^n\mid\A\models\varphi[e(x_1/a_1,\dots,x_n/a_n)]\}
    $$
\end{definition}
Zkráceně totéž zapíšeme také jako $\varphi^\A(\bar x)=\{\bar a\in A^n\mid\A\models\varphi[e(\bar x/\bar a)]\}$.

\begin{example} Uveďme několik příkladů:
    \begin{itemize}
        \item Formule $\neg(\exists y)E(x,y)$ definuje množinu všech \emph{izolovaných} vrcholů v daném grafu.
        \item Uvažme těleso reálných čísel $\underline{\mathbb R}$. Formule $(\exists y)(y\cdot y=x)\land\neg (x=0)$ definuje množinu všech kladných reálných čísel.
        \item Formule $x\leq y\land \neg (x=y)$ definuje v dané uspořádané množině $\langle S,\leq^S\rangle$ relaci \emph{ostrého uspořádání} $<^S$. 
    \end{itemize}
\end{example}

Často se také hodí mluvit o vlastnostech prvků relativně k jiným prvkům dané struktury. To nelze vyjádřit čistě syntakticky, ale můžeme za některé z volných proměnných dosadit prvky struktury jako \emph{parametry}. Zápisem $\varphi(\bar x,\bar y)$ myslíme, že formule $\varphi$ má volné proměnné $x_1,\dots,x_n,y_1,\dots,y_k$ (pro nějaká $n,k$).

\begin{definition}
    Mějme formuli $\varphi(\bar x,\bar y)$, kde $|\bar x|=n$ a $|\bar y|=k$, strukturu $\A$ v témž jazyce, a $k$-tici prvků $\bar b\in A^k$. \emph{Množina definovaná formulí $\varphi(\bar x,\bar y)$ s parametry $\bar b$ ve struktuře $\A$}, značíme $\varphi^{\A,\bar b}(\bar x,\bar y)$, je:
    $$
    \varphi^{\A,\bar b}(\bar x,\bar y)=\{\bar a\in A^n\mid\A\models\varphi[e(\bar x/\bar a,\bar y/\bar b)]\}
    $$
    Pro strukturu $\A$ a podmnožinu $B\subseteq A$ označíme $\mathrm{Df}^n(\A,B)$ množinu všech množin definovatelných ve struktuře $\A$ s parametry pocházejícími z $B$.
\end{definition}


\begin{example}
    Pro  $\varphi(x,y)=E(x,y)$ je $\varphi^{\mathcal G,v}(x,y)$ množina všech sousedů vrcholu $v$.
\end{example}

\begin{observation}
Množina $\mathrm{Df}^n(\A,B)$ je uzavřená na doplněk, průnik, sjednocení, a obsahuje $\emptyset$ a $A^n$. Jde tedy o podalgebru potenční algebry $\mathcal P(A^n)$.
\end{observation}

\subsection{Databázové dotazy}

Definovatelnost nachází přirozenou aplikaci v relačních databázích, např. ve známém dotazovacím jazyce SQL. \emph{Relační databáze} sestává z jedné nebo více \emph{tabulek}, někdy se jim říká \emph{relace}, řádky jedné tabulky jsou \emph{záznamy (records)}, nebo také \emph{tice (tuples)}. Jde tedy v principu o strukturu v čistě relačním jazyce. Představme si databázi obsahující dvě tabulky, Program a Movies, znázorněné na Obrázku \ref{figure:datbase}.

\begin{figure}[htbp]
\begin{multicols}{2}
    \ttfamily\small
    \begin{tabular}{lll}
        cinema         & title          & time   \\ \hline
        Atlas          & Forrest Gump   & 20:00  \\
        Lucerna        & Forrest Gump   & 21:00  \\
        Lucerna        & Philadelphia   & 18:30  \\
        \vdots         & \vdots         & \vdots
    \end{tabular}

    \begin{tabular}{lll}
        title          & director    & actor \\ \hline
        Forrest Gump   & R. Zemeckis & T. Hanks      \\
        Philadelphia   & J. Demme    & T. Hanks      \\
        Batman Returns & T. Burton   & M. Keaton     \\
        \vdots         & \vdots      & \vdots
    \end{tabular}    
\end{multicols}
\caption{Tabulky Program a Movies}
\label{figure:datbase}
\end{figure}

SQL dotaz ve své nejjednodušší formě (pomineme-li např. \emph{agregační funkce}) je v podstatě formule, a výsledkem dotazu je množina definovaná touto formulí (s parametry). Například, kdy a kde můžeme vidět film s Tomem Hanksem?

\begin{quote}
    \textbf{select} Program.cinema, Program.time \textbf{from} Program, Movies \textbf{where}\\ Program.title = Movies.title  \textbf{and} Movies.actor = `T. Hanks'   
\end{quote}

Výsledkem bude množina $\varphi^{\text{Database},\text{`T. Hanks'}} (x_\mathrm{cinema},x_\mathrm{time},y_\mathrm{actor})$ definovaná ve struktuře $\text{Database}=\langle D, \mathrm{Program}, \mathrm{Movies}\rangle$, kde $D=\{\text{`Atlas'},\text{`Lucerna'},\dots,\text{`M. Keaton'}\}$, s parametrem `T. Hanks' následující formulí $\varphi(x_\mathrm{cinema},x_\mathrm{time},y_\mathrm{actor})$ :
$$
(\exists y_\mathrm{title})(\exists y_\mathrm{director})(\mathrm{Program}(x_\mathrm{cinema},y_\mathrm{title},x_\mathrm{time}) \land \mathrm{Movies}(y_\mathrm{title},y_\mathrm{director},\text{`T. Hanks'}))
$$


\section{Vztah výrokové a predikátové logiky}
\label{section:relationship-propositional-predicate-logic}

Nyní si ukážeme, jak lze výrokovou logiku `simulovat' v logice predikátové, a to v teorii Booleových algeber. Nejprve představíme axiomy této teorie:

\begin{definition}[Booleovy algebry]
    \emph{Teorie Booleových algeber} je teorie jazyka $L=\langle -,\landsymb,\lorsymb,\bot,\top\rangle$ s rovností sestávající z následujících axiomů:\footnote{Všimněte si \emph{duality}: záměnou $\land$ s $\lor$ a $\bot$ s $\top$ získáme tytéž axiomy.}
    \begin{multicols}{2}
        \begin{itemize}
            \item \emph{asociativita} $\land$ a $\lor$:
            \begin{align*}
                x\land(y\land z) &=(x\land y)\land z\\
                x\lor(y\lor z) &=(x\lor y)\lor z
            \end{align*}
            \item \emph{komutativita} $\land$ a $\lor$:
            \begin{align*}
                x\land y &= y\land x\\
                x\lor y &= y\lor x
            \end{align*}
            \item \emph{distributivita} $\land$ vůči $\lor$ a $\lor$ vůči $\land$:
            \begin{align*}
                x\land(y\lor z) &= (x\land y)\lor (x\land z)\\
                x\lor(y\land z) &= (x\lor y)\land (x\lor z)
            \end{align*}
            \item \emph{absorpce}:
            \begin{align*}
                x\land(x\lor y) &= x\\
                x\lor(x\land y) &= x
            \end{align*}
            \item \emph{komplementace}:
            \begin{align*}
                x\land(-x) &= \bot \\
                x\lor(-x) &= \top
            \end{align*}
            \item \emph{netrivialita}:
            \begin{align*}
                \neg (\bot &= \top)
            \end{align*}

        \end{itemize}
    \end{multicols}
    Nejmenším modelem je \emph{2-prvková Booleova algebra} $\langle \{0,1\},f_\neg,f_\land,f_\lor,0,1\rangle$. Konečné Booleovy algebry jsou (až na \emph{izomorfismus}) právě $\langle \{0,1\}^n,f_\neg^n,f_\land^n,f_\lor^n,(0,\dots,0),(1,\dots,1)\rangle$, kde $f^n$ znamená, že funkci $f$ aplikujeme po složkách.\footnote{Tyto Booleovy algebry jsou izomorfní \emph{potenčním algebrám} $\mathcal P(\{1,\dots,n\})$, izomorfismus je daný bijekcí mezi podmnožinami a jejich charakteristickými vektory.}

    Výroky tedy můžeme chápat jako \emph{Booleovské termy} (a konstanty $\bot,\top$ představují pravdu a lež), pravdivostní hodnota výroku při daném ohodnocení prvovýroků je potom dána hodnotou odpovídajícího termu v 2-prvkové Booleově algebře.   Kromě toho, \emph{algebra výroků} daného výrokového jazyka nebo teorie je Booleovou algebrou (to platí i pro nekonečné jazyky).

    Na druhou stranu, máme-li \emph{otevřenou} formuli $\varphi$ (bez rovnosti), můžeme reprezentovat atomické výroky pomocí prvovýroků, a získat tak výrok, který platí, právě když platí $\varphi$. Více o tomto směru se dozvíme v Kapitole \ref{chapter:predicate-resolution} (o rezoluci v predikátové logice), kde se nejprve zbavíme kvantifikátorů pomocí tzv. \emph{Skolemizace}.

    Výrokovou logiku bychom také mohli zavést jako fragment logiky predikátové, pokud bychom povolili \emph{nulární relace} (a nulární relační symboly v jazyce): $A^0=\{\emptyset\}$, tedy na libovolné množině jsou právě dvě nulární relace $R^A\subseteq A^0$: $R^A=\emptyset=0$ a $R^A=\{\emptyset\}=\{0\}=1$. To ale dělat nebudeme. 
    
\end{definition}
