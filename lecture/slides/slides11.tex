\documentclass{beamer}

%% slide-specific

\usetheme[progressbar=frametitle]{metropolis}
%\usecolortheme{spruce}
%\metroset{block=fill}

% define Metropolis colors    
\definecolor{mAlert}{HTML}{EB811B}
\definecolor{mExample}{HTML}{14B03D}
\definecolor{mBlock}{HTML}{23373b}

% my blocks
\setlength\fboxsep{0pt}%

\newcommand{\myblock}[1]{\colorbox{mBlock!8}{\begin{minipage}{\linewidth}#1\end{minipage}}}
\newcommand{\myblockmath}[1]{\colorbox{mBlock!8}{\begin{minipage}{\linewidth}\vspace{-6pt}#1\end{minipage}}}
\newcommand{\myblockinline}[1]{\colorbox{mBlock!8}{#1}}
\newcommand{\myexample}[1]{\colorbox{mExample!8}{\begin{minipage}{\linewidth}#1\end{minipage}}}
\newcommand{\myexamplemath}[1]{\colorbox{mExample!8}{\begin{minipage}{\linewidth}\vspace{-6pt}#1\end{minipage}}}
\newcommand{\myexampleinline}[1]{\colorbox{mExample!8}{#1}}
\newcommand{\myalert}[1]{\colorbox{mAlert!8}{\begin{minipage}{\linewidth}#1\end{minipage}}}
\newcommand{\myalertmath}[1]{\colorbox{mAlert!8}{\begin{minipage}{\linewidth}\vspace{-6pt}#1\end{minipage}}}
\newcommand{\myalertinline}[1]{\colorbox{mAlert!8}{#1}}

%% other
\newcommand{\mystructure}[1]{\mathcal{#1}}




%% packages
\usepackage{amsmath,amssymb,amsthm}
\usepackage{booktabs}
\usepackage[czech]{babel}
\usepackage{enumerate}
\usepackage{forest}
\usepackage{multicol}
% \usepackage{tcolorbox}
\usepackage{tikz}
    \usetikzlibrary{arrows.meta}
%\usepackage[unicode]{hyperref}
\usepackage[utf8x]{inputenc}
\usepackage{xfrac}

% %% theorems
% \theoremstyle{plain}
%     \newtheorem{theorem}{Věta}[section]
%     \newtheorem*{theorem-unnumbered}{Věta}
%     \newtheorem{proposition}[theorem]{Tvrzení}
%     \newtheorem{corollary}[theorem]{Důsledek}
%     \newtheorem{lemma}[theorem]{Lemma}
%     \newtheorem{observation}[theorem]{Pozorování}
% \theoremstyle{definition}
%     \newtheorem{definition}[theorem]{Definice}
%     \newtheorem*{algorithm}{Algoritmus}
% \theoremstyle{remark}
%     \newtheorem{remark}[theorem]{Poznámka}
%     \newtheorem{example}[theorem]{Příklad}
%     \newtheorem{exercise}{Cvičení}[chapter]
%     \newtheorem*{solution}{Řešení}

%% macros and definitions
\DeclareMathOperator{\Aut}{Aut}
\DeclareMathOperator{\Conseq}{Csq}
\DeclareMathOperator{\DeLO}{DeLO}
\DeclareMathOperator{\dom}{dom}
\DeclareMathOperator{\Fm}{Fm}
\DeclareMathOperator{\M}{M}
%\DeclareMathOperator{\Proof}{Proof}
\DeclareMathOperator{\rng}{rng}
\DeclareMathOperator{\Term}{Term}
\DeclareMathOperator{\Th}{Th}
\DeclareMathOperator{\Thm}{Thm}
\DeclareMathOperator{\Tree}{Tree}
\DeclareMathOperator{\Var}{Var}
\DeclareMathOperator{\VF}{VF}

\newcommand{\A}{\structure{A}}
\newcommand{\B}{\structure{B}}
\newcommand{\Con}{\mathit{Con}}
\newcommand{\disjointunion}{\mathbin{\dot{\sqcup}}}
\newcommand{\F}{\ensuremath{\mathrm{F}}}
\newcommand{\landsymb}{{\land}}
\newcommand{\lbin}{\mathbin{\square}}
\newcommand{\lbinsymb}{{\lbin}}
\newcommand{\liff}{\mathbin{\leftrightarrow}}
\newcommand{\liffsymb}{{\liff}}
\newcommand{\limplies}{\mathbin{\rightarrow}}
\newcommand{\limpliessymb}{{\limplies}}
\newcommand{\lorsymb}{{\lor}}
\newcommand{\Prf}{\mathit{Prf}}
\newcommand{\proves}{\vdash}
%\newcommand{\structure}[1]{\mathcal{#1}}
\newcommand{\todo}{[TODO]}
\newcommand{\T}{\ensuremath{\mathrm{T}}}
\newcommand{\union}{\mathbin{\cup}}


\title{Jedenáctá přednáška}
\subtitle{NAIL062 Výroková a predikátová logika}
\author{Jakub Bulín (KTIML MFF UK)}
% \institute{KTIML MFF UK}
\date{Zimní semestr 2023}


\begin{document}


\frame{\titlepage}


\begin{frame}{Jedenáctá přednáška}

    \textbf{Program}
        \begin{itemize}
            \item korektnost rezoluce
            \item lifting lemma a úplnost rezoluce
            \item LI-rezoluce a Prolog
            \item elementární ekvivalence
        \end{itemize}

    \textbf{Materiály}

        \href{https://github.com/jbulin-mff-uk/nail062/raw/main/lecture/lecture-notes/lecture-notes.pdf}{\alert{\textbf{Zápisky z přednášky}}}, Sekce 8.6-8.7 z Kapitoly 8, Sekce 9.1 z Kapitoly 9
        %todo Prolog jako samostatná sekce?

\end{frame}


\section{8.6 Korektnost a úplnost}


\begin{frame}{Korektnost rezolučního kroku}

    \myblock{
    \textbf{Tvrzení:}
    Mějme klauzule $C_1$, $C_2$ a jejich rezolventu $C$. Platí-li v~nějaké struktuře $\A$ klauzule $C_1$ a $C_2$, potom v ní platí i $C$.
    }

    \textbf{Důkaz:} Buď $C_1=C_1'\sqcup \{A_1,\dots,A_n\}$, $C_2=C_2'\sqcup \{\neg B_1,\dots,\neg B_m\}$, a $C=C_1'\sigma \cup C_2'\sigma$, kde $S\sigma=\{A_1\sigma\}$ (a $\sigma$ je
    nejobecnější). Klauzule jsou otevřené formule, proto platí i jejich instance:     
    $$
    \A\models C_1\sigma\ \text{ a }\ \A\models C_2\sigma
    $$     
    Po aplikaci unifikace máme: 
    \begin{align*}        
        C_1\sigma&=C_1'\sigma \cup \{A_1\sigma\}\\
        C_2\sigma&=C_2'\sigma \cup \{\neg A_1\sigma\}        
    \end{align*}
    Chceme ukázat, že $\A\models C[e]$ pro lib. ohodnocení $e$. 
    \begin{itemize}
        \item Je-li \alert{$\A\models A_1\sigma[e]$}, potom $\A\not\models\neg A_1\sigma[e]$ a musí \alert{$\A\models C_2'\sigma[e]$}. Tedy i $\A\models C[e]$. 
        \item Je-li \alert{$\A\not\models A_1\sigma[e]$}, musí být \alert{$\A\models C_1'\sigma[e]$} a opět $\A\models C[e]$. \hfill\qedsymbol
    \end{itemize}    

\end{frame}


\begin{frame}{Korektnost rezoluce}

    \myblock{
    \textbf{Věta (O korektnosti rezoluce):}
    Pokud je CNF formule $S$ rezolucí zamítnutelná, potom je nesplnitelná.
    }

    \medskip

    \textbf{Důkaz:}
        Víme, že $S\proves_R\square$, vezměme tedy nějaký rezoluční důkaz $\square$ z $S$. Kdyby existoval model $\A\models S$, díky korektnosti rezolučního pravidla bychom dokázali (indukcí podle délky důkazu) i $\A\models\square$, což ale není možné. \hfill\qedsymbol

\end{frame}


\begin{frame}{Lifting lemma}

    úplnost rezoluce dokážeme převedením na případ výrokové logiky: rezoluční důkaz `na úrovni VL' je možné `zvednout' na úroveň PL

    \medskip

    \myblock{
    \textbf{Lifting lemma:}
        Buďte $C_1$ a $C_2$ klauzule s disj. množ. proměnných, $C^*_1$ a $C^*_2$ jejich základní instance, $C^*$ rezolventa $C^*_1$ a $C^*_2$. Potom $C_1$ a $C_2$ mají rezolventu $C$ takovou, že $C^*$ je základní instance $C$.
    }

    (důkaz na příštím slidu)

    \bigskip

    \myblock{
    \textbf{Důsledek:}
        Buď $S$ CNF formule a označme $S^*$ množinu všech jejích základních instancí. Pokud $S^*\proves_R C^*$ pro nějakou základní klauzuli $C^*$ (`na úrovni VL'), potom existuje klauzule $C$ a základní substituce $\sigma$ taková, že $C^*=C\sigma$ a $S\proves_R C$ (`na úrovni PL').
    }
    
    \textbf{Důkaz:} Snadno z Lifting lemmatu indukcí dle délky důkazu.\hfill\qedsymbol

\end{frame}


\begin{frame}{Důkaz Lifting lemmatu}

    Nechť \alert{$C^*_1=C_1\tau_1$} a \alert{$C^*_2=C_2\tau_2$}, $\tau_1$ a $\tau_2$ zákl. substituce nesdílející žádnou proměnnou. Najdeme rezolventu $C$, že \alert{$C^*=C\tau_1\tau_2$}.

    Buď $C^*$ rezolventa $C_1^*$ a $C_2^*$ přes literál $P(t_1,\dots,t_k)$. Víme, že:
    \begin{align*}
        C_1&=C_1' \sqcup \{A_1,\dots,A_n\},\text{ kde }\{A_1,\dots,A_n\}\tau_1=\{P(t_1,\dots,t_k)\}\\
        C_2&=C_2' \sqcup \{\neg B_1,\dots,\neg B_m\},\{\neg B_1,\dots,\neg B_m\}\tau_2=\{\neg P(t_1,\dots,t_k)\}
    \end{align*}
    Tedy $(\tau_1\tau_2)$ unifikuje $S=\{A_1,\dots,A_n,B_1,\dots,B_m\}$. Buď $\sigma$ nejob. unifikace pro $S$ z Unifikačního algoritmu. Zvolme \alert{$C=C_1'\sigma \cup C_2'\sigma$}.

    \vspace{-24pt}
    
    \begin{align*}
        &\alert{C\tau_1\tau_2
        =} (C_1'\sigma \cup C_2'\sigma)\tau_1\tau_2
        =C_1'\sigma\tau_1\tau_2 \cup C_2'\sigma\tau_1\tau_2
        \textcolor{red}{=}C_1'\tau_1\tau_2 \cup C_2'\tau_1\tau_2\\ &
        \textcolor{blue}{=}C_1'\tau_1 \cup C_2'\tau_2
        =(C_1\setminus\{A_1,\dots,A_n\})\tau_1\cup (C_2\setminus\{\neg B_1,\dots,\neg B_m\})\tau_2\\
        &=(C_1^*\setminus\{P(t_1,\dots,t_k)\})\cup(C_2^*\setminus \{\neg P(t_1,\dots,t_k)\})\alert{=C^*}
    \end{align*}
    
    Zde \textcolor{red}{=} plyne z vlastnosti `navíc' Unif. algoritmu $(\tau_1\tau_2)=\sigma(\tau_1\tau_2)$, \\a \textcolor{blue}{=} z 
    toho, že jde o základní substituce nesdílející proměnnou.\hfill\qedsymbol    

\end{frame}


\begin{frame}{Úplnost rezoluce}

    \myblock{
    \textbf{Věta (O úplnosti rezoluce):}
        Je-li CNF formule $S$ nesplnitelná, potom je zamítnutelná rezolucí.
    }

    \medskip

    \textbf{Důkaz:}
    Množina $S^*$ všech základních instancí klauzulí z $S$ je také nesplnitelná (důsledek Herbrandovy věty). Úplnost \alert{výrokové} rezoluce dává $S^*\proves_R\square$ (`na úrovni VL'). 
    
    Z důsledku Lifting lemmatu dostáváme klauzuli $C$ a základní substituci $\sigma$ takové, že $C\sigma=\square$ a $S\proves_R C$ (`na úrovni PL'). 
    
    Ale protože prázdná klauzule $\square$ je instancí $C$, musí být $C=\square$. Tím jsme našli rezoluční zamítnutí $S\proves_R \square$.        
    \hfill\qedsymbol


    

\end{frame}



\section{8.7 LI-rezoluce}




\section{ČÁST III -- POKROČILÉ PARTIE}


\section{\sc Kapitola 9: Teorie modelů}


\section{9.1 Elementární ekvivalence}


\end{document}


