

\section{Důsledky korektnosti a úplnosti}

Stejně jako ve výrokové logice, Věty o korektnosti a úplnosti dohromady říkají, že \alert{dokazatelnost} je totéž, co \alert{platnost}. To nám umožňuje obdobně zformulovat syntaktické analogie sémantických pojmů a vlastností.

Analogií \alert{důsledků} jsou \alert{teorémy} teorie $T$:
$$
\Thm_L(T)=\{\varphi\mid \varphi\text{ je $L$-sentence a } T\proves\varphi\}
$$

\begin{corollary}[Dokazatelnost = platnost]\label{corollary:corollary-of-soundness-and-completeness-predicate}
    Pro libovolnou teorii $T$ a sentence $\varphi,\psi$ platí:
    \begin{itemize}
        \item $T\proves\varphi$ právě když $T\models\varphi$
        \item $\Thm_L(T)=\Conseq_L(T)$
    \end{itemize}
\end{corollary}

Platí například:
\begin{itemize}
    \item Teorie je \alert{sporná}, jestliže je v ní dokazatelný spor (tj. $T\proves\bot$).
    \item Teorie je \alert{kompletní}, jestliže pro každou sentenci $\varphi$ je buď $T\proves\varphi$ nebo $T\proves\neg\varphi$ (ale ne obojí, jinak by byla sporná).
    \item Věta o dedukci: Pro teorii $T$ a sentence $\varphi,\psi$ platí $T,\varphi\proves\psi$, právě když $T\proves\varphi\to\psi$.
\end{itemize}

Na závěr této sekce si ukážeme několik aplikací Vět o úplnosti a korektnosti.

\subsection{Löwenheim-Skolemova věta}\label{subsection:loewenheim-skolem-theorem}

\begin{theorem}[Löwenheim-Skolemova]
    Je-li $L$ spočetný jazyk bez rovnosti, potom každá bezesporná $L$-teorie má spočetně nekonečný model.
\end{theorem}

\begin{proof}
Vezměme nějaké dokončené (např. systematické) tablo z teorie $T$ s položkou $\F\bot$ v kořeni. Protože $T$ je bezesporná, není v ní dokazatelný spor, tedy tablo musí obsahovat bezespornou větev. Hledaný spočetně nekonečný model je $L$-redukt kanonického modelu pro tuto větev.
\end{proof}

K této větě se ještě vrátíme v Kapitole \ref{chapter:model-theory}, kde si ukážeme silnější verzi zahrnující i jazyky s rovností (v nich je kanonický model spočetný, ale může být i konečný).

\subsection{Věta o kompaktnosti}

Stejně jako ve výrokové logice platí Věta o kompaktnosti, stejný je i její důkaz:

\begin{theorem}[O kompaktnosti]\label{theorem:compactness-theorem-predicate}
    Teorie má model, právě když každá její konečná část má model.    
\end{theorem}
\begin{proof}
Model teorie je zřejmě modelem každé její části. Naopak, pokud $T$ nemá model, je sporná, tedy $T\proves\bot$. Vezměme nějaký \alert{konečný} tablo důkaz $\bot$ z $T$. K jeho konstrukci stačí konečně mnoho axiomů $T$, ty tvoří konečnou podteorii $T'\subseteq T$, která nemá model.
\end{proof}


\subsection{Nestandardní model přirozených čísel}

Na závěr této sekce si ukážeme, že existuje tzv. \alert{nestandardní model} přirozených čísel. Klíčem je Věta o kompaktnosti.
    
Nechť $\underline{\mathbb N}=\langle\mathbb N,S,+,\cdot,0,\leq\rangle$ je standardní model přirozených čísel. Označme $\Th(\underline{\mathbb N})$ množinu všech sentencí \alert{pravdivých} ve struktuře $\underline{\mathbb N}$ (tzv. \alert{teorii struktury} $\underline{\mathbb N}$). Pro $n\in \mathbb N$ definujme \alert{$n$-tý numerál} jako term $\underline n=S(S(\cdots (S(0)\cdots))$, kde $S$ je aplikováno $n$-krát.

Vezměme nový konstantní symbol $c$ a vyjádřeme, že je ostře větší než každý $n$-tý numerál:
$$
T=\Th(\underline{\mathbb N})\cup\{\underline n<c\mid n\in \mathbb N\}
$$
Všimněte si, že každá konečná část teorie $T$ má model. Z věty o kompaktnosti tedy plyne, že i teorie $T$ má model. Říkáme mu \alert{nestandardní model} (označme ho $\A$). Platí v něm tytéž sentence, které platí ve standardním modelu, ale zároveň obsahuje prvek $c^\A$, který je větší než každé $n\in \mathbb N$ (čímž zde myslíme hodnotu termu $\underline n$ v nestandardním modelu $\A$).
    

\section{(draft) Hilbertovský kalkulus v predikátové logice}
\todo

Na závěr kapitoly si ukážeme, jak lze adaptovat Hilbertův kalkulus, představený v Sekci \ref{section:hilbert-calculus-propositional} pro použití v predikátové logice. To není těžké, abychom se vypořádali s kvantifikátory, stačí přidat dvě nová schémata logických axiomů a jedno nové inferenční pravidlo. Opět si ukážeme korektnost tohoto dokazovacího systému, a jen zmíníme, že je také úplný.

% from slides:
\subsubsection*{Hilbertovský kalkul}
    \begin{itemize}
    \item základní logické spojky a kvantifikátory:\ \ $\neg$, $\to$, $(\forall x)$ (ostatní odvozené)
    
    \item dokazují se libovolné formule (nejen sentence)
    
    \item \mdef{logické axiomy} (\myblue{schémata} logických axiomů)
    \vspace{-2mm}\begin{align*}(i)& &\varphi &\to (\psi \to \varphi) \\
    (ii)& &(\varphi\to (\psi \to \chi))&\to ((\varphi \to \psi)\to(\varphi \to \chi))\qquad\qquad\qquad\qquad\phantom{\ } \\
    (iii)& &(\neg \varphi \to \neg \psi)&\to(\psi \to \varphi)\\
    (iv)& &(\forall x)\varphi &\to\varphi(x/t)\quad\quad\quad\ \ \ \text{ je-li $t$ substituovatelný za $x$ do $\varphi$}\\
    (v)& &(\forall x)(\varphi \to \psi)&\to(\varphi \to (\forall x)\psi)\quad\text{není-li $x$ volná proměnná ve $\varphi$}\\
    \end{align*}
    
    \vspace{-6mm}
    kde $\varphi$, $\psi$, $\chi$ jsou libovolné formule (daného jazyka), $t$ je libovolný term a
    \vspace{0.5mm}
    
    $x$ je libovolná proměnná.
    \smallskip
    
    \item je-li jazyk s rovností, mezi logické axiomy patří navíc \myblue{axiomy rovnosti}
    
    \item \mdef{odvozovací (deduktivní) pravidla}
    \vspace{-2mm}
    $$\frac{\varphi,\ \varphi \to \psi}{\psi}\quad\text{\myblue{(modus ponens)},}\qquad\frac{\varphi}{(\forall x)\varphi}\quad\text{\myblue{(generalizace)}}$$
    \end{itemize}
    
    
    
    %%%%%%%%%%%%%%%%%%%%%%%%%%%%%%%%%%%%%%%%%%%%%%%%%%%%%%5
    \subsubsection*{Pojem důkazu}
    \mdef{Důkaz} (\alert{Hilbertova stylu}) formule $\varphi$ z teorie $T$ je \myblue{konečná} posloupnost
    \smallskip
    
    $\varphi_0, \dots, \varphi_n=\varphi$ formulí taková, že pro každé $i\le n$
    \smallskip
    
    \begin{itemize}
    \item $\varphi_i$ je logický axiom nebo $\varphi_i \in T$ (axiom teorie), nebo
    \smallskip
    
    \item $\varphi_i$ lze odvodit z předchozích formulí pomocí odvozovacích pravidel.
    \end{itemize}
    \smallskip
    
    Formule $\varphi$ je \mdef{dokazatelná} v $T$, má-li důkaz z $T$, značíme $T \proves_{H} \varphi$.
    \bigskip
    
    {\bf \myblue{Věta} (o korektnosti Hilbertova kalkulu)}\ \ {\it  Pro každou teorii $T$ a formuli $\varphi$,\ \ $T\proves_H \varphi\ \Rightarrow\ T\models \varphi$.}
    \medskip
    
    {\it \myblue{Důkaz}}
    \begin{itemize}
    \item Je-li $\varphi\in T$ nebo logický axiom, je $T \models \varphi$ (logické axiomy jsou tautologie),
    \item jestliže $T \models \varphi$ a $T \models \varphi \to \psi$, pak $T \models \psi$, \alert{tj. modus ponens je korektní},
    \item jestliže $T \models \varphi$, pak $T \models (\forall x)\varphi$, \alert{tj. pravidlo generalizace je korektní},
    \item tedy každá formule vyskytující se v důkazu z $T$ platí v $T$. $\qed$
    \end{itemize}
    \medskip
    
    {\it \myblue{Poznámka}\ \ Platí i \myblue{úplnost}, tj. $T\models \varphi \Rightarrow T\proves_H \varphi$ pro každou teorii $T$ a formuli $\varphi$.}
% :from slides