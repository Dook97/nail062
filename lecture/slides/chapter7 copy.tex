
\section{(draft) Hilbertovský kalkulus v predikátové logice}
\todo

Na závěr kapitoly si ukážeme, jak lze adaptovat Hilbertův kalkulus, představený v Sekci \ref{section:hilbert-calculus-propositional} pro použití v predikátové logice. To není těžké, abychom se vypořádali s kvantifikátory, stačí přidat dvě nová schémata logických axiomů a jedno nové inferenční pravidlo. Opět si ukážeme korektnost tohoto dokazovacího systému, a jen zmíníme, že je také úplný.

% from slides:
\subsubsection*{Hilbertovský kalkul}
    \begin{itemize}
    \item základní logické spojky a kvantifikátory:\ \ $\neg$, $\to$, $(\forall x)$ (ostatní odvozené)
    
    \item dokazují se libovolné formule (nejen sentence)
    
    \item \mdef{logické axiomy} (\myblue{schémata} logických axiomů)
    \vspace{-2mm}\begin{align*}(i)& &\varphi &\to (\psi \to \varphi) \\
    (ii)& &(\varphi\to (\psi \to \chi))&\to ((\varphi \to \psi)\to(\varphi \to \chi))\qquad\qquad\qquad\qquad\phantom{\ } \\
    (iii)& &(\neg \varphi \to \neg \psi)&\to(\psi \to \varphi)\\
    (iv)& &(\forall x)\varphi &\to\varphi(x/t)\quad\quad\quad\ \ \ \text{ je-li $t$ substituovatelný za $x$ do $\varphi$}\\
    (v)& &(\forall x)(\varphi \to \psi)&\to(\varphi \to (\forall x)\psi)\quad\text{není-li $x$ volná proměnná ve $\varphi$}\\
    \end{align*}
    
    \vspace{-6mm}
    kde $\varphi$, $\psi$, $\chi$ jsou libovolné formule (daného jazyka), $t$ je libovolný term a
    \vspace{0.5mm}
    
    $x$ je libovolná proměnná.
    \smallskip
    
    \item je-li jazyk s rovností, mezi logické axiomy patří navíc \myblue{axiomy rovnosti}
    
    \item \mdef{odvozovací (deduktivní) pravidla}
    \vspace{-2mm}
    $$\frac{\varphi,\ \varphi \to \psi}{\psi}\quad\text{\myblue{(modus ponens)},}\qquad\frac{\varphi}{(\forall x)\varphi}\quad\text{\myblue{(generalizace)}}$$
    \end{itemize}
    
    
    
    %%%%%%%%%%%%%%%%%%%%%%%%%%%%%%%%%%%%%%%%%%%%%%%%%%%%%%5
    \subsubsection*{Pojem důkazu}
    \mdef{Důkaz} (\alert{Hilbertova stylu}) formule $\varphi$ z teorie $T$ je \myblue{konečná} posloupnost
    \smallskip
    
    $\varphi_0, \dots, \varphi_n=\varphi$ formulí taková, že pro každé $i\le n$
    \smallskip
    
    \begin{itemize}
    \item $\varphi_i$ je logický axiom nebo $\varphi_i \in T$ (axiom teorie), nebo
    \smallskip
    
    \item $\varphi_i$ lze odvodit z předchozích formulí pomocí odvozovacích pravidel.
    \end{itemize}
    \smallskip
    
    Formule $\varphi$ je \mdef{dokazatelná} v $T$, má-li důkaz z $T$, značíme $T \proves_{H} \varphi$.
    \bigskip
    
    {\bf \myblue{Věta} (o korektnosti Hilbertova kalkulu)}\ \ {\it  Pro každou teorii $T$ a formuli $\varphi$,\ \ $T\proves_H \varphi\ \Rightarrow\ T\models \varphi$.}
    \medskip
    
    {\it \myblue{Důkaz}}
    \begin{itemize}
    \item Je-li $\varphi\in T$ nebo logický axiom, je $T \models \varphi$ (logické axiomy jsou tautologie),
    \item jestliže $T \models \varphi$ a $T \models \varphi \to \psi$, pak $T \models \psi$, \alert{tj. modus ponens je korektní},
    \item jestliže $T \models \varphi$, pak $T \models (\forall x)\varphi$, \alert{tj. pravidlo generalizace je korektní},
    \item tedy každá formule vyskytující se v důkazu z $T$ platí v $T$. $\qed$
    \end{itemize}
    \medskip
    
    {\it \myblue{Poznámka}\ \ Platí i \myblue{úplnost}, tj. $T\models \varphi \Rightarrow T\proves_H \varphi$ pro každou teorii $T$ a formuli $\varphi$.}
% :from slides