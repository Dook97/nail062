\chapter{Historie logiky}\label{appendix:history}

Historii logiky jako vědního oboru\footnote{Viz \href{https://en.wikipedia.org/wiki/History_of_logic}{Wikipedia}.} lze velmi zhruba rozdělit do několika fází podle hlavní aplikační domény (a povolání většiny praktikujících logiků). Zde uvádíme jen několik nejdůležitějších milníků.


\subsection*{Logika ve filozofii (od 6.\ století př.\ n.\ l.)}

\begin{itemize}
    \item Helénistická filozofie: Aristotelés (384–-322 př.\ n.\ l.), základy predikátové logiky, kvantifikátory (\emph{všichni}/\emph{někteří}), proměnné (\( \alpha \), \( \beta \), \( \gamma \)) zastupující logické formule, dedukce ve formě sylogismů. Stoická škola (3.\ stol.\ př.\ n.\ l.) (výroková logika).

        \begin{quote}\it
            [Zásada vyloučeného sporu] ``Totéž nemůže zároveň náležet a nenáležet témuž a v témž vztahu.'' 
        \end{quote}        

        \begin{quote}\it
            Všichni lidé jsou smrtelní.\\
            Sókratés je člověk.\\
            Závěr: Sókratés je smrtelný.
        \end{quote}

    \item Islámská filozofie a teologie: Avicenna (980-–1037), induktivní uvažování, souvislost implikace a času (inspirace pro pozdější \emph{temporální} logiku).
        
        \begin{quote}\it      
            ``Každý, kdo popírá zásadu vyloučeného sporu, by měl být bit a pálen, dokud nepřizná, že být bit není totéž jako nebýt bit, a být pálen není totéž jako nebýt pálen.''
        \end{quote}
        
        \begin{quote}\it
            ``Bůh vidí celý řetězec příčin a důsledků zvenku (mimo čas), a proto si je vědom každé dílčí události (v čase)''\\  
        \end{quote}

\item Středověká filozofie a teologie: Ockham (1287?--1347), rozdíl mezi \emph{materiální} a \emph{logickou} implikací.
    
        \begin{quote}\it
            ``Si homo currit, Deus est.'' [Pokud člověk běží, Bůh existuje.]
        \end{quote}

        \begin{quote}\it
            ``Logika je ze všech [svobodných] umění ten nejužitečnější nástroj, bez kterého žádná věda nemůže být dokonale poznána.''
        \end{quote}

        \begin{quote}\it
            [Ockhamova břitva] ``Nic by nemělo být předkládáno bez udání důvodu, pokud to není samozřejmé nebo známé ze zkušenosti nebo dokázané autoritou Písma svatého.'' 
        \end{quote}

\end{itemize}


\subsection*{Logika v matematice}

\begin{itemize}
    
    \item kořeny: Thales (dedukce v geometrii), Pythagoras (koncept důkazu), Eukleidés (axiomatizace geometrie), Descartes (algebraizace geometrie)
    
    \item Leibniz (1679--90):  \emph{characteristica universalis} a \emph{calculus ratiocinator}, snaha o vytvoření univerzálního symbolického jazyka a kalkulu lidského myšlení.

        \begin{quote}\it
            ``Jediný způsob, jak napravit naše úvahy, je učinit je stejně hmatatelnými jako úvahy matematiků, abychom na první pohled našli naši chybu, a když mezi lidmi dojde ke sporům, můžeme jednoduše říci: Calculemus! [Počítejme!], bez dalších okolků, abychom viděli, kdo má pravdu.'' 
        \end{quote}
    
    \item algebraická škola (od 1847): Boole (\emph{Mathematical Analysis of Logic}, 1847; \emph{The Laws of Thought}, 1854), DeMorgan, Venn, algebraické zákony vyjadřující logické vztahy, Booleova algebra, booleovské funkce jako sémantika výroků.
    
        \begin{quote}\it
            Např.\ Distributivita konjunkce vůči disjunkci: 
            \[ 
                p \land (q \lor r) \leftrightarrow (p \land q) \lor (p \land r ) 
                \]
            nebo DeMorganovy zákony: 
            \begin{align*}
                \neg(p \land q) &\leftrightarrow \neg p \lor \neg q \\
                \neg(p \lor q) &\leftrightarrow \neg p \land \neg q                
            \end{align*} 
        \end{quote}

    \item logicismus (od 1872): snaha vyjádřit celou matematiku v logickém jazyce:

    \item Cantor (1878): naivní teorie množin, pro každou vlastnost \( \varphi(x) \) máme množinu \( \{x\mid\varphi(x)\} \).
            
    \item Frege: predikátové logiky, syntaxe pokus o axiomatizaci aritmetiky, teorie množin.
            
    \item Schröder: sémantika predikátové logiky, modely jsou \emph{struktury} (např.\ graf, grupa, těleso).
            
    \item Russel (1903): naivní teorie množin je sporná, tzv. \emph{Russelův paradox} ``Platí \( x \in x \) pro množinu \( x = \{y\mid \neg (y \in y)\}\)?'', známý také jako \emph{paradox holiče}.
        
        \begin{quote}\it
            ``Holič holí každého, kdo neholí sám sebe. Holí holič sám sebe?''
        \end{quote}
            
    \item Zermelo, Fraenkel (1908, 1922): axiomatizace ZFC teorie množin (`C' znamená `choice', tzv. \emph{axiom výběru}).
       
    \item \todo
\end{itemize}

\subsection*{Logika v teoretické informatice}

\textbf{\todo}

\subsection*{Logika v aplikované informatice}

\todo